\documentclass{article}
\usepackage[utf8]{inputenc}

\usepackage[T2A]{fontenc}
\usepackage[utf8]{inputenc}
\usepackage[russian]{babel}

\usepackage{tabularx}
\usepackage{amsmath}
\usepackage{pgfplots}
\usepackage{multienum}
\usepackage{geometry}
\geometry{
    left=1cm,right=1cm,top=2cm,bottom=2cm
}
\newcommand*\diff{\mathop{}\!\mathrm{d}}

\newtheorem{definition}{Определение}
\newtheorem{theorem}{Теорема}

\DeclareMathOperator{\sign}{sign}

\usepackage{hyperref}
\hypersetup{
    colorlinks, citecolor=black, filecolor=black, linkcolor=black, urlcolor=black
}

\title{Высшая математика}
\author{Лисид Лаконский}
\date{May 2023}

\begin{document}
\raggedright

\maketitle

\tableofcontents
\pagebreak

\section{Высшая математика — справочный материал к экзамену}

\subsection{Производные функции одной переменной, экстремумы, выпуклость-вогнутость, возрастание-убывание, касательные и оси}

\subsubsection{Производные функции одной переменной}

\paragraph{Свойства производных функций}

\begin{multienumerate}
    \mitemxx{$(c)' = 0$}{$(c u)' = c * u'$}
    \mitemxx{$(u \pm v)' = u' \pm v'$}{$(u * v)' = u' v + u v'$}
    \mitemx{$(\frac{u}{v})' = \frac{u' v - u v'}{v^2}$}
    \mitemx{Если $y = f(u), u = \phi(x)$, то $(f(\phi(x)))' = f'(u) * u'$. \\ Пример: $\cos 3x = -\sin 3x * 3 = -3\sin x$ \\ Еще один пример: $\tg^{2x} e^{x} = 2 \tg e^x * \frac{1}{\cos^2 e^x} * e^x$}
\end{multienumerate}

\paragraph{Таблица производных}

\begin{multienumerate}
    \mitemxx{$(u^{a})' = a * u^{a - 1} * u', a \in R$  \\
    $(\frac{1}{u}) = (u^{-1})' = -1 * \frac{1}{u^2} * u'$ \\
    $(\sqrt{u})' = (u^{\frac{1}{2}})'$ = $\frac{1}{2\sqrt{u}} * u'$}{$(a^{u}) = a^{u} * \ln a * u'$ \\
    $(e^{u})' = e^{u} * u'$}
    \mitemxx{$(\log_{a}{u})' = \frac{1}{u} \log_{a}{e} * u' = \frac{1}{u \ln a} * u'$ \\
    $(\ln u)' = \frac{1}{u} * u', (\ln |u|)' = \frac{1}{u} * u'$}{$(\sin u)' = \cos x$}
    \mitemxx{$(\cos u)' = -\sin x$}{$(\tg u)' = \frac{1}{\cos^2 u} * u'$}
    \mitemxx{$(\ctg u)' = - \frac{1}{\sin^2 u} * u'$}{$(\arcsin u)' = \frac{1}{\sqrt{1 - u^2}} * u'$}
    \mitemxx{$(\arccos u)' = - \frac{1}{\sqrt{1 - u^2}} * u'$}{$(\arctg u)' = \frac{1}{1 + u^2} * u'$}
    \mitemxx{$(\arcctg u)' = - \frac{1}{1 + u^2} * u'$}{$(\sinh u)' = \cosh u * u'$}
    \mitemxx{$(\cosh u)' = \sinh u * u'$}{$(\tanh u)' = \frac{1}{\cosh^2 u} * u'$}
    \mitemxx{$(\coth u)' = - \frac{1}{\sinh^2{u}} * u'$}{ ($u(x)^{v(x)})' = v(x) * u(x)^{v(x) - 1} * u'(x) + u(x)^{v(x)} * \ln u(x) * v'(x)$}
\end{multienumerate}

\subsubsection{Нахождение экстремумов функции одной переменной}

\begin{enumerate}
    \item Находим производную функции
    \item Приравниваем эту производную к нулю
    \item Находим значения переменной получившегося выражения
    \item Разбиваем этими значениями координатную прямую на промежутки (при этом не нужно забывать о точках разрыва, которые также надо наносить на прямую)
    \item Вычисляем, на каких из этих промежутков производная будет положительной, а на каких — отрицательной
\end{enumerate}

\subsubsection{Нахождение интервалов выпуклости и интервалов вогнутости}

Пусть функция $y = f(x)$ дважды дифференцируема на некотором интервале.

Тогда:

\begin{enumerate}
    \item Если вторая производная $f''(x) < 0$ на интервале, то график функции $f(x)$ \textbf{является выпуклым} на данном интервале
    \item Если вторая производная $f''(x) > 0$ на интервале, то график функции $f(x)$ \textbf{является вогнутым} на данном интервале
\end{enumerate}

\subsubsection{Возрастание и убывание функции на интервале}

\paragraph{Определение возрастающей функции}

\begin{enumerate}
    \item если производная функции $y=f(x)$ положительна для любого $x$ из интервала $X$, то функция возрастает на $X$
    \item если производная функции $y=f(x)$ отрицательна для любого $x$ из интервала $X$, то функция убывает на $X$
\end{enumerate}

Таким образом, чтобы определить промежутки возрастания и убывания функции необходимо:

\begin{enumerate}
    \item найти область определения функции
    \item найти производную функции
    \item решить неравенства $f'(x) > 0$ и $f'(x) < 0$ на области определения
    \item к полученным промежуткам добавить граничные точки, в которых функция определена и непрерывна
\end{enumerate}
    
\subsubsection{Касательная к графику функции}

\paragraph{Определение}

\begin{enumerate}
    \item Пусть функция $f: U(x_0) \subset R \rightarrow R$ определена в некоторой окрестности точки $x_0 \in R$, и дифференцируема в ней: $f \in D(x_0)$. Касательной прямой к графику функции $f$ в точке $x_0$ называется график линейной функции, задаваемый уравнением

    $$y = f(x_0) + f'(x_0)(x - x_0), \ \ x \in R$$
    \item Если функция $f$ имеет в точке $x_0$ бесконечную производную $f'(x_0) = \pm \infty$, то касательной прямой в этой точке называется вертикальная прямая, задаваемая уравнением
    $$x = x_0$$
\end{enumerate}

\paragraph{Замечание}

Прямо из определения следует, что график касательной прямой проходит через точку $(x_0, f(x_0))$. Угол $\alpha$ между касательной к кривой и осью Ох удовлетворяет уравнению

$$\tg \alpha = f'(x_0) = k$$

где $\tg$ обозначает тангенс, а $k$ — коэффициент наклона касательной. Производная в точке $x_0$ равна угловому коэффициенту касательной к графику функции $y = f (x)$ в этой точке. 

\subsubsection{Преобразование графиков функций}

\begin{center}
\begin{tabularx}{1\textwidth} {|X|X|}
 \hline
 Функция & Преобразование графика функции $y = f(x)$ \\
 \hline
 $y = f(x) + A$ & Параллельный перенос вдоль оси $OY$ на $A$ единиц вверх, если $А>0$, и на $|A|$ единиц вниз, если $А<0$ \\
 \hline
 $y = f(x - a)$ & Параллельный перенос вдоль оси $OX$ на $a$ единиц вправо, если $a > 0$, на $|a|$ единиц влево, если $a < 0$ \\
 \hline
 $y = k f(x)$ & Растяжение вдоль оси $OY$ относительно оси $OX$ в $k$ раз, если $k > 1$, и сжатие в $1/k$ раз, если $0 < k < 1$ \\
 \hline
 $y = f(k x)$ & Сжатие вдоль оси $OX$ относительно оси $OY$ в $k$ раз, если $k > 1$, и растяжение в $1/k$ раз, если $0 < k < 1$ \\
 \hline
 $y = -f(x)$ & Симметричное отражение относительно оси OX \\
 \hline
 $y = |f(x)|$ & Часть графика, расположенная ниже оси $OX$, симметрично отражается относительно этой оси, остальная его часть остается без изменения. \\
 \hline
 $y = f(-x)$ & Симметричное отражение относительно оси OY \\
 \hline
 $y = f(|x|)$ & Часть графика, расположенная правее оси $OX$, симметрично отражается относительно этой оси, остальная его часть остается без изменения \\
 \hline
\end{tabularx}
\end{center}

\pagebreak
\subsection{Неопределенные интегралы}

$\int f(x) \diff x = F(x) + C$ — \textbf{неопределенный интеграл}, где $f(x)$ называется \textbf{подинтегральной функцией}, а $x$ называется \textbf{переменной интегрирования}

\subsubsection{Свойства неопределенного интеграла}

\begin{multienumerate}
    \mitemxx{$(\int f(x) \diff x)' = (F(x) + C)' = f(x)$}{$\diff (\int f(x) \diff x) = f(x) \diff x$}
    \mitemxx{$\int \diff (F(x)) = F(x) + C$}{$\int (f_1(x) + f_2(x)) \diff x = \int f_1(x) \diff x + \int f_2(x) \diff x$}
    \mitemx{$\int \alpha f(x) \diff x = \alpha \int f(x) \diff x$}
    \mitemx{Если $\int f(x) \diff x = F(x) + C$, то
    \begin{enumerate}
        \item $\int f(\alpha x) \diff x = \frac{1}{a} F(\alpha x) + C$
        \item $\int f(x + b) \diff x = F(x + b) + C$
        \item $\int f(\alpha x + b) \diff x = \frac{1}{a} F(\alpha x + b) + C$
    \end{enumerate}}
\end{multienumerate}

\subsubsection{Таблица неопределенных интегралов}

\begin{multienumerate}
    \mitemxx{$\int x^{n} \diff x = \frac{x^{n + 1}}{n + 1} + C$}{$\int \frac{\diff x}{x} = \ln |x| + C$}
    \mitemxx{$\int \sin x \diff x = - \cos x + C$}{$\int \cos x \diff x = \sin x + C$}
    \mitemxx{$\int \frac{\diff x}{\cos^2 x} = \tg x + C$}{$\int \frac{\diff x}{\sin^2 x} = - \ctg x + C$}
    \mitemxx{$\int \tg x \diff x = - \ln (\cos x) + C$}{$\int \ctg x \diff x = \ln |\sin x| + C$}
    \mitemxx{$\int e^{x} \diff x = e^{x} + C$}{$\int a^{x} \diff x = \frac{a^{x}}{\ln a} + C$}
    \mitemxx{$\int \frac{\diff x}{1 + x^2} = \arctg x + C$}{$\int \frac{\diff x}{a^2 - x^2} = \frac{1}{2 a} \ln | \frac{a + x}{a - x} | + C$}
    \mitemxx{$\int \frac{\diff x}{\sqrt{a^2 - x^2}} = \arcsin \frac{x}{a} + C$}{$\int \frac{\diff x}{\sqrt{x^2 + a}} = \ln |x + \sqrt{x^2 + a}| + C$}
\end{multienumerate}

\subsubsection{Подведение функции под знак дифференциала}

Найти неопределенный интеграл. Выполнить проверку.

$\int \sin (3x + 1) \diff x$

\hfill

Смотрим на таблицу интегралов и находим похожую формулу: $\int \sin x \diff x = - \cos x + C$. Но проблема заключается в том, что у нас под синусом не просто буковка «икс», а сложное выражение. Что делать?

\hfill

 Подводим функцию $(3x + 1)$ под знак дифференциала:

 $\int \sin (3x + 1) \diff x = \frac{1}{3} \int \sin (3x + 1) \diff (3 x + 1)$

 \hfill

Раскрывая дифференциал, легко проверить, что:

$\frac{1}{3} \int \sin (3x + 1) \diff (3x + 1) = \frac{1}{3} \int \sin (3x + 1) * (3x + 1)' \diff x = \frac{1}{3} \int \sin (3x + 1) * (3 + 0) \diff x = \int \sin (3x + 1) \diff x$

\hfill

Теперь можно пользоваться табличной формулой $\int \sin x \diff x = - \cos x + C$:

$\int \sin (3x + 1) \diff x = \frac{1}{3} \int \sin (3x + 1) \diff (3 x + 1) = -\frac{1}{3} \cos (3 x + 1) + C$, где $C = const$

\subsubsection{Метод замены переменной в неопределенном интеграле}

Найти неопределенный интеграл. 

$\int \sin (3x + 1) \diff x$

\hfill

Идея метода замены состоит в том, чтобы \textbf{сложное выражение (или некоторую функцию) заменить одной буквой}

В данном случае напрашивается: $t = 3 x + 1$

\hfill

Действие следующее. После того, как мы подобрали замену, в данном примере, $t = 3 x + 1$, нам нужно найти дифференциал $\diff t$.

Так как $t = 3 x + 1$, то $\diff t = \diff (3x + 1) = (3x + 1)' \diff x = 3 \diff x$

$\diff x = \frac{\diff t}{3}$

\hfill

Таким образом:

$\int \sin (3x + 1) \diff x = \frac{1}{3} \int \sin t \diff t = - \frac{1}{3} \cos t + C$

Вернемся к переменной $x$:

$\frac{1}{3} \int \sin t \diff t = - \frac{1}{3} \cos t + C = -\frac{1}{3} \cos (3x + 1) + C$

\subsubsection{Метод интегрирования по частям}

$\int u \diff v = u v - \int v \diff u$

\begin{enumerate}
    \item \textbf{многочлен} * \textbf{тригонометрическую или показательную функцию}, то \\
    за $u$ выбирают многочлен, $\diff v$ — все, что осталось \\
    \textbf{Пример} $\int (3 x + 1) \cos 5 x \diff x = \frac{(3x + 1)}{5} \sin 5 x - \frac{3}{5} \int \sin 5 x \diff x = \frac{(3 x + 1)}{5} \sin 5 x + \frac{3}{25} \cos 5 x + C$ \\
    $du = 3 d x$, $v = \int \cos 5 x \diff x = \frac{1}{5} \sin 5x$ \\
    \textbf{Другой пример} $\int (3 x^2 + 1) \cos 5 x \diff x = \frac{(3 x^2 + 1)}{5} \sin 5 x + \frac{6}{5} \int x \sin 5 x \diff x$, дальше следует применить метод интегрирования по частям заново
    \item \textbf{многочлен} * \textbf{логарифмическую или обратную тригонометрическую функцию}, то \\
    за $u$ выбирают функцию, а $\diff v$ — все остальное \\
    \textbf{Пример} $\int (3x^2 + 5) \ln x \diff x = (\frac{x^3}{3} + 5 x) \ln x - \int (\frac{x^2}{3} + 5 x) \frac{\diff x}{1} = (\frac{x^3}{3} + 5x) \ln x - \frac{x^3}{9} - 5 x + C$ \\
    $ln x = u \Longrightarrow \frac{\diff x}{x} = \diff u$, $\diff v = (x^2 + 5) \diff x \Longrightarrow v = \int (x^2 + 5) \diff x = \frac{x^3}{3} + 5 x$
    \item \textbf{тригонометрическая функция} * \textbf{показательную функцию}, то \\
    не имеет значения, что выбрать за $u$, а что за $\diff v$, но формулу интегрирования по частям в этом случае \textbf{придется применить два раза подряд} единообразно \\
    \textbf{Пример} $\int \sin 5x e^{x} \diff x = \sin 5 x * e^{x} - 5 \int \cos 5 x * e^{x} \diff x = \dots$ \\
    Пусть $u = \sin 5x \Longrightarrow \diff u = 5 \cos 5 x \diff x$, $\diff v = e^{x} \diff x \Longrightarrow v = e^{x}$ \\
    \textbf{Применим метод интегрирования по частям во второй раз}, теперь $u = \cos 5 x \Longrightarrow \diff u = - 5 \sin 5 x \diff x$, $v = e^{x} \diff x \Longrightarrow v = e^{x}$ \\
    $\dots = \sin 5 x * e^{x} - 5 (\cos 5 x e^{x} + 5 \int \sin 5 x e^{x} \diff x)$ \\
    $y = (\sin 5 x - 5 \cos 5 x) e^{x} - 25 y \Longleftrightarrow 26 y = (\dots) e^{x} \Longleftrightarrow y = \frac{(\sin 5x - 5 \cos 5 x) e^{x}}{26}$, где $y = \int \sin 5 x e^{x} \diff x$
\end{enumerate}

\pagebreak
\subsection{Определенные интегралы}

В общем виде определенный интеграл записывается следующим образом:

$\int\limits_{a}^{b} f(x) \diff x$

\hfill

\textbf{Как решить определенный интеграл?}

$\int\limits_{a}^{b} f(x) \diff x = F(X) \bigg|_{a}^{b} = F(b) - F(a)$

Для того \textbf{чтобы определенный интеграл вообще существовал, достаточно чтобы подынтегральная функция была непрерывной на отрезке интегрирования}.

\hfill

\textbf{Свойства} определенных интегралов:

\begin{enumerate}
    \item $\int\limits_{a}^{b} f(x) \diff x = -\int\limits_{b}^{a} f(x) \diff x$
    \item $\int\limits_{a}^{b} k f(x) \diff x = k \int\limits_{a}^{b} f(x) \diff x$, где $k = const$
    \item $\int\limits_{a}^{b} (f(x) \pm g(x)) \diff x = \int\limits_{a}^{b} f(x) \diff x \pm \int\limits_{a}^{b} g(x) \diff x$
\end{enumerate}

\subsubsection{Замена переменной в определенном интергале}

Для определенного интеграла справедливы все типы замен, что и для неопределенного интеграла.

\textbf{Этапы решения}:

\begin{enumerate}
    \item Выполняем замену
    \item Находим новые пределы интегрирования, записываем новый интеграл с новыми пределами интегрирования в соответствии с заменой
    \item Производим интегрирование
    \item Применяем формулу Ньютона-Лейбница
\end{enumerate}

Никаких \textbf{обратных замен проводить не надо}

\hfill

\textbf{Если мы подводим функцию под знак дифференциала, то менять пределы интегрирования не нужно!
}
\subsubsection{Метод интегрирования по частям в определенном интеграле}

Все выкладки статьи Интегрирование по частям в неопределенном интеграле в полной мере справедливы и для определенного интеграла.

Плюсом идёт только одна деталь, \textbf{в формуле интегрирования по частям добавляются пределы интегрирования}:

$$\int\limits_{a}^{b} u \diff v = u v \bigg|_{a}^{b} - \int\limits_{a}^{b} v \diff u$$

Формулу Ньютона-Лейбница здесь \textbf{необходимо применить дважды}: для произведения $u v$ и, после того, как мы возьмем интеграл $\int\limits_{a}^{b} v \diff u$

\pagebreak
\subsection{Площадь и длина дуги кривой (декартовые, полярные, параметрические координаты)}

\subsubsection{Вычисление площадей в прямоугольных координатах}

$\int\limits_{a}^{b} f(x) \diff x = S_{\text{криволинейной трапеции}}$

Если график несколько раз пересекает ось $OX$, надо разбить его на несколько отрезков

\subsubsection{Вычисление площадей при параметрическом задании кривой}

\begin{equation}
\begin{cases}
    x = \phi(t) \\
    y = \psi(t) = \psi(y(x))
\end{cases}
\end{equation}

$\alpha \le t \le b$, $\phi(\alpha) = a$, $\phi(\beta) = b$

$S = \int\limits_{a}^{b} \phi(x) \diff x = \int\limits_{\alpha}^{\beta} \psi (t) \phi'(t) \diff t$

\subsubsection{Площадь в полярных координатах}

Пусть имеем $\rho = f (\theta)$, различные углы $\alpha = \theta_0$, $\beta = \theta_{n}$, разбивающие график на секторы.

\hfill

$S_{i} = \frac{1}{2} \Delta \Theta \rho^2$

$S = \sum\limits_{i = 1}^{n} = \frac{1}{2} \sum\limits_{i = 1}^{n} (f(\theta_i))^2 \Delta \theta_i$

$S = \lim\limits_{n \to \infty} \frac{1}{2} \sum\limits_{i = 1}^{n} (f(\theta_i))^2 \Delta \theta_i = \frac{1}{2} \int\limits_{\alpha}^{\beta} (f(\theta))^2 \diff \theta$

\subsubsection{Длина дуги кривой}

\begin{enumerate}
    \item Длина дуги кривой в декартовых координатах ($y = f(x)$, $[a; b]$), то $l = \int\limits_{a}^{b} \sqrt{1 + (f'(x))^2} \diff x$
    \item Если $$\begin{cases}
        x = \phi(t) \\
        y = \psi(t), \ \alpha \le t \le \beta
    \end{cases}$$

    То $l = \int\limits_{a}^{b} \sqrt{(\phi'_t)^2 + (\psi'_t)^2} \diff t$
    \item Если имеем полярные координаты ($\rho = f(\theta)$), то $l = \int\limits_{\theta_1}^{\theta_2} \sqrt{f^2(\theta) + (f'(\theta))^2} \diff \theta^2$
\end{enumerate}

\pagebreak
\subsection{Функции нескольких переменных}

Для функций нескольких переменных существуют лишь частные производные; пусть $z = z(x; y)$, то $\frac{\delta z}{\delta x} = \lim\limits_{\Delta x \to 0} \frac{z(x + \Delta x, y) - z(x, y)}{\Delta x}$; $\frac{\delta z}{\delta y} = \lim\limits_{\Delta y \to 0} \frac{z(x, y + \Delta y) - z(x, y)}{\Delta y}$

\hfill

Пусть $z = 2xy + xy^3$. тогда $\frac{\delta z}{\delta x} = \lim\limits_{\Delta x \to 0} \frac{2(x + \Delta x)y + (x + \Delta x)y^3 - 2xy - xy^3}{\Delta x} = \lim\limits_{\Delta x to 0} \frac{2xy + 2 \Delta x y + xy^3 + \Delta x y^3 - 2xy - xy^3}{\Delta x} = 2y + y^3$; $\frac{\delta z}{\delta y} = \lim\limits_{\Delta y \to 0} \frac{2x(y + \Delta y) + x(y + \Delta y)^3 - 2xy - xy^3}{\Delta y} = \lim\limits_{\Delta y \to 0} \frac{2xy + 2x \Delta y + xy^3 + 3xy^2 \Delta y + 3xy (\Delta y)^2 + x (\Delta y)^3 - 2xy - xy^3}{\Delta y} = \lim\limits_{\Delta y \to 0} 2x + 3xy^2 + 3xy \Delta y + x (\Delta y)^2 = 2x + 3xy^2$

\subsubsection{Полный дифференциал}

$du = \frac{\delta u}{\delta x} \delta x + \frac{\delta u}{\delta y} \delta y$ — \textbf{полный дифференциал} функции от двух переменных

\paragraph{Применение полного дифференциала к вычислению приближенных значений}

\hfill

\hfill

$f(x + \Delta x) = f(x) + \delta f \approx f(x_0) + d f$

Например, нам нужно вычислить $1.04^{2.02}$, тогда составим функцию $z = x^y$, $x_0 = 1$, $y_0 = 2$, $\delta x = d x = 0.04$, $\delta y = d y = 0.02$

$z(1.04; 2.02) \approx z(1; 2) + d z = 1 + \frac{\delta z}{\delta x} dx + \frac{\delta z}{\delta y} dy = 1 + yx^{y - 1} dx + x^y \ln x d y = 1 + 2 * 1 * 0.04 + 1^2 \ln 1 * 0.02 = 1.08$

\subsubsection{Частные производные}

Частные производные $\frac{\delta z}{\delta x}$ и $\frac{\delta z}{\delta y}$ тоже являются функциями, и поэтому от них можно брать частные производные.

\begin{theorem}

Если функция и ее частные производные определены и непрерывны в точке $M$ и некоторой ее окрестности, то в этой точке выполняется условие: $\frac{\delta}{\delta x} (\frac{\delta z}{\delta y}) = \frac{\delta}{\delta y} (\frac{\delta z}{\delta x})$

\end{theorem}

Порядок взятия частных производных не имеет значения:

$$\frac{\delta^n z}{\delta x^k \delta y^{n - k}} = \frac{\delta^n z}{\delta y^{n - k} \delta x^{k}}$$

\subsubsection{Производная функции, заданной неявно}

\begin{theorem}
    Пусть непрерывная функция $y = y(x)$ задана неявно уравнением $F(x, y) = 0$, причем сама эта функция и ее первые производные — непрерывные функции в некоторой области, $F'_y \ne 0$ в интересующей нас точке, тогда $y'_x = \frac{-F'_x(x, y)}{F'_y(x, y)}$
\end{theorem}

Например, у нас есть функция $x^2 + x \sin y = 0$ ($F(x, y) = x^2 + x \sin y$), тогда возьмем производные по $x$ и $y$: $F'_x = 2 x y + \sin y$, $F'_y = x^2 + x \cos y$, $y'_x = -\frac{2 x y + \sin y}{x^2 + x \cos y}$

\subsubsection{Производная сложной функции и понятие полной производной}

Пусть $z = z(u; v)$, $u = u(x; y)$, $v = v(x; y)$, и существуют непрерывные частные производные $z$ по $u; v$, $u, v$ по $x; y$, тогда мы можем рассматривать $z$ как функцию от $x$ и $y$: $z = z(u(x, y), v(x, y))$, но не всегда так делать целесообразно и поступать лучше следующим образом:

$$\frac{\delta z}{\delta x} = \frac{\delta z}{\delta u} \frac{\delta u}{\delta x} + \frac{\delta z}{\delta v} \frac{\delta v}{\delta x}, \frac{\delta z}{\delta y} = \frac{\delta z}{\delta u} \frac{\delta u}{\delta y} + \frac{\delta z}{\delta v} \frac{\delta v}{\delta y}$$

Но функция может быть и от большего количества переменных: $z = z(x; y; t)$, $x = x(t)$, $y = y(t)$, по аналогии $z = z(x(t), y(t), t)$ — есть зависимость лишь от $t$, можно говорить о полной производной:

$$
\frac{d z}{d t} = \frac{\delta z}{\delta x} \frac{\delta x}{\delta t} + \frac{\delta z}{\delta y} \frac{\delta y}{\delta t} + \frac{\delta z}{\delta t}
$$

Например, $z = u\sqrt{v} + v u^2$, $u = \sin(x + y)$, $v = \sqrt{x^2 + y}$, $\frac{\delta z}{\delta u} = \sqrt{v} = 2v u$, $\frac{\delta z}{\delta v} = \frac{u}{2 \sqrt{v}} + u^2$, $\frac{\delta u}{\delta x} = \cos (x + y)$, $\frac{\delta u}{\delta y} = \cos (x + y)$, $\frac{\delta v}{\delta x} = \frac{2x}{2\sqrt{x^2 + y}}$, $\frac{\delta v}{\delta y} = \frac{1}{2\sqrt{x^2 + y}}$

$\frac{\delta z}{\delta x} = (\sqrt{v} + 2 v u) \cos (x + y) + (\frac{u}{2\sqrt{v}} + u^2) * \frac{x}{\sqrt{x^2 + y}}$

\hfill

Другой не менее славный пример: $z = t e^{x - 2y} + x t^2$, $x = \sin t$, $y = t^3$, мы желаем посчитать $\frac{\delta z}{\delta t}$, для этого посчитаем много различной фигни: $\frac{\delta z}{\delta x} = t e^{x - 2y} + t^2$, $\frac{\delta z}{\delta y} = t e^{x - 2y} (-2)$, $\frac{\delta z}{\delta t} = e^{x - 2y} + 2 x t$, $\frac{\delta z}{\delta t} = (t e^{x - 2y} + t^2) \cos t - 2 t e^{x - 2 y} * 3t^2 + e^{x -2 y} + 2 xt$

\subsubsection{Производная по направлению}

Пусть $D$ — некоторое пространство, определяющееся функцией $u = u(x; y; z)$, и т. $M(x; y; z)$, т. $M_1(x + \Delta x, y + \Delta y, z + \Delta z)$ лежат в данном пространстве, от нее отложен некоторый $\vec{S} = \{ \cos \alpha, \cos \beta, \cos \gamma \}$, при этом $\Delta S = \sqrt{\Delta x^2 + \Delta y^2 + \Delta z^2}$

$\Delta u = \frac{\delta u}{\delta x} \Delta x + \frac{\delta u}{\delta y} \Delta y + \frac{\delta u}{\delta z} \Delta z + E_1 \Delta x + E_2 \Delta y + E_3 \Delta z$

$\frac{\Delta u}{\Delta s} = \frac{\delta u}{\delta x} + \frac{\Delta x}{\Delta S} + \frac{\delta u}{\delta y} \frac{\Delta y}{\Delta S} + \frac{\delta u}{\delta z} \frac{\Delta z}{\Delta S} + \dots$

$\frac{\delta u}{\delta s} = \lim\limits_{\Delta S \to 0} \frac{\Delta u}{\Delta S} = \frac{\delta u}{\delta x} \cos \alpha + \frac{\delta u}{\delta y} \cos \beta + \frac{\delta u}{\delta z} \cos \gamma$

\paragraph{Пример} $M \vec{M_1} = \{ 3; 4 \} = \{\frac{3}{5}; \frac{4}{5} \}$, $\mu = \frac{1}{\sqrt{9 + 16}} = \frac{1}{5}$

Пусть $u(x; y) = x^2 + 3\sqrt{y}$, т. $M(1; 1)$, т. $M_1(4; 5)$; $M \to M_1$ в точке $M$, в этой точке $\frac{\delta u}{\delta x} = 2x = 2$, $\frac{\delta u}{\delta y} = \frac{3}{2\sqrt{y}} = \frac{3}{2}$

\subsubsection{Градиент функции}

В каждой точке области $D$, в которой задана функция $u = u(x, y, z)$, определим вектор, координатами которого являются значения частных производных, и назовем его \textbf{градиентом функции}:

$$grad u = \frac{\delta u}{\delta x} \overrightarrow{c} + \frac{\delta u}{\delta y} \overrightarrow{j} + \frac{\delta u}{\delta z} \overrightarrow{k} $$

$\frac{\delta u}{\delta s} = \text{Пр}_{\overrightarrow{s}} (grad u)$ — производная по направлению в данной точке имеет наибольшее значение, если направление $\overrightarrow{s}$ совпадает с направлением градиента функции, равное модулю этого градиента.

\subsubsection{Локальный экстремум функции двух переменных}

Функция $z = z(x; y)$ \textbf{имеет локальный максимум} в точке $M_0(x_0; y_0)$, если $z(x_0, y_0) > z(x, y)$ в окрестности точки $M(x, y) \ne M_0(x_0, y_0)$, но достаточно близких к ней.

Функция $z = z(x; y)$ \textbf{имеет локальный минимум} в точке $M_0(x_0; y_0)$, если $z(x_0, y_0) < z(x, y)$ в окрестности точки $M(x, y) \ne M_0(x_0, y_0)$, но достаточно близких к ней.

\begin{theorem}[необходимое условие локального экстремума]
    Если функция $z( = zx; y)$ достигает экстремума в точке $M_0(x_0, y_0)$, то каждая частная производная или обращается в ноль в этой точке, или не существует: $\frac{\delta z}{\delta x} = 0$, $\frac{\delta z}{\delta y} = 0$ (*), и такие точки называются стационарными (точками возможного экстремума)
\end{theorem}

\begin{theorem}[достаточное условие локального экстремума]
    Пусть в некоторой области, содержащей точку $M_0(x_0, y_0)$, выполнены условия (*), и функция $z = z(x; y)$ имеет непрерывные частные производные до третьего порядка включительно: $A = \frac{\delta^2 z}{\delta x^2}$, $B = \frac{\delta^2 z}{\delta x \delta y}$, $C = \frac{\delta^2 z}{\delta y^2}$, $\Delta = \begin{vmatrix}
        A & B \\
        B & C
    \end{vmatrix} = AC - B^2$, то если $\Delta > 0$ — экстремум есть ($A(C) < 0$ — максимум, $A(C) > 0$ — минмум), $\Delta < 0$ — экстремума не существует, $\Delta = 0$ — спорный случай, который стоит рассматривать отдельно
\end{theorem}

\subsubsection{Линия уровня функции, поверхность уровня функции, частные производные}

\begin{definition}
Если заданы два непустых множества $D$ и $G$ и каждому элементу $M$ множества $D$ по определенному правилу ставится в соответствие один и только один элемент множества $G$, то говорят, что на области определения задана функция со множеством значений $G$
\end{definition}

\textbf{Область определения} представляет собой часть координатной плоскости, ограниченной плоской кривой.

\hfill

$Z = \sqrt{1 - x^2 - y^2} \Longleftrightarrow Z^{2} = 1 - x^2 - y^2, 1 - x^2 - y^2 \ge 0 \Longleftrightarrow x^2 + y^2 \le 0$

\begin{definition}
    Линией уровня функции двух переменных называется линия на координатной плоскости, где функция сохраняет постоянное значение
\end{definition}

\begin{definition}
    Поверхностью уровня функции двух переменных называется поверхность, в точках которых функция сохраняет постоянное значение
\end{definition}

$f = \frac{y}{x}, \frac{y}{x} = c \Longleftrightarrow y = cx$ — линия уровня плоскости.

\paragraph{Частные производные}

\paragraph{Пример 1}

\hfill

\hfill

$\frac{\delta^2 z}{\delta x \delta y}$, если $z(x, y) = \frac{x}{3y + 2x^2}$ в т. $V(1, 0)$

$\frac{\delta z}{\delta x} = \frac{1(3y + 2x^2) - x(4x)}{(3y + 2x^2)^2} = \frac{3y - 2x^2}{9y^2 + 12x^2 y + 4x^4}$

$\frac{\delta^2 z}{\delta x \delta y} = \frac{3(9y^2 + 12x^2 y + 4x^4) - (3y - 2x^2)(18y + 12x^2)}{(9y^2 + 12x^2 y + 4x^4)^2} = \frac{3 * 4 - (-2) * 12}{16} = \frac{9}{4}$

\paragraph{Пример 2}

\hfill

\hfill

Найти $\frac{\delta x}{\delta y}$ для $x^2 + 2xyz - \frac{z}{x} - 2yz^2 = 0$ в т. $M(2; 0; 8)$

$\frac{\delta f}{\delta y} = 2xz - 2z^2$; $\frac{\delta f}{\delta z} = 2zy - \frac{1}{x} - 4yz$; $\frac{\delta f}{\delta x} = 2x + 2yz - \frac{(x - z)}{x^2}$

$\frac{\delta z}{\delta y} = \frac{\frac{- \delta f}{\delta y}}{\frac{\delta f}{\delta z}}$


\paragraph{Пример 3}

\hfill

\hfill

Найти $\frac{\delta^2 z}{\delta x \delta y}$, если $z(x, y) = \frac{x^2}{x - 26}$ в т. $M(1; 0)$

$\frac{\delta z}{\delta x} = \frac{(x^2)' * (x - 2y) - x^2(x - 2y)'}{(x - 2y)^2} - \frac{2x * (x - 2y) - x^2}{(x - 2y)^2} = \frac{2x^2 - 4x y}{(x - 2y)^2}$

$\frac{\delta^2 z}{\delta x}{\delta y} = \frac{(2x^2 - 4xy)' * (x - 2y)^2 - (2x^2 - 4xy)((x - 2y)^2)'}{(x - 2y)^4} = \frac{(-4) * (x - 2y)^2 - (2x^2 - 4xy) * (x^2 - 4x y + 4y^2)'}{(x - 2y)^2} = \frac{-4 - 2 * (-4)}{1} = -4 + 8 = 4$

\pagebreak
\subsection{Ряды}

\subsubsection{Основы}

\paragraph{Понятие числового ряда}

В общем виде \textbf{числовой ряд} можно записать так: $\sum\limits_{n = 1}^{\infty} a_{n}$

Здесь:

\begin{enumerate}
    \item $\sum$ — математический знак суммы
    \item $a_{n}$ — \textbf{общий член ряда}
    \item $n$ — переменная-«счетчик»
\end{enumerate}

Слагаемые $a_{1}$, $a_{2}$, $a_{3}$, $\dots$ — это \textbf{числа}, которые называются \textbf{членами} ряда. Если все они неотрицательны, то такой ряд называют \textbf{положительным числовым рядом}.

\paragraph{Сходимость числовых рядов}

Одной из ключевых задач темы является \textbf{исследование ряда на сходимость}. При этом возможны два случая:

\begin{enumerate}
    \item Ряд $\sum\limits_{n = 1}^{\infty} a_{n}$ \textbf{расходится}. Это значит, что бесконечная сумма равна бесконечности:
    $$a_{1} + a_{2} + a_{3} + a_{4} + a_{5} + \dots = \infty$$
    Либо суммы вообще не существует
    \item Ряд $\sum\limits_{n = 1}^{\infty} a_{n}$ \textbf{сходится}. Это значит, что бесконечная сумма равна некоторому конечному числу $S$:
    $$a_{1} + a_{2} + a_{3} + a_{4} + a_{5} + \dots = S$$
\end{enumerate}
    
\paragraph{Необходимый признак сходимости ряда}

\textbf{Если ряд сходится, то его общий член стремится к нулю:} $\lim\limits_{n \to \infty} a_{n} = 0$

Если общий член ряда \textbf{не стремится к нулю}, то ряд \textbf{расходится}

\paragraph{Признаки сравнения для положительных числовых рядов}

Существуют два признака сравнения, один из них я буду называть просто \textbf{признаком сравнения}, другой – \textbf{предельным признаком сравнения}.

\hfill

Сначала рассмотрим \textbf{признак сравнения}, а точнее, первую его часть:

Рассмотрим два положительных числовых ряда $\sum\limits_{n \to \infty} a_{n}$ и $\sum\limits_{n \to \infty} b_{n}$. \textbf{Если известно}, что ряд $\sum\limits_{n \to \infty} b_{n}$ — \textbf{сходится}, и, начиная с некоторого номера $n$, выполнено неравенство $a_{n} \le b_{n}$, то ряд $\sum\limits_{n \to \infty} a_{n}$ \textbf{тоже сходится}.

\hfill

Иными словами: \textbf{Из сходимости ряда с бОльшими членами следует сходимость ряда с меньшими членами}

\paragraph{Предельный признак сравнения числовых положительных рядов}

Рассмотрим два положительных числовых ряда $\sum\limits_{n \to \infty} a_{n}$ и $\sum\limits_{n \to \infty} b_{n}$. Если предел отношения общих членов этих рядов равен \textbf{конечному, отличному от нуля числу} $A$: $\lim\limits_{n \to \infty} \frac{a_{n}}{b_{n}} = A$, \textbf{то оба ряда сходятся или расходятся одновременно}

\subsubsection{Признак Д'Аламбера, признаки Коши}

\paragraph{Признак сходимости Д'Аламбера}

Основные \textbf{предпосылки} для применения признака Д'Аламбера:

\begin{enumerate}
    \item В общий член ряда входит какое-нибудь число в степени, например, $2^{n}$, $3^{n}$, $5^{n}$ и так далее.
    \item В общий член ряда входит факториал
    \item Если в общем члене ряда есть «цепочка множителей», например, $1 * 3 * 5 * \dots * (2n - 1)$
\end{enumerate}

Рассмотрим \textbf{положительный числовой ряд} $\sum\limits_{n = 1}^{\infty} a_{n}$. Если существует предел отношения последующего члена к предыдущему: $\lim\limits_{n \to \infty} \frac{a_{n + 1}}{a_{n}} = D$, то:

\begin{enumerate}
    \item При $D < 1$ ряд \textbf{сходится}. В частности, ряд сходится при $D = 0$
    \item При $D > 1$ ряд \textbf{расходится}. В частности, ряд расходится при $D = \infty$
    \item При $D = 1$ \textbf{признак не дает ответа}. Нужно использовать другой признак.
\end{enumerate}
    
\paragraph{Радикальный признак сходимости Коши}

Рассмотрим \textbf{положительный числовой ряд} $\sum\limits_{n = 1}^{\infty} a_{n}$. Если существует предел: $\lim\limits_{n \to \infty} \sqrt[n]{a_{n}} = D$, то:

\begin{enumerate}
    \item При $D < 1$ ряд \textbf{сходится}. В частности, ряд сходится при $D = 0$
    \item При $D > 1$ ряд \textbf{расходится}. В частности, ряд расходится при $D = \infty$
    \item При $D = 1$ \textbf{признак не дает ответа}. Нужно использовать другой признак. Интересно отметить, что если признак Коши не даёт нам ответа на вопрос о сходимости ряда, то признак Даламбера тоже не даст ответа. Но если признак Даламбера не даёт ответа, то признак Коши вполне может «сработать».
\end{enumerate}

\textbf{Когда нужно использовать радикальный признак Коши?} Радикальный признак Коши обычно использует в тех случаях, когда корень $\sqrt[n]{a_{n}}$ «хорошо» извлекается из общего члена ряда.

\paragraph{Интегральный признак сходимости Коши}

Для того чтобы применять интегральный признак Коши необходимо более или менее уверенно уметь находить производные, интегралы, а также иметь навык вычисления \textbf{несобственного интеграла} первого рода.

\hfill

Рассмотрим \textbf{положительный числовой ряд} $\sum\limits_{n = 1}^{\infty} a_{n}$. Если существует несобственный интеграл $\int\limits_{1}^{+\infty} a_{x} \diff x$, то ряд \textbf{сходится или расходится вместе с этим интегралом}.

\subsubsection{Знакочередующиеся ряды. Признак Лейбница.
Абсолютная и условная сходимость}

\paragraph{Признак Лейбница}

Если члены знакочередующегося ряда \textbf{монотонно убывают} по модулю, то ряд \textbf{сходится}.

Или в два пункта:

\begin{enumerate}
    \item Ряд является знакочередующимся
    \item Члены ряда убывают по модулю: $\lim\limits_{n \to \infty} |a_{n}| = 0$, причём, убывают монотонно
\end{enumerate}

Если выполнены эти условия, то ряд \textbf{сходится}.

\paragraph{Абсолютная и условная сходимость}

Сходимость бывает разной. А именно:

\begin{enumerate}
    \item сходящийся ряд $\sum\limits_{n = 1}^{\infty} a_{n}$ называются \textbf{абсолютно сходящимся}, если сходится ряд $\sum\limits_{n = 1}^{\infty} |a_{n}|$
    \item в противном случае ряд $\sum\limits_{n = 1}^{\infty} a_{n}$ \textbf{сходится условно}
\end{enumerate}

\subsubsection{Функциональные ряды. Степенные ряды. Область сходимости ряда}

\paragraph{Понятие функционального ряда и степенного ряда}

Обычный числовой ряд, вспоминаем, состоит из \textbf{чисел}:

$\sum\limits_{n = 1}^{\infty} a_{n} = a_{1} + a_{2} + a_{3} + a_{4} + a_{5} + \dots$

Функциональный же ряд состоит из \textbf{функций}:

$\sum\limits_{n = 1}^{\infty} u_{n}(x) = u_{1}(x) + u_{2}(x) + u_{3}(x) + u_{4}(x) + u_{5}(x) + \dots$

Наиболее популярной разновидностью функционального ряда является \textbf{степенной ряд}. Членами степенного ряда являются целые положительные степени переменной $x$ либо двучлена $(x - a)$ $(a = const)$, умноженные на числовые коэффициенты:

$$\sum\limits_{n = 0}^{\infty} c_{n} (x - a)^{n} = c_0 + c_1 (x - a) + c_2 (x - a)^2 + c_3 (x - a)^3 + \dots$$

\paragraph{Исследование степенного ряда на сходимость}

Найти область сходимости степенного ряда $\sum\limits_{n = 1}^{\infty} \frac{x^{n}}{n^{2}}$

На первом этапе находим интервал сходимости ряда. Технически нам нужно вычислить предел $\lim\limits_{n \to \infty} | \frac{u_{n + 1} (x)}{u_{n} (x)} |$:

$\lim\limits_{n \to \infty} |\frac{u_{n + 1} (x)}{u_{n} (x)}| = \lim\limits_{n \to \infty} |\frac{\frac{x^{n + 1}}{(n + 1)^2}}{\frac{x^{n}}{n^{2}}}| = \lim\limits_{n \to \infty} |\frac{n^2 * x^{n + 1}}{(n + 1)^2 * x^{n}}| = \lim\limits_{n \to \infty} |\frac{n^2 * x * x^{n}}{(n^2 + 2 n + 1) * x^{n}}| = |x| \lim\limits_{n \to \infty} \frac{n^2}{n^2 + 2n + 1} = \frac{\infty}{\infty} = |x| \lim\limits_{n \to \infty} \frac{\frac{n^{2}}{n^{2}}}{\frac{n^2 + 2n + 1}{n^2}} = |x| \lim\limits_{n \to \infty} \frac{1}{1 + \frac{2}{n} + \frac{1}{n^2}} = |x| * 1 = |x|$

После того, как предел найден, нужно проанализировать, что у нас получилось:

\begin{enumerate}
    \item \textbf{Если в пределе получается ноль}, то алгоритм решения заканчивает свою работу, и мы даём окончательный ответ задания: «ряд сходится при $x \in (-\infty; +\infty)$»
    \item \textbf{Если в пределе получается бесконечность}, то алгоритм решения также заканчивает свою работу, и мы даём окончательный ответ задания: «ряд сходится при $x = 0$ (или при $x = a$, либо $x = -a$)»
    \item \textbf{Если в пределе получается не ноль и не бесконечность}, то у нас самый распространенный на практике случай №1 – ряд сходится на некотором интервале
\end{enumerate}

Как найти интервал сходимости ряда? Составляем неравенство:

$|x| < 1$

В \textbf{ЛЮБОМ задании данного типа} в левой части неравенства должен находиться \textbf{результат вычисления предела}, а в правой части неравенства – \textbf{строго единица}

\hfill

На втором этапе необходимо исследовать сходимость ряда на концах найденного интервала.

И после этого записать результат

\subsubsection{Разложение функций в степенные ряды. Ряд Тейлора. Ряд Маклорена}

Если функция $f(x)$ в некотором интервале раскладывается в степенной ряд по степеням $(x - a)$, то это разложение единственно, задается следующей формулой и называется \textbf{рядом Тейлора}:

$$
f(x) = f(\alpha) + \frac{f'(\alpha)}{1!} (x - \alpha) + \frac{f''(\alpha)}{2!} (x - \alpha)^2 + \frac{f'''(\alpha)}{3!} (x - \alpha)^3 + \dots + \frac{f^{(n)}(\alpha)}{n!} (x - a)^{n} + \dots
$$

На практике процентах в 95-ти приходится иметь дело с частным случаем формулы Тейлора, когда $a = 0$, и данный ряд называется рядом \textbf{Маклорена}:

$$
f(x) = f(\alpha) + \frac{f'(\alpha)}{1!} x + \frac{f''(\alpha)}{2!} x^2 + \frac{f'''(\alpha)}{3!} x^3 + \dots + \frac{f^{(n)}(\alpha)}{n!} x^{n} + \dots
$$


\end{document}