\documentclass{article}
\usepackage[utf8]{inputenc}

\usepackage[T2A]{fontenc}
\usepackage[utf8]{inputenc}
\usepackage[russian]{babel}

\usepackage{amsmath}
\usepackage{pgfplots}
\usepackage{multienum}
\usepackage{geometry}
\geometry{
    left=1cm,right=1cm,top=2cm,bottom=2cm
}
\newcommand*\diff{\mathop{}\!\mathrm{d}}

\newtheorem{definition}{Определение}
\newtheorem{theorem}{Теорема}

\DeclareMathOperator{\sign}{sign}

\usepackage{hyperref}
\hypersetup{
    colorlinks, citecolor=black, filecolor=black, linkcolor=black, urlcolor=black
}

\title{Высшая математика}
\author{Лисид Лаконский}
\date{March 2023}

\begin{document}
\raggedright

\maketitle

\tableofcontents
\pagebreak

\section{Высшая математика - 15.03.2023}

\subsection{Задание на дом}

Кардиоида, астроида, локон Аньези, спираль Архимеда, циклода, леминската, двух-, трех-, четырех- лепестковые розы.

\textbf{Записать уравнения} во всех возможных видах: в декартовых, полярных, параметрических координатах, \textbf{сделать картинки}.

\subsection{Несобственные интегралы}

\subsubsection{Первого рода (с бесконечными пределами)}

Пусть функция $f(x)$ определена при $x$ от $a$ до $\infty$. Если существует конечный предел $\lim \int\limits_{a}^{b} f(x) \diff x$ при $b \to +\infty$, то определен и сходится несобственный интеграл $\int \limits_{a}^{+\infty} f(x) \diff x$.

Если же предел не существует или равен бесконечности, то говорят, что этот интеграл расходится.

\hfill

$\int\limits_{-\infty}^{a} f(x) \diff x = \lim\limits_{b \to -\infty} \int\limits_{b}^{a} f(x) \diff x$

$\int\limits_{-\infty}^{\infty} f(x) \diff x = \int\limits_{-\infty}^{a} + \int\limits_{a}^{+\infty} = \lim\limits_{b_1 \to -\infty} \int\limits_{b_1}^{a} f(x) \diff x + \lim\limits_{b_2 \to +\infty} \int\limits_{a}^{b_2} f(x) \diff x$ — если хоть один интеграл расходится, то весь интеграл тоже расходящийся


\begin{theorem}
    Если для всех $x \ge a$ выполняется $0 \le f(x) \le g(x)$, $f(x)$, $g(x)$ — непрерывные функции, то из сходимости $\int\limits_{a}^{+\infty} g(x) \diff x$ следует сходимость $\int\limits_{a}^{+\infty} f(x) \diff x$

    А из расходимости $\int\limits_{a}^{+\infty} f(x) \diff x$ следует расходимость $\int\limits_{a}^{+\infty} g(x) \diff x$
\end{theorem}

Допустим, $y = \frac{1}{x^2}$, $\int\limits_{1}^{+\infty} \frac{\diff x}{x^2} = \lim\limits_{b \to +\infty} \int\limits_{1}^{b} \frac{\diff x}{x^2} = -\lim\limits_{b \to +\infty} (\frac{1}{b} - 1) = 1$ — сходящийся интеграл

И хотим проверить, является ли сходящимся $\int\limits_{1}^{+\infty} \frac{|\sin x| \diff x}{x^2}$, уверенно заявляем, что $\frac{|\sin x|}{x^2} \le \frac{1}{x^2}$, следовательно интеграл от этой функции тоже является сходящимся

\hfill

Рассмотрим $\int\limits_{a}^{+\infty} \frac{\diff x}{x^p}$, $a > 1$

\begin{enumerate}
    \item $p > 1$ — сходящийся
    \item $p \le 1$ — расходящийся
\end{enumerate}

\begin{theorem}
    Если сходится интеграл от $\int\limits_{a}^{+\infty} |f(x)| \diff x$, то $\int\limits_{a}^{+\infty} f(x) \diff x$ тоже сходится, при этом называется абсолютно сходящимся
\end{theorem}

\subsubsection{Второго рода (от бесконечных функций)}

$\int\limits_{a}^{b} f(x) \diff x = \lim\limits_{\epsilon \to 0} \int\limits_{a + \epsilon}^{b} f(x) \diff x$, где $a$ — «плохая точка», $\lim\limits_{a}^{b} f(x) \diff x = \lim\limits_{\epsilon \to 0} \int\limits_{a}^{b - \epsilon} f(x) \diff x$, если $b$ — «плохая точка»

Если плохая точка находится между $a$ и $b$, то интеграл необходимо разбить надвое: $\int\limits_{a}^{b} f(x) \diff x = \int\limits_{a}^{c} + \int\limits_{c}^{b} = \lim\limits_{\epsilon_1 \to 0} \int\limits_{a}^{c - \epsilon_1} f(x) \diff x + \lim\limits_{\epsilon_2 \to 0} \int\limits_{c + \epsilon_2}^{b} f(x) \diff x$

\hfill

Если обе точки плохие, то $\int\limits_{0}^{+\infty} f(x) \diff x = \int\limits_{0 + \epsilon}^{a} + \int\limits_{a}^{+\infty}$ — первый интеграл — второго рода, второй — первого рода

\hfill

\hfill

$\int\limits_{a}^{c} \frac{\diff x}{(c - x)^{p}}$, где $c$ — плохая точка

\begin{enumerate}
    \item $p < 1$ — сходящийся интеграл
    \item $p \ge 1$ — расходящийся интеграл
\end{enumerate}

\hfill

$\int\limits_{0}^{+\infty} \frac{\diff x}{x \sqrt[3]{x}} = \int\limits_{0}^{1} + \int\limits_{1}^{+\infty} = \lim\limits_{\epsilon \to 0} \int\limits_{\epsilon}^{1} \frac{\diff x}{x^{\frac{4}{3}}} + \lim\limits_{b \to +\infty} \int\limits_{1}^{+\infty} \frac{\diff x}{x^{\frac{4}{3}}}$ — первый интеграл сходится, второй расходится — интеграл расходящийся

\subsection{Вычисление значения определенного интеграла}

\subsubsection{Первый способ, тривиальный}

$\Delta x_i = \frac{b - a}{n}$, $y_0 = f(x_0)$, $y_1 = f(x_1)$, $y_2 = f(x_2)$

$\sum\limits_{i = 1}^{n} f(x_i) * \Delta x_i$

\subsubsection{Второй способ, если мы желаем чуть усложнить себе жизнь, но увеличить точность}

$S = \frac{y_{i - 1} + y_{i}}{2} \Delta x_i$

$\sum\limits_{i = 1}^{n} \frac{f(x_i - 1) + f(x_i)}{2} * \frac{b - a}{n}$

\subsection{Знакоположительные числовые ряды}

Нам знакомы числовые последовательности $a_1$, $a_2$, $a_3$, ..., $a_n$

$a_1 + a_2 + a_3 + a_4 + \dots + a_n$ — есть числовой ряд. Сумма первых $n$ членов $S_{n}$ называется частичной суммой ряда

Если существует конечный предел $\lim\limits_{n \to \infty} S_n = S$ — сумма ряда

\hfill

$b_1 = 1$, $q = \frac{1}{3}$

$1 + \frac{1}{3} + \frac{1}{9} + \dots + \frac{1}{3^{n}} + \dots$, \ $S_n = b_1 \frac{1 - q^{n}}{1 - q} = \frac{1 - \frac{1}{3^{n}}}{1 - \frac{1}{3}}$

$\lim\limits_{n \to \infty} S_{n} = \lim\limits_{n \to \infty} \frac{1 - \frac{1}{3^{n}}}{\frac{2}{3}} = \frac{3}{2}$

\hfill

\hfill

$\sum\limits_{n = 1}^{\infty} \frac{1}{n(n + 1)} = \frac{1}{1 * 2} + \frac{1}{2 * 3} + \frac{1}{3 * 4} + \dots + \frac{1}{(n - 1) n} + \frac{1}{n (n + 1)} + \dots$

$S_n = \frac{1}{1} - \frac{1}{2} + \frac{1}{2} - \frac{1}{3} + \frac{1}{3} - \frac{1}{4} + \dots + \frac{1}{n - 1} - \frac{1}{n} + \frac{1}{n} - \frac{1}{n + 1} = 1 - \frac{1}{n + 1} = \frac{n}{n + 1}$ — формула $n$-ой частичной суммы

$\lim\limits_{n \to \infty} S_{n} = \lim\limits_{n \to \infty} \frac{n}{n + 1} = 1 = S$


\hfill


Иногда нам нет необходимости находить сумму ряда, а нужно лишь определить, является ряд сходящимся или расходящимся.

\begin{theorem}
    Если сходится ряд, получившийся из данного отбрасыванием нескольких его членов (то есть, конечного числа его членов), то будет сходиться и данный ряд
    
    Верное и обратное, что если данный ряд сходится, то будет сходиться и ряд, полученный отбрасыванием нескольких его членов
\end{theorem}

\begin{theorem}
    Если некий ряд $a_1 + a_2 + a_3 + \dots$ сходится и его сумма равняется $S$, то будет сходиться и ряд $ka_1 + ka_2 + ka_3 + \dots$, и его сумма будет равна $k S$
\end{theorem}

\begin{theorem}
    Если есть два сходящихся ряда $a_1 + a_2 + a_3 + \dots = S_1$ и $b_1 + b_2 + b_3 + \dots = S_2$, то будут сходиться и ряды, полученные почленным сложением или вычитанием этих двух рядов:
    \begin{enumerate}
        \item $(a_1 \pm b_1) + (a_2 \pm b_2) + (a_3 \pm b_3) = S_1 \pm S_2$
    \end{enumerate}
\end{theorem}

\subsubsection{Необходимый признак сходимости}

Если ряд сходится, то $\lim\limits_{n \to \infty} a_{n} = 0$, если $\lim\limits_{n \to \infty} a_n \ne 0$, то ряд точно расходится

\hfill

$a_1 + a_2 + a_3 + \dots$, $\lim\limits_{n \to \infty} S_n = S$, $\lim\limits_{n \to \infty} S_{n - 1} = S$

$\lim\limits_{n \to \infty} S_{n} - \lim\limits_{n \to \infty} S_{n - 1} = \lim\limits_{n \to \infty} a_n = 0$

\hfill

Как следствие, если $n$—ый член не стремится к нулю, то ряд точно будет расходящимся. Но мы не должны делать вывод, если $n$—ый член стремится к нулю, что ряд будет точно сходящимся. Мы должны дальше исследовать его на сходимость

\subsubsection{Достаточные признаки сходимости}

\begin{theorem}[Признак сравнения]
    Пусть $u_1 + u_2 + u+3 + \dots$, $v_1 + v_2 + v_3 + \dots$ — знакоположительные числовые ряды, при этом $u_{i} \le v_{i}$

    Тогда из сходимости более большого ряда следует сходимость более маленького ряда

    А из расходимости более маленького ряда следует расходимость более большого ряда
\end{theorem}

\begin{theorem}[Предельный признак сравнения]
    Пусть $u_1 + u_2 + u+3 + \dots$, $v_1 + v_2 + v_3 + \dots$ — знакоположительные числовые ряды, при этом $u_{i} \le v_{i}$

    Если $\lim\limits_{n \to \infty} \frac{u_{n}}{v_{n}} = C \ne 0 \ne \pm \infty$

    То данные ряды ведут себя одинаково: или сходятся, или расходятся одновременно
\end{theorem}

\end{document}