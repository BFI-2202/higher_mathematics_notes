\documentclass{article}
\usepackage[utf8]{inputenc}

\usepackage[T2A]{fontenc}
\usepackage[utf8]{inputenc}
\usepackage[russian]{babel}

\usepackage{amsmath}
\usepackage{pgfplots}
\usepackage{multienum}
\usepackage{geometry}
\geometry{
    left=1cm,right=1cm,top=2cm,bottom=2cm
}
\newcommand*\diff{\mathop{}\!\mathrm{d}}

\newtheorem{definition}{Определение}
\newtheorem{theorem}{Теорема}

\DeclareMathOperator{\sign}{sign}

\usepackage{hyperref}
\hypersetup{
    colorlinks, citecolor=black, filecolor=black, linkcolor=black, urlcolor=black
}

\title{Высшая математика}
\author{Лисид Лаконский}
\date{March 2023}

\begin{document}
\raggedright

\maketitle

\tableofcontents
\pagebreak

\section{Высшая математика - 01.02.2023}

\subsection{Нахождение среднего значения функции в указанном промежутке}

$\mu = \frac{1}{b - a} \int\limits_{a}^{b} f(x) \diff x = F(x)\bigg|_{a}^{b} = F(b) - F(a)$

Если функция $y = f(x)$ непрерывна на отрезке $[a; b]$, то существует точка $c \in [ a; b ]$, что значение $f(c) = \mu$

Если $f(x) > 0$ при $x \in [a; b]$, то площадь криволинейной трапеции, ограниченной линиями $x = a$, $x = b$, $y = 0$ и $y = f(x)$ равна площади прямоугольника с основанием $[a, b]$ и высотой $f(c)$

\subsubsection{Примеры}

\paragraph{Пример №1} Найти среднее значение функции $y = 5x^4 - 2$ на промежутке $[1; 2]$

$\mu = \frac{1}{2 - 1} \int\limits_{1}^{2} (5x^4 - 2) \diff x = (x^5 - 2x) \bigg|_{1}^{2} = (32 - 4) - (1 - 2) = 29$

\subsection{Нахождение точек экстремума и точек перегиба}

Точки экстремума и точки перегиба функции $\Phi (x)$, заданной интегралом с переменным верхним пределом.

$$
\Phi(x) = \int\limits_{a}^{x} f(t) \diff t
$$

Если $f$ непрерывна в точке $x$, то $\Phi'(x) = f(x) \implies \Phi''(x) = f'(x)$

$\Phi(x) = \int\limits_{a}^{x} f(t) \diff t = F(t) \bigg|_{a}^{x} = F(x) - F(a)$


\hfill

$f'' > 0$ - вогнутая, $f'' < 0$ — выпуклая

\subsubsection{Примеры}

\paragraph{Пример №1} $\Phi(x) = \int\limits_{1}^{x} (t - t^3) \diff t = (\frac{t^2}{2} - \frac{t^4}{4}) \bigg|_{1}^{x} = \frac{x^2}{2} - \frac{x^4}{4} - \frac{1}{2} + \frac{1}{4} = \frac{x^2}{2} - \frac{x^4}{4} - \frac{1}{4}$

$x - x^3 = \Phi'(x) \Longleftrightarrow x - x^3 = 0 \Longleftrightarrow x_1 = 0, \ x_{2, 3} = \pm 1$

Изобразим знаки $\Phi'(x)$ и $\Phi(x)$ на координатной прямой с отмеченными точками $x = -1$, $x = 0$, $x = 1$. Найдем точки максимума и минимума; $x_{min} = 0$, $x_{max_1} = -1$, $x_{max_2} = 1$

$\Phi(0) = - \frac{1}{4}$ — min, $\Phi(\pm 1) = 0$ — max

\hfill

$\Phi''(x) = f'(x) = 1 - 3x^2$

$1 - 3x^2 = 0 \Longleftrightarrow x_{1, 2} = \pm \frac{1}{\sqrt{3}}$

Изобразим знаки $\Phi''(x)$ и $\Phi(x)$ на координатной прямой с отмеченными точками $x = -\frac{1}{\sqrt{3}}$, $x = \frac{1}{\sqrt{3}}$. Определим, на каких промежутках график функции вогнут, а на каких выпукл.


$\Phi(-\frac{1}{\sqrt{3}}) = \frac{1}{6} - \frac{1}{36} - \frac{1}{4} = -\frac{1}{9}$, $\Phi(\frac{1}{\sqrt{3}}) = -\frac{1}{9}$

$x = \pm \frac{1}{\sqrt{3}}$ — точки перегиба

\pagebreak
\section{Высшая математика - 15.03.2023}

\subsection{Задание на дом}

Кардиоида, астроида, локон Аньези, спираль Архимеда, циклода, леминската, двух-, трех-, четырех- лепестковые розы.

\textbf{Записать уравнения} во всех возможных видах: в декартовых, полярных, параметрических координатах, \textbf{сделать картинки}.

\subsection{Несобственные интегралы}

\subsubsection{Первого рода (с бесконечными пределами)}

Пусть функция $f(x)$ определена при $x$ от $a$ до $\infty$. Если существует конечный предел $\lim \int\limits_{a}^{b} f(x) \diff x$ при $b \to +\infty$, то определен и сходится несобственный интеграл $\int \limits_{a}^{+\infty} f(x) \diff x$.

Если же предел не существует или равен бесконечности, то говорят, что этот интеграл расходится.

\hfill

$\int\limits_{-\infty}^{a} f(x) \diff x = \lim\limits_{b \to -\infty} \int\limits_{b}^{a} f(x) \diff x$

$\int\limits_{-\infty}^{\infty} f(x) \diff x = \int\limits_{-\infty}^{a} + \int\limits_{a}^{+\infty} = \lim\limits_{b_1 \to -\infty} \int\limits_{b_1}^{a} f(x) \diff x + \lim\limits_{b_2 \to +\infty} \int\limits_{a}^{b_2} f(x) \diff x$ — если хоть один интеграл расходится, то весь интеграл тоже расходящийся


\begin{theorem}
    Если для всех $x \ge a$ выполняется $0 \le f(x) \le g(x)$, $f(x)$, $g(x)$ — непрерывные функции, то из сходимости $\int\limits_{a}^{+\infty} g(x) \diff x$ следует сходимость $\int\limits_{a}^{+\infty} f(x) \diff x$

    А из расходимости $\int\limits_{a}^{+\infty} f(x) \diff x$ следует расходимость $\int\limits_{a}^{+\infty} g(x) \diff x$
\end{theorem}

Допустим, $y = \frac{1}{x^2}$, $\int\limits_{1}^{+\infty} \frac{\diff x}{x^2} = \lim\limits_{b \to +\infty} \int\limits_{1}^{b} \frac{\diff x}{x^2} = -\lim\limits_{b \to +\infty} (\frac{1}{b} - 1) = 1$ — сходящийся интеграл

И хотим проверить, является ли сходящимся $\int\limits_{1}^{+\infty} \frac{|\sin x| \diff x}{x^2}$, уверенно заявляем, что $\frac{|\sin x|}{x^2} \le \frac{1}{x^2}$, следовательно интеграл от этой функции тоже является сходящимся

\hfill

Рассмотрим $\int\limits_{a}^{+\infty} \frac{\diff x}{x^p}$, $a > 1$

\begin{enumerate}
    \item $p > 1$ — сходящийся
    \item $p \le 1$ — расходящийся
\end{enumerate}

\begin{theorem}
    Если сходится интеграл от $\int\limits_{a}^{+\infty} |f(x)| \diff x$, то $\int\limits_{a}^{+\infty} f(x) \diff x$ тоже сходится, при этом называется абсолютно сходящимся
\end{theorem}

\subsubsection{Второго рода (от бесконечных функций)}

$\int\limits_{a}^{b} f(x) \diff x = \lim\limits_{\epsilon \to 0} \int\limits_{a + \epsilon}^{b} f(x) \diff x$, где $a$ — «плохая точка», $\lim\limits_{a}^{b} f(x) \diff x = \lim\limits_{\epsilon \to 0} \int\limits_{a}^{b - \epsilon} f(x) \diff x$, если $b$ — «плохая точка»

Если плохая точка находится между $a$ и $b$, то интеграл необходимо разбить надвое: $\int\limits_{a}^{b} f(x) \diff x = \int\limits_{a}^{c} + \int\limits_{c}^{b} = \lim\limits_{\epsilon_1 \to 0} \int\limits_{a}^{c - \epsilon_1} f(x) \diff x + \lim\limits_{\epsilon_2 \to 0} \int\limits_{c + \epsilon_2}^{b} f(x) \diff x$

\hfill

Если обе точки плохие, то $\int\limits_{0}^{+\infty} f(x) \diff x = \int\limits_{0 + \epsilon}^{a} + \int\limits_{a}^{+\infty}$ — первый интеграл — второго рода, второй — первого рода

\hfill

\hfill

$\int\limits_{a}^{c} \frac{\diff x}{(c - x)^{p}}$, где $c$ — плохая точка

\begin{enumerate}
    \item $p < 1$ — сходящийся интеграл
    \item $p \ge 1$ — расходящийся интеграл
\end{enumerate}

\hfill

$\int\limits_{0}^{+\infty} \frac{\diff x}{x \sqrt[3]{x}} = \int\limits_{0}^{1} + \int\limits_{1}^{+\infty} = \lim\limits_{\epsilon \to 0} \int\limits_{\epsilon}^{1} \frac{\diff x}{x^{\frac{4}{3}}} + \lim\limits_{b \to +\infty} \int\limits_{1}^{+\infty} \frac{\diff x}{x^{\frac{4}{3}}}$ — первый интеграл сходится, второй расходится — интеграл расходящийся

\subsection{Вычисление значения определенного интеграла}

\subsubsection{Первый способ, тривиальный}

$\Delta x_i = \frac{b - a}{n}$, $y_0 = f(x_0)$, $y_1 = f(x_1)$, $y_2 = f(x_2)$

$\sum\limits_{i = 1}^{n} f(x_i) * \Delta x_i$

\subsubsection{Второй способ, если мы желаем чуть усложнить себе жизнь, но увеличить точность}

$S = \frac{y_{i - 1} + y_{i}}{2} \Delta x_i$

$\sum\limits_{i = 1}^{n} \frac{f(x_i - 1) + f(x_i)}{2} * \frac{b - a}{n}$

\subsection{Знакоположительные числовые ряды}

Нам знакомы числовые последовательности $a_1$, $a_2$, $a_3$, ..., $a_n$

$a_1 + a_2 + a_3 + a_4 + \dots + a_n$ — есть числовой ряд. Сумма первых $n$ членов $S_{n}$ называется частичной суммой ряда

Если существует конечный предел $\lim\limits_{n \to \infty} S_n = S$ — сумма ряда

\hfill

$b_1 = 1$, $q = \frac{1}{3}$

$1 + \frac{1}{3} + \frac{1}{9} + \dots + \frac{1}{3^{n}} + \dots$, \ $S_n = b_1 \frac{1 - q^{n}}{1 - q} = \frac{1 - \frac{1}{3^{n}}}{1 - \frac{1}{3}}$

$\lim\limits_{n \to \infty} S_{n} = \lim\limits_{n \to \infty} \frac{1 - \frac{1}{3^{n}}}{\frac{2}{3}} = \frac{3}{2}$

\hfill

\hfill

$\sum\limits_{n = 1}^{\infty} \frac{1}{n(n + 1)} = \frac{1}{1 * 2} + \frac{1}{2 * 3} + \frac{1}{3 * 4} + \dots + \frac{1}{(n - 1) n} + \frac{1}{n (n + 1)} + \dots$

$S_n = \frac{1}{1} - \frac{1}{2} + \frac{1}{2} - \frac{1}{3} + \frac{1}{3} - \frac{1}{4} + \dots + \frac{1}{n - 1} - \frac{1}{n} + \frac{1}{n} - \frac{1}{n + 1} = 1 - \frac{1}{n + 1} = \frac{n}{n + 1}$ — формула $n$-ой частичной суммы

$\lim\limits_{n \to \infty} S_{n} = \lim\limits_{n \to \infty} \frac{n}{n + 1} = 1 = S$


\hfill


Иногда нам нет необходимости находить сумму ряда, а нужно лишь определить, является ряд сходящимся или расходящимся.

\begin{theorem}
    Если сходится ряд, получившийся из данного отбрасыванием нескольких его членов (то есть, конечного числа его членов), то будет сходиться и данный ряд
    
    Верное и обратное, что если данный ряд сходится, то будет сходиться и ряд, полученный отбрасыванием нескольких его членов
\end{theorem}

\begin{theorem}
    Если некий ряд $a_1 + a_2 + a_3 + \dots$ сходится и его сумма равняется $S$, то будет сходиться и ряд $ka_1 + ka_2 + ka_3 + \dots$, и его сумма будет равна $k S$
\end{theorem}

\begin{theorem}
    Если есть два сходящихся ряда $a_1 + a_2 + a_3 + \dots = S_1$ и $b_1 + b_2 + b_3 + \dots = S_2$, то будут сходиться и ряды, полученные почленным сложением или вычитанием этих двух рядов:
    \begin{enumerate}
        \item $(a_1 \pm b_1) + (a_2 \pm b_2) + (a_3 \pm b_3) = S_1 \pm S_2$
    \end{enumerate}
\end{theorem}

\subsubsection{Необходимый признак сходимости}

Если ряд сходится, то $\lim\limits_{n \to \infty} a_{n} = 0$, если $\lim\limits_{n \to \infty} a_n \ne 0$, то ряд точно расходится

\hfill

$a_1 + a_2 + a_3 + \dots$, $\lim\limits_{n \to \infty} S_n = S$, $\lim\limits_{n \to \infty} S_{n - 1} = S$

$\lim\limits_{n \to \infty} S_{n} - \lim\limits_{n \to \infty} S_{n - 1} = \lim\limits_{n \to \infty} a_n = 0$

\hfill

Как следствие, если $n$—ый член не стремится к нулю, то ряд точно будет расходящимся. Но мы не должны делать вывод, если $n$—ый член стремится к нулю, что ряд будет точно сходящимся. Мы должны дальше исследовать его на сходимость

\subsubsection{Достаточные признаки сходимости}

\begin{theorem}[Признак сравнения]
    Пусть $u_1 + u_2 + u+3 + \dots$, $v_1 + v_2 + v_3 + \dots$ — знакоположительные числовые ряды, при этом $u_{i} \le v_{i}$

    Тогда из сходимости более большого ряда следует сходимость более маленького ряда

    А из расходимости более маленького ряда следует расходимость более большого ряда
\end{theorem}

\begin{theorem}[Предельный признак сравнения]
    Пусть $u_1 + u_2 + u+3 + \dots$, $v_1 + v_2 + v_3 + \dots$ — знакоположительные числовые ряды, при этом $u_{i} \le v_{i}$

    Если $\lim\limits_{n \to \infty} \frac{u_{n}}{v_{n}} = C \ne 0 \ne \pm \infty$

    То данные ряды ведут себя одинаково: или сходятся, или расходятся одновременно
\end{theorem}

\pagebreak
\section{Высшая математика - 17.03.2023}

\subsection{Вычисление объема тела вращения}

\subsubsection{Примеры}

\paragraph{Пример №1} Объем тела вращения $y = x^2 - x$, $y = 0$ вокруг оси $OX$

$V = \pi \int\limits_{0}^{1} f^2(x) \diff x = \pi \int\limits_{0}^{1} (x^2-x)^2 \diff x = \pi \int\limits_{0}^{1} (x^4 - 2x^3 + x^2) \diff x = \pi \int\limits_{0}^{1} (x^4 - 2x^3 + x^2) \diff x = \pi (\frac{x^5}{5} - \frac{x^4}{2} + \frac{x^3}{3}) \bigg|_{0}^{1} = \pi (\frac{1}{5} - \frac{1}{2} + \frac{1}{3}) = \frac{\pi}{30}$

\pagebreak
\section{Высшая математика - 22.03.2023}

\subsection{Эталонные ряды}

\begin{enumerate}
    \item $\sum \frac{1}{n^{p}}$, $p > 1$ — ряд сходится, $p \le 1$ — ряд расходится
    \item $\sum \frac{1}{n e_{n}^{p} n}$ — аналогично
\end{enumerate}

\subsection{Исследование рядов на сходимость}

\paragraph{Пример №1}

$\sum \frac{\sin \frac{1}{n}}{n} \sim \sum \frac{1}{n^2}$, так как $\sin \alpha \sim \alpha$

Проверим: $\lim\limits_{n \to \infty} \frac{\frac{\sin \frac{1}{n}}{n}}{\frac{1}{n^2}} = \lim\limits_{n \to \infty} \frac{n \sin \frac{1}{n}}{1} = \lim\limits_{n \to \infty} \frac{\sin \frac{1}{n}}{\frac{1}{n}} = 1 \ne 0 \ne \pm \infty$

\paragraph{Пример №2}

$\sum \frac{\sqrt{n + 1} + \sqrt[3]{n}}{\sqrt[5]{n^12 + n} + \sqrt{n^3}} = \sum \frac{n^{1/2}}{n^{12/5}}  = \sum \frac{1}{n^{19/10}}$

$\lim\limits_{n \to \infty} \frac{\frac{\sqrt{n + 1} + \sqrt[3]{n}}{\sqrt[5]{n^12 + n} + \sqrt{n^3}}}{\frac{1}{n^{19/10}}} = \dots$

\subsection{Очередные признаки сходимости}

\begin{theorem}[Признак Д'Аламбера]

$\exists \lim\limits_{n \to \infty} \frac{u_{n + 1}}{u_{n}} = C$

Если $C > 1$ — ряд расходится, $C < 1$ — ряд сходится, если $C = 1$ — признак неприменим

\end{theorem}

\paragraph{Пример №1}

$\sum\limits_{n = 1}^{\infty} \frac{n}{2(n + 1) 3^{2 n - 1}}$, $a_{n} = \frac{n}{2(n + 1) 3^{2 n - 1}}$, $a_{n + 1} = \frac{n + 1}{2(n + 2) 3^{2 n + 1}}$

$\lim\limits_{n \to \infty} \frac{\frac{n + 1}{2(n + 2) 3^{2 n + 1}}}{\frac{n}{2(n + 1) 3^{2 n - 1}}} = \lim\limits_{n \to \infty} \frac{(n + 1) 3^{2 n - 1} 2 (n + 1)}{2 (n + 2) 3^{2 n + 1} n} = \lim\limits{n \to \infty} \frac{3^{2 n - 1}}{3^{2 n - 1} * 3^2} = \frac{1}{9} \le 1$ — ряд является сходящимся

\paragraph{Пример №2}

$\sum\limits_{n = 1}^{\infty} \frac{n}{3n^2 - 5}$, $a_{n} = \frac{n}{3n^2 - 5}$, $a_{n + 1} = \frac{n + 1}{3 (n + 1)^2 - 5}$

$\lim\limits_{n \to \infty} \frac{\frac{n + 1}{3 (n + 1)^2 - 5}}{\frac{n}{3n^2 - 5}} = \lim\limits_{n \to \infty} \frac{(n + 1_) (3n^2 - 5)}{(3n^2 + 6n - 2) n} - 1$ — данный признак неприменим, применяем признак сравнения

$\sum\limits_{n = 1}^{\infty} \frac{n}{3n^2 - 5} \sim \sum \frac{1}{n}$, $\lim\limits_{n \to \infty} \frac{\frac{n}{3n^2 - 5}}{\frac{1}{n}} = \frac{1}{3}$ — данные ряды ведут себя одинаково, так как один расходящийся — другой тоже расходящийся.


\begin{theorem}[Признак Коши (радикальный)]

$\exists \lim\limits_{n \to \infty} \sqrt[n]{u_{n}} = C$

Если $C > 1$ — ряд расходится, $C < 1$ — ряд сходится, если $C = 1$ — признак неприменим

\end{theorem}

\paragraph{Пример №1} $\sum\limits_{n = 1}^{\infty} \frac{(\frac{n + 1}{n})^{n^2}}{3^{n}}$

$\lim\limits{n \to \infty} \sqrt[n]{\frac{(\frac{n + 1}{n})^{n^2}}{3^{n}}} = \lim\limits_{n \to \infty} \frac{(\frac{n + 1}{n})^{n}}{3} = \frac{1}{3} \lim\limits_{n \to \infty} (1 + \frac{1}{n})^{n} = \frac{e}{3} \le 1$ — ряд сходится

\begin{theorem}[Интегральный признак Коши]

Пусть для знакоположительного ряда выполняется условие $u_1 \ge u_2 \ge u_3 \ge \dots$

Можно ввести такую $f(x)$, что $f(1) = u_1$, $f(2) = u_2$, $\dots$

Тогда, если существует $\int\limits_{1}^{\infty} f(x) \diff x$ и он сходится, то будет сходиться и ряд $\sum\limits_{n = 1}^{\infty} u_{n}$

\end{theorem}

\paragraph{Пример №1} $\sum\limits_{n = 1}^{\infty} \frac{1}{n^2} \to \int\limits_{1}^{\infty} \frac{\diff x}{x^2}$

$\int\limits_{1}^{\infty} \frac{\diff x}{x^2} = \lim\limits_{\beta \to \infty} \int\limits_{1}^{\beta} x^{-2} \diff x = \lim\limits_{\beta \to \infty} (-\frac{1}{x}) \bigg|_{1}^{\beta} = -\lim\limits_{\beta \to \infty} (\frac{1}{\beta} - 1) = 1$

\paragraph{Пример №2} $\sum\limits_{n = 1}^{\infty} \frac{1}{n} \to \int\limits_{1}^{\infty} \frac{\diff x}{x} = \lim\limits_{\beta \to \infty} \int\limits_{1}^{\beta} \frac{\diff x}{x} = \lim\limits_{\beta \to \infty} \ln |x| \bigg|_{1}^{\beta} = \lim\limits_{\beta \to \infty} (\ln \beta - \ln 1)$ — расходится, ряд является расходящимся

\paragraph{Пример №2} $\sum\limits_{n = 1}^{\infty} \frac{1}{\sqrt{n}} \to \int\limits_{1}^{\infty} x^{-1/2} \diff x = \lim\limits_{\beta \to \infty} \int\limits_{1}^{\beta} x^{-1/2} \diff x = \lim\limits_{\beta \to \infty} 2 \sqrt{x} \bigg|_{1}^{\beta} = \lim\limits_{\beta \to \infty} (2\sqrt{\beta} - 2)$ — расходится

\subsubsection{Замечания о применении признаков сходимости}

\begin{enumerate}

\item При исследовании рядов, общий член которых представляет собой логарифмическую функцию, мы можем пользоваться следующим знанием:

Если $p \in R$, $q > 0$, то $\exists n_0 \in N$, $n \ge n_0 \implies \ln^{p} n < n^{q}$

\item $n!$

$n \ge 4$, $2^{n} < n! < n^{n}$, $n \ln 2 < \ln(n!) < n \ln n$


\item $\lim\limits_{n \to \infty} \sqrt[n]{n} = 1$

\item \textbf{Формула Стирлинга}

$n! \sim (\frac{n}{e})^{n} \sqrt{2 \pi n}$

$\sum \frac{n! e^{n}}{n^{n + p}} = \sum\limits_{n}^{\infty} \frac{\sqrt{2 \pi} n^{1/2}}{n^{p}} = \sum \frac{\sqrt{2\pi}}{n^{p - 1/2}}$, $p - \frac{1}{2} > 1$, $p > \frac{3}{2}$

\item Если $\alpha$ ($\alpha \to 0$) — малый угол, то $\sin \alpha$, $\tg \alpha$, $\arcsin \alpha$, $\arccos \alpha \to \alpha$

\end{enumerate}

\subsection{Знакопеременные и знакочередующиеся ряды}

$u_1 - u_2 + u_3 - u_4 + u_5 - \dots$ — \textbf{знакочередующийся ряд}, сами $u_1, u_2, u_3 \dots > 0$

\begin{theorem}[Теорема Лейбница]

Если в знакочередующемся ряде выполнены условия $u_1 > u_2 > u_3 > \dots$ и $\lim\limits_{n \to \infty} u_n = 0$, то знакочередующийся ряд является сходящимся, его сумма положительна и не превосходит $u_1$

Доказательство: найдем частичную сумму четного числа элементов, $S_{2 m} = (u_1 - u_2) + (u_3 - u_4) + \dots + (u_{2 m - 1} - u_{2 m })$, $S_{2 m} > 0$

$S_{2 m} = u_1 - (u_2 - u_3) - (u_4 - u_5) - \dots - u_{2 m}$, $0 < S_{2 m} < u_1$

$S_{2 m + 1} = S_{2 m} + u_{2 m + 1}$, $\lim S_{2 m + 1} = \lim S_{2  m} + 0$, $S_{2 m + 1}$ также удовлетворяет всем условиям

Так что $0 < S < u_1$

\hfill

\textbf{Замечание}. Теорема работает, даже если данные неравенства выполнены не с первого, а с некоторого члена.

\end{theorem}

\paragraph{Пример №1} $\sum \frac{(-1)^{n + 1}}{n^2} = 1 - \frac{1}{4} + \frac{1}{9} - \frac{1}{16} + \dots \frac{(-1)^{n}}{n^2} \dots$, $1 > \frac{1}{4} > \frac{1}{9} > \frac{1}{16}$

$\frac{1}{n^2} - \frac{1}{(n + 1)^2} = \frac{(n + 1)^2 - n^2}{n^2 (n + 1)^2} = \frac{(n + 1 - n) (n + 1 + n)}{n^2 (n + 1)^2} = \frac{2 n + 1}{n^2 (n + 1)^2} > 0$

Следовательно, первое условие теоремы выполняется. Второе условие тоже выполняется: $\lim\limits_{n \to \infty} \frac{1}{n^2} = 0$

Следовательно, ряд сходится

\paragraph{Пример №2} $\sum \frac{(-1)^{n}}{n} = 1 - \frac{1}{2} + \frac{1}{3} - \frac{1}{4} + \frac{1}{5} - \dots$

$\lim\limits_{n \to \infty} \frac{1}{n} = 0$

Такой ряд тоже сходится

\paragraph{Пример №3} $\sum \frac{(-1)^{n + 1} \ln^{2} n}{n}$, $f(x) = \frac{\ln^2 x}{x}$, $f'(x) = \frac{2 \ln x * \frac{1}{x} * x - \ln^{2} x * 1}{x^2} = \frac{\ln x (2 - \ln x)}{x^{2}}$

Исследуем поведение данной производной, $\ln x > 2$, $x > e^{2}$

Будем рассматривать $u_8 > u_9 > u_{10} \dots$

$\lim\limits_{n \to \infty} \frac{\ln^{2} n}{n} = \lim \frac{2 \ln n * \frac{1}{n}}{1} = 2 \lim \frac{(\ln n)'}{(n)'} = 2 \lim \frac{1}{n} = 0$

Все условия признака выполнены, следовательно ряд является сходящимся

\subsubsection{Куча признаков и определений}

\begin{definition}

Знакопеременный ряд называется \textbf{абсолютно сходящимся}, если сходится ряд из модулей

\end{definition}

\begin{theorem}

Абсолютно сходящийся ряд является сходящимся

\end{theorem}

\begin{theorem}

Если ряды $\sum a_{n}$ и $\sum b_{n}$ являются абсолютно сходящимися, то для любых чисел $\alpha$ и $\beta$ ряд $\sum (\alpha a_n + \beta b_n)$ тоже является абсолютно сходящимся

\end{theorem}

\begin{definition}

Если ряд сходится абсолютно, он останется сходящимся при любой перестановке его членов, и сумма ряда не зависит от порядка этих самых членов

\textbf{Замечание.} Для сходимости по Лейбницу это может и не выполняться

\end{definition}

\begin{theorem}[Признак Дирихле]

Пусть для $\sum a_{m} * b_{n}$ последовательность $a_{n} \ge a_{n + 1}$ монотонно стремится к нулю и $\lim\limits_{n \to \infty} a_{n} = 0$, а последовательность частичных сумм для $b_n$ ограничена, то есть $\exists M > 0 \forall n \in N$

$|B_{n}| = |\sum\limits_{i = 1}^{n} b_i| \le M$

\end{theorem}

\begin{theorem}[Признак Абеля]

$\sum a_{m} * b_{n}$

\begin{enumerate}
    \item Последовательность ${ a_m }$ — монотонна и ограничена
    \item $\sum b_{n}$ — сходящийся
\end{enumerate}

Тогда $\sum a_{n} b_{n}$ — сходящийся

\end{theorem}

Если знакопеременный ряд сходится, но абсолютной сходимости нет, то говорят об \textbf{условной сходимости}

\paragraph{Пример №1} $\sum\limits_{n = 1}^{\infty} (-1)^{n + 1} \frac{n^3}{2^{n}}$

Проверим на абсолютную сходимость, исследуем $\sum \frac{n^3}{2^{n}}$, $\lim\limits_{n \to \infty} \frac{(n + 1)^2 2^{n}}{2^{n + 1} n^3} = \frac{1}{2} < 1$ — ряд сходится абсолютно

\paragraph{Пример №2} $\sum \frac{(-1)^{n}}{\sqrt{n}}$ — нет надежды на абсолютную сходимость, так как $\sum \frac{1}{\sqrt{n}}$ — расходящийся

Сходимость будет, но лишь условная. Абсолютной сходимости нет

\pagebreak
\section{Высшая математика - 27.03.2023}

\subsection{Разбор заданий по определенным интегралам}

\subsubsection{Задание №3 — задание №4}

\paragraph{В декартовой системе координат}

Смотри прикрепленное изображение №1

$S = \int\limits_{a}^{b} (f_2(x) - f_1(x)) \diff x$, $y = f_2(x)$, $y = f_1(x)$

\paragraph{В полярной системе координат}

Смотри прикрепленное изображение №2

$S = \frac{1}{2} \int\limits_{\alpha}^{\beta} (r_2^2(\phi) - r_1^2(\phi)) \diff \phi$, $r = r_2(\phi)$, $r = r_1(\phi)$

\hfill

Смотри прикрепленное изображение №3

$S = \frac{1}{2} \int\limits_{\alpha}^{\beta} r^2(\phi) \diff \phi$

\paragraph{Примеры}

\textbf{Пример №1}

Пусть $y = x^2$, $y = \frac{x^2}{2} + 1$. Найти площадь, ограниченную ими.

$x^2 = \frac{x^2}{2} + 1 \Longleftrightarrow x = \pm \sqrt{2}$

$S = \int\limits_{-\sqrt{2}}^{\sqrt{2}} (\frac{x^2}{2} + 1 - x^2) \diff x = \int\limits_{-\sqrt{2}}^{\sqrt{2}} (1 - \frac{x^2}{2}) \diff x = (x - \frac{x^3}{6}) \bigg|_{-\sqrt{2}}^{\sqrt{2}} = (\sqrt{2} - \frac{2\sqrt{2}}{6}) - (-\sqrt{2} - \frac{-2\sqrt{2}}{6}) = 2\sqrt{2} - \frac{2\sqrt{2}}{3} = \frac{4\sqrt{2}}{3}$

\hfill

\textbf{Пример №2}

Пусть $r = a \sqrt{\sin 4\phi}$, $\sin 4\phi \ge 0$, $r \ge 0$, $a > 0$, $0 \le 4\phi \le \pi \Longleftrightarrow 0 \le \phi \le \frac{\pi}{4}$

$S = \frac{1}{2} \int\limits_{0}^{\pi/4} (a \sqrt{\sin 4\phi})^2 \diff \phi = \frac{1}{2} \int\limits_{0}^{\pi/4} a^2 \sin 4 \phi \diff \phi = \frac{a^2}{2} (- \frac{\cos 4\phi}{4}) \bigg|_{0}^{\pi/4} = \frac{a^2}{2} (- \frac{(-1)}{4} - (- \frac{1}{4})) = \frac{a^2}{4}$ 

\subsubsection{Задание №5}

См. прикрепленное изображение №4

$V_{x} = \pi \int\limits_{a}^{b} (f_2^2(x) - f_1^2(x)) \diff x$, $V_y = \pi \int\limits_{c}^{d} (\phi_2^2(y) - \phi_1^2(y)) \diff y$

\paragraph{Примеры}

\textbf{Пример №1}

Пусть $y = x^2$, $y = \frac{x^2}{2} + 1$, $x = 0$

$V_x = \pi \int\limits_{0}^{\sqrt{2}} ((\frac{x^2}{2} + 1)^2 - (x^2)^2) \diff x = \pi \int\limits_{0}^{\sqrt{2}} (\frac{x^4}{4} + x^2 + 1 - x^4) \diff x = \pi (\frac{x^3}{3} + x - \frac{3x^5}{20}) \bigg|_{0}^{\sqrt{2}} = \pi (\frac{2\sqrt{2}}{3} + \sqrt{2} - \frac{3\sqrt{2}}{5}) = \pi (\frac{10\sqrt{2} + 15\sqrt{2} - 9\sqrt{2}}{15}) = \frac{16\sqrt{2}}{15} \pi$

$x = \sqrt{y}$, $x = \sqrt{2y - 2}$

$V_y = \pi \int\limits_{0}^{1} ((\sqrt{y})^2 - (\sqrt{2y - 2})^2) + \pi \int\limits_{1}^{2} ((\sqrt{y})^2 - (\sqrt{2y - 2})^2) = \pi (\frac{y^2}{2}) \bigg|_{0}^{1} + \pi (2y - \frac{y^2}{2}) \bigg|_{1}^{2} = \frac{\pi}{2} + \pi (4 - 2 - (2 - \frac{1}{2})) = \pi$

\subsubsection{Задание №11}

\paragraph{Примеры}

\textbf{Пример №1}

$Z = \sqrt{y - x^2} - \ln (x - y + 1) + \frac{x}{y}$. Изобразить область определения данной функции.

\begin{equation}
    \begin{cases}
        y - x^2 \ge 0 \\
        x - y + 1 > 0 \\
        y \ne 0
    \end{cases} \Longrightarrow
    \begin{cases}
        y \ge x^2 \\
        y < x + 1 \\
        y \ne 0
    \end{cases}
\end{equation}

Нарисовать эту фигню, что выше в виде системы представлена, после чего подумать и заштриховать то, что надо.

\subsection{Длина дуги кривой}

\subsubsection{В декартовой системе координат}

$l = \int\limits_{a}^{b} \sqrt{1 + (f'(x))^2} \diff x$

\paragraph{Пример №1}

$y = a \cosh \frac{x}{a}$, $x \in [-a; a]$

$y' = \sinh \frac{x}{a}$, $l = \int\limits_{-a}^{a} \sqrt{1 + (\sinh \frac{x}{a})^2} \diff x = \int\limits_{-a}^{a} \sqrt{1 + \sinh^2 \frac{x}{a}} \diff x = \int\limits_{-a}^{a} \sqrt{\cosh^2 \frac{x}{a}} \diff x = \int\limits_{-a}^{a} \cosh \frac{x}{a} \diff x = a \sinh \frac{x}{a} \bigg|_{-a}^{a} = (a \sinh 1) - (a \sinh (-1)) = 2 a \sinh 1$

\subsubsection{В полярной системе координат}

$l = \int\limits_{\alpha}^{\beta} \sqrt{r^2 (\phi) + (r'(\phi))^2} \diff \phi$

\paragraph{Пример №1}

$r = a (1 - \cos \phi)$, $a > 0$

$r' = a \sin \phi$

$l = 2 \int\limits_{0}^{\pi} \sqrt{(a (1 - \cos \phi))^2 + (a \sin \phi)^2} \diff \phi = 2 \int\limits_{0}^{\pi} \sqrt{(a - a \cos \phi)^2 + (a \sin \phi)^2} \diff \phi = 2 \int\limits_{0}^{\pi} \sqrt{a^2 - 2a^2 \cos^2 \phi + a^2 \sin^2 \phi} \diff \phi = 2 \int\limits_{0}^{\pi} \sqrt{a^2 ( 1 - 2\cos \phi + \cos^2 \phi + \sin^2 \phi ) } \diff \phi = 2a \int\limits_{0}^{\pi} \sqrt{2 - 2 \cos \phi } \diff \phi = 2 a \int\limits_{0}^{\pi} \sqrt{4 (\frac{1 - \cos \phi}{2})} \diff \phi = 4 a \int \limits_{0}^{\pi} \sqrt{\sin^2 \frac{\phi}{2}} \diff \phi = 4 a \int\limits_{0}^{\pi} \sin \frac{\phi}{2} \diff \phi = 4 a ( - 2 \cos \frac{\phi}{2} ) \bigg|_{0}^{\pi} = 4 a ( - (-2)) = 8 agith$

\subsubsection{В параметрической системе координат}

\begin{equation}
\begin{cases}
    x = x(t) \\
    y = y(t)
\end{cases}, t \in [t_1, t_2]
\end{equation}

$l = \int\limits_{t_1}^{t_2} \sqrt{(x'(t))^2 + (y'(t))^2} \diff t$

\paragraph{Пример №1}

Найти длину четверти окружности радиуса $a$

\begin{equation}
    \begin{cases}
        x = a \cos t \\
        y = a \sin t
    \end{cases}, t \in [0; \frac{\pi}{2}]
\end{equation}

$(x') = - a \sin t$, $(y') = a \cos t$

$l = \int\limits_{0}^{\pi/2} \sqrt{(-a \sin t)^2 + (a \cos t)^2} \diff t = \int\limits_{0}^{\pi/2} \sqrt{a^2 (\sin^2 t + \cos^2 t)} \diff t = a \int\limits_{0}^{\pi/2} \diff t = a (t \bigg|_{0}^{\pi/2}) = \frac{a \pi}{2}$

\pagebreak
\section{Высшая математика - 29.03.2023}

\subsection{Примеры применения признаков сходимости для исследования рядов}

\paragraph{Пример №1}

$\sum\limits_{n = 1}^{\infty} \frac{2 n + 1}{(n^2 + n) 2^n}$

$\lim\limits_{n \to \infty} \frac{a_{n + 1}}{a_{n}} = \lim\limits_{n \to \infty} \frac{\frac{2 ( n + 1 ) + 1}{((n + 1)^2 + (n + 1)) 2^{n + 1}}}{\frac{2 n + 1}{(n^2 + n) 2^n}} = \lim\limits_{n \to \infty} \frac{(2 n + 3) (n^2 + n) 2^{n}}{(n^2 + 3 n + 2) 2^{n} * 2 (2 n + 1)} = \frac{1}{2} < 1$ — ряд сходится

\paragraph{Пример №2}

$\sum\limits_{n = 1}^{\infty} \frac{(-1)^{n}}{(4 n - 1) 3^{n}}$, $\sum\limits_{n = 1}^{\infty} \frac{1}{(4 n -1) 3^{n}}$

$\lim\limits_{n \to \infty} \frac{a_{n + 1}}{a_{n}} = \lim\limits_{n \to \infty} \frac{(4 n - 1) 3^{n}}{(4 n + 3) 3^{n} * 3} = \frac{1}{3} < 1$ — сходится абсолютно

\paragraph{Пример №3}

$\sum \frac{(-1)^{n}}{4 n - 1}$, $\sum \frac{1}{4 n - 1} \sim \sum \frac{1}{n}$

$\lim\limits_{n \to \infty} \frac{\frac{1}{4 n - 1}}{\frac{1}{n}} = \frac{1}{4} \ne 0 \ne \pm \infty$ — нет абсолютной сходимости, есть сходимость по Лейбницу

\subsection{Функциональные ряды}

$\sum u_{n}(x) = u_1(x) + u_2(x) +u_3(x) + \dots + u_n(x)$, $S_{n}(x) = u_1(x) + u_2(x) +u_3(x) + \dots + u_n(x)$

$\lim\limits_{n \to \infty} S_{n} (x) = S(x)$

\begin{definition}
    \textbf{Областью сходимости функционального ряда} называется множество тех значений $x$, при которых ряд будет сходящимся.

    Тогда $\lim\limits_{n \to \infty} S_{n} (x) = S(x)$
\end{definition}

\paragraph{Пример №1}

$\sum \frac{(x - 2)^{2 n + 1}}{3^{n} * (n + 5)}$

$\lim\limits_{n \to \infty} | \frac{u_{ n + 1 } (x)}{u_{n} (x)} | = \lim | \frac{(x - 2)^{2 (n + 1) + 1} * 3^{n} * ( n + 5)}{3^{n + 1} (n + 1 + 5) (x - 2)^{2 n + 1}} | = \frac{1}{3} | x - 2 |^{2} < 1$

$|x - 2|^2 < 3 \Longleftrightarrow - \sqrt{3} < x - 2 < \sqrt{3} \Longleftrightarrow 2 - \sqrt{3} < x < 2 + \sqrt{3}$ — ряд сходится при этих условиях

Отдельно нужно проверить граничные значения:

$x = 2 + \sqrt{3}$, $\sum \frac{(2 + \sqrt{3} - 2)^{2 n + 1}}{3^{n} (n + 5)} = \sum\limits_{n = 1}^{\infty} \frac{\sqrt{3}}{n + 5} \sim \frac{1}{n}$ — расходящийся

$x = 2 - \sqrt{3}$,  $\sum \frac{(2 - \sqrt{3} - 2)^{2 n + 1}}{3^{n} (n + 5)} = \sum\limits_{n = 1}^{\infty} \frac{\sqrt{3}}{n + 5} \sim \frac{1}{n}$ — тоже расходящийся

\subsubsection{Сходимость функциональных рядов}

\begin{definition}
    Функциональный ряд называется равномерно сходящимся на некотором множестве $D$, если $\forall \ \epsilon > 0 \ \exists \ N_0$, не зависящее от $\epsilon$, что при $n > N_0$, и всех $x \in D$, выполняется следующее неравенство:

    $$| S(x) - S_{n} (x) | < \epsilon$$

    Если ряд является равномерно сходящимся, то он является и сходящимся
\end{definition}

\begin{definition}
    Функциональный ряд называется абсолютно сходящимся, если сходится ряд $\sum | u_{n} (x) |$
\end{definition}

\begin{theorem}[Мажорантный признак Вейерштрасса]
    Функциональный ряд сходится абсолютно и равномерно на множестве $D$, если существует сходящийся числовой ряд с положительными членами, и при том сходящийся, такой что

    $$
    |u_{i} (x)| \le a_{i} 
    $$

    Для всех $x \in D$
\end{theorem}

\paragraph{Пример №1}

$\sum\limits_{n = 1}^{\infty} \frac{\cos n x}{3^{n}} \sim \sum\limits_{n = 1}^{\infty} \frac{1}{3^{n}}$, $| \frac{\cos n x}{3^{n}} | \le \frac{1}{3^{n}}$

$S_{n} = b_{1} \frac{1 - q^{n}}{1 - q}$

$\lim\limits_{n \to \infty} \frac{1}{3} \frac{1 - \frac{1}{3^{n}}}{1 - \frac{1}{3}} = \frac{1}{2}$ — является сходящейся, так что $\sum\limits_{n = 1}^{\infty} \frac{\cos n x}{3^{n}}$ является абсолютно и равномерно сходящимся, $\sum\limits_{n = 1}^{\infty} \frac{1}{3^{n}}$ мажорирует данный ряд

\hfill

Для мажорируемых рядов \textbf{справедливы следующие теоремы}:

\begin{theorem}
    Сумма ряда из непрерывных функций, мажорируемого на $[a; b]$ есть функция, непрерывная на этом отрезке
\end{theorem}

\subsubsection{Почленное интегрирование и дифференцирование рядов}

\begin{theorem}[О почленном интегрировании]

Пусть $u_1 (x)$, $u_2 (x)$, $\dots$ — непрерывные функции и ряд из $u_{n} (x)$ является мажорируемым на интервале $[a; b]$, $S (x)$ — сумма этого ряда, тогда

$$
\int\limits_{a}^{x} s(t) \diff t = \int\limits_{a}^{x} u_1(x) \diff x + \int\limits_{a}^{x} u_2 (x) \diff x + \dots + \int \limits_{a}^{x} u_{n} (x) \diff x
$$

Если ряд не является мажорируемым, то почленное интегрирование не всегда возможно.

\end{theorem}

\begin{theorem}[О почленном дифференцировании]

$\sum u_{n} (x)$, $u_1 (x), u_2(x), \dots$ — имеют непрерывные производные на $[a; b]$

$\sum u_n (x) = S(x)$ — сумма ряда

Пусть ряд из производных является мажорируемым на $[a; b]$, тогда сумма ряда из производных будет являться производной от суммы исходного ряда:

$$
\sum u_{n}'(x) = S'(x)
$$

\end{theorem}

\paragraph{Пример №1}

$x + \frac{x^{5}}{5} + \frac{x^{9}}{9} + \dots + \frac{x^{4 n - 3}}{4 n - 3} + \dots = S$

$S_{'}(x) = 1 + x^{4} + x^{8} + \dots + x^{4 n - 4} + \dots$, $|x| < 1$, геометрическая прогрессия $b_{1} = 1$, $q = x^4$

$S_{'}(x) = \frac{1}{1 - x^{4}}$

$\int \frac{\diff x}{1 - x^{4}} = \int \frac{1}{(1 - x^2) (1 + x^2)} = \int \frac{1 / 2}{1 + x^2} \diff x + \int \frac{1 / 2}{1 - x^2} \diff x = \frac{1}{2} \arctg x + \frac{1}{2} * \frac{1}{2} \ln | \frac{1 + x}{1 - x} |$ — \textbf{искомая сумма ряда}

\end{document}