\documentclass{article}
\usepackage[utf8]{inputenc}

\usepackage[T2A]{fontenc}
\usepackage[utf8]{inputenc}
\usepackage[russian]{babel}

\usepackage{amsmath}
\usepackage{pgfplots}
\usepackage{multienum}
\usepackage{geometry}
\geometry{
    left=1cm,right=1cm,top=2cm,bottom=2cm
}
\newcommand*\diff{\mathop{}\!\mathrm{d}}

\newtheorem{definition}{Определение}
\newtheorem{theorem}{Теорема}

\DeclareMathOperator{\sign}{sign}

\usepackage{hyperref}
\hypersetup{
    colorlinks, citecolor=black, filecolor=black, linkcolor=black, urlcolor=black
}

\title{Высшая математика}
\author{Лисид Лаконский}
\date{March 2023}

\begin{document}
\raggedright

\maketitle

\tableofcontents
\pagebreak

\section{Высшая математика - 17.03.2023}

\subsection{Вычисление объема тела вращения}

\subsubsection{Примеры}

\paragraph{Пример №1} Объем тела вращения $y = x^2 - x$, $y = 0$ вокруг оси $OX$

$V = \pi \int\limits_{0}^{1} f^2(x) \diff x = \pi \int\limits_{0}^{1} (x^2-x)^2 \diff x = \pi \int\limits_{0}^{1} (x^4 - 2x^3 + x^2) \diff x = \pi \int\limits_{0}^{1} (x^4 - 2x^3 + x^2) \diff x = \pi (\frac{x^5}{5} - \frac{x^4}{2} + \frac{x^3}{3}) \bigg|_{0}^{1} = \pi (\frac{1}{5} - \frac{1}{2} + \frac{1}{3}) = \frac{\pi}{30}$

\end{document}