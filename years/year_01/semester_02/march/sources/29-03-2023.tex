\documentclass{article}
\usepackage[utf8]{inputenc}

\usepackage[T2A]{fontenc}
\usepackage[utf8]{inputenc}
\usepackage[russian]{babel}

\usepackage{amsmath}
\usepackage{pgfplots}
\usepackage{multienum}
\usepackage{geometry}
\geometry{
    left=1cm,right=1cm,top=2cm,bottom=2cm
}
\newcommand*\diff{\mathop{}\!\mathrm{d}}

\newtheorem{definition}{Определение}
\newtheorem{theorem}{Теорема}

\DeclareMathOperator{\sign}{sign}

\usepackage{hyperref}
\hypersetup{
    colorlinks, citecolor=black, filecolor=black, linkcolor=black, urlcolor=black
}

\title{Высшая математика}
\author{Лисид Лаконский}
\date{March 2023}

\begin{document}
\raggedright

\maketitle

\tableofcontents
\pagebreak

\section{Высшая математика - 29.03.2023}

\subsection{Примеры применения признаков сходимости для исследования рядов}

\paragraph{Пример №1}

$\sum\limits_{n = 1}^{\infty} \frac{2 n + 1}{(n^2 + n) 2^n}$

$\lim\limits_{n \to \infty} \frac{a_{n + 1}}{a_{n}} = \lim\limits_{n \to \infty} \frac{\frac{2 ( n + 1 ) + 1}{((n + 1)^2 + (n + 1)) 2^{n + 1}}}{\frac{2 n + 1}{(n^2 + n) 2^n}} = \lim\limits_{n \to \infty} \frac{(2 n + 3) (n^2 + n) 2^{n}}{(n^2 + 3 n + 2) 2^{n} * 2 (2 n + 1)} = \frac{1}{2} < 1$ — ряд сходится

\paragraph{Пример №2}

$\sum\limits_{n = 1}^{\infty} \frac{(-1)^{n}}{(4 n - 1) 3^{n}}$, $\sum\limits_{n = 1}^{\infty} \frac{1}{(4 n -1) 3^{n}}$

$\lim\limits_{n \to \infty} \frac{a_{n + 1}}{a_{n}} = \lim\limits_{n \to \infty} \frac{(4 n - 1) 3^{n}}{(4 n + 3) 3^{n} * 3} = \frac{1}{3} < 1$ — сходится абсолютно

\paragraph{Пример №3}

$\sum \frac{(-1)^{n}}{4 n - 1}$, $\sum \frac{1}{4 n - 1} \sim \sum \frac{1}{n}$

$\lim\limits_{n \to \infty} \frac{\frac{1}{4 n - 1}}{\frac{1}{n}} = \frac{1}{4} \ne 0 \ne \pm \infty$ — нет абсолютной сходимости, есть сходимость по Лейбницу

\subsection{Функциональные ряды}

$\sum u_{n}(x) = u_1(x) + u_2(x) +u_3(x) + \dots + u_n(x)$, $S_{n}(x) = u_1(x) + u_2(x) +u_3(x) + \dots + u_n(x)$

$\lim\limits_{n \to \infty} S_{n} (x) = S(x)$

\begin{definition}
    \textbf{Областью сходимости функционального ряда} называется множество тех значений $x$, при которых ряд будет сходящимся.

    Тогда $\lim\limits_{n \to \infty} S_{n} (x) = S(x)$
\end{definition}

\paragraph{Пример №1}

$\sum \frac{(x - 2)^{2 n + 1}}{3^{n} * (n + 5)}$

$\lim\limits_{n \to \infty} | \frac{u_{ n + 1 } (x)}{u_{n} (x)} | = \lim | \frac{(x - 2)^{2 (n + 1) + 1} * 3^{n} * ( n + 5)}{3^{n + 1} (n + 1 + 5) (x - 2)^{2 n + 1}} | = \frac{1}{3} | x - 2 |^{2} < 1$

$|x - 2|^2 < 3 \Longleftrightarrow - \sqrt{3} < x - 2 < \sqrt{3} \Longleftrightarrow 2 - \sqrt{3} < x < 2 + \sqrt{3}$ — ряд сходится при этих условиях

Отдельно нужно проверить граничные значения:

$x = 2 + \sqrt{3}$, $\sum \frac{(2 + \sqrt{3} - 2)^{2 n + 1}}{3^{n} (n + 5)} = \sum\limits_{n = 1}^{\infty} \frac{\sqrt{3}}{n + 5} \sim \frac{1}{n}$ — расходящийся

$x = 2 - \sqrt{3}$,  $\sum \frac{(2 - \sqrt{3} - 2)^{2 n + 1}}{3^{n} (n + 5)} = \sum\limits_{n = 1}^{\infty} \frac{\sqrt{3}}{n + 5} \sim \frac{1}{n}$ — тоже расходящийся

\subsubsection{Сходимость функциональных рядов}

\begin{definition}
    Функциональный ряд называется равномерно сходящимся на некотором множестве $D$, если $\forall \ \epsilon > 0 \ \exists \ N_0$, не зависящее от $\epsilon$, что при $n > N_0$, и всех $x \in D$, выполняется следующее неравенство:

    $$| S(x) - S_{n} (x) | < \epsilon$$

    Если ряд является равномерно сходящимся, то он является и сходящимся
\end{definition}

\begin{definition}
    Функциональный ряд называется абсолютно сходящимся, если сходится ряд $\sum | u_{n} (x) |$
\end{definition}

\begin{theorem}[Мажорантный признак Вейерштрасса]
    Функциональный ряд сходится абсолютно и равномерно на множестве $D$, если существует сходящийся числовой ряд с положительными членами, и при том сходящийся, такой что

    $$
    |u_{i} (x)| \le a_{i} 
    $$

    Для всех $x \in D$
\end{theorem}

\paragraph{Пример №1}

$\sum\limits_{n = 1}^{\infty} \frac{\cos n x}{3^{n}} \sim \sum\limits_{n = 1}^{\infty} \frac{1}{3^{n}}$, $| \frac{\cos n x}{3^{n}} | \le \frac{1}{3^{n}}$

$S_{n} = b_{1} \frac{1 - q^{n}}{1 - q}$

$\lim\limits_{n \to \infty} \frac{1}{3} \frac{1 - \frac{1}{3^{n}}}{1 - \frac{1}{3}} = \frac{1}{2}$ — является сходящейся, так что $\sum\limits_{n = 1}^{\infty} \frac{\cos n x}{3^{n}}$ является абсолютно и равномерно сходящимся, $\sum\limits_{n = 1}^{\infty} \frac{1}{3^{n}}$ мажорирует данный ряд

\hfill

Для мажорируемых рядов \textbf{справедливы следующие теоремы}:

\begin{theorem}
    Сумма ряда из непрерывных функций, мажорируемого на $[a; b]$ есть функция, непрерывная на этом отрезке
\end{theorem}

\subsubsection{Почленное интегрирование и дифференцирование рядов}

\begin{theorem}[О почленном интегрировании]

Пусть $u_1 (x)$, $u_2 (x)$, $\dots$ — непрерывные функции и ряд из $u_{n} (x)$ является мажорируемым на интервале $[a; b]$, $S (x)$ — сумма этого ряда, тогда

$$
\int\limits_{a}^{x} s(t) \diff t = \int\limits_{a}^{x} u_1(x) \diff x + \int\limits_{a}^{x} u_2 (x) \diff x + \dots + \int \limits_{a}^{x} u_{n} (x) \diff x
$$

Если ряд не является мажорируемым, то почленное интегрирование не всегда возможно.

\end{theorem}

\begin{theorem}[О почленном дифференцировании]

$\sum u_{n} (x)$, $u_1 (x), u_2(x), \dots$ — имеют непрерывные производные на $[a; b]$

$\sum u_n (x) = S(x)$ — сумма ряда

Пусть ряд из производных является мажорируемым на $[a; b]$, тогда сумма ряда из производных будет являться производной от суммы исходного ряда:

$$
\sum u_{n}'(x) = S'(x)
$$

\end{theorem}

\paragraph{Пример №1}

$x + \frac{x^{5}}{5} + \frac{x^{9}}{9} + \dots + \frac{x^{4 n - 3}}{4 n - 3} + \dots = S$

$S_{'}(x) = 1 + x^{4} + x^{8} + \dots + x^{4 n - 4} + \dots$, $|x| < 1$, геометрическая прогрессия $b_{1} = 1$, $q = x^4$

$S_{'}(x) = \frac{1}{1 - x^{4}}$

$\int \frac{\diff x}{1 - x^{4}} = \int \frac{1}{(1 - x^2) (1 + x^2)} = \int \frac{1 / 2}{1 + x^2} \diff x + \int \frac{1 / 2}{1 - x^2} \diff x = \frac{1}{2} \arctg x + \frac{1}{2} * \frac{1}{2} \ln | \frac{1 + x}{1 - x} |$ — \textbf{искомая сумма ряда}

\end{document}