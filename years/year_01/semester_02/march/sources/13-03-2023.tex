\documentclass{article}
\usepackage[utf8]{inputenc}

\usepackage[T2A]{fontenc}
\usepackage[utf8]{inputenc}
\usepackage[russian]{babel}

\usepackage{amsmath}
\usepackage{pgfplots}
\usepackage{multienum}
\usepackage{geometry}
\geometry{
    left=1cm,right=1cm,top=2cm,bottom=2cm
}
\newcommand*\diff{\mathop{}\!\mathrm{d}}

\newtheorem{definition}{Определение}
\newtheorem{theorem}{Теорема}

\DeclareMathOperator{\sign}{sign}

\usepackage{hyperref}
\hypersetup{
    colorlinks, citecolor=black, filecolor=black, linkcolor=black, urlcolor=black
}

\title{Высшая математика}
\author{Лисид Лаконский}
\date{March 2023}

\begin{document}
\raggedright

\maketitle

\tableofcontents
\pagebreak

\section{Высшая математика - 01.02.2023}

\subsection{Нахождение среднего значения функции в указанном промежутке}

$\mu = \frac{1}{b - a} \int\limits_{a}^{b} f(x) \diff x = F(x)\bigg|_{a}^{b} = F(b) - F(a)$

Если функция $y = f(x)$ непрерывна на отрезке $[a; b]$, то существует точка $c \in [ a; b ]$, что значение $f(c) = \mu$

Если $f(x) > 0$ при $x \in [a; b]$, то площадь криволинейной трапеции, ограниченной линиями $x = a$, $x = b$, $y = 0$ и $y = f(x)$ равна площади прямоугольника с основанием $[a, b]$ и высотой $f(c)$

\subsubsection{Примеры}

\paragraph{Пример №1} Найти среднее значение функции $y = 5x^4 - 2$ на промежутке $[1; 2]$

$\mu = \frac{1}{2 - 1} \int\limits_{1}^{2} (5x^4 - 2) \diff x = (x^5 - 2x) \bigg|_{1}^{2} = (32 - 4) - (1 - 2) = 29$

\subsection{Нахождение точек экстремума и точек перегиба}

Точки экстремума и точки перегиба функции $\Phi (x)$, заданной интегралом с переменным верхним пределом.

$$
\Phi(x) = \int\limits_{a}^{x} f(t) \diff t
$$

Если $f$ непрерывна в точке $x$, то $\Phi'(x) = f(x) \implies \Phi''(x) = f'(x)$

$\Phi(x) = \int\limits_{a}^{x} f(t) \diff t = F(t) \bigg|_{a}^{x} = F(x) - F(a)$


\hfill

$f'' > 0$ - вогнутая, $f'' < 0$ — выпуклая

\subsubsection{Примеры}

\paragraph{Пример №1} $\Phi(x) = \int\limits_{1}^{x} (t - t^3) \diff t = (\frac{t^2}{2} - \frac{t^4}{4}) \bigg|_{1}^{x} = \frac{x^2}{2} - \frac{x^4}{4} - \frac{1}{2} + \frac{1}{4} = \frac{x^2}{2} - \frac{x^4}{4} - \frac{1}{4}$

$x - x^3 = \Phi'(x) \Longleftrightarrow x - x^3 = 0 \Longleftrightarrow x_1 = 0, \ x_{2, 3} = \pm 1$

Изобразим знаки $\Phi'(x)$ и $\Phi(x)$ на координатной прямой с отмеченными точками $x = -1$, $x = 0$, $x = 1$. Найдем точки максимума и минимума; $x_{min} = 0$, $x_{max_1} = -1$, $x_{max_2} = 1$

$\Phi(0) = - \frac{1}{4}$ — min, $\Phi(\pm 1) = 0$ — max

\hfill

$\Phi''(x) = f'(x) = 1 - 3x^2$

$1 - 3x^2 = 0 \Longleftrightarrow x_{1, 2} = \pm \frac{1}{\sqrt{3}}$

Изобразим знаки $\Phi''(x)$ и $\Phi(x)$ на координатной прямой с отмеченными точками $x = -\frac{1}{\sqrt{3}}$, $x = \frac{1}{\sqrt{3}}$. Определим, на каких промежутках график функции вогнут, а на каких выпукл.


$\Phi(-\frac{1}{\sqrt{3}}) = \frac{1}{6} - \frac{1}{36} - \frac{1}{4} = -\frac{1}{9}$, $\Phi(\frac{1}{\sqrt{3}}) = -\frac{1}{9}$

$x = \pm \frac{1}{\sqrt{3}}$ — точки перегиба


\end{document}