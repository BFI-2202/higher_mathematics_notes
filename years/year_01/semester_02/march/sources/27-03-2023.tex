\documentclass{article}
\usepackage[utf8]{inputenc}

\usepackage[T2A]{fontenc}
\usepackage[utf8]{inputenc}
\usepackage[russian]{babel}

\usepackage{amsmath}
\usepackage{pgfplots}
\usepackage{multienum}
\usepackage{geometry}
\geometry{
    left=1cm,right=1cm,top=2cm,bottom=2cm
}
\newcommand*\diff{\mathop{}\!\mathrm{d}}

\newtheorem{definition}{Определение}
\newtheorem{theorem}{Теорема}

\DeclareMathOperator{\sign}{sign}

\usepackage{hyperref}
\hypersetup{
    colorlinks, citecolor=black, filecolor=black, linkcolor=black, urlcolor=black
}

\title{Высшая математика}
\author{Лисид Лаконский}
\date{March 2023}

\begin{document}
\raggedright

\maketitle

\tableofcontents
\pagebreak

\section{Высшая математика - 27.03.2023}

\subsection{Разбор заданий по определенным интегралам}

\subsubsection{Задание №3 — задание №4}

\paragraph{В декартовой системе координат}

Смотри прикрепленное изображение №1

$S = \int\limits_{a}^{b} (f_2(x) - f_1(x)) \diff x$, $y = f_2(x)$, $y = f_1(x)$

\paragraph{В полярной системе координат}

Смотри прикрепленное изображение №2

$S = \frac{1}{2} \int\limits_{\alpha}^{\beta} (r_2^2(\phi) - r_1^2(\phi)) \diff \phi$, $r = r_2(\phi)$, $r = r_1(\phi)$

\hfill

Смотри прикрепленное изображение №3

$S = \frac{1}{2} \int\limits_{\alpha}^{\beta} r^2(\phi) \diff \phi$

\paragraph{Примеры}

\textbf{Пример №1}

Пусть $y = x^2$, $y = \frac{x^2}{2} + 1$. Найти площадь, ограниченную ими.

$x^2 = \frac{x^2}{2} + 1 \Longleftrightarrow x = \pm \sqrt{2}$

$S = \int\limits_{-\sqrt{2}}^{\sqrt{2}} (\frac{x^2}{2} + 1 - x^2) \diff x = \int\limits_{-\sqrt{2}}^{\sqrt{2}} (1 - \frac{x^2}{2}) \diff x = (x - \frac{x^3}{6}) \bigg|_{-\sqrt{2}}^{\sqrt{2}} = (\sqrt{2} - \frac{2\sqrt{2}}{6}) - (-\sqrt{2} - \frac{-2\sqrt{2}}{6}) = 2\sqrt{2} - \frac{2\sqrt{2}}{3} = \frac{4\sqrt{2}}{3}$

\hfill

\textbf{Пример №2}

Пусть $r = a \sqrt{\sin 4\phi}$, $\sin 4\phi \ge 0$, $r \ge 0$, $a > 0$, $0 \le 4\phi \le \pi \Longleftrightarrow 0 \le \phi \le \frac{\pi}{4}$

$S = \frac{1}{2} \int\limits_{0}^{\pi/4} (a \sqrt{\sin 4\phi})^2 \diff \phi = \frac{1}{2} \int\limits_{0}^{\pi/4} a^2 \sin 4 \phi \diff \phi = \frac{a^2}{2} (- \frac{\cos 4\phi}{4}) \bigg|_{0}^{\pi/4} = \frac{a^2}{2} (- \frac{(-1)}{4} - (- \frac{1}{4})) = \frac{a^2}{4}$ 

\subsubsection{Задание №5}

См. прикрепленное изображение №4

$V_{x} = \pi \int\limits_{a}^{b} (f_2^2(x) - f_1^2(x)) \diff x$, $V_y = \pi \int\limits_{c}^{d} (\phi_2^2(y) - \phi_1^2(y)) \diff y$

\paragraph{Примеры}

\textbf{Пример №1}

Пусть $y = x^2$, $y = \frac{x^2}{2} + 1$, $x = 0$

$V_x = \pi \int\limits_{0}^{\sqrt{2}} ((\frac{x^2}{2} + 1)^2 - (x^2)^2) \diff x = \pi \int\limits_{0}^{\sqrt{2}} (\frac{x^4}{4} + x^2 + 1 - x^4) \diff x = \pi (\frac{x^3}{3} + x - \frac{3x^5}{20}) \bigg|_{0}^{\sqrt{2}} = \pi (\frac{2\sqrt{2}}{3} + \sqrt{2} - \frac{3\sqrt{2}}{5}) = \pi (\frac{10\sqrt{2} + 15\sqrt{2} - 9\sqrt{2}}{15}) = \frac{16\sqrt{2}}{15} \pi$

$x = \sqrt{y}$, $x = \sqrt{2y - 2}$

$V_y = \pi \int\limits_{0}^{1} ((\sqrt{y})^2 - (\sqrt{2y - 2})^2) + \pi \int\limits_{1}^{2} ((\sqrt{y})^2 - (\sqrt{2y - 2})^2) = \pi (\frac{y^2}{2}) \bigg|_{0}^{1} + \pi (2y - \frac{y^2}{2}) \bigg|_{1}^{2} = \frac{\pi}{2} + \pi (4 - 2 - (2 - \frac{1}{2})) = \pi$

\subsubsection{Задание №11}

\paragraph{Примеры}

\textbf{Пример №1}

$Z = \sqrt{y - x^2} - \ln (x - y + 1) + \frac{x}{y}$. Изобразить область определения данной функции.

\begin{equation}
	\begin{cases}
		y - x^2 \ge 0 \\
		x - y + 1 > 0 \\
		y \ne 0
	\end{cases} \Longrightarrow
	\begin{cases}
		y \ge x^2 \\
		y < x + 1 \\
		y \ne 0
	\end{cases}
\end{equation}

Нарисовать эту фигню, что выше в виде системы представлена, после чего подумать и заштриховать то, что надо.

\subsection{Длина дуги кривой}

\subsubsection{В декартовой системе координат}

$l = \int\limits_{a}^{b} \sqrt{1 + (f'(x))^2} \diff x$

\paragraph{Пример №1}

$y = a \cosh \frac{x}{a}$, $x \in [-a; a]$

$y' = \sinh \frac{x}{a}$, $l = \int\limits_{-a}^{a} \sqrt{1 + (\sinh \frac{x}{a})^2} \diff x = \int\limits_{-a}^{a} \sqrt{1 + \sinh^2 \frac{x}{a}} \diff x = \int\limits_{-a}^{a} \sqrt{\cosh^2 \frac{x}{a}} \diff x = \int\limits_{-a}^{a} \cosh \frac{x}{a} \diff x = a \sinh \frac{x}{a} \bigg|_{-a}^{a} = (a \sinh 1) - (a \sinh (-1)) = 2 a \sinh 1$

\subsubsection{В полярной системе координат}

$l = \int\limits_{\alpha}^{\beta} \sqrt{r^2 (\phi) + (r'(\phi))^2} \diff \phi$

\paragraph{Пример №1}

$r = a (1 - \cos \phi)$, $a > 0$

$r' = a \sin \phi$

$l = 2 \int\limits_{0}^{\pi} \sqrt{(a (1 - \cos \phi))^2 + (a \sin \phi)^2} \diff \phi = 2 \int\limits_{0}^{\pi} \sqrt{(a - a \cos \phi)^2 + (a \sin \phi)^2} \diff \phi = 2 \int\limits_{0}^{\pi} \sqrt{a^2 - 2a^2 \cos^2 \phi + a^2 \sin^2 \phi} \diff \phi = 2 \int\limits_{0}^{\pi} \sqrt{a^2 ( 1 - 2\cos \phi + \cos^2 \phi + \sin^2 \phi ) } \diff \phi = 2a \int\limits_{0}^{\pi} \sqrt{2 - 2 \cos \phi } \diff \phi = 2 a \int\limits_{0}^{\pi} \sqrt{4 (\frac{1 - \cos \phi}{2})} \diff \phi = 4 a \int \limits_{0}^{\pi} \sqrt{\sin^2 \frac{\phi}{2}} \diff \phi = 4 a \int\limits_{0}^{\pi} \sin \frac{\phi}{2} \diff \phi = 4 a ( - 2 \cos \frac{\phi}{2} ) \bigg|_{0}^{\pi} = 4 a ( - (-2)) = 8 agith$

\subsubsection{В параметрической системе координат}

\begin{equation}
\begin{cases}
	x = x(t) \\
	y = y(t)
\end{cases}, t \in [t_1, t_2]
\end{equation}

$l = \int\limits_{t_1}^{t_2} \sqrt{(x'(t))^2 + (y'(t))^2} \diff t$

\paragraph{Пример №1}

Найти длину четверти окружности радиуса $a$

\begin{equation}
	\begin{cases}
		x = a \cos t \\
		y = a \sin t
	\end{cases}, t \in [0; \frac{\pi}{2}]
\end{equation}

$(x') = - a \sin t$, $(y') = a \cos t$

$l = \int\limits_{0}^{\pi/2} \sqrt{(-a \sin t)^2 + (a \cos t)^2} \diff t = \int\limits_{0}^{\pi/2} \sqrt{a^2 (\sin^2 t + \cos^2 t)} \diff t = a \int\limits_{0}^{\pi/2} \diff t = a (t \bigg|_{0}^{\pi/2}) = \frac{a \pi}{2}$

\end{document}