\documentclass{article}
\usepackage[utf8]{inputenc}

\usepackage[T2A]{fontenc}
\usepackage[utf8]{inputenc}
\usepackage[russian]{babel}

\usepackage{amsmath}
\usepackage{pgfplots}
\usepackage{multienum}
\newcommand*\diff{\mathop{}\!\mathrm{d}}

\DeclareMathOperator{\sign}{sign}

\usepackage{hyperref}
\hypersetup{
    colorlinks, citecolor=black, filecolor=black, linkcolor=black, urlcolor=black
}

\title{Высшая математика}
\author{Лисид Лаконский}
\date{January 2023}

\begin{document}
\raggedright

\maketitle

\tableofcontents
\pagebreak

\section{Высшая математика - 31.01.2023}

\subsection{Интегралы}

$\int f(x) \diff x$, $(F(x))' = f(x)$

\textbf{Пример}: $\int x^2 \diff x = \frac{x^3}{3} + C$, $\int_{1}^{2} x^2 = \frac{8}{3} - \frac{1}{3} = \frac{7}{3}$

\subsubsection{Таблица интегралов}

\begin{multienumerate}
    \mitemxx{$\int x^{a} \diff x = \frac{x^{a + 1}}{a + 1} + C, a \ne 1$}{$\int \frac{1}{x} \diff x = \ln |x| + C$}
    \mitemxx{$\int a^{x} \diff x = \frac{a^{x}}{\ln a} + C$}{$\int e^{x} \diff x = e^{x} + C$}
    \mitemxx{$\int \cos x \diff x = \sin x + C$}{$\int \sin x \diff x = - \cos x + C$}
    \mitemxx{$\int \frac{\diff x}{\cos^2 x} \diff x = \tg x + C$}{$\int \frac{\diff x}{\sin^2 x} \diff x = - \ctg x + C$}
    \mitemxx{$\int \cosh x \diff x = \sinh + C$}{$\int \sinh \diff x = \cosh x + C$}
    \mitemxx{$\int \frac{\diff x}{\cosh^2 x} = \tanh x + C$}{$\int \frac{\diff x}{\sinh^2 x} = - \coth x + C$}
    \mitemxx{$\int \frac{\diff x}{\sqrt{a^2 - x^2}} = \arcsin \frac{x}{a} + C, a > 0$}{$\int \frac{\diff x}{a^2 + x^2} = \frac{1}{a} \arctg \frac{x}{a} + C$}
    \mitemxx{$\int \frac{\diff x}{\sqrt{x^2 \pm a^2}} = \ln |x + \sqrt{x^2 \pm a^2}| + C$}{$\int \frac{\diff x}{a^2 - x^2} = \frac{1}{2 a} \ln | \frac{a + x}{a - x}| + C$}
    \mitemx{$\int \frac{\diff x}{\sin x} = \ln |\tg \frac{x}{2}| + C$}
\end{multienumerate}

\subsubsection{Свойства интегралов}

\begin{multienumerate}
    \mitemxx{$[\int f(x) \diff x)]' = f(x)$}{$\diff \int f(x) \diff x = \int f(x) \diff x$}
    \mitemx{$\diff \int f(x) \diff x = [ \int f(x) \diff x]' \diff x = \int f(x) \diff x = F(x) + C$}
    \mitemxx{
        $\int F'(x) \diff x = F(x) + C$ \\
        $\int \diff F(x) = F(x) + C$
    }{$\int a f(x) \diff x = a \int f(x) \diff x$}
    \mitemx{$\int [f_1 (x) \pm f_2 (x)] \diff x = \int f_1 (x) \diff x \pm \int f_2 (x) \diff x$}
\end{multienumerate}

\subsubsection{Примеры решения}

\paragraph{Первый пример}

$\int \frac{x^2 \sqrt{x} - 3 x + 2 \sqrt{x} - 5}{x} \diff x = \int x^{\frac{3}{2}} \diff x - \int 3 \diff x + \int 2 x^{-\frac{3}{2}} \diff x - \int \frac{5}{x} \diff x = \frac{2}{5} x^{\frac{5}{2}} - 3x + 4 \sqrt{x} - 5 \ln |x| + C$

\paragraph{Второй пример}

$\int \tg^2 x \diff x = \int \frac{\sin^2 x}{\cos^2 x} \diff x = \int \frac{1 - \cos^2 x}{\cos^2 x} \diff x = \int \frac{1}{\cos^2 x} \diff x - \int \frac{\cos^2 x}{\cos^2 x} \diff x = \tg x - x + C$

\paragraph{Третий пример}

$\int x \ln x \diff x = \int \frac{\ln x}{\frac{1}{x}} \diff x = \frac{2 x^{\frac{1}{2}}}{x} = 2 x^{-\frac{1}{2}}$

\paragraph{Четвертый пример}

$\int \frac{\cos^2 x}{\sin 2 x} \diff x = \int \frac{\cos^2 x}{2 \cos x - \sin x} \diff x = \frac{1}{2} \int \frac{\cos x}{\sin x} \diff x = \begin{vmatrix}
    \sin x = t \\
    \cos \diff x = \diff t
\end{vmatrix} = \frac{1}{2} \int \frac{\diff t}{t} = \ln |t| + C = \ln |\sin x| + C$

\paragraph{Пятый пример}

$\int \frac{e^{2 x} \diff x}{e^{2 x} + 2} = \begin{vmatrix}
    e^{2 x} + 2 = t \\
    e^{2 x} \diff x = \diff t
\end{vmatrix} = \int \frac{\frac{1}{2} \diff t}{t} = 2 \ln t + C = 2 \ln (e^{2 x} + 2) + C$

\paragraph{Шестой пример}

$\int (x^{-\frac{3}{4}} + (4x^2 - 8)^{-\frac{1}{2}}) \diff x = \int (x^{-\frac{3}{4}}) \diff x + \int (4x^2 - 8)^{-\frac{1}{2}} \diff x = \frac{x^{\frac{1}{4}}}{\frac{1}{4}} + C_1 + \frac{1}{2} \int \frac{\diff x}{\sqrt{x^2 - 2}} = 4 x^{\frac{1}{4}} + C_1 + \frac{1}{2} \ln |x + \sqrt{x^2 - 2}| + C_2 = 4 \sqrt{x} + \frac{1}{2} \ln |x + \sqrt{x^2 - 2}| + C$

\paragraph{Седьмой пример}

$\int (2 \sin^2 2x - 1) \diff x = \int (2 \sin^2 2x) \diff x - \int (1) \diff x = \dots \text{ (решение оставляется читателю)}$

\end{document}
