\documentclass{article}
\usepackage[utf8]{inputenc}

\usepackage[T2A]{fontenc}
\usepackage[utf8]{inputenc}
\usepackage[russian]{babel}

\usepackage{amsmath}
\usepackage{pgfplots}
\usepackage{multienum}
\usepackage{geometry}
\geometry{
    left=1cm,right=1cm,top=2cm,bottom=2cm
}
\newcommand*\diff{\mathop{}\!\mathrm{d}}

\newtheorem{definition}{Определение}
\newtheorem{theorem}{Теорема}

\DeclareMathOperator{\sign}{sign}

\usepackage{hyperref}
\hypersetup{
    colorlinks, citecolor=black, filecolor=black, linkcolor=black, urlcolor=black
}

\title{Высшая математика}
\author{Лисид Лаконский}
\date{February 2023}

\begin{document}
\raggedright

\maketitle

\tableofcontents
\pagebreak

\section{Высшая математика - 17.02.2023}

\subsection{Подстановки Эйлера}

$\int R (x, \sqrt{ax^2 + bx + c} \diff x$

\begin{enumerate}
    \item Если $a > 0$, то $\sqrt{a x^2 + bx + c} = t \pm x \sqrt{a}$
    \item Если $c > 0$, то $\sqrt{a x^2 + bx + c} = t x \pm \sqrt{c}$
    \item $a x^2 + b x + c = a (x - \alpha) (x - \beta)$, $a$ — действительный корень, то можно выполнить замену $\sqrt{a x^2 + b x + c} = (x - \alpha) t$, аналогичную замену можно провести, если $\beta$ — действительный корень
\end{enumerate}

\paragraph{Пример. } $\int \frac{x \diff x}{(\sqrt{7x - 10 - x^2})^3}$ = (*)

\hfill

Решим квадратное уравнение: $x^2 - 7x - 10 = 0 \Longleftrightarrow (x - 2)(x - 5) = 0$

$\sqrt{-(x - 2)(x - 5)} = (x - 2) t \Longleftrightarrow (x - 2) (5 - x) = (x - 2)^2 t^2 \Longleftrightarrow (5 - x) = x t^2 - 2t^2$

Отсюда выразим $x = \frac{2t^2 + 5}{t^2 + 1}$, $\diff x = \frac{4t (t^2 + 1) - (2t^2 + 5) 2 t}{(t^2 + 1)} \diff t = \frac{-6 t \diff t}{(t^2 + 1)^2} = (\frac{2t^2 + t}{t^2 + 1} - 2) t = \frac{2t^2 + 5 - 2t^2 - 2}{t^2 + 1} = \frac{3t}{t^2 + 1}$

\hfill

(*) = $\int \frac{\frac{2t^2 + 5}{t^2 + 1} \frac{(-6t) \diff t}{(t^2 + 1)^2}}{(\frac{3t}{t^2 + 1})^3} = \int \frac{(2t^2 + 5) \diff t}{t^2} = -\frac{2}{9} \int (2 + \frac{5}{t^2}) \diff t = - \frac{2}{9} (2t - \frac{5}{t}) + C$, \ $t = \frac{\sqrt{7x-10-x^2}}{x-2}$

\subsection{Определенный интеграл}

\subsubsection{Определение и геометрическое значение}

\begin{definition}
    Если при любых разбиениях отрезка $[a; b]$ таких, что наибольшее значение $\Delta X_{i} \to 0$ и любом выборе точек $\xi_{i}$ существует $\lim\limits_{max \ \Delta X_{i} \to 0} \sum\limits_{i = 1}^{n} f(\xi_{i}) \Delta x_{i}$, то он называется определенным интегралом $S = \int\limits_{a}^{b} f(x) \diff x$ ($a < b$)

    Если этот предел существует, то функция считается интегрируемой на отрезке $[a; b]$

    Если $b < a$, то $\int\limits^{b}_{a} f(x) \diff x = - \int\limits_b^{a} f(x) \diff x$
\end{definition}

Определённый интеграл от неотрицательной функции $\int\limits_{a}^{b} f(x) \diff x$ \textbf{численно равен площади фигуры}, ограниченной осью абсцисс, прямыми $x = a$ и $x = b$ и графиком функции $f(x)$

\subsubsection{Основные свойства определенного интеграла}

\begin{multienumerate}
    \mitemxx{$\int\limits_{a}^b A f(x) \diff x = A \int\limits_a^b f(x) \diff x$}{$\int\limits_a^b (f_1 \pm f_2) \diff x = \int f_1 \diff x \pm \int f_2 \diff x$}
    \mitemx{Если $f(x) \le g(x)$, то $\int\limits_a^b f(x) \diff x \le \int\limits_a^b g(x) \diff x$}
    \mitemx{Если $m$ и $M$ — наименьшее и наибольшее $f(x)$ на $[a; b]$ ($a \le b$), то $m (b - a) \le \int\limits_a^b f(x) \diff x \le M (b - a)$}
    \mitemx{\textbf{Теорема о среднем}. Если $f(x)$ непрерывна на $[a; b] \ \exists \ C \in [a; b]$, то $\int\limits_a^b f(x) \diff x = (b - a) f(c)$ }
    \mitemx{При любом расположении $a$, $b$ и $c$ справедливо $\int\limits_a^b f(x) \diff x = \int\limits_a^c f(x) \diff x + \int\limits_c^b f(x) \diff x$}
\end{multienumerate}

\subsubsection{Вычисление определенного интеграла. Формула Ньютона-Лейбница}

\begin{theorem}
Если $f(x)$ — непрерывная функция, $\Phi(x) = \int\limits_a^x f(t) \diff t$, то имеет место $\Phi'(x) = f(x)$
\end{theorem}

\begin{theorem}[Формула Ньютона-Лейбница]
Если $F(x)$ — какая-либо первообразная для $f(x)$, то справедливо $\int\limits_{a}^{b} f(x) \diff x = F(b) - F(a)b$
\end{theorem}

\paragraph{Пример 1. } $\int\limits_1^{\sqrt{3}} \frac{\diff x}{1 + x^2} = (*)$, $\int \frac{\diff x}{1 + x^2} = \arctg x$

(*) = $\arctg x \bigg|_1^{\sqrt{3}} = \arctg \sqrt{3} - \arctg 1 = \frac{\pi}{3} - \frac{\pi}{4} = \frac{\pi}{12}$

\paragraph{Пример 2. } $\int\limits_0^2 |1 - x| \diff x = \int\limits_0^1 (1 - x) \diff x + \int\limits_1^2 (x - 1) \diff x = (x - \frac{x^2}{2}) \bigg|_{0}^{1} + (\frac{x^2}{2} - x) \bigg|_{1}^{2} = 1$

\subsubsection{Замена переменной в определенном интеграле}

\begin{theorem}
Пусть дан интеграл $\int\limits_{a}^{b} f(x) \diff x$, где $f(x)$ — непрерывная на $[a; b]$ функция

Вводим $t$, исходя из формулы $x = \phi(t)$. Если

\begin{multienumerate}
    \mitemxxx{$\phi(\alpha) = a$, $\phi(\beta) = b$}{$\phi$, $\phi$' непрерывна на $[a; b]$}{$f(\phi(t))$ определена и непрерывна на $[\alpha; \beta]$}
\end{multienumerate}

то $\int\limits_{a}^{b} f(x) \diff x = \int\limits_{\alpha}^{\beta} f(\phi(t)) \phi'(t) \diff t$
\end{theorem}

\paragraph{Пример 1.} $\int\limits_{2}^{4} \frac{x \diff x}{\sqrt{2 + 4x}} = (*)$, выполним замену $t = \sqrt{2 + 4x} \implies x = \frac{t^2 - 2}{4}$, $\diff x = \frac{t \diff t}{2}$

(*) = $\int\limits_{\sqrt{6}}^{\sqrt{12}} \frac{t^2 - 2}{4t} * \frac{1}{2} \diff t = \frac{1}{8} \int\limits_{\sqrt{6}}^{\sqrt{18}} (t^2 - 2) \diff t = \frac{1}{8} (\frac{t^3}{3} - 2t) \bigg|_{\sqrt{6}}^{\sqrt{18}} = \frac{1}{8} (\frac{18 \sqrt{18}}{3} - 2 \sqrt{18} - \frac{6\sqrt{6}}{3} + 2 \sqrt{6}) = \frac{3\sqrt{2}}{2}$

\paragraph{Пример 2. } $\int\limits_{3/4}^{4/3} \frac{\diff x}{x\sqrt{x^2 + 1}} = (*)$, выполним замену $t = \frac{1}{x} \implies x = \frac{1}{t}$, $\diff x = -\frac{1}{t^2} \diff t$

(*) = $
\int\limits_{4/3}^{3/4} \frac{-\frac{1}{t^2} \diff t}{\frac{1}{t} \frac{\sqrt{1 + t^2}}{t}} = \int\limits_{3/4}^{4/3} \frac{\frac{1}{t^2} \diff t}{\frac{1}{t} \frac{\sqrt{1 + t^2}}{t}} = \ln |t + \sqrt{1 + t^2}| \bigg|_{3/4}^{4/3} = \ln |\frac{4}{3} + \sqrt{1 + \frac{16}{9}} - \ln |\frac{3}{4} + \sqrt{1 + \frac{9}{16}}| = \int\limits_{\sqrt{6}}^{\sqrt{12}} \frac{t^2 - 2}{4t} * \frac{1}{2} \diff t = \frac{1}{8} \int\limits_{\sqrt{6}}^{\sqrt{18}} (t^2 - 2) \diff t = \frac{1}{8} (\frac{t^3}{3} - 2t) \bigg|_{\sqrt{6}}^{\sqrt{18}} = \ln \frac{3}{2}
$


\subsubsection{Интегрирование по частям в неопределенном интеграле}

$\int\limits_{a}^{b} (u v)' \diff x = \int\limits_{a}^{b} u' v \diff x + \int\limits_{a}^{b} u v' \diff x$

$\int\limits_{a}^{b} u \diff v = u v \bigg|_{a}^{b} - \int\limits_{a}^{b} v \diff u$

\paragraph{Пример 1. } $\int\limits_{0}^{\pi/4} x \cos 2x \diff x = (*)$, пусть $u = x$, $\diff v = \cos 2 x \diff x$, $\diff u = \diff x$, $v = \frac{1}{2} \sin 2x$

(*) = $[ x * \frac{1}{2} \sin 2 x - \frac{1}{2} \int \sin 2 x \diff x] \bigg|_{0}^{\pi/4} = [\frac{x}{2} \sin 2x + \frac{1}{4} \cos 2x] \bigg|_{0}^{\pi/4} = \frac{\pi}{8} * 1 - 0 + 0 - \frac{1}{4} * 1 = \frac{\pi}{8} - \frac{1}{4}$

\subsubsection{Упрощение интегралов, основанное на свойстве симметрии подынтегральных функций}

\begin{enumerate}
    \item Если функция $f(x)$ является четной на симметричном интервале $[-a; a]$, то $\int\limits_{-a}^{a} f(x) \diff x = 2 \int\limits_{0}^{a} f(x) \diff x$
    \item Если $f(x)$ является нечетной на $[-a; a]$, то $\int\limits_{-a}^{a} f(x) \diff x = 0$
    \item Если $f(x)$ является периодической функцией (то есть, $f(x) = f(x + T)$), то $\int\limits_{a}^{b} = \int\limits_{a + n T}^{b + n T} f(x) \diff x$
\end{enumerate}

\end{document}