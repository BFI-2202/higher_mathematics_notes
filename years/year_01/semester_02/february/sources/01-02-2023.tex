\documentclass{article}
\usepackage[utf8]{inputenc}

\usepackage[T2A]{fontenc}
\usepackage[utf8]{inputenc}
\usepackage[russian]{babel}

\usepackage{amsmath}
\usepackage{pgfplots}
\usepackage{multienum}
\usepackage{geometry}
\geometry{
    left=1cm,right=1cm,top=2cm,bottom=2cm
}
\newcommand*\diff{\mathop{}\!\mathrm{d}}

\newtheorem{definition}{Определение}
\newtheorem{theorem}{Теорема}

\DeclareMathOperator{\sign}{sign}

\usepackage{hyperref}
\hypersetup{
    colorlinks, citecolor=black, filecolor=black, linkcolor=black, urlcolor=black
}

\title{Высшая математика}
\author{Лисид Лаконский}
\date{February 2023}

\begin{document}
\raggedright

\maketitle

\tableofcontents
\pagebreak

\section{Высшая математика - 01.02.2023}

\subsection{Дифференциал функции}

$f'(x) = \lim_{x \to 0} \frac{\Delta f}{\Delta x}$, $\frac{\Delta f}{\Delta x} = f'(x) + \alpha(x)$, где $\alpha$ — бесконечно малая

$\Delta f = \underbrace{f'(x) \Delta x}_{\delta f \text{ дифференциал }} + \alpha (x) \Delta x$

\begin{definition}
\textbf{Дифференциалом} функции называется главная часть приращения функции, линейная относительно $\Delta x$

Дифференциал независимосй переменной: $\delta x = \Delta x$
\end{definition}

\subsubsection{Инвариантность формы дифференциала первого порядка}

$\delta f = f'(x) \Delta x$

Если $f = f(u(x))$, то $\delta f = f'_u u'_x \delta x = f'(u) \delta u$

\subsection{Первообразная и неопределенный интеграл}

\begin{definition}
Функция $F(x)$ называется \textbf{первообразной} от $f(x)$ на $[a; b]$, если на всех точках данного отрезка выполняется условие, что $F'(x) = f(x)$
\end{definition}

\begin{theorem}
Если $F_1(x)$ и $F_2(x)$ первообразные для $f(x)$ на некотором отрезке, то $F_1(x) - F_2(x) = const$

\textbf{Следовательно}, $F(x) + C$ — также первообразная для данной функции 
\end{theorem}

\hfill

$\int f(x) \diff x = F(x) + C$ — \textbf{неопределенный интеграл}, где $f(x)$ называется \textbf{подинтегральной функцией}, а $x$ называется \textbf{переменной интегрирования}

\subsubsection{Свойства неопределенного интеграла}

\begin{multienumerate}
    \mitemxx{$(\int f(x) \diff x)' = (F(x) + C)' = f(x)$}{$\diff (\int f(x) \diff x) = f(x) \diff x$}
    \mitemxx{$\int \diff (F(x)) = F(x) + C$}{$\int (f_1(x) + f_2(x)) \diff x = \int f_1(x) \diff x + \int f_2(x) \diff x$}
    \mitemx{$\int \alpha f(x) \diff x = \alpha \int f(x) \diff x$}
    \mitemx{Если $\int f(x) \diff x = F(x) + C$, то
    \begin{enumerate}
        \item $\int f(\alpha x) \diff x = \frac{1}{a} F(\alpha x) + C$
        \item $\int f(x + b) \diff x = F(x + b) + C$
        \item $\int f(\alpha x + b) \diff x = \frac{1}{a} F(\alpha x + b) + C$
    \end{enumerate}}
\end{multienumerate}

\subsubsection{Таблица неопределенных интегралов}

\begin{multienumerate}
    \mitemxx{$\int x^{n} \diff x = \frac{x^{n + 1}}{n + 1} + C$}{$\int \frac{\diff x}{x} = \ln |x| + C$}
    \mitemxx{$\int \sin x \diff x = - \cos x + C$}{$\int \cos x \diff x = \sin x + C$}
    \mitemxx{$\int \frac{\diff x}{\cos^2 x} = \tg x + C$}{$\int \frac{\diff x}{\sin^2 x} = - \ctg x + C$}
    \mitemxx{$\int \tg x \diff x = - \ln (\cos x) + C$}{$\int \ctg x \diff x = \ln |\sin x| + C$}
    \mitemxx{$\int e^{x} \diff x = e^{x} + C$}{$\int a^{x} \diff x = \frac{a^{x}}{\ln a} + C$}
    \mitemxx{$\int \frac{\diff x}{1 + x^2} = \arctg x + C$}{$\int \frac{\diff x}{a^2 - x^2} = \frac{1}{2 a} \ln | \frac{a + x}{a - x} | + C$}
    \mitemxx{$\int \frac{\diff x}{\sqrt{a^2 - x^2}} = \arcsin \frac{x}{a} + C$}{$\int \frac{\diff x}{\sqrt{x^2 + a}} = \ln |x + \sqrt{x^2 + a}| + C$}
\end{multienumerate}

\subsubsection{Метод подведения под знак дифференциала}

$\int f(x) \diff x = F(x) + C$, $\int f(y) \diff y = F(y) + C$

$\int f(y(x)) \diff (y(x)) = F(y(x)) + C$

\hfill

$\int x^2 \diff x = \frac{x^3}{3} + C$, $\int (x + 5)^2 \diff (x + 5) = \frac{(x + 5)^3}{3} + C$, $\int (\sin t)^2 \diff (\sin t) = \frac{(\sin t)^3}{3} + C$

Но если нам нужно найти $\int (2x + 7)^2 \diff x$, то \textbf{преобразуем} следующим образом: $\frac{1}{2} \int (2 x + 7)^2 2 \diff x = \frac{1}{2} \int (2 x + 7)^2 \diff (2 x + 7) = \frac{1}{2} \frac{(2 x + 7)^3}{3} + C$, \textbf{так как} $\diff (2 x + 7) = 2 \diff x$

\paragraph{Первый пример}

$\int \sqrt{x + 7} \diff x = \int \sqrt{x + 7} \diff (x + 7) = \frac{2}{3} (x + 7)^{\frac{3}{2}} + C$

\paragraph{Второй пример}

$\int x \sqrt{x^2 + 7} \diff x$, $\diff (x^2 + 7) = 2 x \diff x$, $\int x \sqrt{x^2 + 7} \diff x = \frac{1}{2} \int \sqrt{x^2 + 7} \diff (x^2 + 7) = \dots$

\paragraph{Третий пример}

$\int \frac{\diff x}{3x + 5} = \frac{1}{3} \int \frac{3 \diff x}{3 x + 5} = \frac{1}{3} \int \frac{\diff (3x + 5)}{3x + 5} = \frac{1}{3} \ln |3 x + 5| + C$, так как $\diff (3 x + 5) = 3 \diff x$

\paragraph{Четвертый пример}

$\int \frac{2 x \diff x}{x^2 + 1} = \int \frac{\diff (x^2 + 1)}{x^2 + 1} = \ln |x^2 + 1| + C$, так как $\diff (x^2 + 1) = 2 x \diff x$

\paragraph{Пятый пример}

$\int \frac{(2 x + 5) \diff x)}{x^2 + 5 x + 11} = \int \frac{\diff (x^2 + 5x + 11)}{(x^2 + 5x + 11)} = \ln |x^2 + 5x + 11| + C$

\paragraph{Шестой пример}

$\int \frac{\diff x}{x^2 + 4x + 5} = \int \frac{\diff x}{(x^2 + 4x + 4) + 1} = \int \frac{\diff (x + 2)}{(x + 2)^2 + 1} = \arctg (x + 2) + C$

\paragraph{Седьмой пример}

$\int e^{\sin x} \cos x \diff x = \int e^{\sin x} \diff (\sin x) = e^{\sin x} + C$, $\diff (\sin x) = \cos x \diff x$

Можно пойти другим путем: $\int e^{\sin x} \cos x \diff x = \int d (e^{\sin x}) = e^{\sin x} + C$, $\diff (e^{\sin x}) = e^{\sin x} * \cos \diff x$

\subsubsection{Метод замены переменной в неопределенном интеграле}

Пусть имеем $\int t(x) \diff x$, можем выполнить замену: $\begin{vmatrix}
    x = \phi (t) \\
    \diff x = \phi'(t) \diff t
\end{vmatrix} = \int f(\phi(t)) \phi'(t) \diff t$

\textbf{Замечание}: иногда будет удобней сразу выполнить замену $t = \phi(x)$

\hfill

\paragraph{Первый пример}

$\int \sqrt{1 - x^2} \diff x$, выполним тригонометрическую подстановку $x = \sin t$, $\diff x = \cos t \diff t$, тогда $\int \sqrt{1 - x^2} \diff x = \int \sqrt{1 - \sin^2 t} * \cos t \diff t = \int \cos^2 t \diff t = \dots$ 

Воспользуемся \textbf{формулами понижения степени}: $\dots = \frac{1}{2} \int (1 + \cos 2 t) \diff t = \frac{1}{2} (t + \frac{1}{2} \sin 2t) + C = \dots$

Выполним \textbf{обратную замену}: $\dots = \frac{1}{2} (\arcsin x + x \sqrt{1 - x^2}) + C$, так как $t = \arcsin x$, $x = \sin t$, $\diff x = \cos t \diff t$, $\cos t = \sqrt{1 - \sin^2 t}$

\paragraph{Второй пример}

$\int \frac{x \diff x}{1 + x^4} = \int \frac{x \diff x}{1 + (x^2)^2} = \begin{vmatrix}
    x = \sqrt{t} \\
    \diff x = \frac{1}{2 \sqrt{t}}
\end{vmatrix} = \frac{1}{2} \int \frac{\diff t}{1 + t^2} = \frac{1}{2} \arctg t + C = \frac{1}{2} \arctg x^2 + C$

Данный пример можно было бы также решить методом подведения под знак дифференциала: $\int \frac{x \diff x}{1 + x^4} = \frac{1}{2} \arctg x^2 + C$, так как $\diff (x^2) = 2 x \diff x$

\paragraph{Третий пример}

$\int e^{\sin x} \cos x \diff x = \begin{vmatrix} 
    \sin x = t \Longleftrightarrow x = \arcsin t \\
    \diff x = \frac{\diff t}{\sqrt{1 - t^2}} \\
    \cos x = \sqrt{1 - \sin^2 x} = \sqrt{1 - t^2}
\end{vmatrix} = \int e^{t} * \sqrt{1 - t^2} \frac{\diff t}{\sqrt{1 - t^2}} = \int e^{t} \diff t = e^{t} + C = e^{\sin x} + C$

\end{document}
