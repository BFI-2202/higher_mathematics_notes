\documentclass{article}
\usepackage[utf8]{inputenc}

\usepackage[T2A]{fontenc}
\usepackage[utf8]{inputenc}
\usepackage[russian]{babel}

\usepackage{amsmath}
\usepackage{pgfplots}
\usepackage{multienum}
\usepackage{geometry}
\geometry{
    left=1cm,right=1cm,top=2cm,bottom=2cm
}
\newcommand*\diff{\mathop{}\!\mathrm{d}}

\newtheorem{definition}{Определение}
\newtheorem{theorem}{Теорема}

\DeclareMathOperator{\sign}{sign}

\usepackage{hyperref}
\hypersetup{
    colorlinks, citecolor=black, filecolor=black, linkcolor=black, urlcolor=black
}

\title{Высшая математика}
\author{Лисид Лаконский}
\date{February 2023}

\begin{document}
\raggedright

\maketitle

\tableofcontents
\pagebreak

$\int \frac{2 x \diff x}{x^2 + 1} = \int \frac{\diff (x^2 + 1)}{x^2 + 1} = \ln |x^2 + 1| + C$, так как $\diff (x^2 + 1) = 2 x \diff x$

\paragraph{Пятый пример}

$\int \frac{(2 x + 5) \diff x)}{x^2 + 5 x + 11} = \int \frac{\diff (x^2 + 5x + 11)}{(x^2 + 5x + 11)} = \ln |x^2 + 5x + 11| + C$

\paragraph{Шестой пример}

$\int \frac{\diff x}{x^2 + 4x + 5} = \int \frac{\diff x}{(x^2 + 4x + 4) + 1} = \int \frac{\diff (x + 2)}{(x + 2)^2 + 1} = \arctg (x + 2) + C$

\paragraph{Седьмой пример}

$\int e^{\sin x} \cos x \diff x = \int e^{\sin x} \diff (\sin x) = e^{\sin x} + C$, $\diff (\sin x) = \cos x \diff x$

Можно пойти другим путем: $\int e^{\sin x} \cos x \diff x = \int d (e^{\sin x}) = e^{\sin x} + C$, $\diff (e^{\sin x}) = e^{\sin x} * \cos \diff x$

\subsubsection{Метод замены переменной в неопределенном интеграле}

Пусть имеем $\int t(x) \diff x$, можем выполнить замену: $\begin{vmatrix}
    x = \phi (t) \\
    \diff x = \phi'(t) \diff t
\end{vmatrix} = \int f(\phi(t)) \phi'(t) \diff t$

\textbf{Замечание}: иногда будет удобней сразу выполнить замену $t = \phi(x)$

\hfill

\paragraph{Первый пример}

$\int \sqrt{1 - x^2} \diff x$, выполним тригонометрическую подстановку $x = \sin t$, $\diff x = \cos t \diff t$, тогда $\int \sqrt{1 - x^2} \diff x = \int \sqrt{1 - \sin^2 t} * \cos t \diff t = \int \cos^2 t \diff t = \dots$ 

Воспользуемся \textbf{формулами понижения степени}: $\dots = \frac{1}{2} \int (1 + \cos 2 t) \diff t = \frac{1}{2} (t + \frac{1}{2} \sin 2t) + C = \dots$

Выполним \textbf{обратную замену}: $\dots = \frac{1}{2} (\arcsin x + x \sqrt{1 - x^2}) + C$, так как $t = \arcsin x$, $x = \sin t$, $\diff x = \cos t \diff t$, $\cos t = \sqrt{1 - \sin^2 t}$

\paragraph{Второй пример}

$\int \frac{x \diff x}{1 + x^4} = \int \frac{x \diff x}{1 + (x^2)^2} = \begin{vmatrix}
    x = \sqrt{t} \\
    \diff x = \frac{1}{2 \sqrt{t}}
\end{vmatrix} = \frac{1}{2} \int \frac{\diff t}{1 + t^2} = \frac{1}{2} \arctg t + C = \frac{1}{2} \arctg x^2 + C$

Данный пример можно было бы также решить методом подведения под знак дифференциала: $\int \frac{x \diff x}{1 + x^4} = \frac{1}{2} \arctg x^2 + C$, так как $\diff (x^2) = 2 x \diff x$

\paragraph{Третий пример}

$\int e^{\sin x} \cos x \diff x = \begin{vmatrix} 
    \sin x = t \Longleftrightarrow x = \arcsin t \\
    \diff x = \frac{\diff t}{\sqrt{1 - t^2}} \\
    \cos x = \sqrt{1 - \sin^2 x} = \sqrt{1 - t^2}
\end{vmatrix} = \int e^{t} * \sqrt{1 - t^2} \frac{\diff t}{\sqrt{1 - t^2}} = \int e^{t} \diff t = e^{t} + C = e^{\sin x} + C$

\pagebreak
\section{Высшая математика - 08.02.2023}

\subsection{Метод интегрирования по частям}

$\int u \diff v = u v - \int v \diff u$

\begin{enumerate}
    \item \textbf{многочлен} * \textbf{тригонометрическую или показательную функцию}, то \\
    за $u$ выбирают многочлен, $\diff v$ — все, что осталось \\
    \textbf{Пример} $\int (3 x + 1) \cos 5 x \diff x = \frac{(3x + 1)}{5} \sin 5 x - \frac{3}{5} \int \sin 5 x \diff x = \frac{(3 x + 1)}{5} \sin 5 x + \frac{3}{25} \cos 5 x + C$ \\
    $du = 3 d x$, $v = \int \cos 5 x \diff x = \frac{1}{5} \sin 5x$ \\
    \textbf{Другой пример} $\int (3 x^2 + 1) \cos 5 x \diff x = \frac{(3 x^2 + 1)}{5} \sin 5 x + \frac{6}{5} \int x \sin 5 x \diff x$, дальше следует применить метод интегрирования по частям заново
    \item \textbf{многочлен} * \textbf{логарифмическую или обратную тригонометрическую функцию}, то \\
    за $u$ выбирают функцию, а $\diff v$ — все остальное \\
    \textbf{Пример} $\int (3x^2 + 5) \ln x \diff x = (\frac{x^3}{3} + 5 x) \ln x - \int (\frac{x^2}{3} + 5 x) \frac{\diff x}{1} = (\frac{x^3}{3} + 5x) \ln x - \frac{x^3}{9} - 5 x + C$ \\
    $ln x = u \Longrightarrow \frac{\diff x}{x} = \diff u$, $\diff v = (x^2 + 5) \diff x \Longrightarrow v = \int (x^2 + 5) \diff x = \frac{x^3}{3} + 5 x$
    \item \textbf{тригонометрическая функция} * \textbf{показательную функцию}, то \\
    не имеет значения, что выбрать за $u$, а что за $\diff v$, но формулу интегрирования по частям в этом случае \textbf{придется применить два раза подряд} единообразно \\
    \textbf{Пример} $\int \sin 5x e^{x} \diff x = \sin 5 x * e^{x} - 5 \int \cos 5 x * e^{x} \diff x = \dots$ \\
    Пусть $u = \sin 5x \Longrightarrow \diff u = 5 \cos 5 x \diff x$, $\diff v = e^{x} \diff x \Longrightarrow v = e^{x}$ \\
    \textbf{Применим метод интегрирования по частям во второй раз}, теперь $u = \cos 5 x \Longrightarrow \diff u = - 5 \sin 5 x \diff x$, $v = e^{x} \diff x \Longrightarrow v = e^{x}$ \\
    $\dots = \sin 5 x * e^{x} - 5 (\cos 5 x e^{x} + 5 \int \sin 5 x e^{x} \diff x)$ \\
    $y = (\sin 5 x - 5 \cos 5 x) e^{x} - 25 y \Longleftrightarrow 26 y = (\dots) e^{x} \Longleftrightarrow y = \frac{(\sin 5x - 5 \cos 5 x) e^{x}}{26}$, где $y = \int \sin 5 x e^{x} \diff x$
\end{enumerate}

\paragraph{Применения метода интегрирования по частям к произвольным интегралам}

$\int \sqrt{1 - x^2} \diff x = \dots$, пусть $u = \sqrt{1 - x^2} \Longrightarrow \diff u = \frac{1(-2x) \diff x}{2\sqrt{1 - x^2}}$, а $v = \diff x \Longrightarrow v = x$

$\dots = x \sqrt{1 - x^2} - \int \frac{1 - x^2 - 1}{\sqrt{1 - x^2}} \diff x = x \sqrt{1 - x^2} - \int \sqrt{1 - x^2} \diff x + \int \frac{\diff x}{\sqrt{1 - x^2}}$, пусть $y = \int \sqrt{1 - x^2} \diff x$, тогда

$y = x \sqrt{1 - x^2} - y + \arcsin x \Longleftrightarrow y = \frac{x \sqrt{1 - x^2} + \arcsin x}{2}$

\subsection{Рекуррентные формулы}

\subsubsection{Рекуррентная формула №1}

$y_{n} = \int \frac{\diff x}{(x^2 + a^2)^{n}}$

\hfill

$\int \frac{\diff x}{(x^2 + a^2)^{n}} = \dots$, пусть $u = \frac{1}{(x^2 + a^2)^{n}} \Longrightarrow \diff u = - 2 n x (x^2 + a^2)^{-n - 1} \diff x$, а $\diff v = \diff x \Longrightarrow x = v$

$\dots = \frac{x}{(x^2 + a^2)^{n}} + 2 n \int \frac{(x^2 + a^{2}) - a^{2}}{(x^2 + a^2)^{n + 1}} \diff x \Longrightarrow \frac{x}{(x^2 + a^2)^{n}} + 2 n \int \frac{\diff x}{(x^2 + a^2)^{n}} - 2 n a^2 \int \frac{\diff x}{(x^2 + a^2)^{n + 1}}$

\hfill 

$y_n = \int \frac{\diff x}{(x^2 + a^2)^{n}}$, $y_{n + 1} = \int \frac{\diff x}{(x^2 + a^2)^{n + 1}}$, $y_n = \frac{x}{(x^2 + a^2)^{n}} + 2 n y_n - 2 n a ^2 y_{n + 1} \Longleftrightarrow 2 n a^2 y_{n + 1} = \frac{x}{(x^2 + a^2)^{n}} + y_n (2 n - 1) \Longleftrightarrow y_{n + 1} = \frac{1}{2 n a^2} \frac{x}{(x^2 + a^2)^{n}} + \frac{2 n - 1}{2 n a^2} y_n$ — \textbf{рекуррентная формула}

\hfill

Например, $\int \frac{\diff x}{(x^2 + a^2)^{2}} = \frac{1}{2 a^2} \frac{x}{(x^2 + a^2)} + \frac{1}{2a^2} * \frac{1}{a} \arctg \frac{x}{a}$

\subsubsection{Рекуррентная формула №2}

$y_{n, - m} = \int \frac{\sin^{n} x}{\cos^{m} x} \diff x$

$y_{n, - m} = \frac{\sin^{n - 1} x}{(m - 1) \cos^{m - 1} x} - \frac{n - 1}{m - 1} y_{n - 2, 2 - m}$

\subsubsection{Рекуррентная формула №3}

$y_{n} = \int (a^2 - x^2)^{n} \diff x$

$y_{n} = \frac{x (a^2 - x^2)^{n}}{2 n + 1} + \frac{2 n a^2}{2 n + 1} y_{n - 1}$

\subsection{Интегрирование функций, содержащих квадратный трехчлен}

$\int \frac{\diff x}{a x^2 + b x + c} = \frac{1}{a} \int \frac{\diff x}{x^2 + 2 \frac{b}{2 a} x + (\frac{b}{2 a})^2 - (\frac{b}{2 a})^2 + C} = \frac{1}{a} \int \frac{\diff x}{(x + \frac{b}{2 a}) + (C - (\frac{b}{2 a})^2)}$

\subsubsection{Пример №1}

$\int \frac{\diff x}{x^2 + 2x + 2} = \int \frac{\diff x}{(x^2 + 2 x + 1) - 1 + 2} = \int \frac{\diff (x + 1)}{(x + 1)^2 + 1} = \arctg (x + 1) + C$

\subsubsection{Пример №2}

$\int \frac{(2 x + 3) \diff x}{x^2 + 3x + 5} = \ln |x^2 + 3x + 5|$, так как $(2 x + 3) \diff x = \diff (x^2 + 3x + 5) + C$

\subsubsection{Пример №3}

$\int \frac{(2 x + 4) \diff x}{x^2 + 3x + 5} = \int \frac{(2 x + 3) \diff x}{x^2 + 3x + 5} + \int \frac{\diff x}{x^2 + 2 \frac{3 x}{2} + \frac{9}{4} - \frac{9}{4} + 5} = \int \frac{\diff x + \frac{3}{2}}{(x + \frac{3}{2})^2 + (\frac{\sqrt{11}}{2})^{2}} = \frac{2}{\sqrt{11}} \arctg \frac{x + \frac{3}{2}}{\frac{\sqrt{11}}{2}} + C $

\subsection{Интегрирование рациональных дробей}

Комплексные корни многочлена с действительными коэффициентами \textbf{являются попарно-сопряженными}: $a + i b$, $a - i b$

$(x - (a + i b))(x - (a - i b)) = x^2 - x (a + i b) - x (a - i b) + (a^2 - (i b)^2) = x^2 - 2 a x + (a^2 + b^2) = (x^2 + p x + q)$

\hfill

Интегрирование рациональных дробей будет сводиться к интегрированию элементарных дробей: каждую рациональную дробь мы можем свести к линейной комбинации из элементарных дробей

\textbf{Виды элементарных дробей}:

\begin{multienumerate}
    \mitemxx{$\frac{1}{x - a}$}{$\frac{1}{(x - a)^{n}}$}
    \mitemxx{$\frac{1}{x^2 + p x + 1}$}{$\frac{i}{(x^2 + px + q)^{m}}$}
\end{multienumerate}

\textbf{Правила сведения рациональной дроби к линейной комбинации из элементарных дробей}:

\begin{enumerate}
    \item Если знаменатель имеет только действительные различные корни \\
    То, например, $\frac{2 x - 3}{(x - 4)(x + 5)} = \frac{A}{x - 4} + \frac{B}{x + 5} = \frac{A(x + 5) + B(x - 4)}{(x - 4)(x + 5)} = \frac{A x + 5 A + B x - 4 B}{(x - 4)(x + 5)}$ (\textbf{метод неопределенных коэффициентов}) \\
    \textbf{Собираем коэффициенты} при $x$: A + B = 2, при свободных членах: $5 A - 4 B = -3$, \textbf{решаем данную систему любым угодным нам способом}, получается $A = \frac{5}{9}$, а $B = 2 - \frac{5}{9} = \frac{13}{9}$ \\
    $\int \frac{2 x - 3}{(x - 4)(x + 5)} \diff x = \frac{5}{9} \int \frac{\diff x}{x - 4} + \frac{13}{9} \int \frac{\diff x}{x + 5}= \frac{5}{9} \ln |x - 4| + \frac{13}{9} \ln |x + 5|$
    \item Если знаменатель имеет действительные кратные корни \\
    То, например, $\frac{7 x - 8}{(x - 4)^{2} (x+5)^{3}} = \frac{A}{x - 4} + \frac{B}{(x - 4)^2} + \frac{C}{x + 5} + \frac{D}{(x + 5)^2} + \frac{E}{(x + 5)^3}$
    \item Если знаменатель имеет комплексные корни (различные) \\
    То, например, $\frac{2 x - 3}{(x^2 + 1) (x^2 + 2 x + 10)} = \frac{A x + B}{x^2 + 1} + \frac{C x + D}{x^2 + 2x + 10}$ \\
    $\int \frac{A x + B}{x^2 + 1} \diff x = A \int \frac{x \diff x}{x^2 + 1} + B \int \frac{\diff x}{x^2 + 1} = \dots$
    \item Если знаменатель имеет комплексные кратные корни \\
    То. например, $\frac{2 x - 3}{(x^2 + 1)^2 (x^2 + 2x + 10)^2} = \frac{A x + B}{x^2 + 1} + \frac{C x + D}{(x^2 + 1)^2} + \frac{M x + N}{x^2 + 2x + 10} + \frac{P x + Q}{(x^2 + 2x + 10)^2}$
\end{enumerate}

\end{document}
