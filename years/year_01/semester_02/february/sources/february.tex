\documentclass{article}
\usepackage[utf8]{inputenc}

\usepackage[T2A]{fontenc}
\usepackage[utf8]{inputenc}
\usepackage[russian]{babel}

\usepackage{amsmath}
\usepackage{pgfplots}
\usepackage{multienum}
\usepackage{geometry}
\geometry{
    left=1cm,right=1cm,top=2cm,bottom=2cm
}
\newcommand*\diff{\mathop{}\!\mathrm{d}}

\newtheorem{definition}{Определение}
\newtheorem{theorem}{Теорема}

\DeclareMathOperator{\sign}{sign}

\usepackage{hyperref}
\hypersetup{
    colorlinks, citecolor=black, filecolor=black, linkcolor=black, urlcolor=black
}

\title{Высшая математика}
\author{Лисид Лаконский}
\date{February 2023}

\begin{document}
\raggedright

\maketitle

\tableofcontents
\pagebreak

\section{Высшая математика - 01.02.2023}

\subsection{Дифференциал функции}

$f'(x) = \lim_{x \to 0} \frac{\Delta f}{\Delta x}$, $\frac{\Delta f}{\Delta x} = f'(x) + \alpha(x)$, где $\alpha$ — бесконечно малая

$\Delta f = \underbrace{f'(x) \Delta x}_{\delta f \text{ дифференциал }} + \alpha (x) \Delta x$

\begin{definition}
\textbf{Дифференциалом} функции называется главная часть приращения функции, линейная относительно $\Delta x$

Дифференциал независимосй переменной: $\delta x = \Delta x$
\end{definition}

\subsubsection{Инвариантность формы дифференциала первого порядка}

$\delta f = f'(x) \Delta x$

Если $f = f(u(x))$, то $\delta f = f'_u u'_x \delta x = f'(u) \delta u$

\subsection{Первообразная и неопределенный интеграл}

\begin{definition}
Функция $F(x)$ называется \textbf{первообразной} от $f(x)$ на $[a; b]$, если на всех точках данного отрезка выполняется условие, что $F'(x) = f(x)$
\end{definition}

\begin{theorem}
Если $F_1(x)$ и $F_2(x)$ первообразные для $f(x)$ на некотором отрезке, то $F_1(x) - F_2(x) = const$

\textbf{Следовательно}, $F(x) + C$ — также первообразная для данной функции 
\end{theorem}

\hfill

$\int f(x) \diff x = F(x) + C$ — \textbf{неопределенный интеграл}, где $f(x)$ называется \textbf{подинтегральной функцией}, а $x$ называется \textbf{переменной интегрирования}

\subsubsection{Свойства неопределенного интеграла}

\begin{multienumerate}
    \mitemxx{$(\int f(x) \diff x)' = (F(x) + C)' = f(x)$}{$\diff (\int f(x) \diff x) = f(x) \diff x$}
    \mitemxx{$\int \diff (F(x)) = F(x) + C$}{$\int (f_1(x) + f_2(x)) \diff x = \int f_1(x) \diff x + \int f_2(x) \diff x$}
    \mitemx{$\int \alpha f(x) \diff x = \alpha \int f(x) \diff x$}
    \mitemx{Если $\int f(x) \diff x = F(x) + C$, то
    \begin{enumerate}
        \item $\int f(\alpha x) \diff x = \frac{1}{a} F(\alpha x) + C$
        \item $\int f(x + b) \diff x = F(x + b) + C$
        \item $\int f(\alpha x + b) \diff x = \frac{1}{a} F(\alpha x + b) + C$
    \end{enumerate}}
\end{multienumerate}

\subsubsection{Таблица неопределенных интегралов}

\begin{multienumerate}
    \mitemxx{$\int x^{n} \diff x = \frac{x^{n + 1}}{n + 1} + C$}{$\int \frac{\diff x}{x} = \ln |x| + C$}
    \mitemxx{$\int \sin x \diff x = - \cos x + C$}{$\int \cos x \diff x = \sin x + C$}
    \mitemxx{$\int \frac{\diff x}{\cos^2 x} = \tg x + C$}{$\int \frac{\diff x}{\sin^2 x} = - \ctg x + C$}
    \mitemxx{$\int \tg x \diff x = - \ln (\cos x) + C$}{$\int \ctg x \diff x = \ln |\sin x| + C$}
    \mitemxx{$\int e^{x} \diff x = e^{x} + C$}{$\int a^{x} \diff x = \frac{a^{x}}{\ln a} + C$}
    \mitemxx{$\int \frac{\diff x}{1 + x^2} = \arctg x + C$}{$\int \frac{\diff x}{a^2 - x^2} = \frac{1}{2 a} \ln | \frac{a + x}{a - x} | + C$}
    \mitemxx{$\int \frac{\diff x}{\sqrt{a^2 - x^2}} = \arcsin \frac{x}{a} + C$}{$\int \frac{\diff x}{\sqrt{x^2 + a}} = \ln |x + \sqrt{x^2 + a}| + C$}
\end{multienumerate}

\subsubsection{Метод подведения под знак дифференциала}

$\int f(x) \diff x = F(x) + C$, $\int f(y) \diff y = F(y) + C$

$\int f(y(x)) \diff (y(x)) = F(y(x)) + C$

\hfill

$\int x^2 \diff x = \frac{x^3}{3} + C$, $\int (x + 5)^2 \diff (x + 5) = \frac{(x + 5)^3}{3} + C$, $\int (\sin t)^2 \diff (\sin t) = \frac{(\sin t)^3}{3} + C$

Но если нам нужно найти $\int (2x + 7)^2 \diff x$, то \textbf{преобразуем} следующим образом: $\frac{1}{2} \int (2 x + 7)^2 2 \diff x = \frac{1}{2} \int (2 x + 7)^2 \diff (2 x + 7) = \frac{1}{2} \frac{(2 x + 7)^3}{3} + C$, \textbf{так как} $\diff (2 x + 7) = 2 \diff x$

\paragraph{Первый пример}

$\int \sqrt{x + 7} \diff x = \int \sqrt{x + 7} \diff (x + 7) = \frac{2}{3} (x + 7)^{\frac{3}{2}} + C$

\paragraph{Второй пример}

$\int x \sqrt{x^2 + 7} \diff x$, $\diff (x^2 + 7) = 2 x \diff x$, $\int x \sqrt{x^2 + 7} \diff x = \frac{1}{2} \int \sqrt{x^2 + 7} \diff (x^2 + 7) = \dots$

\paragraph{Третий пример}

$\int \frac{\diff x}{3x + 5} = \frac{1}{3} \int \frac{3 \diff x}{3 x + 5} = \frac{1}{3} \int \frac{\diff (3x + 5)}{3x + 5} = \frac{1}{3} \ln |3 x + 5| + C$, так как $\diff (3 x + 5) = 3 \diff x$

\paragraph{Четвертый пример}

$\int \frac{2 x \diff x}{x^2 + 1} = \int \frac{\diff (x^2 + 1)}{x^2 + 1} = \ln |x^2 + 1| + C$, так как $\diff (x^2 + 1) = 2 x \diff x$

\paragraph{Пятый пример}

$\int \frac{(2 x + 5) \diff x)}{x^2 + 5 x + 11} = \int \frac{\diff (x^2 + 5x + 11)}{(x^2 + 5x + 11)} = \ln |x^2 + 5x + 11| + C$

\paragraph{Шестой пример}

$\int \frac{\diff x}{x^2 + 4x + 5} = \int \frac{\diff x}{(x^2 + 4x + 4) + 1} = \int \frac{\diff (x + 2)}{(x + 2)^2 + 1} = \arctg (x + 2) + C$

\paragraph{Седьмой пример}

$\int e^{\sin x} \cos x \diff x = \int e^{\sin x} \diff (\sin x) = e^{\sin x} + C$, $\diff (\sin x) = \cos x \diff x$

Можно пойти другим путем: $\int e^{\sin x} \cos x \diff x = \int d (e^{\sin x}) = e^{\sin x} + C$, $\diff (e^{\sin x}) = e^{\sin x} * \cos \diff x$

\subsubsection{Метод замены переменной в неопределенном интеграле}

Пусть имеем $\int t(x) \diff x$, можем выполнить замену: $\begin{vmatrix}
    x = \phi (t) \\
    \diff x = \phi'(t) \diff t
\end{vmatrix} = \int f(\phi(t)) \phi'(t) \diff t$

\textbf{Замечание}: иногда будет удобней сразу выполнить замену $t = \phi(x)$

\hfill

\paragraph{Первый пример}

$\int \sqrt{1 - x^2} \diff x$, выполним тригонометрическую подстановку $x = \sin t$, $\diff x = \cos t \diff t$, тогда $\int \sqrt{1 - x^2} \diff x = \int \sqrt{1 - \sin^2 t} * \cos t \diff t = \int \cos^2 t \diff t = \dots$ 

Воспользуемся \textbf{формулами понижения степени}: $\dots = \frac{1}{2} \int (1 + \cos 2 t) \diff t = \frac{1}{2} (t + \frac{1}{2} \sin 2t) + C = \dots$

Выполним \textbf{обратную замену}: $\dots = \frac{1}{2} (\arcsin x + x \sqrt{1 - x^2}) + C$, так как $t = \arcsin x$, $x = \sin t$, $\diff x = \cos t \diff t$, $\cos t = \sqrt{1 - \sin^2 t}$

\paragraph{Второй пример}

$\int \frac{x \diff x}{1 + x^4} = \int \frac{x \diff x}{1 + (x^2)^2} = \begin{vmatrix}
    x = \sqrt{t} \\
    \diff x = \frac{1}{2 \sqrt{t}}
\end{vmatrix} = \frac{1}{2} \int \frac{\diff t}{1 + t^2} = \frac{1}{2} \arctg t + C = \frac{1}{2} \arctg x^2 + C$

Данный пример можно было бы также решить методом подведения под знак дифференциала: $\int \frac{x \diff x}{1 + x^4} = \frac{1}{2} \arctg x^2 + C$, так как $\diff (x^2) = 2 x \diff x$

\paragraph{Третий пример}

$\int e^{\sin x} \cos x \diff x = \begin{vmatrix} 
    \sin x = t \Longleftrightarrow x = \arcsin t \\
    \diff x = \frac{\diff t}{\sqrt{1 - t^2}} \\
    \cos x = \sqrt{1 - \sin^2 x} = \sqrt{1 - t^2}
\end{vmatrix} = \int e^{t} * \sqrt{1 - t^2} \frac{\diff t}{\sqrt{1 - t^2}} = \int e^{t} \diff t = e^{t} + C = e^{\sin x} + C$

\pagebreak
\section{Высшая математика - 08.02.2023}

\subsection{Метод интегрирования по частям}

$\int u \diff v = u v - \int v \diff u$

\begin{enumerate}
    \item \textbf{многочлен} * \textbf{тригонометрическую или показательную функцию}, то \\
    за $u$ выбирают многочлен, $\diff v$ — все, что осталось \\
    \textbf{Пример} $\int (3 x + 1) \cos 5 x \diff x = \frac{(3x + 1)}{5} \sin 5 x - \frac{3}{5} \int \sin 5 x \diff x = \frac{(3 x + 1)}{5} \sin 5 x + \frac{3}{25} \cos 5 x + C$ \\
    $du = 3 d x$, $v = \int \cos 5 x \diff x = \frac{1}{5} \sin 5x$ \\
    \textbf{Другой пример} $\int (3 x^2 + 1) \cos 5 x \diff x = \frac{(3 x^2 + 1)}{5} \sin 5 x + \frac{6}{5} \int x \sin 5 x \diff x$, дальше следует применить метод интегрирования по частям заново
    \item \textbf{многочлен} * \textbf{логарифмическую или обратную тригонометрическую функцию}, то \\
    за $u$ выбирают функцию, а $\diff v$ — все остальное \\
    \textbf{Пример} $\int (3x^2 + 5) \ln x \diff x = (\frac{x^3}{3} + 5 x) \ln x - \int (\frac{x^2}{3} + 5 x) \frac{\diff x}{1} = (\frac{x^3}{3} + 5x) \ln x - \frac{x^3}{9} - 5 x + C$ \\
    $ln x = u \Longrightarrow \frac{\diff x}{x} = \diff u$, $\diff v = (x^2 + 5) \diff x \Longrightarrow v = \int (x^2 + 5) \diff x = \frac{x^3}{3} + 5 x$
    \item \textbf{тригонометрическая функция} * \textbf{показательную функцию}, то \\
    не имеет значения, что выбрать за $u$, а что за $\diff v$, но формулу интегрирования по частям в этом случае \textbf{придется применить два раза подряд} единообразно \\
    \textbf{Пример} $\int \sin 5x e^{x} \diff x = \sin 5 x * e^{x} - 5 \int \cos 5 x * e^{x} \diff x = \dots$ \\
    Пусть $u = \sin 5x \Longrightarrow \diff u = 5 \cos 5 x \diff x$, $\diff v = e^{x} \diff x \Longrightarrow v = e^{x}$ \\
    \textbf{Применим метод интегрирования по частям во второй раз}, теперь $u = \cos 5 x \Longrightarrow \diff u = - 5 \sin 5 x \diff x$, $v = e^{x} \diff x \Longrightarrow v = e^{x}$ \\
    $\dots = \sin 5 x * e^{x} - 5 (\cos 5 x e^{x} + 5 \int \sin 5 x e^{x} \diff x)$ \\
    $y = (\sin 5 x - 5 \cos 5 x) e^{x} - 25 y \Longleftrightarrow 26 y = (\dots) e^{x} \Longleftrightarrow y = \frac{(\sin 5x - 5 \cos 5 x) e^{x}}{26}$, где $y = \int \sin 5 x e^{x} \diff x$
\end{enumerate}

\paragraph{Применения метода интегрирования по частям к произвольным интегралам}

$\int \sqrt{1 - x^2} \diff x = \dots$, пусть $u = \sqrt{1 - x^2} \Longrightarrow \diff u = \frac{1(-2x) \diff x}{2\sqrt{1 - x^2}}$, а $v = \diff x \Longrightarrow v = x$

$\dots = x \sqrt{1 - x^2} - \int \frac{1 - x^2 - 1}{\sqrt{1 - x^2}} \diff x = x \sqrt{1 - x^2} - \int \sqrt{1 - x^2} \diff x + \int \frac{\diff x}{\sqrt{1 - x^2}}$, пусть $y = \int \sqrt{1 - x^2} \diff x$, тогда

$y = x \sqrt{1 - x^2} - y + \arcsin x \Longleftrightarrow y = \frac{x \sqrt{1 - x^2} + \arcsin x}{2}$

\subsection{Рекуррентные формулы}

\subsubsection{Рекуррентная формула №1}

$y_{n} = \int \frac{\diff x}{(x^2 + a^2)^{n}}$

\hfill

$\int \frac{\diff x}{(x^2 + a^2)^{n}} = \dots$, пусть $u = \frac{1}{(x^2 + a^2)^{n}} \Longrightarrow \diff u = - 2 n x (x^2 + a^2)^{-n - 1} \diff x$, а $\diff v = \diff x \Longrightarrow x = v$

$\dots = \frac{x}{(x^2 + a^2)^{n}} + 2 n \int \frac{(x^2 + a^{2}) - a^{2}}{(x^2 + a^2)^{n + 1}} \diff x \Longrightarrow \frac{x}{(x^2 + a^2)^{n}} + 2 n \int \frac{\diff x}{(x^2 + a^2)^{n}} - 2 n a^2 \int \frac{\diff x}{(x^2 + a^2)^{n + 1}}$

\hfill 

$y_n = \int \frac{\diff x}{(x^2 + a^2)^{n}}$, $y_{n + 1} = \int \frac{\diff x}{(x^2 + a^2)^{n + 1}}$, $y_n = \frac{x}{(x^2 + a^2)^{n}} + 2 n y_n - 2 n a ^2 y_{n + 1} \Longleftrightarrow 2 n a^2 y_{n + 1} = \frac{x}{(x^2 + a^2)^{n}} + y_n (2 n - 1) \Longleftrightarrow y_{n + 1} = \frac{1}{2 n a^2} \frac{x}{(x^2 + a^2)^{n}} + \frac{2 n - 1}{2 n a^2} y_n$ — \textbf{рекуррентная формула}

\hfill

Например, $\int \frac{\diff x}{(x^2 + a^2)^{2}} = \frac{1}{2 a^2} \frac{x}{(x^2 + a^2)} + \frac{1}{2a^2} * \frac{1}{a} \arctg \frac{x}{a}$

\subsubsection{Рекуррентная формула №2}

$y_{n, - m} = \int \frac{\sin^{n} x}{\cos^{m} x} \diff x$

$y_{n, - m} = \frac{\sin^{n - 1} x}{(m - 1) \cos^{m - 1} x} - \frac{n - 1}{m - 1} y_{n - 2, 2 - m}$

\subsubsection{Рекуррентная формула №3}

$y_{n} = \int (a^2 - x^2)^{n} \diff x$

$y_{n} = \frac{x (a^2 - x^2)^{n}}{2 n + 1} + \frac{2 n a^2}{2 n + 1} y_{n - 1}$

\subsection{Интегрирование функций, содержащих квадратный трехчлен}

$\int \frac{\diff x}{a x^2 + b x + c} = \frac{1}{a} \int \frac{\diff x}{x^2 + 2 \frac{b}{2 a} x + (\frac{b}{2 a})^2 - (\frac{b}{2 a})^2 + C} = \frac{1}{a} \int \frac{\diff x}{(x + \frac{b}{2 a}) + (C - (\frac{b}{2 a})^2)}$

\subsubsection{Пример №1}

$\int \frac{\diff x}{x^2 + 2x + 2} = \int \frac{\diff x}{(x^2 + 2 x + 1) - 1 + 2} = \int \frac{\diff (x + 1)}{(x + 1)^2 + 1} = \arctg (x + 1) + C$

\subsubsection{Пример №2}

$\int \frac{(2 x + 3) \diff x}{x^2 + 3x + 5} = \ln |x^2 + 3x + 5|$, так как $(2 x + 3) \diff x = \diff (x^2 + 3x + 5) + C$

\subsubsection{Пример №3}

$\int \frac{(2 x + 4) \diff x}{x^2 + 3x + 5} = \int \frac{(2 x + 3) \diff x}{x^2 + 3x + 5} + \int \frac{\diff x}{x^2 + 2 \frac{3 x}{2} + \frac{9}{4} - \frac{9}{4} + 5} = \int \frac{\diff x + \frac{3}{2}}{(x + \frac{3}{2})^2 + (\frac{\sqrt{11}}{2})^{2}} = \frac{2}{\sqrt{11}} \arctg \frac{x + \frac{3}{2}}{\frac{\sqrt{11}}{2}} + C $

\subsection{Интегрирование рациональных дробей}

Комплексные корни многочлена с действительными коэффициентами \textbf{являются попарно-сопряженными}: $a + i b$, $a - i b$

$(x - (a + i b))(x - (a - i b)) = x^2 - x (a + i b) - x (a - i b) + (a^2 - (i b)^2) = x^2 - 2 a x + (a^2 + b^2) = (x^2 + p x + q)$

\hfill

Интегрирование рациональных дробей будет сводиться к интегрированию элементарных дробей: каждую рациональную дробь мы можем свести к линейной комбинации из элементарных дробей

\textbf{Виды элементарных дробей}:

\begin{multienumerate}
    \mitemxx{$\frac{1}{x - a}$}{$\frac{1}{(x - a)^{n}}$}
    \mitemxx{$\frac{1}{x^2 + p x + 1}$}{$\frac{i}{(x^2 + px + q)^{m}}$}
\end{multienumerate}

\textbf{Правила сведения рациональной дроби к линейной комбинации из элементарных дробей}:

\begin{enumerate}
    \item Если знаменатель имеет только действительные различные корни \\
    То, например, $\frac{2 x - 3}{(x - 4)(x + 5)} = \frac{A}{x - 4} + \frac{B}{x + 5} = \frac{A(x + 5) + B(x - 4)}{(x - 4)(x + 5)} = \frac{A x + 5 A + B x - 4 B}{(x - 4)(x + 5)}$ (\textbf{метод неопределенных коэффициентов}) \\
    \textbf{Собираем коэффициенты} при $x$: A + B = 2, при свободных членах: $5 A - 4 B = -3$, \textbf{решаем данную систему любым угодным нам способом}, получается $A = \frac{5}{9}$, а $B = 2 - \frac{5}{9} = \frac{13}{9}$ \\
    $\int \frac{2 x - 3}{(x - 4)(x + 5)} \diff x = \frac{5}{9} \int \frac{\diff x}{x - 4} + \frac{13}{9} \int \frac{\diff x}{x + 5}= \frac{5}{9} \ln |x - 4| + \frac{13}{9} \ln |x + 5|$
    \item Если знаменатель имеет действительные кратные корни \\
    То, например, $\frac{7 x - 8}{(x - 4)^{2} (x+5)^{3}} = \frac{A}{x - 4} + \frac{B}{(x - 4)^2} + \frac{C}{x + 5} + \frac{D}{(x + 5)^2} + \frac{E}{(x + 5)^3}$
    \item Если знаменатель имеет комплексные корни (различные) \\
    То, например, $\frac{2 x - 3}{(x^2 + 1) (x^2 + 2 x + 10)} = \frac{A x + B}{x^2 + 1} + \frac{C x + D}{x^2 + 2x + 10}$ \\
    $\int \frac{A x + B}{x^2 + 1} \diff x = A \int \frac{x \diff x}{x^2 + 1} + B \int \frac{\diff x}{x^2 + 1} = \dots$
    \item Если знаменатель имеет комплексные кратные корни \\
    То. например, $\frac{2 x - 3}{(x^2 + 1)^2 (x^2 + 2x + 10)^2} = \frac{A x + B}{x^2 + 1} + \frac{C x + D}{(x^2 + 1)^2} + \frac{M x + N}{x^2 + 2x + 10} + \frac{P x + Q}{(x^2 + 2x + 10)^2}$
\end{enumerate}

\pagebreak
\section{Высшая математика - 13.02.2023}

\subsection{Интегрирование по частям}

$\int u \diff v = u v - \int v \diff u$

Допустим, имеем $\int x e^{-3 x} \diff x$, представим за $u$ ту функцию, от которой проще взять производную; $u = x$, $\diff v = e^{- 3 x} \diff x$

$\begin{vmatrix}
    u = x & dv = e^{- 3 x} \diff x \\
    1 * \diff u = 1 * \diff x & v = - \frac{e^{- 3 x}}{3}
\end{vmatrix} = - \frac{x e^{- 3 x}}{3} = -\frac{1}{3} \int e^{- 3 x} \diff x = - \frac{x e^{-3 x}}{3} - \frac{1}{3} * (-\frac{e^{- 3 x}}{3}) + C = -\frac{x e^{- 3 x}}{3} + \frac{e^{- 3 x}}{9} + C$

\subsection{Интегрирование многочленов}

$
\int \frac{5 x + 7}{4x^2 - 6x + 10} \diff x = \int \frac{\frac{5}{8} (8x - 6) + \frac{43}{4}}{4x^2 - 6x + 10} \diff x = \frac{5}{8} \int \frac{8x - 6}{4x^2 - 6x + 10} \diff x + \frac{43}{4} \int \frac{\diff x}{4x^2 - 6x + 10}
$
($4x^2 - 6x + 10 = 4 (x^2 - \frac{3}{2} x + \frac{5}{2}) = 4((x - \frac{3}{4})^2 - \frac{9}{16} + \frac{5}{2}) = 4((x - \frac{3}{4})^2 + \frac{31}{16})$)
$
= \frac{43}{16} \int \frac{\diff x}{(x - \frac{x}{4})^2 + \frac{31}{16}} = \dots
$ (табличное значение, тривиально посчитать дальше)

\subsection{Задания}

\subsubsection{Задание №1}

$\int \frac{\sin^3 x}{\cos^2 x} \diff x = \int \frac{\sin x * \sin^2 x}{\cos^2 x}$

\subsubsection{Задание №2}

$\int \frac{e^{2 x} \diff x}{2 + e^{2 x}}$

\subsubsection{Задание №3}

$
\int \frac{\diff x}{x^2 - 6x + 5} = \int \frac{\diff x}{(x - 3)^2 - 4}
= \frac{1}{4} \ln | \frac{(x - 3) - 2}{(x - 3) + 2} | + C = \frac{1}{4} \ln | \frac{x - 5}{x - 1} | + C
$


\subsubsection{Задание №4}

\paragraph{Один вариант}

$
\int \frac{3x - 1}{\sqrt{x^2 - x + 1}} \diff x
= \int \frac{\frac{3}{2} (2 x - 1) + \frac{1}{2}}{\sqrt{x^2 - x + 1}} \diff x
= \frac{3}{2} \int \frac{2x - 1}{\sqrt{x^2 - x + 1}} + \frac{1}{2} \int \frac{\diff x}{\sqrt{x^2 - x + 1}}
= \frac{3}{2} \int \frac{\diff t}{\sqrt{t}}
= \frac{3}{2} \ln |\sqrt{t}| + C_1  + \frac{2}{3} \arcctg \frac{4(x - \frac{1}{2})}{3} + C_2
= \frac{3}{2} \ln |\sqrt{x^2 + x + 1}|  + \frac{2}{3} \arcctg \frac{4(x - \frac{1}{2})}{3} + C
$

\paragraph{Другой вариант}

$\int \frac{4 x + 3}{\sqrt{1 - x - 3x^2}} \diff x = $

\paragraph{Еще другой вариант}

$\int \frac{2 x + 3}{\sqrt{7 - 6x - x^2}} \diff x = \int \frac{2 x + 6}{\sqrt{-(x^2 + 6x - 7)}} \diff x = $

\subsubsection{Задание №5}

$\int x^2 e^{\frac{x}{2}} \diff x = \begin{vmatrix}
    x^2 = u \\
    e^{\frac{x}{2}} \diff x = \diff v
\end{vmatrix} = \begin{vmatrix}
    \diff u = 2 x \diff x \\
    v = 2 e^{\frac{x}{2}}
\end{vmatrix} = x^2 * 2 e^{\frac{1}{2}} - \int 2 e^{\frac{x}{2}} * 2 x \diff x = x^2 * 2 e^{\frac{x}{2}} - 4 \int e^{\frac{x}{2}} * x \diff x = \begin{vmatrix}
    x = u \\
    e^{\frac{1}{2}} \diff x = \diff v
\end{vmatrix} = \begin{vmatrix}
    \diff u = \diff x \\
    v = 2 e^{\frac{1}{2}}
\end{vmatrix} = x^{2} * 2 e^{\frac{1}{2}} - 4 (2x e^{\frac{x}{2}} - \int 2 e^{\frac{x}{2}} \diff x) = x^2 2 e^{\frac{x}{2}} - 8 x e^{\frac{x}{2}} + 15 e^{\frac{x}{2}} + C$

\subsubsection{Задание №6}

\paragraph{Один вариант}

$
\int e^{-x} \sin 2 x \diff x
$

\paragraph{Другой вариант}

$
\int e^{- 2 x} \sin x \diff x 
$

\subsubsection{Задание №7}

$
\int \ln (1 - x) \diff x = \begin{vmatrix}
    u = \ln (1 - x) & \diff u = - \frac{1}{1 - x} \diff x \\
    \diff v = \diff x & v = x
\end{vmatrix} = \ln (1 - x) * x + \int x * (\frac{1}{1 - x}) \diff x = x \ln (1 - x) - \int \frac{-x}{1 - x} \diff x = x \ln (1 - x) - \int \frac{1 - x + 1}{1 - x} \diff x = x \ln (1 - x) - \int \frac{1 - x}{1 - x} * x + \int \frac{\diff x}{-1 + x} = x \ln (1 - x) - x + \ln |x - 1| + C
$

\subsubsection{Задание №8}

$\int x \sin^2 4x \diff x$

\pagebreak
\section{Высшая математика - 14.02.2023}

\subsection{Задание №6}

$\int e^{\frac{x}{2}} \cos 5 x \diff x = \begin{vmatrix}
    u = e^{\frac{x}{2}} & dv = \cos 5 x \diff x \\
    \diff u = \frac{e^{\frac{x}{2}}}{2} & v = \frac{\cos 5 x}{5}
\end{vmatrix} = e^{\frac{x}{2}} \frac{\cos 5 x}{5} - \frac{1}{10} \int \cos 5 x e^{\frac{x}{2}} \diff x = \begin{vmatrix}
    u = e^{\frac{x}{2}} & dv = \cos 5 x \diff x \\
    \diff u = \frac{e^{\frac{x}{2}}}{2} & v = \frac{\cos 5 x}{5}
\end{vmatrix} = \frac{\cos 5 x}{5} - \frac{1}{10} (\frac{e^{\frac{x}{2}} \cos 5 x}{5} - \frac{1}{10} \int \cos 5 x e^{\frac{x}{2}} \diff x) = \frac{\cos 5 x}{5} - \frac{e^{\frac{x}{2}} \cos 5 x}{50} + \frac{1}{100} \int \cos 5 x e^{\frac{x}{2}} \diff x$

\hfill

Дальше все решается тривиально: переносим $\frac{1}{100} \int \cos 5 x e^{\frac{x}{2}} \diff x$ в левую сторону, что-то вроде того

\subsection{Задание №9 — задание №11}

$\int \frac{(x^2 + 1)^3}{x(x - 1)^2} \diff x = \int \frac{x^6 + 3x^4 + 3x^2 + 1}{x^3 - 2 x^2 + x} \diff x$ = (*)

\hfill

Сделаем из данной дроби правильную дробь, выполнив деление числителя на знаменатель:

(*) = $\int (x^3 + 2x^2 + 6x + 10) \diff x + \int \frac{17x^2 - 10 x + 1}{x^3 - 2x^2 + x} \diff x = (*)$

\hfill

$\int \frac{17x^2 - 10 x + 1}{x^3 - 2x^2 + x} \diff x = \int \frac{17x^2 - 10 x + 1}{x(x - 1)^2} \diff x = \frac{A}{x} + \frac{B}{x - 1} + \frac{C}{(x - 1)^2} \Longleftrightarrow \int \frac{17x^2 - 10 x + 1}{x(x - 1)^2} \diff x = \frac{A(x-1)^2 + Bx(x-1) + Cx}{x(x-1)^2} \Longleftrightarrow 17x^2 - 10x + 1 = Ax^2 - 2A x + A + B x^2 - B x + C x$

$\begin{cases}
    17 = A + B \\
    -10 = -2A - B + C \\
    A = 1 \ B = 16 \ C = 8
\end{cases}$

\hfill

\hfill

(*) = $\int x^3 \diff x + 2 \int x^2 \diff x + 6 \int x \diff x + 10 \int \diff x + \int \frac{\diff x}{x} + 16 \int \frac{\diff x}{x - 1} + 8 \int \frac{\diff x}{(x - 1)^2} = \dots$ (все интегралы табличные, дальнейшее решение тривиально и предоставляется читателю)

\pagebreak
\section{Высшая математика - 14.02.2023}

\subsection{Интегрирование рациональных дробей}

$
\frac{2x - 3}{(x+1)^2 (x+2)^3} = \frac{A}{x + 1} + \frac{B}{(x + 1)^2} + \frac{C}{x + 2} + \frac{D}{(x + 2)^2} + \frac{F}{(x + 2)^3}
$

\hfill

$
\frac{2x - 3}{x (x + 1)^2 (x^2 + 4)} = \frac{A}{x} + \frac{B}{x + 1} + \frac{C}{(x + 1)^2} + \frac{D x + F}{x^2 + 4} = \frac{A x^4 + 4 A x^2 + 2 A x^3 + 8 A x + A x^2 + 4A + B x^4 + B x^3 + 4 B x^2 + 4 B x + C x^3 + 4 C x D x^4 + 2 D x^3 + D x^2 + F x^3 + 2 F x^2 + F x}{x (x+1)^2 (x^2 + 4)}
$ = (*)

При $x^4$: $A + B + D = 0$

$x^3$: $2A + B + C + 2D + F = 0$

$x^2$: $5A + 4B + D + 2F = 0$

$x$: $8 A + 4 B + 4C + F = 2$

Свободные члены: $4A = -3$

\hfill

Решение данной системы уравнений оставляется в качестве упражнения читателю, мы же его опустим. Лишь заметим, что ее можно решать как угодно

$A = -\frac{3}{4}$, $B = 1$, $C = 1$, $D = - \frac{1}{4}$

\hfill

(*) = $
\int \frac{2x - 3}{x (x + 1)^2 (x^2 + 4)} \diff x = - \frac{3}{4} \int \frac{\diff x}{x} + \frac{\diff (x + 1)}{x + 1} + \int \frac{\diff x}{(x + 1)^2} - \frac{1}{4} \int \frac{2 x \diff x}{x^2 + 4} = -\frac{3}{4} \ln |x| + \ln |x + 1| - \frac{1}{x + 1} - \frac{1}{8} \ln |x^2 + 4| + C
$

\textbf{Данные преобразования могут выполняться лишь с правильными дробями}

\subsection{Интегрирование дробно-степенных функций}

Если стоящая под знаком интеграла функция зависит от $x$ в дробных степенях, то мы находим общий знаменатель этих степеней и $x$ в соответствующей степени обозначаем за $t$

\begin{enumerate}
    \item $\frac{x^{\frac{1}{2}} \diff x}{x^{\frac{1}{3}} + 1} = \begin{vmatrix}
        x^{\frac{1}{6}} = t & x = t^6 & \diff x = 6 t^5 \diff t \\
        x^{\frac{1}{2}} = \sqrt{3} = t^3 & x^{\frac{1}{3}} = t^2
    \end{vmatrix} = \int \frac{t^3 6 t^5 \diff t}{t^2 + 1} = 6 \int \frac{t^8}{t^2 + t} \diff t = 6 \int \frac{(t^8 - 1) + 1}{t^2 + t} \diff t = 6 (\int (\frac{(t^4 + 1) (t^2 - 1) (t^2 + 1)}{t^2 + 1} + \frac{1}{t^2 + 1}) \diff t = 6 [ \int (t^6 - t^4 + t^2 - 1 + \frac{1}{t^2 + 1}) \diff t ] = 6 (\frac{t^7}{7} - \frac{t^5}{5} + \frac{t^3}{3} - t + \arctg t) + C \Longrightarrow 6 (\frac{x^{\frac{7}{6}}}{7} - \frac{x^{\frac{5}{6}}}{5} + \frac{x^{\frac{1}{2}}}{3} - x^{\frac{1}{6}} + \arctg (x^\frac{1}{6})) + C$
\end{enumerate}

\hfill

Если стоящая под знаком интеграла функция зависит от $x$ и дробно-линейной функции в какой-то дробной степени, то $(\frac{a x + b}{c x + d})^{\frac{1}{v}}$, где $v$ — общий знаменатель этих степеней, мы обозначим за $t$

$\int R (x, \frac{a x + b}{c x + d}^{\frac{m}{n}}, \dots (\frac{a x + b}{c x + d})^{\frac{p}{q}}))$, $t = (\frac{a x + b}{c x + d})^{\frac{1}{v}}$

\begin{enumerate}
    \item $\int \frac{2}{(2 - x)^2} \sqrt[3]{\frac{2 - x}{2 + x}} \diff x = \begin{vmatrix}
        t = \sqrt[3]{\frac{2 - x}{2 + x}} & t^3 = \frac{2 - x}{2 + x} \\
        x = \frac{2 - 2t^3}{t^3 + 1} & \diff x = \frac{-6t^2 (t^3 + 1) - 3t^2 (2 - 2t^3)}{(t^3 + 1)^2} \diff t = \frac{-12 t^2 \diff t}{(t^3 + 1)^2}
    \end{vmatrix} = \int \frac{2t (\frac{-12 t^2}{(t^3 + 1)^2}) \diff t}{(2 - \frac{2 - 2t^3}{t^3 + 1})^2} = -\frac{24}{16} \int \frac{\frac{t^3 \diff t}{(t^3 + 1)^2}}{\frac{t^5}{(t^3 + 1)^2}} = -\frac{3}{2} \int \frac{\diff t}{t^3} = \frac{3}{4} \sqrt[3]{(\frac{2 - x}{2 + x})^2} + C$
\end{enumerate}

\subsection{Применение тригонометрических подстановок к интегрированию иррациональных функций}

\begin{enumerate}
    \item Если имеем $\int R (x, \sqrt{m^2 x^2 + n^2}) \diff x$, то выполняем замену $x = \frac{n}{m} \tg t$
    \item Если имеем $\int R(x, \sqrt{m^2 x^2 - n^2}) \diff x$, то выполняем замену $x = \frac{n}{m} \frac{1}{\cos t}$
    \item Если имеем $\int R(x, \sqrt{n^2 - m^2 x^2}) \diff x$, то выполняем замену $x = \frac{n}{m} \sin t$
\end{enumerate}

Например, $\int \frac{\diff x}{\sqrt{(4 - x^2)^3}} = \begin{vmatrix}
    x = 2 \sin t & \diff x = 2 \cos t \diff t
\end{vmatrix} = \int \frac{2 \cos t \diff t}{\sqrt{(4 - 4 \sin^2 t)^3}} = \int \frac{2 \cos t \diff t}{8 \cos^3 t \diff t} = \frac{1}{4} \int \frac{\diff t}{\cos^2 t} = \frac{1}{4} \tg t = \frac{1}{4} \frac{\sin t}{\cos t} = \frac{1}{4} \frac{\sin t}{\sqrt{1 - \sin^2 t}} = \frac{1}{4} \frac{\frac{x}{2}}{\sqrt{1 - \frac{x^2}{4}}} = \frac{x}{4\sqrt{4 - x^2}} + C$

\subsection{Интегрирование тригонометрических функций}

Если под знаком интеграла стоит произведение тригонометрических функций, то его желательно преобразовать в сумму или разность. Помним из школьного курса тригонометрии:

\begin{enumerate}
    \item $\sin \alpha * \sin \beta = \frac{1}{2} (\cos (\alpha - \beta) - \cos (\alpha + \beta))$
    \item $\sin \alpha * \cos \beta = \frac{1}{2} (\sin (\alpha - \beta) + \sin (\alpha + \beta))$
    \item $\cos \alpha * \cos \beta = \frac{1}{2} (\cos (\alpha - \beta) + \cos (\alpha + \beta))$
\end{enumerate}

Также бывают полезны формулы понижения степени:

\begin{enumerate}
    \item $\sin^2 \alpha = \frac{1 - \cos 2 \alpha}{2}$
    \item $\cos^2 \alpha = \frac{1 + \cos 2 a}{2}$
\end{enumerate}

Если имеем $\int \sin^{n} x \cos^{m} x \diff x$, где $n$, $m$ — четные степени, то мы понижаем степени до того, как не сможем воспользоваться табличными интегралами

\hfill

Если $m$ и (или) $n$ нечетно, то мы «откусываем» от нечетной степени и убираем под знак дифференциала

Например, $\int \sin^4 x \cos^3 x \diff x = \int sin^4 x \cos^2 x * \cos x \diff x = \int sin^4 x (1 - \sin^2 x) \diff (\sin x) = \int (t^4 - t^6) \diff t = \frac{t^5}{5} - \frac{t^7}{7} + C = \frac{sin^5 x}{5} - \frac{\sin^7 x}{7} + C$

\subsubsection{Универсальная тригонометрическая подстановка}

$\tg \frac{x}{2} = t$, $x = 2 \arctg t$, $\diff x = \frac{2 \diff t}{1 + t^2}$

$\sin x = \frac{2 \tg \frac{x}{2}}{1 + \tg^2 \frac{x}{2}} = \frac{2 t}{1 + t^2}$, $\cos x = \frac{1 - t^2}{1 + t^2}$, $\tg x = \frac{2 t}{1 - t^2}$

\hfill

Например, $\int \frac{\diff x}{\sin x (2 + \cos x - 2 \sin x)} = \int \frac{\frac{2 \diff t}{1 + t^2}}{\frac{2t}{1 + t^2} (2 + \frac{1 - t^2}{1 + t^2} - \frac{2 * 2 t}{1 + t^2})} = \int \frac{(t^2 + 1) \diff t}{t (t^2 - 4t + 3)}$ = (*)

$\frac{t^2 + 1}{t(t - 1)(t - 3)} = \frac{A}{t} + \frac{B}{t - 1} + \frac{C}{t - 3}$

Посчитаем так, как считали. Найдем что $A = \frac{1}{3}$, $B = \frac{5}{3}$, $C = -1$

(*) = $\frac{1}{3} \int \frac{\diff t}{t} + \frac{5}{3} \int \frac{\diff t}{t - 1} - \int \frac{\diff t}{t - 3} = \frac{1}{3} \ln |\tg \frac{x}{2}| + \frac{5}{3} \ln |\tg \frac{x}{2} - 1| - \ln |\tg \frac{x}{2} - 3| + C$

\hfill

Если косинус и синус в дроби входят в виде $\cos^2 x$ и $\sin^2 x$, то мы можем делать замену не универсальную, а \textbf{обозначать} $\tg x = z$, $x = \arctg z$, $\diff x = \frac{\diff z}{1 + z^2}$

$\cos^2 x = \frac{1}{1 + t^2}$, $\sin^2 x = \frac{t^2}{1 + t^2}$

\pagebreak
\section{Высшая математика - 17.02.2023}

\subsection{Подстановки Эйлера}

$\int R (x, \sqrt{ax^2 + bx + c} \diff x$

\begin{enumerate}
    \item Если $a > 0$, то $\sqrt{a x^2 + bx + c} = t \pm x \sqrt{a}$
    \item Если $c > 0$, то $\sqrt{a x^2 + bx + c} = t x \pm \sqrt{c}$
    \item $a x^2 + b x + c = a (x - \alpha) (x - \beta)$, $a$ — действительный корень, то можно выполнить замену $\sqrt{a x^2 + b x + c} = (x - \alpha) t$, аналогичную замену можно провести, если $\beta$ — действительный корень
\end{enumerate}

\paragraph{Пример. } $\int \frac{x \diff x}{(\sqrt{7x - 10 - x^2})^3}$ = (*)

\hfill

Решим квадратное уравнение: $x^2 - 7x - 10 = 0 \Longleftrightarrow (x - 2)(x - 5) = 0$

$\sqrt{-(x - 2)(x - 5)} = (x - 2) t \Longleftrightarrow (x - 2) (5 - x) = (x - 2)^2 t^2 \Longleftrightarrow (5 - x) = x t^2 - 2t^2$

Отсюда выразим $x = \frac{2t^2 + 5}{t^2 + 1}$, $\diff x = \frac{4t (t^2 + 1) - (2t^2 + 5) 2 t}{(t^2 + 1)} \diff t = \frac{-6 t \diff t}{(t^2 + 1)^2} = (\frac{2t^2 + t}{t^2 + 1} - 2) t = \frac{2t^2 + 5 - 2t^2 - 2}{t^2 + 1} = \frac{3t}{t^2 + 1}$

\hfill

(*) = $\int \frac{\frac{2t^2 + 5}{t^2 + 1} \frac{(-6t) \diff t}{(t^2 + 1)^2}}{(\frac{3t}{t^2 + 1})^3} = \int \frac{(2t^2 + 5) \diff t}{t^2} = -\frac{2}{9} \int (2 + \frac{5}{t^2}) \diff t = - \frac{2}{9} (2t - \frac{5}{t}) + C$, \ $t = \frac{\sqrt{7x-10-x^2}}{x-2}$

\subsection{Определенный интеграл}

\subsubsection{Определение и геометрическое значение}

\begin{definition}
    Если при любых разбиениях отрезка $[a; b]$ таких, что наибольшее значение $\Delta X_{i} \to 0$ и любом выборе точек $\xi_{i}$ существует $\lim\limits_{max \ \Delta X_{i} \to 0} \sum\limits_{i = 1}^{n} f(\xi_{i}) \Delta x_{i}$, то он называется определенным интегралом $S = \int\limits_{a}^{b} f(x) \diff x$ ($a < b$)

    Если этот предел существует, то функция считается интегрируемой на отрезке $[a; b]$

    Если $b < a$, то $\int\limits^{b}_{a} f(x) \diff x = - \int\limits_b^{a} f(x) \diff x$
\end{definition}

Определённый интеграл от неотрицательной функции $\int\limits_{a}^{b} f(x) \diff x$ \textbf{численно равен площади фигуры}, ограниченной осью абсцисс, прямыми $x = a$ и $x = b$ и графиком функции $f(x)$

\subsubsection{Основные свойства определенного интеграла}

\begin{multienumerate}
    \mitemxx{$\int\limits_{a}^b A f(x) \diff x = A \int\limits_a^b f(x) \diff x$}{$\int\limits_a^b (f_1 \pm f_2) \diff x = \int f_1 \diff x \pm \int f_2 \diff x$}
    \mitemx{Если $f(x) \le g(x)$, то $\int\limits_a^b f(x) \diff x \le \int\limits_a^b g(x) \diff x$}
    \mitemx{Если $m$ и $M$ — наименьшее и наибольшее $f(x)$ на $[a; b]$ ($a \le b$), то $m (b - a) \le \int\limits_a^b f(x) \diff x \le M (b - a)$}
    \mitemx{\textbf{Теорема о среднем}. Если $f(x)$ непрерывна на $[a; b] \ \exists \ C \in [a; b]$, то $\int\limits_a^b f(x) \diff x = (b - a) f(c)$ }
    \mitemx{При любом расположении $a$, $b$ и $c$ справедливо $\int\limits_a^b f(x) \diff x = \int\limits_a^c f(x) \diff x + \int\limits_c^b f(x) \diff x$}
\end{multienumerate}

\subsubsection{Вычисление определенного интеграла. Формула Ньютона-Лейбница}

\begin{theorem}
Если $f(x)$ — непрерывная функция, $\Phi(x) = \int\limits_a^x f(t) \diff t$, то имеет место $\Phi'(x) = f(x)$
\end{theorem}

\begin{theorem}[Формула Ньютона-Лейбница]
Если $F(x)$ — какая-либо первообразная для $f(x)$, то справедливо $\int\limits_{a}^{b} f(x) \diff x = F(b) - F(a)b$
\end{theorem}

\paragraph{Пример 1. } $\int\limits_1^{\sqrt{3}} \frac{\diff x}{1 + x^2} = (*)$, $\int \frac{\diff x}{1 + x^2} = \arctg x$

(*) = $\arctg x \bigg|_1^{\sqrt{3}} = \arctg \sqrt{3} - \arctg 1 = \frac{\pi}{3} - \frac{\pi}{4} = \frac{\pi}{12}$

\paragraph{Пример 2. } $\int\limits_0^2 |1 - x| \diff x = \int\limits_0^1 (1 - x) \diff x + \int\limits_1^2 (x - 1) \diff x = (x - \frac{x^2}{2}) \bigg|_{0}^{1} + (\frac{x^2}{2} - x) \bigg|_{1}^{2} = 1$

\subsubsection{Замена переменной в определенном интеграле}

\begin{theorem}
Пусть дан интеграл $\int\limits_{a}^{b} f(x) \diff x$, где $f(x)$ — непрерывная на $[a; b]$ функция

Вводим $t$, исходя из формулы $x = \phi(t)$. Если

\begin{multienumerate}
    \mitemxxx{$\phi(\alpha) = a$, $\phi(\beta) = b$}{$\phi$, $\phi$' непрерывна на $[a; b]$}{$f(\phi(t))$ определена и непрерывна на $[\alpha; \beta]$}
\end{multienumerate}

то $\int\limits_{a}^{b} f(x) \diff x = \int\limits_{\alpha}^{\beta} f(\phi(t)) \phi'(t) \diff t$
\end{theorem}

\paragraph{Пример 1.} $\int\limits_{2}^{4} \frac{x \diff x}{\sqrt{2 + 4x}} = (*)$, выполним замену $t = \sqrt{2 + 4x} \implies x = \frac{t^2 - 2}{4}$, $\diff x = \frac{t \diff t}{2}$

(*) = $\int\limits_{\sqrt{6}}^{\sqrt{12}} \frac{t^2 - 2}{4t} * \frac{1}{2} \diff t = \frac{1}{8} \int\limits_{\sqrt{6}}^{\sqrt{18}} (t^2 - 2) \diff t = \frac{1}{8} (\frac{t^3}{3} - 2t) \bigg|_{\sqrt{6}}^{\sqrt{18}} = \frac{1}{8} (\frac{18 \sqrt{18}}{3} - 2 \sqrt{18} - \frac{6\sqrt{6}}{3} + 2 \sqrt{6}) = \frac{3\sqrt{2}}{2}$

\paragraph{Пример 2. } $\int\limits_{3/4}^{4/3} \frac{\diff x}{x\sqrt{x^2 + 1}} = (*)$, выполним замену $t = \frac{1}{x} \implies x = \frac{1}{t}$, $\diff x = -\frac{1}{t^2} \diff t$

(*) = $
\int\limits_{4/3}^{3/4} \frac{-\frac{1}{t^2} \diff t}{\frac{1}{t} \frac{\sqrt{1 + t^2}}{t}} = \int\limits_{3/4}^{4/3} \frac{\frac{1}{t^2} \diff t}{\frac{1}{t} \frac{\sqrt{1 + t^2}}{t}} = \ln |t + \sqrt{1 + t^2}| \bigg|_{3/4}^{4/3} = \ln |\frac{4}{3} + \sqrt{1 + \frac{16}{9}} - \ln |\frac{3}{4} + \sqrt{1 + \frac{9}{16}}| = \int\limits_{\sqrt{6}}^{\sqrt{12}} \frac{t^2 - 2}{4t} * \frac{1}{2} \diff t = \frac{1}{8} \int\limits_{\sqrt{6}}^{\sqrt{18}} (t^2 - 2) \diff t = \frac{1}{8} (\frac{t^3}{3} - 2t) \bigg|_{\sqrt{6}}^{\sqrt{18}} = \ln \frac{3}{2}
$


\subsubsection{Интегрирование по частям в неопределенном интеграле}

$\int\limits_{a}^{b} (u v)' \diff x = \int\limits_{a}^{b} u' v \diff x + \int\limits_{a}^{b} u v' \diff x$

$\int\limits_{a}^{b} u \diff v = u v \bigg|_{a}^{b} - \int\limits_{a}^{b} v \diff u$

\paragraph{Пример 1. } $\int\limits_{0}^{\pi/4} x \cos 2x \diff x = (*)$, пусть $u = x$, $\diff v = \cos 2 x \diff x$, $\diff u = \diff x$, $v = \frac{1}{2} \sin 2x$

(*) = $[ x * \frac{1}{2} \sin 2 x - \frac{1}{2} \int \sin 2 x \diff x] \bigg|_{0}^{\pi/4} = [\frac{x}{2} \sin 2x + \frac{1}{4} \cos 2x] \bigg|_{0}^{\pi/4} = \frac{\pi}{8} * 1 - 0 + 0 - \frac{1}{4} * 1 = \frac{\pi}{8} - \frac{1}{4}$

\subsubsection{Упрощение интегралов, основанное на свойстве симметрии подынтегральных функций}

\begin{enumerate}
    \item Если функция $f(x)$ является четной на симметричном интервале $[-a; a]$, то $\int\limits_{-a}^{a} f(x) \diff x = 2 \int\limits_{0}^{a} f(x) \diff x$
    \item Если $f(x)$ является нечетной на $[-a; a]$, то $\int\limits_{-a}^{a} f(x) \diff x = 0$
    \item Если $f(x)$ является периодической функцией (то есть, $f(x) = f(x + T)$), то $\int\limits_{a}^{b} = \int\limits_{a + n T}^{b + n T} f(x) \diff x$
\end{enumerate}

\end{document}

