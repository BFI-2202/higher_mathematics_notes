\documentclass{article}
\usepackage[utf8]{inputenc}

\usepackage[T2A]{fontenc}
\usepackage[utf8]{inputenc}
\usepackage[russian]{babel}

\usepackage{amsmath}
\usepackage{pgfplots}
\usepackage{multienum}
\usepackage{geometry}
\geometry{
    left=1cm,right=1cm,top=2cm,bottom=2cm
}
\newcommand*\diff{\mathop{}\!\mathrm{d}}

\newtheorem{definition}{Определение}
\newtheorem{theorem}{Теорема}

\DeclareMathOperator{\sign}{sign}

\usepackage{hyperref}
\hypersetup{
    colorlinks, citecolor=black, filecolor=black, linkcolor=black, urlcolor=black
}

\title{Высшая математика}
\author{Лисид Лаконский}
\date{April 2023}

\begin{document}
\raggedright

\maketitle

\tableofcontents
\pagebreak

\section{Высшая математика - 12 апреля 2023 г.}

\subsection{Примеры решения знакоположительных числовых рядов}

Если $\lim\limits_{n \to \infty} \ne 0$, то ряд точно является расходящимся

Иначе проверяем достаточные признаки: 
\begin{multienumerate}
    \mitemxx{Первый признак сравнения}{Второй признак сравнения}
    \mitemxx{Признак Д'Аламбера}{Признак Коши (радикальный)}
    \mitemx{Признак Коши (интегральный)}
\end{multienumerate}

\paragraph{Пример №1}

$\sum \frac{2 n + 1}{n + 1}$ — не выполняется необходимый признак, следовательно, расходится

\paragraph{Пример №2}

$\sum \sin \frac{\pi}{2\sqrt{n}} \sim \sum \frac{\pi}{2\sqrt{n}}$

Проверим: $\lim \frac{\sin \frac{\pi}{2 \sqrt{n}}}{\frac{\pi}{2 \sqrt{n}}} = 1$ — следовательно, ряды ведут себя одинаково

$\sum \frac{\pi}{2\sqrt{n}} \sim \sum \frac{1}{\sqrt{n}}$ — расходится, так как $\frac{1}{2} < 1$

\paragraph{Пример №3}

$\sum \frac{n^3}{3^n}$

$\lim\limits_{n \to \infty} \frac{a_{n + 1}}{a_{n}} = \lim\limits_{n \to \infty} \frac{\frac{(n + 1)^3}{3^{n + 1}}}{\frac{n^3}{3^n}} = \lim\limits_{n \to \infty} \frac{(n + 1)^3}{3 * n^3} = \frac{1}{3} < 1$ — ряд сходящийся

\paragraph{Пример №4}

$\sum \frac{2^{n}}{n^2 + n} \sim \sum \frac{2^{n}}{n^2}$

$\lim\limits_{n \to \infty} \frac{a_{n + 1}}{a_{n}} = \lim\limits_{n \to \infty} \frac{\frac{2^{n + 1}}{(n + 1)^2}}{\frac{2^{n}}{n^2}} = \lim\limits_{n \to \infty} \frac{2^n *2}{(n + 1)^2} * \frac{n^2}{2^{n}} = 2 > 1$ — расходящийся

\paragraph{Пример №5}

$\sum 2^{- n} (\frac{n + 1}{n})^{n^2}$

$\lim\limits_{n \to \infty} \sqrt[n]{2^{- n} (\frac{n + 1}{n})^{n^2}} = \lim\limits_{n \to \infty} (2^{-1} * (\frac{n + 1}{n})^{n}) = \frac{1}{2} \lim (1 + \frac{1}{n})^{n} = \frac{e}{2} > 1$ — расходящийся

\paragraph{Пример №6}

$\sum (\frac{n}{n + 1})^{n^2}$

$\lim\limits_{n \to \infty} \sqrt[n]{(\frac{n}{n + 1})^{n^2}} = \lim\limits_{n \to \infty} (\frac{n}{n + 1})^{n} = \frac{1}{e} < 1$ — ряд сходится

\paragraph{Пример №7}

$\sum \frac{1}{n^2} \sin \frac{1}{n}$

$\int\limits_{1}^{\infty} \frac{1}{x^2} \sin \frac{1}{x} \diff x = \lim\limits_{B \to \infty} \int\limits_{1}^{B} \frac{1}{x^2} \sin \frac{1}{x} \diff x = - \lim\limits_{B \to \infty} \int\limits_{1}^{B} \sin \frac{1}{x} \diff (\frac{1}{x}) = \lim\limits_{B \to \infty} \cos \frac{1}{x} \bigg|_{1}^{B} = \lim\limits_{B \to \infty} (\cos \frac{1}{B} - \cos 1) = 1 - \cos 1$ — интеграл сходится, следовательно, ряд тоже сходится

\paragraph{Пример №8}

$\sum n e^{-\frac{n^2}{2}}$

$\int\limits_{1}^{\infty} x e^{-\frac{x^2}{2}} \diff x = \lim\limits_{B \to \infty} \int\limits_{1}^{B} x e^{-\frac{x^2}{2}} \diff x = \dots$

$d (e^{-\frac{x^2}{2}}) = e^{-\frac{x^2}{2}} * (-x)$

$\dots = - \lim\limits_{B \to \infty} \int\limits_{1}^{B} d(e^{-x^2/2}) = - \lim\limits_{B \to \infty} e^{-x^2/2} \bigg|_{1}^{B} = \frac{1}{\sqrt{e}}$ — сходящийся ряд

\pagebreak
\subsection{Примеры решения знакочередующихся (знакопеременных) числовых рядов}

Если сходится $\sum |a_{n}|$, то ряд из $\sum a_{n}$ \textbf{сходится абсолютно}

Признак Лейбница
\begin{enumerate}
    \item $\lim\limits_{n \to \infty} a_{n} = 0$
    \item $a_1 > a_2 > a_3 > \dots > a_{n} > a_{n + 1}$
\end{enumerate}

\paragraph{Пример №9}

$\sum (-1)^{n + 1} \frac{n}{2^{n}}$

$\sum \frac{n}{2^{n}}$, проверим $\lim\limits_{n \to \infty} \frac{\frac{n + 1}{2^{n + 1}}}{\frac{n}{2^{n}}} = \lim\limits_{n \to \infty} \frac{n + 1}{2 * n} = \frac{1}{2} < 1$ — ряд сходится абсолютно

\paragraph{Пример №10}

$\sum (-1)^{n + 1} \frac{3^{n}}{n^2}$

$\sum \frac{3^{n}}{n^2}$, проверим $\lim\limits_{n \to \infty} \frac{\frac{3^{n + 1}}{(n + 1)^2}}{\frac{3^{n}}{n^2}} = 3$ — абсолютной сходимости нет

Проверим сходимость по Лейбницу: $\lim\limits_{n \to \infty} \frac{3^{n}}{n^2} = \lim\limits_{n \to \infty} \frac{3^{n} \ln 3}{2 n} = \lim\limits_{n \to \infty} \frac{3^{n} (\ln 3)^2}{2} = \infty$ — не выполнен необходимый признак, никакой сходимости нет

\paragraph{Пример №11}

$\sum (-1)^{n + 1} \frac{n}{6 n - 5}$

$\lim\limits_{n \to \infty} \frac{n}{6 n - 5} = \frac{1}{6}$ — тоже расходящийся ряд по необходимому признаку

\paragraph{Пример №12}

$\sum \frac{1}{\sqrt{n}}$

Можем легко проверить, что абсолютной сходимости нет, но есть сходимость по Лейбницу: $\lim\limits_{n \to \infty} \frac{1}{\sqrt{n}} = 0$, $1 > \frac{1}{\sqrt{2}} > \frac{1}{\sqrt{3}} > \dots$

\pagebreak
\subsection{Примеры решения функциональных рядов}

\paragraph{Пример №1}

$\sum \frac{(-1)^{n} * n}{n^2 + 1} (x + 2)^{n}$

$\lim \limits_{n \to \infty} | \frac{a_{n + 1} (x)}{a_{n} (x)} | = \lim\limits_{n \to \infty} | \frac{\frac{(-1)^{n + 1} * (n + 1)}{(n + 1)^2 + 1} (x + 2)^{n + 1}}{\frac{(-1)^{n} * n}{n^2 + 1} (x + 2)^{n}} | = | x + 2 | \lim\limits_{n \to \infty} | \frac{n + 1}{(n + 1)^2 + 1} * \frac{n^2 + 1}{n} | = | x + 2 | < 1$

$-1 < x + 2 < 1 \Longleftrightarrow -3 < x < -1$ — \textbf{область сходимости} данного ряда

Проверим граничные значения:

Если $x = -1$, то $\sum \frac{(-1)^{n} n}{n^2 + 1}$ — условно сходится

Если $x = -3$, то $\sum \frac{(-1)^n n}{n^2 + 1} (-1)^{n} = \sum \frac{n}{n^2 + 1}$ — расходящийся

Обновим границы: $-3 < x \le -1$

\paragraph{Пример №2}

$\sum \frac{(-1)^{n} n!}{n^2} (x - 1)^{n}$

Ряд из модулей: $\sum \frac{n!}{n^2} (x - 1)^{n}$, $\lim\limits_{n \to \infty} \frac{a_{n + 1}}{a_{n}} = \lim\limits_{n \to \infty} | \frac{(n + 1)! (x - 1)^{n + 1}}{(n + 1)^2 n! (x - 1)^{n}} | = |x - 1| \lim\limits_{n \to \infty} (n + 1) < 1$ только при $x = 1$

Область сходимости: $x = 1$

\subsection{Примеры разложения в ряд}

\paragraph{Пример №1}

$\ln (3 x - 2) = \ln (3 (x - 2 + 2) - 2) = \ln (3 (x - 2) + 4) = \ln 4 (1 + \frac{3 ( x - 2)}{4}) = \ln 4 + \ln (1 + \frac{3}{4}(x - 2)) = \ln 4 + \frac{3}{4} (x - 2) - \frac{1}{2} (\frac{3}{4} (x - 2))^2 + \frac{1}{3} (\frac{3}{4} (x - 2))^3 - \dots$

\pagebreak
\subsection{Ряды Фурье}

$f(x) = \frac{a_0}{2} + \sum (a_{n} \cos n x + b_{n} \sin n x)$

$a_0 = \frac{1}{\pi} \int\limits_{-\pi}^{\pi} f(x) \diff x$, $a_{k} = \frac{1}{\pi} \int\limits_{-\pi}^{\pi} f(x) \cos k x \diff x$, $b_{k} = \frac{1}{\pi} \int\limits_{-\pi}^{\pi} f(x) \sin k x \diff x$

\subsubsection{Разложение в ряд Фурье четных и нечетных функций}

\begin{enumerate}
    \item $f(-x) = f(x)$ четная $[-\pi; \pi]$ \\
    $a_0 = \frac{2}{\pi} \int\limits_{0}^{\pi} f(x) \diff x$, $a_{n} = \frac{2}{\pi} \int\limits_{0}^{\pi} f(x) \cos n x \diff x$, $b_{n} = 0$
    \item $f(-x) = - f(x)$ нечетная $[-\pi; \pi]$ \\
    $a_{n} = 0$, $b_{n} = \frac{2}{\pi} \int\limits_{0}^{\pi} f(x) \sin n x \diff x$
\end{enumerate}

\paragraph{Пример №1}

$f(x) = x^2$, промежуток $[-\pi; \pi]$

$a_0 = \frac{2}{\pi} \int\limits_{0}^{\pi} x^2 \diff x = \frac{2}{\pi} * \frac{x^3}{3} \bigg|_{0}^{\pi} = \frac{2}{3} \pi^2$

$a_{n} = \frac{2}{\pi} \int\limits_{0}^{\pi} f(x) \cos n x \diff x = \frac{2}{\pi} \int\limits_{0}^{\pi} x^2 \cos n n x \diff x = \begin{vmatrix}
    u = x^2 & \diff v = \cos n x \diff x \\
    \diff u = 2x \diff x & v = \frac{1}{n} \sin n x
\end{vmatrix} = \frac{2}{\pi} (\frac{x^2}{n} \sin n x - \frac{2}{n} (-\frac{x}{n} \cos n x + \frac{1}{n^2} \sin n x)) \bigg|_{0}^{\pi} = \frac{4}{\pi n^2} (\pi \cos \pi n - 0) = \frac{4 \cos \pi n}{n^2} = \frac{4}{n^2} (-1)^{n}$

\hfill

$x^2 = \frac{\pi^2}{3} + 4 \sum \frac{(-1)^{n}}{n^2} \cos n x = \frac{\pi^2}{3} - 4 (\frac{\cos x}{1} - \frac{\cos 2 x}{2^2} + \frac{\cos 3 x}{3^2} - \dots)$

$0 = \frac{\pi^2}{3} - 4 (1 - \frac{1}{2^2} + \frac{1}{3^2} - \frac{1}{4^2} + \dots)$

\hfill

Ряды Фурье с \textbf{произвольным периодом} (например, на $[-l; l]$) записываются следующим образом:

$$f(x) = \frac{a_0}{2} + \sum (a_{n} \frac{\cos \pi n x}{x} + b_{n} \frac{\sin \pi n x}{l})$$

И тогда $a_{n} = \frac{1}{l} \int\limits_{- l}^{l} f(x) \cos \frac{\pi n x}{l} \diff x$, $b_{n} = \frac{1}{l} \int\limits_{- l}^{l} f(x) \sin \frac{\pi n x}{l} \diff x$



\end{document}