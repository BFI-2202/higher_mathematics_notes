\documentclass{article}
\usepackage[utf8]{inputenc}

\usepackage[T2A]{fontenc}
\usepackage[utf8]{inputenc}
\usepackage[russian]{babel}

\usepackage{amsmath}
\usepackage{pgfplots}
\usepackage{multienum}
\usepackage{geometry}
\geometry{
    left=1cm,right=1cm,top=2cm,bottom=2cm
}
\newcommand*\diff{\mathop{}\!\mathrm{d}}

\newtheorem{definition}{Определение}
\newtheorem{theorem}{Теорема}

\DeclareMathOperator{\sign}{sign}

\usepackage{hyperref}
\hypersetup{
    colorlinks, citecolor=black, filecolor=black, linkcolor=black, urlcolor=black
}

\title{Высшая математика}
\author{Лисид Лаконский}
\date{April 2023}

\begin{document}
\raggedright

\maketitle

\tableofcontents
\pagebreak

\section{Высшая математика - 5 апреля 2023 г.}

\subsection{Степенные ряды как частный случай функциональных рядов}

$a_0 + a_1 x + a_2 x^2 + a_3 x^3 + \dots$

$a_0 + a_1 (x - x_0) + a_2 (x - x_0)^2 + a_3 (x - x_0)^3 + \dots$

\begin{theorem}
Если степенной ряд сходится при некотором значении $x' \ne 0$, то он будет абсолютно сходиться $\forall |x| < |x'|$

Если степенной ряд расходится при некотором значении $x' \ne 0$, то он будет расходиться $\forall |x| > |x'|$
\end{theorem}

\begin{definition}
\textbf{Интервалом сходимости} степенного ряда называется интервал $(-R; R)$, что для всякого $x$, находящегося внутри этого интервала, ряд абсолютно сходится, а для находящегося снаружи — расходится
\end{definition}

$\lim\limits_{n \to \infty} |\frac{u_{n + 1} (x)}{u_{n} (x)}| = \lim\limits_{n \to \infty} |\frac{a_{n + 1} x^{n + 1}}{a_{n} x^{n}}| = |x| \lim\limits_{n \to \infty} |\frac{a_{n + 1}}{a_{n}}| < 1$

$|x| < \frac{1}{\lim\limits_{n \to \infty} |\frac{a_{n + 1}}{a_{n}}|} = \lim\limits_{n \to \infty} \frac{a_{n}}{a_{n + 1}}$ — \textbf{радиус сходимости}

$|x| < R$, $-R < x < R$

\hfill

$\sum \frac{x^{2 n - 1}}{9^{n}} = \frac{x^1}{9} + \frac{x^3}{81} + \frac{x^5}{729} + \dots$

$\lim\limits_{n \to \infty} |\frac{u_{n + 1} (x)}{u_{n} (x)}| = \lim\limits_{n \to \infty} | \frac{x^{2 n + 1} * 9^{n}}{9^{n} * 9 * x^{2 n - 1}} | = \frac{|x|^2}{9} < 1$, $|x^2| < 9$, $-3 < x < 3$ — \textbf{интервал сходимости}

Проверим, что происходит при $x = \pm 3$:

$\sum\limits^{x = 3} \frac{3^{2 n - 1}}{9^{n}} = \sum \frac{1}{3}$

$\sum\limits^{x = -3} \frac{(-3)^{2 n - 1}}{9^{n}}$ — ряд не является сходящимся

\hfill

Степенной ряд является \textbf{мажорируемым} на любом отрезке, целиком лежащем внутри его области сходимости.

\hfill

Если \textbf{пределы интегрирования} тоже лежат внутри интервала сходимости, то \textbf{интеграл от суммы ряда будет равняться сумме отдельных интегралов от элементов данного ряда}.

\hfill

Если степенной ряд имеет интервал сходимости $(-R; R)$, то ряд, полученный \textbf{почленным дифференцированием} этого ряда, \textbf{имеет тот же интервал сходимости}, и сумма этого ряда будет равна производной суммы исходного ряда, если $x \in (-R; R)$

\subsection{Формулы приближенных вычислений}

\begin{multienumerate}
    \mitemxxx{$\sin x = x - \frac{x^3}{3!} + \frac{x^5}{5} - \dots$}{$\cos x = 1 - \frac{x^2}{2!} + \frac{x^4}{4!} + \dots$}{$e^{x} = 1 + x + \frac{x^2}{2!} + \frac{x^3}{3!} + \dots$}
    \mitemxxx{$\cosh x = \frac{e^{x} + e^{-x}}{2} = 1 + \frac{x^2}{2!} + \frac{x^4}{4!} + \dots$}{$\sinh h = \frac{e^{x} - e^{-x}}{2} = x + \frac{x^3}{3!} + \frac{x^5}{5!} + \dots$}{$\ln (1 + x) = x - \frac{x^2}{2} + \frac{x^3}{3} - \frac{x^4}{4} + \dots$}
    \mitemxx{$\frac{1}{1 + x} = 1 - x + x^2 - x^3 + \dots$}{$(1 + x)^{m} = 1 + m x + \frac{m (m - 1)}{2!} x^2 + \frac{m (m - 1) (m - 2)}{3!} x^3 + \dots$}
\end{multienumerate}

\textbf{Например},

$\frac{1}{\sqrt[5]{e^3}} = e^{-3/5} = 1 - \frac{3}{5} + \frac{1}{2} (\frac{9}{25}) - \frac{1}{6} (\frac{27}{125}) + \dots = 1 - 0.6 + 0.18 - 0.036 = 0.544$, с точностью $0.0054$ — первый отбрасываемый член

\hfill

\textbf{Другой пример},

$\sqrt[5]{36} = (32 + 4)^{1/5} = 32^{1/5} (1 + \frac{1}{8})^{1/5} = 2 (1 + \frac{1}{8})^{1/5}$

$(1 + x)^{1/5} = 1 + \frac{1}{5} x + \frac{\frac{1}{5} (-\frac{4}{5}) x^2}{2!} + \frac{\frac{1}{5} (-\frac{4}{5}) (-\frac{9}{5}) x^3}{3!}$, $(1 + \frac{1}{8})^{1/5} = 1 + \frac{1}{5} * \frac{1}{8} - \frac{4}{25 * 2} * (\frac{1}{8})^2 + \frac{36}{125 * 6} * (\frac{1}{3})^3$

Дальнейшее решение тривиально и оставляется в качестве упражнения читателю

\hfill

\textbf{Третий пример},

$\int\limits_{0}^{1} \sqrt{x} e^{-x} \diff x = \int\limits_{0}^{1} \sqrt{x} (1 - x + \frac{x^2}{2!} - \frac{x^3}{3!} + \dots) \diff x = \int\limits_{0}^{1} (x^{1/2} - x^{3/2} + \frac{1}{2!} x^{5/2} - \frac{1}{3!} x^{7/2} + \dots) \diff x = ( \frac{2 x^{3/2}}{3} - \frac{2x^{5/2}}{5} + \frac{1}{2!} * \frac{2 x^{7/2}}{7} - \frac{1}{3!} * \frac{2 x^{9/2}}{9} + \dots) \bigg|_{0}^{1} = \frac{2}{3} - \frac{2}{5} + \frac{1}{7} - \frac{1}{54} + \dots$

\subsection{Ряды Фурье}

Функциональный ряд вида $\frac{a_0}{2} + \sum\limits_{n = 1}^{\infty} (a_{n} \cos n x + b_{n} \sin n x) $ называется \textbf{тригонометрическим рядом}, где $a_0$, $a_{n}$, $b_{n}$ — \textbf{коэффициенты тригонометрического ряда}

\hfill

Пусть периодическая функция $f(x)$ с периодом $2 \pi$ такова, что она \textbf{представляется тригонометрическим рядом, сходящимся к данной функции} на интервале $[-\pi; \pi]$, то есть \textbf{является суммой данного ряда}, тогда

$a_0 = \frac{1}{\pi} \int\limits_{-\pi}^{\pi} f(x) \diff x$, $a_{k} = \frac{1}{\pi} \int\limits_{-\pi}^{\pi} f(x) \cos k x \diff x$, $b_{k} = \frac{1}{\pi} \int\limits_{-\pi}^{\pi} f(x) \sin k x \diff x$

\hfill

Попробуем с помощью этих формул составить тригонометрический ряд Фурье для функции $$f(x) = \begin{cases}
    -1; - \pi < x < 0 \\
    1; 0 \le x \le \pi
\end{cases}$$

$a_0 = \frac{1}{\pi} \int\limits_{-\pi}^{\pi} f(x) \diff x = \frac{1}{\pi} (\int\limits_{-\pi}^{0} (-1) \diff x + \int\limits_{0}^{\pi} (1) \diff x) = (-x) \bigg|_{-\pi}^{0} + x \bigg|_{0}^{\pi} = 0 - \pi + \pi - 0 = 0$

$a_{k} = \frac{1}{\pi} \int\limits_{-\pi}^{\pi} f(x) \cos x \diff x = \frac{1}{\pi} (-\int\limits_{-\pi}^{0} \cos k x \diff x + \int\limits_{0}^{\pi} \cos k x \diff x) = \frac{1}{\pi} (-\frac{1}{k} \sin k x \bigg|^{0}_{-\pi} + \frac{1}{k} \sin k x \bigg|_{0}^{\pi}) = 0$

$b_{k} = \frac{1}{\pi} \int\limits_{-\pi}^{\pi} f(x) \sin k x \diff x = \frac{1}{\pi} (-\int\limits_{-\pi}^{0} \sin k x \diff x + \int\limits_{0}^{\pi} \sin k x \diff x) = \frac{1}{\pi k} (\cos k x \bigg|_{-\pi}^{0} - \cos k x \bigg|_{0}^{\pi}) = \frac{1}{\pi k} (\cos 0 - \cos k \pi - \cos k \pi + \cos 0) = \frac{1}{\pi k} (1 - \cos k \pi - \cos k \pi + 1) = \begin{cases}
    0, k \text{ — четное} \\
    \frac{4}{\pi k}, k \text{ — нечетное}
\end{cases}$

$\frac{a_0}{2} + \sum\limits_{n = 1}^{\infty} (a_{n} \cos n x + b_{n} \sin n x) = \frac{4}{\pi} \sum\limits_{k = 1} \frac{\sin n x}{n} = \frac{4}{\pi} (\sin x + \frac{\sin 3 x}{3} + \frac{\sin 5 x}{5})$. где $n$ — нечетное

\hfill

\begin{definition}

Функция $f(x)$ называется \textbf{кусочно-монотонной} на $[a; b]$, если этот отрезок можно разбить конечным числом точек на интервалы, так что на каждом из этих интервалов эта функция либо не возрастает, либо не убывает.

Таким образом, функция может иметь разрывы, но эти разрывы могут быть только первого рода.

\end{definition}

\begin{theorem}

Если $f(x)$ — периодическая функция ($T = 2 \pi$), являющаяся кусочно-монотонной и ограниченной на отрезке $[-\pi; \pi]$, то ряд Фурье ($\frac{a_0}{2} + \sum\limits_{n = 1}^{\infty} (a_{n} \cos n x + b_{n} \sin n x)$), построенный для этой функции, сходится во всех точках.

Сумма полученного ряда равняется значению $f(x)$ в точках непрерывности функции. А в точках, где функция имеет разрыв, сумма равна среднему арифметическому между пределом справа и пределом слева для функции $f(x)$

\end{theorem}

\section{Высшая математика - 10 апреля 2023 г.}

\subsection{Уравнения касательной плоскости и нормали в заданной точке}

$F(x, y, z) = 0$

Исходная точка $M(x_0; y_0; z_0) \in F$, если $F(x_0; y_0; z_0) = 0$

\hfill

\textbf{Уравнение касательной} $A(x - x_0) + B(y - y_0) + C(z - z_0) = 0$

\textbf{Уравнение нормали}: $\frac{x - x_0}{A} = \frac{y - y_0}{B} = \frac{z - z_0}{C}$

\hfill

$A = \frac{\delta F}{\delta x} \bigg|_{M}$, $B = \frac{\delta F}{\delta y} \bigg|_{M}$, $C = \frac{\delta F}{\delta z} \bigg|_{M}$

Если $A, B, C = 0$ или $A, B, C = 1$, то мы оставляем такую дробь в уравнении нормали

\paragraph{Пример №1} Написать уравнение касательной и нормали к плоскости $x + y^2 + z^2 = 5$, $M(1; 0; 2)$

$F(x, y, z): \ x + y^2 + z^2 - 5 = 0$

$A = \frac{\delta F}{\delta x} \bigg|_{M} = 1$, $B = \frac{\delta F}{\delta y} \bigg|_{M} = 0$, $C = \frac{\delta F}{\delta z} \bigg|_{M} = -4$

\textbf{Уравнение нормали}: $\frac{x - 1}{1} = \frac{y - 0}{0} = \frac{z + 2}{4}$

\textbf{Уравнение касательной}: $1 (x - 1) + 0 (y - 0) - 4 (z + 2) = 0 \Longleftrightarrow x - 4z - 9 = 0$

\subsection{Производные по направлению}

$u = f(x, y, z)$ по направлению $\vec{l} = \{ m, n, p \}$ в точке $M (x_0; y_0; z_0)$

\begin{enumerate}
    \item $\vec{grad} \ u = \frac{\delta u}{\delta x} * \vec{i} + \frac{\delta u}{\delta y} * \vec{j} + \frac{\delta u}{\delta z} * \vec{k} = \{ \frac{\delta u}{\delta x}; \frac{\delta u}{\delta y}; \frac{\delta u}{\delta z} \}$
    \item $\vec{e_{l}} = \frac{\vec{l}}{|\vec{l}|} = \{ \frac{m}{\sqrt{m^2 + n^2 + p^2}}; \frac{n}{\sqrt{m^2 + n^2 + p^2}}; \frac{p}{\sqrt{m^2 + n^2 + p^2}} \} = \{ \cos \alpha; \cos \beta; \cos \gamma \}$
    \item $\frac{\delta u}{\delta l} = \frac{\delta u}{\delta x} \bigg|_{M} * \cos \alpha + \frac{\delta u}{\delta y} \bigg|_{M} * \cos \beta + \frac{\delta u}{\delta z} \bigg|_{M} * \cos \gamma$
\end{enumerate}

Модуль градиента функции в какой-либо точке это максимально возможное значение производной этой функции в этой точке.

\paragraph{Пример №1} Пусть $u = x y z$, $M = (-1; 0; 1)$, $\vec{l} = \{ -1; 1; -2 \}$

\begin{enumerate}
    \item $\vec{\text{grad}} u = \{ y z; x z; x y \} = \{ 0; -1; 0 \}$
    \item $\vec{e_{l}} = \{ \frac{-1}{\sqrt{6}}; \frac{\sqrt{1}}{\sqrt{6}}; \frac{-2}{\sqrt{6}} \}$
    \item \textbf{Ответ}: $\frac{\delta u}{\delta l} = - \frac{1}{\sqrt{6}}$
\end{enumerate}

\subsection{Угол между градиентами функции}

Допустим, имеем $\vec{a} = \{ a_{x}; a_{y} \}$, $\vec{b} = \{ b_{x}; b_{y} \}$

$\cos \phi = \frac{\vec{a} \vec{b}}{|\vec{a}| |\vec{b}|} = \frac{a_{x} b_{x} + a_{y} b_{y}}{\sqrt{a_{x}^2 + a_{y}^2} * \sqrt{b_{x}^2 + b_{y}^2}}$

\paragraph{Пример №1}

$z = x^2 y$ с градиентом в точках $A (1; 1)$, $B (2; 0)$

$\vec{grad} \ z = \{ 2x y; x^2 \}$, $\vec{grad} \ z \bigg|_{A} = \{ 2; 1 \}$, $\vec{grad} \ z \bigg|_{B} = \{ 0; 4 \}$

\textbf{Ответ}: $\cos \phi = \frac{4}{\sqrt{5} * \sqrt{16}} = \frac{1}{\sqrt{5}}$

\subsection{Нахождение функции двух переменных по ее полному дифференциалу}

Полный дифференциал — сумма частных производных функции

Допустим, имеем $z(x, y)$, $\diff z = \frac{\delta z}{\delta x} \diff x + \frac{\delta z}{\delta y} \diff y$

Для того, чтобы найти исходную функцию, нам необходимо:

\begin{enumerate}
    \item $P (x, y) = \frac{\delta z}{\delta x}$, $Q (x, y) = \frac{\delta z}{\delta y}$
    \item Проверить равенство $\frac{\delta P}{\delta y} = \frac{\delta Q}{\delta x}$. Если равенство соблюдается — у дифференциала есть исходная функция, если не соблюдается — дальнейшего решения нет
    \item $z = \int P (x, y) \diff x + C_1 (y) = \Phi_1 (x, y) + C_1 (y)$, \textbf{либо} \\
    $z = \int Q(x, y) \diff y + C_2 (x) = \Phi_2 (x, y) + C_2 (x)$, где $C_1, C_2$ — функции, которые не рассматривались в рамках интегрирования
    \item $C_1 (y) = \int (Q (x, y) - \frac{\delta \Phi_1}{\delta y}) \diff y$, \textbf{либо} \\
    $C_2 (x) = \int (P(x, y) - \frac{\delta \Phi_2}{\delta x}) \diff x$
\end{enumerate}

\paragraph{Пример №1} $\diff z = (y^2 - 1) \diff x + (2 x y + 3 y) \diff y$

\begin{enumerate}
    \item $P = y^2 - 1$, $Q = 2 x y + 3 y$
    \item $\frac{\delta P}{\delta y}  = 2 y$, $\frac{\delta Q}{\delta x} = 2 y$, $2y = 2y$ — равенство соблюдается, следовательно, исходная функция существует
    \item $z = \int (y^2 - 1) \diff x + C_1(y) = y^2 x - x + C_1 (y) = \dots$, \\ $C_1 (y) = \int (2 x y + 3 y - 2 y x) \diff y = \frac{3 y^2}{2} + C$, $\dots = y^2 x - x + \frac{3 y^2}{2} + C$ 
\end{enumerate}

$Q (x, y) = \frac{\delta \Phi_1}{\delta y} + \frac{\diff (C_1 (y))}{\delta y} \Longrightarrow \frac{\diff (C_1 (y))}{\diff y} = Q - \frac{\delta \Phi_1}{\delta y}$

\subsection{Экстремумы функции двух переменных}

\begin{enumerate}
    \item $\begin{cases}
        \frac{\delta f}{\delta x} = 0  \text{ или не существует }\\
        \frac{\delta f}{\delta y} = 0 \text{ или не существует }
    \end{cases}$
    \item $\Delta = \begin{vmatrix}
            \frac{\delta^2 f}{\delta x^2} & \frac{\delta^2 f}{\delta x \delta y} \\
            \frac{\delta^2 f}{\delta x \delta y} & \frac{\delta^2 f}{\delta y^2}
        \end{vmatrix} M_0
    $
    \begin{enumerate}
        \item если $\Delta > 0$ и $\frac{\delta f}{\delta x} \bigg|_{M_0} < 0$,то $M$ — точка максимума
        \item если $\Delta > 0$ и $\frac{\delta^2 f}{\delta x} \bigg|_{M_0} < 0$,то $M$ — точка минимума
        \item если $\Delta < 0$, то в точке $M_0$ нет экстремума
        \item если $\Delta = 0$, то неизвестно
    \end{enumerate}
\end{enumerate}

\paragraph{Пример №1}

$z = \frac{x^3}{3} + x y + \frac{y^2}{2} - x - y + 14$

\begin{enumerate}
    \item $
    \begin{cases}
        x^2 + y - 1 = 0 \\
        x + y - 1 = 0
    \end{cases}
    $, решая, находим $x = 0$, $y = 1$; $x = 1$, $y = 0$
    \item $\frac{\delta^2 f}{\delta x^2} = 2 x \bigg|_{x = 0, y = 1} = 0$, $2x \bigg|_{x = 1, y = 0} = 2$ \\
    $\frac{\delta^2 f}{\delta y^2} = 1$, $\frac{\delta^2 f}{\delta x \delta y} = 1$ \\
    $\Delta = \begin{vmatrix}
        0 & 1 \\
        1 & 1
    \end{vmatrix} = -1$, $\Delta = \begin{vmatrix}
        2 & 1 \\
        1 & 1
    \end{vmatrix} = 1$ \\
    Следовательно, в точке $M_1 (0; 1)$ \textbf{нет экстремума} , $M_2 (1; 0)$ — \textbf{точка минимума}
\end{enumerate}

\pagebreak
\section{Высшая математика - 12 апреля 2023 г.}

\subsection{Примеры решения знакоположительных числовых рядов}

Если $\lim\limits_{n \to \infty} \ne 0$, то ряд точно является расходящимся

Иначе проверяем достаточные признаки: 
\begin{multienumerate}
    \mitemxx{Первый признак сравнения}{Второй признак сравнения}
    \mitemxx{Признак Д'Аламбера}{Признак Коши (радикальный)}
    \mitemx{Признак Коши (интегральный)}
\end{multienumerate}

\paragraph{Пример №1}

$\sum \frac{2 n + 1}{n + 1}$ — не выполняется необходимый признак, следовательно, расходится

\paragraph{Пример №2}

$\sum \sin \frac{\pi}{2\sqrt{n}} \sim \sum \frac{\pi}{2\sqrt{n}}$

Проверим: $\lim \frac{\sin \frac{\pi}{2 \sqrt{n}}}{\frac{\pi}{2 \sqrt{n}}} = 1$ — следовательно, ряды ведут себя одинаково

$\sum \frac{\pi}{2\sqrt{n}} \sim \sum \frac{1}{\sqrt{n}}$ — расходится, так как $\frac{1}{2} < 1$

\paragraph{Пример №3}

$\sum \frac{n^3}{3^n}$

$\lim\limits_{n \to \infty} \frac{a_{n + 1}}{a_{n}} = \lim\limits_{n \to \infty} \frac{\frac{(n + 1)^3}{3^{n + 1}}}{\frac{n^3}{3^n}} = \lim\limits_{n \to \infty} \frac{(n + 1)^3}{3 * n^3} = \frac{1}{3} < 1$ — ряд сходящийся

\paragraph{Пример №4}

$\sum \frac{2^{n}}{n^2 + n} \sim \sum \frac{2^{n}}{n^2}$

$\lim\limits_{n \to \infty} \frac{a_{n + 1}}{a_{n}} = \lim\limits_{n \to \infty} \frac{\frac{2^{n + 1}}{(n + 1)^2}}{\frac{2^{n}}{n^2}} = \lim\limits_{n \to \infty} \frac{2^n *2}{(n + 1)^2} * \frac{n^2}{2^{n}} = 2 > 1$ — расходящийся

\paragraph{Пример №5}

$\sum 2^{- n} (\frac{n + 1}{n})^{n^2}$

$\lim\limits_{n \to \infty} \sqrt[n]{2^{- n} (\frac{n + 1}{n})^{n^2}} = \lim\limits_{n \to \infty} (2^{-1} * (\frac{n + 1}{n})^{n}) = \frac{1}{2} \lim (1 + \frac{1}{n})^{n} = \frac{e}{2} > 1$ — расходящийся

\paragraph{Пример №6}

$\sum (\frac{n}{n + 1})^{n^2}$

$\lim\limits_{n \to \infty} \sqrt[n]{(\frac{n}{n + 1})^{n^2}} = \lim\limits_{n \to \infty} (\frac{n}{n + 1})^{n} = \frac{1}{e} < 1$ — ряд сходится

\paragraph{Пример №7}

$\sum \frac{1}{n^2} \sin \frac{1}{n}$

$\int\limits_{1}^{\infty} \frac{1}{x^2} \sin \frac{1}{x} \diff x = \lim\limits_{B \to \infty} \int\limits_{1}^{B} \frac{1}{x^2} \sin \frac{1}{x} \diff x = - \lim\limits_{B \to \infty} \int\limits_{1}^{B} \sin \frac{1}{x} \diff (\frac{1}{x}) = \lim\limits_{B \to \infty} \cos \frac{1}{x} \bigg|_{1}^{B} = \lim\limits_{B \to \infty} (\cos \frac{1}{B} - \cos 1) = 1 - \cos 1$ — интеграл сходится, следовательно, ряд тоже сходится

\paragraph{Пример №8}

$\sum n e^{-\frac{n^2}{2}}$

$\int\limits_{1}^{\infty} x e^{-\frac{x^2}{2}} \diff x = \lim\limits_{B \to \infty} \int\limits_{1}^{B} x e^{-\frac{x^2}{2}} \diff x = \dots$

$d (e^{-\frac{x^2}{2}}) = e^{-\frac{x^2}{2}} * (-x)$

$\dots = - \lim\limits_{B \to \infty} \int\limits_{1}^{B} d(e^{-x^2/2}) = - \lim\limits_{B \to \infty} e^{-x^2/2} \bigg|_{1}^{B} = \frac{1}{\sqrt{e}}$ — сходящийся ряд

\subsection{Примеры решения знакочередующихся (знакопеременных) числовых рядов}

Если сходится $\sum |a_{n}|$, то ряд из $\sum a_{n}$ \textbf{сходится абсолютно}

Признак Лейбница
\begin{enumerate}
    \item $\lim\limits_{n \to \infty} a_{n} = 0$
    \item $a_1 > a_2 > a_3 > \dots > a_{n} > a_{n + 1}$
\end{enumerate}

\paragraph{Пример №9}

$\sum (-1)^{n + 1} \frac{n}{2^{n}}$

$\sum \frac{n}{2^{n}}$, проверим $\lim\limits_{n \to \infty} \frac{\frac{n + 1}{2^{n + 1}}}{\frac{n}{2^{n}}} = \lim\limits_{n \to \infty} \frac{n + 1}{2 * n} = \frac{1}{2} < 1$ — ряд сходится абсолютно

\paragraph{Пример №10}

$\sum (-1)^{n + 1} \frac{3^{n}}{n^2}$

$\sum \frac{3^{n}}{n^2}$, проверим $\lim\limits_{n \to \infty} \frac{\frac{3^{n + 1}}{(n + 1)^2}}{\frac{3^{n}}{n^2}} = 3$ — абсолютной сходимости нет

Проверим сходимость по Лейбницу: $\lim\limits_{n \to \infty} \frac{3^{n}}{n^2} = \lim\limits_{n \to \infty} \frac{3^{n} \ln 3}{2 n} = \lim\limits_{n \to \infty} \frac{3^{n} (\ln 3)^2}{2} = \infty$ — не выполнен необходимый признак, никакой сходимости нет

\paragraph{Пример №11}

$\sum (-1)^{n + 1} \frac{n}{6 n - 5}$

$\lim\limits_{n \to \infty} \frac{n}{6 n - 5} = \frac{1}{6}$ — тоже расходящийся ряд по необходимому признаку

\paragraph{Пример №12}

$\sum \frac{1}{\sqrt{n}}$

Можем легко проверить, что абсолютной сходимости нет, но есть сходимость по Лейбницу: $\lim\limits_{n \to \infty} \frac{1}{\sqrt{n}} = 0$, $1 > \frac{1}{\sqrt{2}} > \frac{1}{\sqrt{3}} > \dots$

\subsection{Примеры решения функциональных рядов}

\paragraph{Пример №1}

$\sum \frac{(-1)^{n} * n}{n^2 + 1} (x + 2)^{n}$

$\lim \limits_{n \to \infty} | \frac{a_{n + 1} (x)}{a_{n} (x)} | = \lim\limits_{n \to \infty} | \frac{\frac{(-1)^{n + 1} * (n + 1)}{(n + 1)^2 + 1} (x + 2)^{n + 1}}{\frac{(-1)^{n} * n}{n^2 + 1} (x + 2)^{n}} | = | x + 2 | \lim\limits_{n \to \infty} | \frac{n + 1}{(n + 1)^2 + 1} * \frac{n^2 + 1}{n} | = | x + 2 | < 1$

$-1 < x + 2 < 1 \Longleftrightarrow -3 < x < -1$ — \textbf{область сходимости} данного ряда

Проверим граничные значения:

Если $x = -1$, то $\sum \frac{(-1)^{n} n}{n^2 + 1}$ — условно сходится

Если $x = -3$, то $\sum \frac{(-1)^n n}{n^2 + 1} (-1)^{n} = \sum \frac{n}{n^2 + 1}$ — расходящийся

Обновим границы: $-3 < x \le -1$

\paragraph{Пример №2}

$\sum \frac{(-1)^{n} n!}{n^2} (x - 1)^{n}$

Ряд из модулей: $\sum \frac{n!}{n^2} (x - 1)^{n}$, $\lim\limits_{n \to \infty} \frac{a_{n + 1}}{a_{n}} = \lim\limits_{n \to \infty} | \frac{(n + 1)! (x - 1)^{n + 1}}{(n + 1)^2 n! (x - 1)^{n}} | = |x - 1| \lim\limits_{n \to \infty} (n + 1) < 1$ только при $x = 1$

Область сходимости: $x = 1$

\subsection{Примеры разложения в ряд}

\paragraph{Пример №1}

$\ln (3 x - 2) = \ln (3 (x - 2 + 2) - 2) = \ln (3 (x - 2) + 4) = \ln 4 (1 + \frac{3 ( x - 2)}{4}) = \ln 4 + \ln (1 + \frac{3}{4}(x - 2)) = \ln 4 + \frac{3}{4} (x - 2) - \frac{1}{2} (\frac{3}{4} (x - 2))^2 + \frac{1}{3} (\frac{3}{4} (x - 2))^3 - \dots$

\subsection{Ряды Фурье}

$f(x) = \frac{a_0}{2} + \sum (a_{n} \cos n x + b_{n} \sin n x)$

$a_0 = \frac{1}{\pi} \int\limits_{-\pi}^{\pi} f(x) \diff x$, $a_{k} = \frac{1}{\pi} \int\limits_{-\pi}^{\pi} f(x) \cos k x \diff x$, $b_{k} = \frac{1}{\pi} \int\limits_{-\pi}^{\pi} f(x) \sin k x \diff x$

\subsubsection{Разложение в ряд Фурье четных и нечетных функций}

\begin{enumerate}
    \item $f(-x) = f(x)$ четная $[-\pi; \pi]$ \\
    $a_0 = \frac{2}{\pi} \int\limits_{0}^{\pi} f(x) \diff x$, $a_{n} = \frac{2}{\pi} \int\limits_{0}^{\pi} f(x) \cos n x \diff x$, $b_{n} = 0$
    \item $f(-x) = - f(x)$ нечетная $[-\pi; \pi]$ \\
    $a_{n} = 0$, $b_{n} = \frac{2}{\pi} \int\limits_{0}^{\pi} f(x) \sin n x \diff x$
\end{enumerate}

\paragraph{Пример №1}

$f(x) = x^2$, промежуток $[-\pi; \pi]$

$a_0 = \frac{2}{\pi} \int\limits_{0}^{\pi} x^2 \diff x = \frac{2}{\pi} * \frac{x^3}{3} \bigg|_{0}^{\pi} = \frac{2}{3} \pi^2$

$a_{n} = \frac{2}{\pi} \int\limits_{0}^{\pi} f(x) \cos n x \diff x = \frac{2}{\pi} \int\limits_{0}^{\pi} x^2 \cos n n x \diff x = \begin{vmatrix}
    u = x^2 & \diff v = \cos n x \diff x \\
    \diff u = 2x \diff x & v = \frac{1}{n} \sin n x
\end{vmatrix} = \frac{2}{\pi} (\frac{x^2}{n} \sin n x - \frac{2}{n} (-\frac{x}{n} \cos n x + \frac{1}{n^2} \sin n x)) \bigg|_{0}^{\pi} = \frac{4}{\pi n^2} (\pi \cos \pi n - 0) = \frac{4 \cos \pi n}{n^2} = \frac{4}{n^2} (-1)^{n}$

\hfill

$x^2 = \frac{\pi^2}{3} + 4 \sum \frac{(-1)^{n}}{n^2} \cos n x = \frac{\pi^2}{3} - 4 (\frac{\cos x}{1} - \frac{\cos 2 x}{2^2} + \frac{\cos 3 x}{3^2} - \dots)$

$0 = \frac{\pi^2}{3} - 4 (1 - \frac{1}{2^2} + \frac{1}{3^2} - \frac{1}{4^2} + \dots)$

\hfill

Ряды Фурье с \textbf{произвольным периодом} (например, на $[-l; l]$) записываются следующим образом:

$$f(x) = \frac{a_0}{2} + \sum (a_{n} \frac{\cos \pi n x}{x} + b_{n} \frac{\sin \pi n x}{l})$$

И тогда $a_{n} = \frac{1}{l} \int\limits_{- l}^{l} f(x) \cos \frac{\pi n x}{l} \diff x$, $b_{n} = \frac{1}{l} \int\limits_{- l}^{l} f(x) \sin \frac{\pi n x}{l} \diff x$

\end{document}