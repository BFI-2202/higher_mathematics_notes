\documentclass{article}
\usepackage[utf8]{inputenc}

\usepackage[T2A]{fontenc}
\usepackage[utf8]{inputenc}
\usepackage[russian]{babel}

\usepackage{amsmath}
\usepackage{pgfplots}
\usepackage{multienum}
\newcommand*\diff{\mathop{}\!\mathrm{d}}

\title{Высшая математика}
\author{Лисид Лаконский}
\date{October 2022}

\begin{document}

\maketitle

\tableofcontents
\pagebreak

\section{Высшая математика - 28.10.2022}

\subsection{Производная неявно заданной функции}

\begin{flushleft}

\textbf{Неявно заданная функция} - это функция, заданная неявно (очень полезное определение).

Например, $y = y(x)$ - явный вид; $y^2 + xy - \sin x = 0$, $F(x, y)$ - неявный вид.

\hfill

Берем производную всего выражения, при этом помним, что $y$ является функцией от $x$.

\subsubsection{Примеры}

\parbox{0.5\textwidth}{
\textbf{Пример 1.}

$y^2 + x y - \sin x = 0$

$2y * y'_x + 1 * y + x * y'_x - \cos x = 0$

$2y * y' + x y' = \cos x - y$

$y'(2y + x) = \cos x - y$

$y' = \frac{\cos x - y}{2y + x}$

\bigskip
\textbf{Пример 2.}

$2y * y'_x + 1 * y + x * y'_x - \cos x = 0$

$2y' * y' + 2y * y'' + y' * y' + xy'' + \sin x = 0$

$y''(2y + x) = -(\sin x + 3y')$

$y'' = -\frac{(\sin x + 3y')}{2y + x}$
}
\hfill
\parbox{0.4\textwidth}{
\textbf{Пример 3.}

$y * \tg x + x^3 y^2 - x^2 = 0$

$y' * \tg x + y * \frac{1}{\cos^2 x} + 3x^2 * y^2 + x^3 * 2y * y' - 2x = 0$

$y'(\tg x + 2x^3y) = 2x - 3 x^2 y^2 - \frac{y}{\cos^2 x}$

$y' = \frac{2x - 3 x^2 y^2 - \frac{y}{\cos^2 x}}{\tg x + 2 x^3 y}$
}

\pagebreak
\subsection{Производная параметрически заданной функции}

\begin{equation}
    \begin{cases}
        x = x(t) \\
        y = y(t)
    \end{cases}
\end{equation}

$y'_x = \frac{y'_t}{x'_t}$ - производная функции, заданной параметрически

$y''_{xx} = (y'_x)' = \frac{(y'_x)'_t}{x'_t} = \frac{y''_{t t} * x'_t - y'_t * x''_{t t}}{(x'_t)^3}$ - вторая производная функции, заданной параметрически

\subsubsection{Примеры}

\textbf{Пример 1.}

\begin{equation}
    \begin{cases}
        x = 3 \cos t \\
        y = 4 \sin t
    \end{cases}
\end{equation}

\textbf{Editor note: добавить табличку с значениями $t$ и график}

\hfill

Эту же функцию можно задать неявно: $\frac{x^2}{9} + \frac{y^2}{16} = 1$

\hfill

\textbf{Пример 2.}

\begin{equation}
    \begin{cases}
        x = 3t \\
        y = 6t - t^2
    \end{cases}
\end{equation}

Выразим из уравнения $t$, получим: $y = 2x - \frac{x^2}{9}$ - уравнение параболы.

\hfill

$x'_t = 3$, $y'_t = 6 - 2t$, $y'_x = \frac{6 - 2t}{3} = 2 - \frac{2t}{3} = 2 - \frac{2x}{9}$

\hfill

\textbf{Пример 3.}

\begin{equation}
    \begin{cases}
        x = \cos t \\
        y = \sin t
    \end{cases}
\end{equation}

$y = \sqrt{1 - x^2} = \frac{-2x}{2\sqrt{1 - x^2}} = -\frac{x}{\sqrt{1 - x^2}}$

$y'_x = \frac{y'_t}{x'_t} = -\frac{\cos t}{\sin t} = -\frac{\cos t}{\sqrt{1 - \cos^2 t}}$

\pagebreak
\subsection{Метод логарифмического дифференцирования}

\subsubsection{Примеры}

\textbf{Пример 1.}

Имеем функцию $y(x) = \frac{\sqrt{x + 1} * \sqrt[3]{2x + 5}}{(x^2 + 6)^5 (x - 4)^6}$.

Пользуясь свойствами логарифмов, максимально упростим данную запись:

\hfill

$\ln y = \ln \frac{\sqrt{} \sqrt[3]{}}{( )^5 ( )^6}$

$\ln y = \frac{1}{2} \ln (x + 1) + \frac{1}{3} \ln (2x + 5) - 5 \ln (x^2 + 6) - 6 \ln (x - 4)$

\hfill

Берем производную от обеих частей этого равенства, помня о том, что $\ln y$ является сложной функцией:

\hfill

$\frac{1}{y} * y'(x) = \frac{1}{2} \frac{1}{x + 1} + \frac{1}{3} \frac{2}{2x + 5} - 5 \frac{2x}{x^2 + 6} - 6 \frac{1}{x - 4}$

$y'(x) = y(...) = \frac{\sqrt{x + 1} \sqrt[3]{2x + 5}}{(x^2 + 6)^5 (x - 4)^6} (\frac{1}{2(x+1)} + \frac{2}{3(2x + 5)} - \frac{10x}{x^2 + 6} - \frac{6}{x - 4})$

\hfill

\textbf{Пример 2.}

Имеем функцию $y = x^{\tg x}$

\hfill

$\ln y(x) = \ln x^{\tg x}$

$\ln y(x) = \tg x * \ln x$

$\frac{1}{y} * y'(x) = \frac{1}{\cos^2 x} \ln x + \tg x * \frac{1}{x}$

$y'(x) = y(...) = x^{\tg x} (\frac{\ln x}{\cos^2 x} + \frac{\tg x}{x})$

\pagebreak
\subsection{Производные и дифференциалы высших порядков}

Имеем функцию $y = f(x)$, определенную на интервале $[ a; b ]$.

Предполагаем, что ее производная не имеет никаких необычных свойств на данном отрезке: не имеет острых углов, разрывов и так далее.

В этом случае мы эту производную $y' = f'(x)$, если она дифференцируема на отрезке $[ a; b ]$, можем дифференцировать.

\hfill

\rule{\textwidth}{0.2pt}
\parbox{0.33\textwidth}{
$y = e^x$

$y' = e^x$

$y'' = e^x$

$y''' = e^x$

$y'''' = e^x$

$y''''' = e^x$

$y^{(n)} = e^x$
}
\parbox{0.33\textwidth}{
$y = \frac{1}{x}$

$y' = - \frac{1}{x^2}$

$y'' = \frac{2}{x^3}$

$y''' = - \frac{2 * 3}{x^4}$

$y'''' = \frac{2 * 3 * 4}{x^5}$

$y''''' = -\frac{2 * 3 * 4 * 5}{x^6}$

$y^{(n)} = (-1)^n \frac{n!}{x^{n+1}}$
}
\parbox{0.3\textwidth}{
$y = \sin x$

$y' = \cos x$

$y'' = - \sin x$

$y''' = - \cos x$

$y'''' = \sin x$

$y''''' = \cos x$

$y^{(n)} = \sin (x + \frac{\pi n}{2})$
}

\rule{\textwidth}{0.2pt}

\hfill

Работая с производными и дифференциалами высших порядков, следует пользоваться следующими свойствами:

\begin{multienumerate}
    \mitemxx{$(u \pm v)^{(n)} = u^{(n)} \pm v^{(n)}$}{$(C u)^{(n)} = C u^{(n)}$}
    \mitemx{$(u v)^{n} = u^{(n)} v + n u^{(n - 1)} + \frac{n (n - 1)}{2!} u^{(n - 2)} v'' + ... + n u' v^{(n - 1)} + u v^{(n)} = \sum^{n}_{k = 1} = C_{n}^{k} u^{(k)} v^{(n - k)}$, где $C_{n}^{k} = \frac{n!}{k!(n-k)!}$ - формула Лейбница}
\end{multienumerate}

\subsubsection{Примеры}

\textbf{Пример 1.}

Найти производную 10-го порядка функции $f(x) = (3x^2 + 2x + 1) \sin x$

\hfill

\rule{\textwidth}{0.2pt}

\parbox{0.43\textwidth}{
$u = \sin x$

$u' = \cos x$   
}
\parbox{0.45\textwidth}{
$v = 3x^2 + 2x + 1$

$v' = 6x + 2$

$v'' = 6$
}

\rule{\textwidth}{0.2pt}

\hfill

$f'(x) = (\sin x)^{(10)} (3x + 2x + 1) + 10 (\sin x)^{(9)} (6x + 2) + \frac{10 * 9}{2} (\sin x)^{(8)} * 6$

\hfill

$u^{(8)} = \sin x$, $u^{(9)} = \cos x$, $u^{10} = (-\sin x)$

\hfill

$f'(x) = (-\sin x) (3x + 2x + 1) + 10 (\cos x) (6x + 2) + \frac{10 * 9}{2} (\sin x) * 6$

\pagebreak
\subsection{Дифференциалы высших порядков}

$\diff (\diff y) = \diff (f'(x) \diff x) = (f' (x) \diff x)' \diff x = \diff^2 f = f''(x) \diff x^2$ - дифференциал второго порядка

$\diff^{n} f = f^{(n)} (x) \diff x^{(n)}$

\hfill

Если для дифференциала первого порядка можно говорить об инвариантности формы, то для дифференциалов высших порядков инвариантности формы нет, и вышеизложенные равенства верны только если $x$ - независимая переменная.

\end{flushleft}

\end{document}