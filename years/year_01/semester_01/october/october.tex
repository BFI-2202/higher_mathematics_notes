\documentclass{article}
\usepackage[utf8]{inputenc}

\usepackage[T2A]{fontenc}
\usepackage[utf8]{inputenc}
\usepackage[russian]{babel}

\usepackage{amsmath}
\usepackage{pgfplots}
\usepackage{multienum}
\newcommand*\diff{\mathop{}\!\mathrm{d}}

\usepackage{hyperref}
\hypersetup{
    colorlinks, citecolor=black, filecolor=black, linkcolor=black, urlcolor=black
}

\title{Высшая математика}
\author{Лисид Лаконский}
\date{October 2022}

\begin{document}

\maketitle

\tableofcontents
\pagebreak

\section{Высшая математика - 03.10.2022}

\subsection{Предел функции}

\begin{enumerate}
    \item Любую константу мы можем вынести за предел
    \item Предел от суммы двух функций $f(x) + g(x)$ дает в нам результате разложения сумму двух пределов
    \item Предел от произведения двух функций разлагается на произведение двух пределов
    \item Предел частного от двух функций ($g(x) \ne 0$) равен частному двух пределов, если нет неопределенности
\end{enumerate}

\subsection{Виды неопределенностей}

\begin{flushleft}

Основные виды неопределенностей: $\frac{0}{0}$, $\frac{\infty}{\infty}$, $(0 * \infty)$, ($\infty - \infty)$, $1^{\infty}$, $0^0$, $\infty^0$

Раскрывать неопределенности позволяет:

\begin{enumerate}
    \item Упрощение вида функции (преобразование выражения с использованием формул сокращенного умножения, тригонометрических формул, домножения на сопряженные выражения с последующим сокращением и тому подобное)
    \item Использование замечательных пределов
    \item Применение правила Лопиталя
    \item Использование замены бесконечно малого выражения ему эквивалентным
\end{enumerate}

\subsubsection{Неопределенность вида $\frac{0}{0}$}

Пробуем преобразовать и упростить выражение. Если есть выражение вида $\frac{\sin (kx)}{kx}$ или $\frac{kx}{\sin (kx)}$, то применяем первый замечательный предел. Если не помогает, то используем правило Лопиталя или таблицу эквивалентных бесконечно малых.

Правила раскрытия неопределенности:

\begin{enumerate}
    \item Для того, чтобы определить предел дробно-рациональной функции ($\lim\limits{x \to a} f(x)$), надо числитель и знаменатель дроби разделить на $x - a$ и перейти к пределу. Если и после этого числитель и знаменатель новой дроби имеют пределы, равные нулю, то надо произвести повторное деление на $x - a$
    \item Для того, чтобы определить предел, в котором числитель или знаменатель иррациональны, следует избавиться от иррациональности, умножив и числитель и знаменатель дроби на одно и то же выражение, приводящее к формулам сокращенного умножения. Неопределенность устраняется после сокращения дроби.
\end{enumerate}

В случае, когда под знаком предела стоят тригонометрические функции, используется первый замечательный предел: $\lim\limits_{x \to 0} \frac{\sin x}{x} = 1$

Его различные формы: $\lim\limits_{x \to 0} \frac{x}{\sin x} = 1$, $\lim\limits_{x \to 0} \frac{\tg x}{x} = 1$, $\lim\limits_{x \to 0} \frac{x}{\tg x} = 1$, $\lim\limits_{x \to 0} \frac{\arcsin x}{x} = 1$, $\lim\limits_{x \to 0} \frac{x}{\arcsin x} = 1$, $\lim\limits_{x \to 0} \frac{\arctg x}{x} = 1$, $\lim\limits_{x \to 0} \frac{x}{\arctg x} = 1$

\subsubsection{Неопределенность вида $\frac{\infty}{\infty}$}

Правила раскрытия неопределенности:

\begin{enumerate}
    \item Чтобы раскрыть неопределенность вида $\frac{\infty}{\infty}$ заданную отношением двух многочленов, надо и числитель и знаменатель почленно разделить на переменную величину в наибольшей степени.
    \item Для раскрытия неопределенности вида $\frac{\infty}{\infty}$, заданную отношением иррациональных функций, надо и числитель и знаменатель почленно разделить на переменную величину в наибольшей степени с учетом степеней корней.
\end{enumerate}

Если не помогает, то используем правило Лопиталя.

$
\begin{aligned}
\lim\limits{x \to \infty} \frac{a_n x^n + a_{n - 1} x^{n - 1} + \dots + a_1 x + a_0}{b_m x^m + b^{m - 1} x^{m - 1} + \dots + b_1 x + b_0} = \begin{cases}
    0, n < m \\
    \frac{a_n}{b_m}, n = m
    \infty, n > m
\end{cases}
\end{aligned}
$

\paragraph{Пример №1} Найти предел $\lim\limits_{x \to \infty} \frac{3x^3 - x^2 + 14}{x^2 - 4}$

$\lim\limits_{x \to \infty} \frac{3x^3 - x^2 + 14}{x^2 - 4} = \{ \frac{\infty}{\infty} \} = \infty$, так как $n = 3$, $m = 2$, $n > m$

\subsubsection{Неопределенность вида $(0 * \infty)$ или $(\infty - \infty)$}

Преобразуем неопределенность к виду $\frac{0}{0}$ или $\frac{\infty}{\infty}$, затем разбираемся с новой неопределенностью.

\hfill

Пусть $\lim\limits{x \to a} f(x) = 0$, $\lim\limits_{x \to a} g(x) = \infty$, тогда $\lim\limits{x \to a} f(x) g(x) = \{ 0 * \infty \} = \begin{cases}
    \lim\limits_{x \to a} \frac{f(x)}{\frac{1}{g(x)}} = \{ \frac{0}{0} \} \\
    \textbf{или} \\
    \lim\limits_{x \to a} \frac{g(x)}{\frac{1}{f(x)}} = \{ \frac{\infty}{\infty} \}
\end{cases}$

\hfill

Неопределенность вида $(\infty - \infty)$, получающаяся в результате алгебраической суммы двух дробей, устраняется или сводится к неопределенности вида $\frac{0}{0}$ путем приведения дроби к общему знаменателю.

Пусть $\lim\limits_{x \to a} f(x) = \infty$, $\lim\limits_{x \to a} g(x) = \infty$, тогда $\lim\limits_{x \to a} (f(x) - g(x)) = \{ \infty - \infty \} = \lim\limits_{x \to a} \frac{1}{\frac{1}{f(x)}} - \frac{1}{\frac{1}{g(x)}} = \lim\limits_{x \to a} \frac{\frac{1}{g(x)} - \frac{1}{f(x)}}{\frac{1}{f(x)} * \frac{1}{g(x)}} = \{ \frac{0}{0} \}$

\hfill

Неопределенность вида $(\infty - \infty)$, получающаяся в результате алгебраической суммы иррациональных выражений, устраняется или сводится к неопределенности вида $\frac{\infty}{\infty}$ путем домножения и деления на одно и то же выражение, приводящее к формулам сокращенного умножения. В случае квадратных корней разность домножается на сопряженное выражение и применяются формулы сокращенного умножения.

\subsubsection{Неопределенность вида $1^{\infty}$}

Применяем второй замечательный предел: $\lim\limits_{x \to \infty} (1 + \frac{1}{x})^x = e$

Его различные формы: $\lim\limits_{x \to 0} (1 + x)^{\frac{1}{x}} = e$, $\lim\limits_{x \to 0} \frac{\ln (1 + x)}{x} = \{ \frac{0}{0} \} = 1$, $\lim\limits_{x \to 0} \frac{a^x - 1}{x} = \{ \frac{0}{0} \} = \ln a$, $\lim\limits_{x \to 0} \frac{e^x - 1}{x} = \{ \frac{0}{0} \} = 1$, $\lim\limits_{x \to 0} \frac{(1 + x)^p - 1}{x} = \{ \frac{0}{0} \} = p$

\subsubsection{Неопределенность вида $0^0$ или $\infty^0$}

Логарифмируем выражение и используем равенство $\lim\limits_{x \to x_0} ln(f(x)) = \ln (\lim\limits_{x \to x_0} f(x))$

\end{flushleft}

\pagebreak
\section{Высшая математика - 14.10.2022}

\subsection{Бесконечно большие и бесконечно малые функции}

\begin{flushleft}

Функция называется бесконечно малой при $x \to x_0$, если $\lim\limits_{x \to x_0} f(x) = 0$.

Функция называется бесконечно большой при $x \to x_0$, если $\lim\limits_{x \to x_0} f(x) = \inf$.

\textbf{Теорема 1.} $\alpha + \beta, \alpha - \beta$ - бесконечно малые, если $\alpha, \beta$ - бесконечно малые

\textbf{Теорема 2.} Произведение бесконечно малой на ограниченую функцию является бесконечно малой

\textbf{Определение.} Если $\alpha(x), \beta(x)$ бесконечно малы при $x \to x_0$, то $\exists \lim\limits_{x \to x_0} \frac{\alpha(x)}{\beta(x)} = $ const $ \ne 0 \ne \pm \inf$, то $\alpha$ и $\beta$ - бесконечно малые одного порядка

Если $\lim\limits_{x \to x_0} \frac{\alpha(x)}{\beta(x)} = 1$, то $\alpha, \beta$ - эквивалентные бесконечно малые

\hfill

Если $\lim\limits_{x \to x_0} \frac{\alpha(x)}{\beta(x)} = 0$, то $\alpha$ - бесконечно малое более высокого порядка малости по сравнению с $\beta$.

Если, наоборот, $\lim\limits_{x \to x_0} \frac{\alpha(x)}{\beta(x)} = \inf$, то говорят, что $\beta$ более высокого порядка малости, чем $\alpha$.

Например, $\alpha = x^3 + 2x^2, \beta = 2x + 3x^2, \lim\limits_{x \to 0} \frac{x^3 + 2x^2}{3x^2 + 2x} = \lim\limits_{x \ to 0} \frac{x^2(x+2)}{x(3x + 2)} = \lim\limits_{x \to 0} \frac{x + 2}{3x + 2}$

\hfill

$\exists \lim\limits_{x \to x_0} \frac{\beta(x)}{\alpha^k(x)} = C \ne 0 \ne \pm \inf$, то $\beta, \alpha^k$ - бесконечно малые одного порядка.

Например, $\alpha = sin^3 x, \beta = x, \lim\limits_{x \to 0} \frac{\sin^3 x}{x^3} = \lim (\frac{\sin x}{x})^3 = 1 \ne 0 \ne \pm \inf$, $\sin^3 x$ величина такого же порядка малости, как $x^3$.

\subsubsection{Применение бесконечно малых к вычислению пределов}

Если при $x \to x_0$  $f(x) \sim f_1(x), g(x) \sim g_1(x)$, то $\exists \lim\limits_{x \to x_0} \frac{f_1(x)}{g_1(x)} \implies \exists \lim\limits_{x \to x_0} \frac{f(x)}{g(x)} = \lim\limits_{x \to x_0} \frac{f_1(x)}{g_1(x)}$ 

\hfill

Например, $\lim\limits_{x \to 0} \frac{\sin^3 (4x)}{x^2 + 3x} = ...$

Допустим, $\lim\limits_{x \to 0} \frac{x^2 + 3x}{x} = \lim (x+3) = 3$.

Допустим, $\lim\limits_{x \to 0} \frac{x^2 + 3x}{x^2} = \lim\limits_{x \to 0} \frac{x(x + 3)}{x} = \inf$.

Посчитали без толку, теперь продолжим, $... = \lim\limits_{x \to 0} \frac{(4x)^3}{x(x + 3)} = \lim\limits_{x \to 0} \frac{64x^2}{x + 3} = 0$.

\subsubsection{Таблица эквивалентных бесконечно малых}

$\sin x \sim x, \tg x \sim x, \arcsin x \sim x, \arctg x \sim x, \ln (x + 1) \sim x, e^x - 1 \sim x, a^x - 1 \sim x \ln a, \sqrt[n]{1 + x} - 1 \sim \frac{x}{n}, 1 - \cos x \sim \frac{x^2}{2}, \cos x ~ 1 - \frac{x^2}{2}$

Это все подходит к умножению или делению, но никак не к сложению или вычитанию.

\subsubsection{Некоторые соображения и примеры}

При $x \to \inf$ $f(x) = x^3 + 2x + 1$ больший вклад вносит $x^3$, при $x \to 0$ $f(x) = x^3 + x^2$ больший вклад вносит $x^2$

\hfill 

\textbf{Пример.} $\lim\limits_{x \to 0} \frac{\sqrt[3]{1 + 4x} - 1}{\ln (1 + 2x)} = \lim\limits_{x \to 0} \frac{\frac{4x}{3}}{2x} = \frac{2}{3}$ - применение таблицы эквивалентных бесконечно малых.

\hfill

$O(x)$ - бесконечно малая более высокая порядка малости.

\subsection{Производные и дифференциалы функции}

Тут есть рисунок, который мне тяжело воспроизвести. Поэтому его тут нет. Но на нем показаны $\Delta x$ (приращение аргумента), $\Delta f$ (приращение функции), касательная к функции.

$\diff f = f' (x) \diff x$ - дифференциал функции, $\Delta f = \diff f + O(\Delta x)$

\hfill

Производной функции называется $\lim\limits_{\Delta x \to 0} \frac{\Delta f(x)}{\Delta x} = \lim\limits_{\Delta x \to 0} \frac{f(x + \Delta x) - f(x)}{\Delta x}$. Производная равна пределу приращения функции к приращению аргумента, когда приращение аргумента стремится к нулю.

\hfill

\textbf{Пример.} Пусть у нас $y = x^3 + 2x - 1$. Попробуем вычислить производную.

$y(x + \Delta x) = (x + \Delta x)^3 + 2(x + \Delta x) - 1 = x^3 + 3x^2 * \Delta x + 3x(\Delta x)^2 + (\Delta x)^3 + 2x + 2\Delta x - 1$

$y'(x) = \lim\limits_{\Delta x \to 0} \frac{x^3 + 3x^2 * \Delta x + 3x (\Delta x)^2 + (\Delta x)^3 + 2x + 2\Delta x - 1 - x^2 - 3x + 1}{\Delta x} = \lim\limits_{\Delta x \to 0} (3x^2 + 3x(\Delta x) + (\Delta x)^2 + 2) = 3x^2 + 2$

Поздравляю, вы написали такую простыню. Вы великолепны.

\hfill

\textbf{Другой пример.} Попробуем доказать, что производная $y'(\sin x) = \cos x$

$y(x + \Delta x) = \sin(x + \Delta x)$

$y'(x) = \lim\limits_{\Delta x \to 0} \frac{\sin(x + \Delta x) - \sin x}{\Delta x} = \text{вспоминайте формулы} = ... = \lim\limits_{\Delta x \to 0} \frac{2 \sin \frac{x + \Delta x - x}{2} \cos \frac{x + \Delta x + x}{2}}{\Delta x} = \lim\limits_{x \to 0} \frac{2 \sin \frac{\Delta x}{2} \cos (x + \frac{\Delta x}{2})}{\Delta x} = \lim\limits_{x \to 0} \frac{2 \cos x}{2} = \cos x$

\subsubsection{Свойства производных}

\begin{enumerate}
    \item $(u \pm v)' = u' \pm v'$
    \item $(u * v)' = u' v + v u'$, $(u * v * w)' = u' v w + u * v' w + u * v * w'$
    \item $(cu)' = c u'$
    \item $(\frac{u}{v})' = \frac{u' v - u v'}{v^2}$
\end{enumerate}

\subsubsection{Дифференцируемость функций}

Если функция $y = f(x)$ имеет производную в точке $x_0$, то есть $\exists \lim\limits_{x \to x_0} \frac{\Delta f(x)}{\Delta x} = \lim\limits_{x \to x_0} \frac{f(x_0 + \Delta x) - f(x_0)}{\Delta x}$, то функция дифференцируема в точке $x_0$.

\hfill

\textbf{Теорема.} Если функций $y = f(x)$ дифференцируема в точке $x_0$, то она в этой точке непрерывна.

\textbf{Замечание.} Обратное высказывание может быть и неверным.

\hfill

Пример функции непрерывной в какой-то точке, но не дифференцируемой в ней.

\begin{tikzpicture}
  \begin{axis}[ 
    xlabel=$x$,
    ylabel={$|x|$}
  ] 
    \addplot { abs(x) }; 
  \end{axis}
\end{tikzpicture}

\subsubsection{Геометрический смысл производной}

Производная - это тангенс угла наклона касательной...

$\tg \alpha = \lim\limits_{\Delta x \to 0} \frac{\Delta y}{\Delta x}$

\subsubsection{Уравнение касательной и нормали к графику функции}

Пусть у нас есть $y = kx + b$, дана какая-то точка $M_0(x_0; y_0)$

\hfill

\textbf{Уравнение касательной.} $y - y_0 = y'(x_0)(x - x_0)$

\hfill

\textbf{Уравнение нормали.} $y - y_0 = -\frac{1}{y'(x_0)} (x - x_0)$

\subsubsection{Производная сложной функции}

Пусть функция $u = u(x)$ имеет в некоторой точке производную $u'_x(x)$, а функция $y = y(u)$ имеет при соответствующем значении $u$ производную $y'_u$.

Тогда сложная функция $y(x) = y(u(x))$ имеет проивзодную $y'_x = y'_u * u'_x$

\hfill

\textbf{Пример 0.} Например, у нас есть $y(x) = y(g(f(x)))$, то $y'_x = y'_g * g'_f * f'_x$

\hfill

\textbf{Пример 1.} $y = 2x^2 + 3x, y' = 4x + 3$

\hfill

\textbf{Пример 2.} $y = \cos (2x^2 + 3x), y' = sin (2x^2 + 3x) * (4x + 3)$

\hfill

\textbf{Пример 3.} $y = \sqrt{\cos{(2x^2 + 3x}}, y' = \frac{1}{2\sqrt{\cos{2x^2 + 3x}}} * (-\sin (2x^2 + 3x)) * (4x + 3)$

\hfill

\textbf{Пример 4.} $y = \tg \sqrt{\cos (2x^2 + 3x)}, y' = \frac{1}{\cos^2 (\sqrt{\cos (2x^2 + 3x)}} * \frac{1}{2 \sqrt{\cos (2x^2 + 3x)}} * (-\sin (2x^2 + 3x)) * (4x + 3)$

\subsubsection{Обратная функция и ее производная}

Пусть у нас есть функция $y = f(x)$, $x = a, x = b$, а $y(a) = c, y(b) = d$, где $[a; b]$ - область определения,  $[c; d]$ - область изменения функции.

\hfill

\textbf{Теорема.} Если для $y = f(x)$ существует обратная функция $x = \phi (y)$, у которой $\phi'(y) \ne 0$ в некоторой точке $y_0$, то $f'(x) = \frac{1}{\phi'(y)}$

\hfill

\textbf{Пример 1.} $y = \arcsin x$, функция обратная к ней $x = \sin y, x' = cos y$.

Таким образом, $(\arcsin x)' = \frac{1}{\cos x} = \frac{1}{\sqrt{1 - \sin^2 y}} = \frac{1}{\sqrt{1 - \sin^2(\arcsin x)}} = \frac{1}{\sqrt{1 - x^2}}$

\end{flushleft}

\pagebreak
\section{Высшая математика - 18.10.2022}

\subsection{Асимптоты функции}

\begin{flushleft}

Асимптоты функции могут быть:
\begin{itemize}
    \item Вертикальные
    \item Наклонные (в том числе горизонтальные)
\end{itemize}

\subsubsection{Вертикальные асимптоты}

Если функция $f(x)$ имеет точку разрыва, в которой хотя бы один односторонний предел бесконечен, то вертикальная прямая, параллельная оси ординат, проходящая через эту точку, называется \textbf{вертикальной асимптотой}.

\hfill

Вертикальных асимптот у функции может быть бесконечное множество.

\hfill

Например, $f(x) = \tg x, x = \frac{\pi}{2} + \pi k, k \in Z$

\subsubsection{Наклонные асимптоты}

Если следующие пределы: $\lim\limits_{x \to +\infty} \frac{f(x)}{x} = k, \lim\limits_{x \to +\infty} (f(x) - k x) = b$ существуют и конечны, то прямая, заданная уравнением $y = k x + b$ является наклонной асимптотой функции $f(x)$ при $x \to \infty$.

\hfill

Если $k = 0$, то асимптота называется горизонтальной.

\hfill

Наклонных асимптот у функции может быть только две.

\subsubsection{Примеры}

\textbf{Пример 1.} Найти асимптоты функции $f(x) = \frac{x}{1 + e^{-x}}$.

\hfill

Найдем вертикальные асимптоты данной функции. Для начала найдем точки разрыва.

Данная функция \textbf{непрерывна}, так как знаменатель не может быть равен нулю.

Следовательно, вертикальных асимптот нет.

\hfill 

Найдем наклонные асимптоты: посчитаем пределы.

$k_{+} = \lim\limits_{x \to +\infty} \frac{x}{x(1+e^{-x})} = 1, k_{-} = \lim\limits_{k \to -\inf} \frac{x}{x(1+e^{-x})} = 0$

$b_{+} = \lim\limits_{x \to +\infty} (\frac{x}{1+e^{-x}} - x) = \lim\limits_{x \to +\infty} \frac{x - x - x e^{-x}}{1 + e^{-x}} = 0, b_{-} = \lim\limits_{x \to -\infty} (\frac{x}{1 + e^x}) = 0$

\hfill

При $x \to +\infty, y = x$ - наклонная асимптота.

При $x \to -\infty, y = 0$ - горизонтальная асимптота.

\subsection{Производные функции}

\textbf{Определение.} Если для $f(x)$ существует предел $\lim\limits_{\Delta x \to 0} \frac{\Delta y}{\Delta x} = \lim\limits_{\Delta x \to 0} \frac{f(x + \Delta x) - f(x)}{\Delta x}$, то он называется производной функции $y = f(x)$ в точке $x$, и обозначается $y' = f'(x) = \frac{\diff f(x)}{\diff x} = \frac{\diff}{\diff x}; f(x) = \frac{\diff y}{\diff x}$.

\subsubsection{Свойства производных функции}

Принятые обозначения: $c$ - константа, $u, v$ - функции.

\begin{multienumerate}
    \mitemxx{$(c)' = 0$}{$(c u)' = c * u'$}
    \mitemxx{$(u \pm v)' = u' \pm v'$}{$(u * v)' = u' v + u v'$}
    \mitemx{$(\frac{u}{v})' = \frac{u' v - u v'}{v^2}$}
    \mitemx{Если $y = f(u), u = \phi(x)$, то $(f(\phi(x)))' = f'(u) * u'$. \\ Пример: $\cos 3x = -\sin 3x * 3 = -3\sin x$ \\ Еще один пример: $\tg^{2x} e^{x} = 2 \tg e^x * \frac{1}{\cos^2 e^x} * e^x$}
\end{multienumerate}

\subsubsection{Таблица производных}

\begin{multienumerate}
    \mitemxx{$(u^{a})' = a * u^{a - 1} * u', a \in R$  \\
    $(\frac{1}{u}) = (u^{-1})' = -1 * \frac{1}{u^2} * u'$ \\
    $(\sqrt{u})' = (u^{\frac{1}{2}})'$ = $\frac{1}{2\sqrt{u}} * u'$}{$(a^{u}) = a^{u} * \ln a * u'$ \\
    $(e^{u})' = e^{u} * u'$}
    \mitemxx{$(\log_{a}{u})' = \frac{1}{u} \log_{a}{e} * u' = \frac{1}{u \ln a} * u'$ \\
    $(\ln u)' = \frac{1}{u} * u', (\ln |u|)' = \frac{1}{u} * u'$}{$(\sin u)' = \cos x$}
    \mitemxx{$(\cos u)' = -\sin x$}{$(\tg u)' = \frac{1}{\cos^2 u} * u'$}
    \mitemxx{$(\ctg u)' = - \frac{1}{\sin^2 u} * u'$}{$(\arcsin u)' = \frac{1}{\sqrt{1 - u^2}} * u'$}
    \mitemxx{$(\arccos u)' = - \frac{1}{\sqrt{1 - u^2}} * u'$}{$(\arctg u)' = \frac{1}{1 + u^2} * u'$}
    \mitemxx{$(\arcctg u)' = - \frac{1}{1 + u^2} * u'$}{$(\sinh u)' = \cosh u * u'$}
    \mitemxx{$(\cosh u)' = \sinh u * u'$}{$(\tanh u)' = \frac{1}{\cosh^2 u} * u'$}
    \mitemxx{$(\coth u)' = - \frac{1}{\sinh^2{u}} * u'$}{ ($u(x)^{v(x)})' = v(x) * u(x)^{v(x) - 1} * u'(x) + u(x)^{v(x)} * \ln u(x) * v'(x)$}
\end{multienumerate}

\subsubsection{Гиперболические функции}


\begin{multienumerate}
    \mitemxx{$\cosh u = \frac{e^u + e^{-u}}{2}$}{$\sinh u = \frac{e^u - u^{-u}}{2}$}
    \mitemxx{$\tanh u = \frac{\sinh u}{\cosh u}$}{$\coth u = \frac{\cosh u}{\sinh u}$}
\end{multienumerate}


\subsubsection{Уравнение гиперболы}

$$\frac{x^2}{a^2} - \frac{y^2}{b^2} = 1$$

\begin{equation}
    \begin{cases}
        x = a \coth t \\
        y = b \cosh t
    \end{cases}
\end{equation}

\subsubsection{Показательно-степенная функция}

Производную показательно-степенной функции можно найти следующим образом:

$(u^v)' = v * u^{v - 1} * u' + u^v \ln u * v'$

\subsubsection{Примеры}

\textbf{Пример 1.} $y = \tg 3x + 5x^2$, $y' = \frac{3}{\cos^{2} 3x} + 10x$

\hfill

\textbf{Пример 2.} $y = \cos (3x^2 + x)$, $y' = -\sin(3x^2 + x) * (6x + 1)$

\hfill

\textbf{Пример 3.} $y = x^3 * \cos x$, $y' = 3x^2 \cos x - x^3 \sin x$

\hfill

\textbf{Пример 4.} $y = \frac{x^2 + 1}{x^2 - 1}$, $y' = \frac{2x(x^2 - 1) - 2x(x^2 + 1)}{(x^2 - 1)^2} = \frac{-4x}{(x^2 - 1)^2}$

\hfill

\textbf{Пример 5.} $y = \ln (2x^2 + x - 1)$, $y' = \frac{1}{2x^2 + x - 1} * (4x + 1)$

\hfill

\textbf{Пример 6.} $y = \tg^3 (x + e^{-x^2})$, $y' = 2 \tg^3 (x + e^{-x^2}) * \frac{1}{\cos^2(x+e^{-x^2}} * (1 + e^{-x^2})$

\hfill

\textbf{Пример 7. } $y = (\cos x)^{x^2}$, $y' = x^2 (\cos x)^{x^2 - 1} * (-\sin x) + (\cos x)^{x^2} * \ln (\cos x) * 2x$

\hfill

\textbf{Пример 8.} $y = 2 \sqrt[3]{x} + \frac{3}{x^2}$, $y' = 2 * \frac{1}{3} * x^{\frac{1}{3} - 1} * x^{-1} = \frac{2}{3\sqrt[3]{x^2}} - \frac{6}{x^3}$

\hfill

\textbf{Пример 9.} $y = (x^2 + 5x + 7)^{8}$, $y' = 8(x^2 + 5x + 7)^{7} * (2x + 5)$

\end{flushleft}

\pagebreak
\section{Высшая математика - 26.10.2022}

\subsection{Производная функции, заданной параметрически}

\begin{flushleft}

\begin{equation}
    \begin{cases}
        x = x(t) \\
        y = y(t), y'_x = \frac{(y_t)'}{(x_t)'}, y''_{xx} = \frac{(y_t)''}{(x_t)'}
    \end{cases}
\end{equation}

Вторую производную функции, заданной параметрически, также можно найти следующим образом:

$y''_{x^2} = \frac{y''_{t^2} * x_t' - y'_t * x''_{t^2}}{(x_t')^3}$

\subsubsection{Примеры}

\textbf{Пример 1.}

\begin{equation}
    \begin{cases}
        x = \arctg t \\
        y = t^2 + 2
    \end{cases}
\end{equation}

$y = f'(x) = \frac{\diff y}{\diff x} = y_x' = \frac{2t}{\frac{1}{1 + t^2}} = 2t + 2t^3$

$y''_{xx} = \frac{2 + 6t^2}{\frac{1}{1 + t^2}} = 2(1 + 3t^2)(1 + t^2)$

\hfill

Решим ее другим способом: $y_t' = 2t$, $y''_{t^2} = 2$, $x'_t = \frac{1}{1 + t^2} = (1 + t^2)^{-1}$, $x''_{t^2} = -\frac{1 * 2t}{(1 + t^2)^2}$

$y''_{x^2} = \frac{2 * \frac{1}{1 + t^2} - 2t * (-\frac{2t}{(1+t^2)^2}}{(\frac{1}{1 + t^2})^3} = ... = 2(1 + 3t^2)(1 + t^2)$

\subsection{Производная обратной функции}

$y = f(x)$ и $x = \phi(y)$ - обратные, то $x'_y = \frac{1}{y'_{x}}$

\subsection{Производная функции, заданной неявно}

$F(x, y) = 0$ - функция, заданная неявно

Для нахождения $y'$ дифференцируем $F(x, y) = 0$, считая $x$ независимой переменной, а $y$ - функцией

\subsection{Уравнение касательной и нормали к графику}

Уравнение касательной к графику дифференцируемой функции $y = f(x)$ в точке $x_0$ выглядит следующим образом:

$y = f(x_0) + f'(x_0)(x - x_0)$

Уравнение нормали к графику:

\begin{equation}
    \begin{cases}
        y = f(x_0) - \frac{1}{f'(x_0)} * (x - x_0) \\
        x = x_0 \text{ при } f'(x_0) = 0
    \end{cases}
\end{equation}

\subsubsection{Примеры}

\textbf{Пример 1.} $\sqrt{x} + \sqrt{y} = 3$, т. $x_0$ $(x_0; y) = (1; 4)$  

\hfill

$y = f(x), \sqrt{x} + \sqrt{f(x)} = 3, (\sqrt{x} + \sqrt{f(x)})' = (3)', \frac{1}{2\sqrt{x}} + \frac{1 * f'(x)}{2\sqrt{f(x)}} = 0$

$f'(x) = \frac{\sqrt{f(x)}}{\sqrt{x}} = -\frac{\sqrt{f(x_0)}}{\sqrt{x}} = -\frac{\sqrt{y}}{\sqrt{x}} = -2$, $y = 0.5x + 3.5$ - уравнение нормали. 

\hfill

$y = 4 + (-2)(x - 1) = -2x + 6$ - уравнение касательной

\end{flushleft}

\pagebreak
\section{Высшая математика - 28.10.2022}

\subsection{Производная неявно заданной функции}

\begin{flushleft}

\textbf{Неявно заданная функция} - это функция, заданная неявно (очень полезное определение).

Например, $y = y(x)$ - явный вид; $y^2 + xy - \sin x = 0$, $F(x, y)$ - неявный вид.

\hfill

Берем производную всего выражения, при этом помним, что $y$ является функцией от $x$.

\subsubsection{Примеры}

\parbox{0.5\textwidth}{
\textbf{Пример 1.}

$y^2 + x y - \sin x = 0$

$2y * y'_x + 1 * y + x * y'_x - \cos x = 0$

$2y * y' + x y' = \cos x - y$

$y'(2y + x) = \cos x - y$

$y' = \frac{\cos x - y}{2y + x}$

\bigskip
\textbf{Пример 2.}

$2y * y'_x + 1 * y + x * y'_x - \cos x = 0$

$2y' * y' + 2y * y'' + y' * y' + xy'' + \sin x = 0$

$y''(2y + x) = -(\sin x + 3y')$

$y'' = -\frac{(\sin x + 3y')}{2y + x}$
}
\hfill
\parbox{0.4\textwidth}{
\textbf{Пример 3.}

$y * \tg x + x^3 y^2 - x^2 = 0$

$y' * \tg x + y * \frac{1}{\cos^2 x} + 3x^2 * y^2 + x^3 * 2y * y' - 2x = 0$

$y'(\tg x + 2x^3y) = 2x - 3 x^2 y^2 - \frac{y}{\cos^2 x}$

$y' = \frac{2x - 3 x^2 y^2 - \frac{y}{\cos^2 x}}{\tg x + 2 x^3 y}$
}

\subsection{Производная параметрически заданной функции}

\begin{equation}
    \begin{cases}
        x = x(t) \\
        y = y(t)
    \end{cases}
\end{equation}

$y'_x = \frac{y'_t}{x'_t}$ - производная функции, заданной параметрически

$y''_{xx} = (y'_x)' = \frac{(y'_x)'_t}{x'_t} = \frac{y''_{t t} * x'_t - y'_t * x''_{t t}}{(x'_t)^3}$ - вторая производная функции, заданной параметрически

\subsubsection{Примеры}

\textbf{Пример 1.}

\begin{equation}
    \begin{cases}
        x = 3 \cos t \\
        y = 4 \sin t
    \end{cases}
\end{equation}

\textbf{Editor note: добавить табличку с значениями $t$ и график}

\hfill

Эту же функцию можно задать неявно: $\frac{x^2}{9} + \frac{y^2}{16} = 1$

\hfill

\textbf{Пример 2.}

\begin{equation}
    \begin{cases}
        x = 3t \\
        y = 6t - t^2
    \end{cases}
\end{equation}

Выразим из уравнения $t$, получим: $y = 2x - \frac{x^2}{9}$ - уравнение параболы.

\hfill

$x'_t = 3$, $y'_t = 6 - 2t$, $y'_x = \frac{6 - 2t}{3} = 2 - \frac{2t}{3} = 2 - \frac{2x}{9}$

\hfill

\textbf{Пример 3.}

\begin{equation}
    \begin{cases}
        x = \cos t \\
        y = \sin t
    \end{cases}
\end{equation}

$y = \sqrt{1 - x^2} = \frac{-2x}{2\sqrt{1 - x^2}} = -\frac{x}{\sqrt{1 - x^2}}$

$y'_x = \frac{y'_t}{x'_t} = -\frac{\cos t}{\sin t} = -\frac{\cos t}{\sqrt{1 - \cos^2 t}}$

\subsection{Метод логарифмического дифференцирования}

\subsubsection{Примеры}

\textbf{Пример 1.}

Имеем функцию $y(x) = \frac{\sqrt{x + 1} * \sqrt[3]{2x + 5}}{(x^2 + 6)^5 (x - 4)^6}$.

Пользуясь свойствами логарифмов, максимально упростим данную запись:

\hfill

$\ln y = \ln \frac{\sqrt{} \sqrt[3]{}}{( )^5 ( )^6}$

$\ln y = \frac{1}{2} \ln (x + 1) + \frac{1}{3} \ln (2x + 5) - 5 \ln (x^2 + 6) - 6 \ln (x - 4)$

\hfill

Берем производную от обеих частей этого равенства, помня о том, что $\ln y$ является сложной функцией:

\hfill

$\frac{1}{y} * y'(x) = \frac{1}{2} \frac{1}{x + 1} + \frac{1}{3} \frac{2}{2x + 5} - 5 \frac{2x}{x^2 + 6} - 6 \frac{1}{x - 4}$

$y'(x) = y(...) = \frac{\sqrt{x + 1} \sqrt[3]{2x + 5}}{(x^2 + 6)^5 (x - 4)^6} (\frac{1}{2(x+1)} + \frac{2}{3(2x + 5)} - \frac{10x}{x^2 + 6} - \frac{6}{x - 4})$

\hfill

\textbf{Пример 2.}

Имеем функцию $y = x^{\tg x}$

\hfill

$\ln y(x) = \ln x^{\tg x}$

$\ln y(x) = \tg x * \ln x$

$\frac{1}{y} * y'(x) = \frac{1}{\cos^2 x} \ln x + \tg x * \frac{1}{x}$

$y'(x) = y(...) = x^{\tg x} (\frac{\ln x}{\cos^2 x} + \frac{\tg x}{x})$

\subsection{Производные и дифференциалы высших порядков}

Имеем функцию $y = f(x)$, определенную на интервале $[ a; b ]$.

Предполагаем, что ее производная не имеет никаких необычных свойств на данном отрезке: не имеет острых углов, разрывов и так далее.

В этом случае мы эту производную $y' = f'(x)$, если она дифференцируема на отрезке $[ a; b ]$, можем дифференцировать.

\hfill

\rule{\textwidth}{0.2pt}
\parbox{0.33\textwidth}{
$y = e^x$

$y' = e^x$

$y'' = e^x$

$y''' = e^x$

$y'''' = e^x$

$y''''' = e^x$

$y^{(n)} = e^x$
}
\parbox{0.33\textwidth}{
$y = \frac{1}{x}$

$y' = - \frac{1}{x^2}$

$y'' = \frac{2}{x^3}$

$y''' = - \frac{2 * 3}{x^4}$

$y'''' = \frac{2 * 3 * 4}{x^5}$

$y''''' = -\frac{2 * 3 * 4 * 5}{x^6}$

$y^{(n)} = (-1)^n \frac{n!}{x^{n+1}}$
}
\parbox{0.3\textwidth}{
$y = \sin x$

$y' = \cos x$

$y'' = - \sin x$

$y''' = - \cos x$

$y'''' = \sin x$

$y''''' = \cos x$

$y^{(n)} = \sin (x + \frac{\pi n}{2})$
}

\rule{\textwidth}{0.2pt}

\hfill

Работая с производными и дифференциалами высших порядков, следует пользоваться следующими свойствами:

\begin{multienumerate}
    \mitemxx{$(u \pm v)^{(n)} = u^{(n)} \pm v^{(n)}$}{$(C u)^{(n)} = C u^{(n)}$}
    \mitemx{$(u v)^{n} = u^{(n)} v + n u^{(n - 1)} + \frac{n (n - 1)}{2!} u^{(n - 2)} v'' + ... + n u' v^{(n - 1)} + u v^{(n)} = \sum^{n}_{k = 1} = C_{n}^{k} u^{(k)} v^{(n - k)}$, где $C_{n}^{k} = \frac{n!}{k!(n-k)!}$ - формула Лейбница}
\end{multienumerate}

\subsubsection{Примеры}

\textbf{Пример 1.}

Найти производную 10-го порядка функции $f(x) = (3x^2 + 2x + 1) \sin x$

\hfill

\rule{\textwidth}{0.2pt}

\parbox{0.43\textwidth}{
$u = \sin x$

$u' = \cos x$   
}
\parbox{0.45\textwidth}{
$v = 3x^2 + 2x + 1$

$v' = 6x + 2$

$v'' = 6$
}

\rule{\textwidth}{0.2pt}

\hfill

$f'(x) = (\sin x)^{(10)} (3x + 2x + 1) + 10 (\sin x)^{(9)} (6x + 2) + \frac{10 * 9}{2} (\sin x)^{(8)} * 6$

\hfill

$u^{(8)} = \sin x$, $u^{(9)} = \cos x$, $u^{10} = (-\sin x)$

\hfill

$f'(x) = (-\sin x) (3x + 2x + 1) + 10 (\cos x) (6x + 2) + \frac{10 * 9}{2} (\sin x) * 6$

\subsection{Дифференциалы высших порядков}

$\diff (\diff y) = \diff (f'(x) \diff x) = (f' (x) \diff x)' \diff x = \diff^2 f = f''(x) \diff x^2$ - дифференциал второго порядка

$\diff^{n} f = f^{(n)} (x) \diff x^{(n)}$

\hfill

Если для дифференциала первого порядка можно говорить об инвариантности формы, то для дифференциалов высших порядков инвариантности формы нет, и вышеизложенные равенства верны только если $x$ - независимая переменная.

\end{flushleft}

\end{document}