\documentclass{article}
\usepackage[utf8]{inputenc}

\usepackage[T2A]{fontenc}
\usepackage[utf8]{inputenc}
\usepackage[russian]{babel}

\usepackage{amsmath}

\title{Высшая математика}
\author{Лисид Лаконский}
\date{October 2022}

\begin{document}

\maketitle

\tableofcontents
\pagebreak

\section{Высшая математика - 03.10.2022}

\subsection{Предел функции}

\begin{enumerate}
    \item Любую константу мы можем вынести за предел
    \item Предел от суммы двух функций $f(x) + g(x)$ дает в нам результате разложения сумму двух пределов
    \item Предел от произведения двух функций разлагается на произведение двух пределов
    \item Предел частного от двух функций ($g(x) \ne 0$) равен частному двух пределов, если нет неопределенности
\end{enumerate}

\subsection{Виды неопределенностей}

Неопределенности бывают следующие: $\frac{\inf}{\inf}$, $\frac{0}{0}$, $\frac{\inf}{-\inf}$, $\frac{0}{\inf}$, $1^{\inf}$, $0^{0}$, $\inf^{0}$

\subsubsection{Неопределенность типа $\frac{0}{0}$}

$\lim\limits_{x \to a} \frac{f(x)}{g(x)}$

Пусть $f(x)$ and $g(x)$ are многочлены, $k_1$, $k_2 \ge 1$.

$f(x) = (x - a)^{k_1} f_{1}(x)$

$g(x) = (x-a)^{k_2} f_{1}(x)$

$\lim\limits_{x \to a} \frac{f(x)}{g(x)} = \lim{x \to a} \frac{(x- a)^{k_1} f_{1}(x)}{(x-a)^{k_2}g_{1}(x)} = \lim{x \to a} (x - a)^{k_1 - k_2} * \lim{x \to a} \frac{f_{1}(x)}{g_{1}(x)}$

Результат выражения выше равен $0$ при $k_1 > k_2$, $A$ при $k_1 = k_2$ и $\inf$ иначе.  

\subsubsection{Неопределенность вида $\frac{\inf}{\inf}$}

В числителе и знаменателе многочлены, пределы которых стремятся к $\inf$.

Если $a = \inf$, тогда предел будет $\lim\limits_{x \to \inf} \frac{a_{x}x^{m} + ... + a_0}{b_{n}x^{n} + ... + b_0}$ равен нулю при $m < n$, $\frac{a^m}{b^n}$при $m = n$, иначе $\inf$

\subsubsection{Неопределенность вида $\frac{0}{0}, \frac{\inf}{\inf}$}

$\lim\limits_{x \to 0} \frac{\sin{x}}{x} = 1$, т.к. при $x \to 0$ обе эти функции являюстя бесконечно малыми, отношение эквивалентных велчиин дает $1$

Если предел функции $f(x)$ при $x \to a$ равен нулю, то функция $f(x)$ называется бесконечно малой величиной в окресности точки $a$

Две бесконечно малые величины $f(x), g(x)$ называются эквивалетнтными бесконечно-малыми величинами в окрестности точки $a$, если предел их отношения равен единице

Пример:

$$
\lim\limits_{x \to 0} \frac{1 - \cos{2x}}{\cos 7x - \cos 3x} = \lim\limits_{x \to 0} \frac{1 - \cos{2x}}{2 * \sin{\frac{10x}{2}} * \sin{\frac{-5x}{2}}} = \frac{sin^2{x}}{2 * \sin{5x} * \sin{2x}} = \frac{sin^2{x}}{2 * \sin{5x} * \sin{2x}} = \frac{1}{10}
$$

\subsubsection{Неопределенность вида $1^{\inf}$}

$$\lim\limits_{x \to 0} (1 + x)^{\frac{1}{x}} = e$$

Пример:

$$
\lim\limits_{x \to \frac{\pi}{2}} (\sin x)^{\tg x}
$$

\pagebreak

\section{Высшая математика - 12.10.2022}

\subsection{Непрерывность функции}

\textbf{Опр. 1.} Функция $y = f(x)$ называется непрерывной в точке $x_0$, если $f(x)$ определена в некоторой окрестности точки $x_0$ и существует предел этой функции при $x$, стремящемся к $x_0$, равный $f(x_0)$.

\subsubsection{Свойства непрерывных функций}

Пусть $f(x)$ и $g(x)$ - непрерывные в точке $x_0$ функции, тогда:

\begin{enumerate}
    \item Функция, полученная в результате сложения и вычитания двух непрерывных в данной точке функций также будет непрерывна в рассматриваемой точке $x_0$
    \item Функция, которая стала результатом произведения двух непрерывных функций, тоже будет непрерывна в точке $x_0$
    \item Функция $\frac{f(x)}{g(x)}$ будет непрерывна в точке $x_0$, если $g(x) \ne 0$
    \item Для того, чтобы $g = f(x)$ была непрерывна в точке $x_0$, необходимо и достаточно, чтобы $\lim_{\Delta x \to 0} \Delta y = 0, \Delta y = f(x_0 - \Delta x) - f(x_0)$
    \item Основные элементарные функции: $a^x, x^a, \log_{a} x, \sin x, \cos x, \tan x, \cot x, \arctan x, \arcsin x, ...$ непрерывны на всей области определения
    \item Пусть $y=f(x)$ непрерывна на $[a; b]$, и на концах этого отрезка принимает значения разных знаков, тогда между точками $a$ и $b$ находится хотя бы одна т. $x = c$, при которой $f(c) = 0$, $a < c < b$
\end{enumerate}

\subsubsection{Пример}

$x^3 + x^2 + x - 1 = 0, x_0 = c, (a, b) = (\frac{1}{2}; 1)$

$f(\frac{1}{2}) = -\frac{1}{8}, f(1) = 2$, следовательно $\exists x_0 = c, f(c) = 0, \frac{1}{2} < c < 1$

\subsection{Точки разрыва функции}

\textbf{Опр. 2.} Точка $x_0 \in R$ называется точкой разрыва функции $f(x)$, определенной в некоторой окрестности точки $x_0$, кроме, может быть, самого $x_0$, если равенство $\lim_{x \to x_0} f(x) \ne f(x_0)$

То есть, либо $x_0 \notin D_f$ и значение $f(x_0)$ не определено, либо $\lim_{x \to x_0} f(x)$ не существует, либо обе части равенства определены, но не равны между собой.

\subsubsection{Типы точек разрыва}

\textbf{1.} $x_0$ - точка разрыва 1-го рода, если существуют конечные односторонние пределы $f(x_0 - 0) = \lim\limits_{x \to x_0 - 0} f(x), f(x_0 + 0) = \lim\limits_{x \to x_0 + 0} f(x)$

Если $\lim\limits_{x \to x_0 + 0} f(x) = \lim\limits_{x \to x_0 - 0}$, то $x_0$ - устранимая точка разрыва первого рода

\textbf{2.} $x_0$ - точка разрыва второго рода, если выполнено хотя бы одно из условий: $\lim\limits_{x \to x_0} f(x) = \pm \inf, \lim\limits_{x \to x_0 + 0} = \pm \inf, \lim\limits_{x \to x_0-0} f(x) = \pm \inf$

\subsubsection{Первый пример}

$f(x) = \frac{x}{\sin x}$, так как результат частного двух простых функций, то она непрерывна при $\sin x \ne 0$, то есть точками разрыва являются нули функции $\sin x$: $x = \pi k, k \in Z$

При $x = 0$: $\lim\limits_{x \to 0} \frac{x}{\sin x} = 1$, $f(0)$ не существует, следовательно функция сама по себе в этой точке не непрерывна.

Рассмотрим два конечных односторонних предела, $\lim\limits_{x \to 0 + 0} \frac{x}{\sin x} = 1, \lim\limits_{x \to 0 - 0} \frac{x}{\sin x} = 1$

Односторонние разрывы равны между собой, следовательно, $x = 0$ - устранимая точка разрыва первого рода.

При $x = \pi$: $\lim\limits_{x \to \pi} \frac{x}{\sin x} = \frac{\pi}{0} = \inf$, $f(\pi)$ не существует

\subsubsection{Второй пример}

\begin{equation}
    \begin{cases}
        x^2 + 1, x \le 0 \\
        x + 1, 0 < x \le 1 \\
        2x - 1, x > 1
    \end{cases}
\end{equation}

Рассмотрим первый случай, $x = 0$: $\lim\limits_{x \to 0 - 0} (x^2 + 1) = 1, \lim\limits_{x \to 0 + 0} (x + 1) = 1, y(0) = 1$, таким образом точка $x = 0$ - точка непрерывности нашей функции, разрыва нет.

Рассмотрим второй случай, $x = 1$: $\lim\limits_{x \to 1 - 0} (2x - 1) = 1, \lim\limits_{x \to 1 + 0} (x + 1) = 2$, таким образом точка $x = 1$ - неустранимая точка разрыва первого рода.

\subsubsection{Третий пример}

Исследовать точки $x = 3, x = 1$ функции $y = 4^\frac{1}{x - 1}$

\textbf{1)} $x = 3, \lim\limits_{x \to 3} 4^\frac{1}{x - 1} = 2 = y(3)$, следовательно данная точка - точка непрерывности данной функции

\textbf{2)} $x = 1, \lim\limits_{x \to 1} 4^\frac{1}{x - 1} = \inf$, следовательно данная точка - точка разрыва второго рода.

$\lim\limits_{x \to 1 - 0} 4^{\frac{1}{x - 1}} = 4^{-\inf} = \frac{1}{4^{\inf}} = 0, \lim\limits_{x \to 1 + 0} 4^\frac{1}{x - 1} = +\inf$

\end{document}
