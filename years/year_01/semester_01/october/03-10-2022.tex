\documentclass{article}
\usepackage[utf8]{inputenc}

\usepackage[T2A]{fontenc}
\usepackage[utf8]{inputenc}
\usepackage[russian]{babel}

\usepackage{amsmath}

\title{Высшая математика}
\author{Лисид Лаконский}
\date{October 2022}

\begin{document}

\maketitle

\tableofcontents
\pagebreak

\section{Высшая математика - 03.10.2022}

\subsection{Предел функции}

\begin{enumerate}
    \item Любую константу мы можем вынести за предел
    \item Предел от суммы двух функций $f(x) + g(x)$ дает в нам результате разложения сумму двух пределов
    \item Предел от произведения двух функций разлагается на произведение двух пределов
    \item Предел частного от двух функций ($g(x) \ne 0$) равен частному двух пределов, если нет неопределенности
\end{enumerate}

\subsection{Виды неопределенностей}

Неопределенности бывают следующие: $\frac{\inf}{\inf}$, $\frac{0}{0}$, $\frac{\inf}{-\inf}$, $\frac{0}{\inf}$, $1^{\inf}$, $0^{0}$, $\inf^{0}$

\subsubsection{Неопределенность типа $\frac{0}{0}$}

$\lim\limits_{x \to a} \frac{f(x)}{g(x)}$

Пусть $f(x)$ and $g(x)$ are многочлены, $k_1$, $k_2 \ge 1$.

$f(x) = (x - a)^{k_1} f_{1}(x)$

$g(x) = (x-a)^{k_2} f_{1}(x)$

$\lim\limits_{x \to a} \frac{f(x)}{g(x)} = \lim{x \to a} \frac{(x- a)^{k_1} f_{1}(x)}{(x-a)^{k_2}g_{1}(x)} = \lim{x \to a} (x - a)^{k_1 - k_2} * \lim{x \to a} \frac{f_{1}(x)}{g_{1}(x)}$

Результат выражения выше равен $0$ при $k_1 > k_2$, $A$ при $k_1 = k_2$ и $\inf$ иначе.  

\subsubsection{Неопределенность вида $\frac{\inf}{\inf}$}

В числителе и знаменателе многочлены, пределы которых стремятся к $\inf$.

Если $a = \inf$, тогда предел будет $\lim\limits_{x \to \inf} \frac{a_{x}x^{m} + ... + a_0}{b_{n}x^{n} + ... + b_0}$ равен нулю при $m < n$, $\frac{a^m}{b^n}$при $m = n$, иначе $\inf$

\subsubsection{Неопределенность вида $\frac{0}{0}, \frac{\inf}{\inf}$}

$\lim\limits_{x \to 0} \frac{\sin{x}}{x} = 1$, т.к. при $x \to 0$ обе эти функции являюстя бесконечно малыми, отношение эквивалентных велчиин дает $1$

Если предел функции $f(x)$ при $x \to a$ равен нулю, то функция $f(x)$ называется бесконечно малой величиной в окресности точки $a$

Две бесконечно малые величины $f(x), g(x)$ называются эквивалетнтными бесконечно-малыми величинами в окрестности точки $a$, если предел их отношения равен единице

Пример:

$$
\lim\limits_{x \to 0} \frac{1 - \cos{2x}}{\cos 7x - \cos 3x} = \lim\limits_{x \to 0} \frac{1 - \cos{2x}}{2 * \sin{\frac{10x}{2}} * \sin{\frac{-5x}{2}}} = \frac{sin^2{x}}{2 * \sin{5x} * \sin{2x}} = \frac{sin^2{x}}{2 * \sin{5x} * \sin{2x}} = \frac{1}{10}
$$

\subsubsection{Неопределенность вида $1^{\inf}$}

$$\lim\limits_{x \to 0} (1 + x)^{\frac{1}{x}} = e$$

Пример:

$$
\lim\limits_{x \to \frac{\pi}{2}} (\sin x)^{\tg x}
$$

\end{document}
