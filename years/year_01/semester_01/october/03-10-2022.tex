\documentclass{article}
\usepackage[utf8]{inputenc}

\usepackage[T2A]{fontenc}
\usepackage[utf8]{inputenc}
\usepackage[russian]{babel}

\usepackage{amsmath}

\usepackage{hyperref}
\hypersetup{
    colorlinks, citecolor=black, filecolor=black, linkcolor=black, urlcolor=black
}

\title{Высшая математика}
\author{Лисид Лаконский}
\date{October 2022}

\begin{document}

\maketitle

\tableofcontents
\pagebreak

\section{Высшая математика - 03.10.2022}

\subsection{Предел функции}

\begin{enumerate}
    \item Любую константу мы можем вынести за предел
    \item Предел от суммы двух функций $f(x) + g(x)$ дает в нам результате разложения сумму двух пределов
    \item Предел от произведения двух функций разлагается на произведение двух пределов
    \item Предел частного от двух функций ($g(x) \ne 0$) равен частному двух пределов, если нет неопределенности
\end{enumerate}

\subsection{Виды неопределенностей}

\begin{flushleft}

Основные виды неопределенностей: $\frac{0}{0}$, $\frac{\infty}{\infty}$, $(0 * \infty)$, ($\infty - \infty)$, $1^{\infty}$, $0^0$, $\infty^0$

Раскрывать неопределенности позволяет:

\begin{enumerate}
    \item Упрощение вида функции (преобразование выражения с использованием формул сокращенного умножения, тригонометрических формул, домножения на сопряженные выражения с последующим сокращением и тому подобное)
    \item Использование замечательных пределов
    \item Применение правила Лопиталя
    \item Использование замены бесконечно малого выражения ему эквивалентным
\end{enumerate}

\subsubsection{Неопределенность вида $\frac{0}{0}$}

Пробуем преобразовать и упростить выражение. Если есть выражение вида $\frac{\sin (kx)}{kx}$ или $\frac{kx}{\sin (kx)}$, то применяем первый замечательный предел. Если не помогает, то используем правило Лопиталя или таблицу эквивалентных бесконечно малых.

Правила раскрытия неопределенности:

\begin{enumerate}
    \item Для того, чтобы определить предел дробно-рациональной функции ($\lim\limits{x \to a} f(x)$), надо числитель и знаменатель дроби разделить на $x - a$ и перейти к пределу. Если и после этого числитель и знаменатель новой дроби имеют пределы, равные нулю, то надо произвести повторное деление на $x - a$
    \item Для того, чтобы определить предел, в котором числитель или знаменатель иррациональны, следует избавиться от иррациональности, умножив и числитель и знаменатель дроби на одно и то же выражение, приводящее к формулам сокращенного умножения. Неопределенность устраняется после сокращения дроби.
\end{enumerate}

В случае, когда под знаком предела стоят тригонометрические функции, используется первый замечательный предел: $\lim\limits_{x \to 0} \frac{\sin x}{x} = 1$

Его различные формы: $\lim\limits_{x \to 0} \frac{x}{\sin x} = 1$, $\lim\limits_{x \to 0} \frac{\tg x}{x} = 1$, $\lim\limits_{x \to 0} \frac{x}{\tg x} = 1$, $\lim\limits_{x \to 0} \frac{\arcsin x}{x} = 1$, $\lim\limits_{x \to 0} \frac{x}{\arcsin x} = 1$, $\lim\limits_{x \to 0} \frac{\arctg x}{x} = 1$, $\lim\limits_{x \to 0} \frac{x}{\arctg x} = 1$

\subsubsection{Неопределенность вида $\frac{\infty}{\infty}$}

Правила раскрытия неопределенности:

\begin{enumerate}
    \item Чтобы раскрыть неопределенность вида $\frac{\infty}{\infty}$ заданную отношением двух многочленов, надо и числитель и знаменатель почленно разделить на переменную величину в наибольшей степени.
    \item Для раскрытия неопределенности вида $\frac{\infty}{\infty}$, заданную отношением иррациональных функций, надо и числитель и знаменатель почленно разделить на переменную величину в наибольшей степени с учетом степеней корней.
\end{enumerate}

Если не помогает, то используем правило Лопиталя.

$
\begin{aligned}
\lim\limits{x \to \infty} \frac{a_n x^n + a_{n - 1} x^{n - 1} + \dots + a_1 x + a_0}{b_m x^m + b^{m - 1} x^{m - 1} + \dots + b_1 x + b_0} = \begin{cases}
    0, n < m \\
    \frac{a_n}{b_m}, n = m
    \infty, n > m
\end{cases}
\end{aligned}
$

\paragraph{Пример №1} Найти предел $\lim\limits_{x \to \infty} \frac{3x^3 - x^2 + 14}{x^2 - 4}$

$\lim\limits_{x \to \infty} \frac{3x^3 - x^2 + 14}{x^2 - 4} = \{ \frac{\infty}{\infty} \} = \infty$, так как $n = 3$, $m = 2$, $n > m$

\subsubsection{Неопределенность вида $(0 * \infty)$ или $(\infty - \infty)$}

Преобразуем неопределенность к виду $\frac{0}{0}$ или $\frac{\infty}{\infty}$, затем разбираемся с новой неопределенностью.

\hfill

Пусть $\lim\limits{x \to a} f(x) = 0$, $\lim\limits_{x \to a} g(x) = \infty$, тогда $\lim\limits{x \to a} f(x) g(x) = \{ 0 * \infty \} = \begin{cases}
    \lim\limits_{x \to a} \frac{f(x)}{\frac{1}{g(x)}} = \{ \frac{0}{0} \} \\
    \textbf{или} \\
    \lim\limits_{x \to a} \frac{g(x)}{\frac{1}{f(x)}} = \{ \frac{\infty}{\infty} \}
\end{cases}$

\hfill

Неопределенность вида $(\infty - \infty)$, получающаяся в результате алгебраической суммы двух дробей, устраняется или сводится к неопределенности вида $\frac{0}{0}$ путем приведения дроби к общему знаменателю.

Пусть $\lim\limits_{x \to a} f(x) = \infty$, $\lim\limits_{x \to a} g(x) = \infty$, тогда $\lim\limits_{x \to a} (f(x) - g(x)) = \{ \infty - \infty \} = \lim\limits_{x \to a} \frac{1}{\frac{1}{f(x)}} - \frac{1}{\frac{1}{g(x)}} = \lim\limits_{x \to a} \frac{\frac{1}{g(x)} - \frac{1}{f(x)}}{\frac{1}{f(x)} * \frac{1}{g(x)}} = \{ \frac{0}{0} \}$

\hfill

Неопределенность вида $(\infty - \infty)$, получающаяся в результате алгебраической суммы иррациональных выражений, устраняется или сводится к неопределенности вида $\frac{\infty}{\infty}$ путем домножения и деления на одно и то же выражение, приводящее к формулам сокращенного умножения. В случае квадратных корней разность домножается на сопряженное выражение и применяются формулы сокращенного умножения.

\subsubsection{Неопределенность вида $1^{\infty}$}

Применяем второй замечательный предел: $\lim\limits_{x \to \infty} (1 + \frac{1}{x})^x = e$

Его различные формы: $\lim\limits_{x \to 0} (1 + x)^{\frac{1}{x}} = e$, $\lim\limits_{x \to 0} \frac{\ln (1 + x)}{x} = \{ \frac{0}{0} \} = 1$, $\lim\limits_{x \to 0} \frac{a^x - 1}{x} = \{ \frac{0}{0} \} = \ln a$, $\lim\limits_{x \to 0} \frac{e^x - 1}{x} = \{ \frac{0}{0} \} = 1$, $\lim\limits_{x \to 0} \frac{(1 + x)^p - 1}{x} = \{ \frac{0}{0} \} = p$

\subsubsection{Неопределенность вида $0^0$ или $\infty^0$}

Логарифмируем выражение и используем равенство $\lim\limits_{x \to x_0} ln(f(x)) = \ln (\lim\limits_{x \to x_0} f(x))$

\end{flushleft}

\end{document}
