\documentclass{article}
\usepackage[utf8]{inputenc}

\usepackage[T2A]{fontenc}
\usepackage[utf8]{inputenc}
\usepackage[russian]{babel}

\usepackage{amsmath}
\usepackage{pgfplots}
\newcommand*\diff{\mathop{}\!\mathrm{d}}

\title{Высшая математика}
\author{Лисид Лаконский}
\date{October 2022}

\begin{document}

\maketitle

\tableofcontents
\pagebreak

\section{Высшая математика - 14.10.2022}

\subsection{Бесконечно большие и бесконечно малые функции}

\begin{flushleft}

Функция называется бесконечно малой при $x \to x_0$, если $\lim\limits_{x \to x_0} f(x) = 0$.

Функция называется бесконечно большой при $x \to x_0$, если $\lim\limits_{x \to x_0} f(x) = \inf$.

\textbf{Теорема 1.} $\alpha + \beta, \alpha - \beta$ - бесконечно малые, если $\alpha, \beta$ - бесконечно малые

\textbf{Теорема 2.} Произведение бесконечно малой на ограниченую функцию является бесконечно малой

\textbf{Определение.} Если $\alpha(x), \beta(x)$ бесконечно малы при $x \to x_0$, то $\exists \lim\limits_{x \to x_0} \frac{\alpha(x)}{\beta(x)} = $ const $ \ne 0 \ne \pm \inf$, то $\alpha$ и $\beta$ - бесконечно малые одного порядка

Если $\lim\limits_{x \to x_0} \frac{\alpha(x)}{\beta(x)} = 1$, то $\alpha, \beta$ - эквивалентные бесконечно малые

\hfill

Если $\lim\limits_{x \to x_0} \frac{\alpha(x)}{\beta(x)} = 0$, то $\alpha$ - бесконечно малое более высокого порядка малости по сравнению с $\beta$.

Если, наоборот, $\lim\limits_{x \to x_0} \frac{\alpha(x)}{\beta(x)} = \inf$, то говорят, что $\beta$ более высокого порядка малости, чем $\alpha$.

Например, $\alpha = x^3 + 2x^2, \beta = 2x + 3x^2, \lim\limits_{x \to 0} \frac{x^3 + 2x^2}{3x^2 + 2x} = \lim\limits_{x \ to 0} \frac{x^2(x+2)}{x(3x + 2)} = \lim\limits_{x \to 0} \frac{x + 2}{3x + 2}$

\hfill

$\exists \lim\limits_{x \to x_0} \frac{\beta(x)}{\alpha^k(x)} = C \ne 0 \ne \pm \inf$, то $\beta, \alpha^k$ - бесконечно малые одного порядка.

Например, $\alpha = sin^3 x, \beta = x, \lim\limits_{x \to 0} \frac{\sin^3 x}{x^3} = \lim (\frac{\sin x}{x})^3 = 1 \ne 0 \ne \pm \inf$, $\sin^3 x$ величина такого же порядка малости, как $x^3$.

\subsubsection{Применение бесконечно малых к вычислению пределов}

Если при $x \to x_0$  $f(x) \sim f_1(x), g(x) \sim g_1(x)$, то $\exists \lim\limits_{x \to x_0} \frac{f_1(x)}{g_1(x)} \implies \exists \lim\limits_{x \to x_0} \frac{f(x)}{g(x)} = \lim\limits_{x \to x_0} \frac{f_1(x)}{g_1(x)}$ 

\hfill

Например, $\lim\limits_{x \to 0} \frac{\sin^3 (4x)}{x^2 + 3x} = ...$

Допустим, $\lim\limits_{x \to 0} \frac{x^2 + 3x}{x} = \lim (x+3) = 3$.

Допустим, $\lim\limits_{x \to 0} \frac{x^2 + 3x}{x^2} = \lim\limits_{x \to 0} \frac{x(x + 3)}{x} = \inf$.

Посчитали без толку, теперь продолжим, $... = \lim\limits_{x \to 0} \frac{(4x)^3}{x(x + 3)} = \lim\limits_{x \to 0} \frac{64x^2}{x + 3} = 0$.

\subsubsection{Таблица эквивалентных бесконечно малых}

$\sin x \sim x, \tg x \sim x, \arcsin x \sim x, \arctg x \sim x, \ln (x + 1) \sim x, e^x - 1 \sim x, a^x - 1 \sim x \ln a, \sqrt[n]{1 + x} - 1 \sim \frac{x}{n}, 1 - \cos x \sim \frac{x^2}{2}, \cos x ~ 1 - \frac{x^2}{2}$

Это все подходит к умножению или делению, но никак не к сложению или вычитанию.

\subsubsection{Некоторые соображения и примеры}

При $x \to \inf$ $f(x) = x^3 + 2x + 1$ больший вклад вносит $x^3$, при $x \to 0$ $f(x) = x^3 + x^2$ больший вклад вносит $x^2$

\hfill 

\textbf{Пример.} $\lim\limits_{x \to 0} \frac{\sqrt[3]{1 + 4x} - 1}{\ln (1 + 2x)} = \lim\limits_{x \to 0} \frac{\frac{4x}{3}}{2x} = \frac{2}{3}$ - применение таблицы эквивалентных бесконечно малых.

\hfill

$O(x)$ - бесконечно малая более высокая порядка малости.

\subsection{Производные и дифференциалы функции}

Тут есть рисунок, который мне тяжело воспроизвести. Поэтому его тут нет. Но на нем показаны $\Delta x$ (приращение аргумента), $\Delta f$ (приращение функции), касательная к функции.

$\diff f = f' (x) \diff x$ - дифференциал функции, $\Delta f = \diff f + O(\Delta x)$

\hfill

Производной функции называется $\lim\limits_{\Delta x \to 0} \frac{\Delta f(x)}{\Delta x} = \lim\limits_{\Delta x \to 0} \frac{f(x + \Delta x) - f(x)}{\Delta x}$. Производная равна пределу приращения функции к приращению аргумента, когда приращение аргумента стремится к нулю.

\hfill

\textbf{Пример.} Пусть у нас $y = x^3 + 2x - 1$. Попробуем вычислить производную.

$y(x + \Delta x) = (x + \Delta x)^3 + 2(x + \Delta x) - 1 = x^3 + 3x^2 * \Delta x + 3x(\Delta x)^2 + (\Delta x)^3 + 2x + 2\Delta x - 1$

$y'(x) = \lim\limits_{\Delta x \to 0} \frac{x^3 + 3x^2 * \Delta x + 3x (\Delta x)^2 + (\Delta x)^3 + 2x + 2\Delta x - 1 - x^2 - 3x + 1}{\Delta x} = \lim\limits_{\Delta x \to 0} (3x^2 + 3x(\Delta x) + (\Delta x)^2 + 2) = 3x^2 + 2$

Поздравляю, вы написали такую простыню. Вы великолепны.

\hfill

\textbf{Другой пример.} Попробуем доказать, что производная $y'(\sin x) = \cos x$

$y(x + \Delta x) = \sin(x + \Delta x)$

$y'(x) = \lim\limits_{\Delta x \to 0} \frac{\sin(x + \Delta x) - \sin x}{\Delta x} = \text{вспоминайте формулы} = ... = \lim\limits_{\Delta x \to 0} \frac{2 \sin \frac{x + \Delta x - x}{2} \cos \frac{x + \Delta x + x}{2}}{\Delta x} = \lim\limits_{x \to 0} \frac{2 \sin \frac{\Delta x}{2} \cos (x + \frac{\Delta x}{2})}{\Delta x} = \lim\limits_{x \to 0} \frac{2 \cos x}{2} = \cos x$

\subsubsection{Свойства производных}

\begin{enumerate}
    \item $(u \pm v)' = u' \pm v'$
    \item $(u * v)' = u' v + v u'$, $(u * v * w)' = u' v w + u * v' w + u * v * w'$
    \item $(cu)' = c u'$
    \item $(\frac{u}{v})' = \frac{u' v - u v'}{v^2}$
\end{enumerate}

\subsubsection{Дифференцируемость функций}

Если функция $y = f(x)$ имеет производную в точке $x_0$, то есть $\exists \lim\limits_{x \to x_0} \frac{\Delta f(x)}{\Delta x} = \lim\limits_{x \to x_0} \frac{f(x_0 + \Delta x) - f(x_0)}{\Delta x}$, то функция дифференцируема в точке $x_0$.

\hfill

\textbf{Теорема.} Если функций $y = f(x)$ дифференцируема в точке $x_0$, то она в этой точке непрерывна.

\textbf{Замечание.} Обратное высказывание может быть и неверным.

\hfill

Пример функции непрерывной в какой-то точке, но не дифференцируемой в ней.

\begin{tikzpicture}
  \begin{axis}[ 
    xlabel=$x$,
    ylabel={$|x|$}
  ] 
    \addplot { abs(x) }; 
  \end{axis}
\end{tikzpicture}

\subsubsection{Геометрический смысл производной}

Производная - это тангенс угла наклона касательной...

$\tg \alpha = \lim\limits_{\Delta x \to 0} \frac{\Delta y}{\Delta x}$

\subsubsection{Уравнение касательной и нормали к графику функции}

Пусть у нас есть $y = kx + b$, дана какая-то точка $M_0(x_0; y_0)$

\hfill

\textbf{Уравнение касательной.} $y - y_0 = y'(x_0)(x - x_0)$

\hfill

\textbf{Уравнение нормали.} $y - y_0 = -\frac{1}{y'(x_0)} (x - x_0)$

\subsubsection{Производная сложной функции}

Пусть функция $u = u(x)$ имеет в некоторой точке производную $u'_x(x)$, а функция $y = y(u)$ имеет при соответствующем значении $u$ производную $y'_u$.

Тогда сложная функция $y(x) = y(u(x))$ имеет проивзодную $y'_x = y'_u * u'_x$

\hfill

\textbf{Пример 0.} Например, у нас есть $y(x) = y(g(f(x)))$, то $y'_x = y'_g * g'_f * f'_x$

\hfill

\textbf{Пример 1.} $y = 2x^2 + 3x, y' = 4x + 3$

\hfill

\textbf{Пример 2.} $y = \cos (2x^2 + 3x), y' = sin (2x^2 + 3x) * (4x + 3)$

\hfill

\textbf{Пример 3.} $y = \sqrt{\cos{(2x^2 + 3x}}, y' = \frac{1}{2\sqrt{\cos{2x^2 + 3x}}} * (-\sin (2x^2 + 3x)) * (4x + 3)$

\hfill

\textbf{Пример 4.} $y = \tg \sqrt{\cos (2x^2 + 3x)}, y' = \frac{1}{\cos^2 (\sqrt{\cos (2x^2 + 3x)}} * \frac{1}{2 \sqrt{\cos (2x^2 + 3x)}} * (-\sin (2x^2 + 3x)) * (4x + 3)$

\subsubsection{Обратная функция и ее производная}

Пусть у нас есть функция $y = f(x)$, $x = a, x = b$, а $y(a) = c, y(b) = d$, где $[a; b]$ - область определения,  $[c; d]$ - область изменения функции.

\hfill

\textbf{Теорема.} Если для $y = f(x)$ существует обратная функция $x = \phi (y)$, у которой $\phi'(y) \ne 0$ в некоторой точке $y_0$, то $f'(x) = \frac{1}{\phi'(y)}$

\hfill

\textbf{Пример 1.} $y = \arcsin x$, функция обратная к ней $x = \sin y, x' = cos y$.

Таким образом, $(\arcsin x)' = \frac{1}{\cos x} = \frac{1}{\sqrt{1 - \sin^2 y}} = \frac{1}{\sqrt{1 - \sin^2(\arcsin x)}} = \frac{1}{\sqrt{1 - x^2}}$

\end{flushleft}

\end{document}