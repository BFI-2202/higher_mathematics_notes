\documentclass{article}
\usepackage[utf8]{inputenc}

\usepackage[T2A]{fontenc}
\usepackage[utf8]{inputenc}
\usepackage[russian]{babel}

\usepackage{amsmath}
\usepackage{pgfplots}
\usepackage{multienum}
\newcommand*\diff{\mathop{}\!\mathrm{d}}

\title{Высшая математика}
\author{Лисид Лаконский}
\date{October 2022}

\begin{document}

\maketitle

\tableofcontents
\pagebreak

\section{Высшая математика - 18.10.2022}

\subsection{Асимптоты функции}

\begin{flushleft}

Асимптоты функции могут быть:
\begin{itemize}
    \item Вертикальные
    \item Наклонные (в том числе горизонтальные)
\end{itemize}

\subsubsection{Вертикальные асимптоты}

Если функция $f(x)$ имеет точку разрыва, в которой хотя бы один односторонний предел бесконечен, то вертикальная прямая, параллельная оси ординат, проходящая через эту точку, называется \textbf{вертикальной асимптотой}.

\hfill

Вертикальных асимптот у функции может быть бесконечное множество.

\hfill

Например, $f(x) = \tg x, x = \frac{\pi}{2} + \pi k, k \in Z$

\subsubsection{Наклонные асимптоты}

Если следующие пределы: $\lim\limits_{x \to +\infty} \frac{f(x)}{x} = k, \lim\limits_{x \to +\infty} (f(x) - k x) = b$ существуют и конечны, то прямая, заданная уравнением $y = k x + b$ является наклонной асимптотой функции $f(x)$ при $x \to \infty$.

\hfill

Если $k = 0$, то асимптота называется горизонтальной.

\hfill

Наклонных асимптот у функции может быть только две.

\subsubsection{Примеры}

\textbf{Пример 1.} Найти асимптоты функции $f(x) = \frac{x}{1 + e^{-x}}$.

\hfill

Найдем вертикальные асимптоты данной функции. Для начала найдем точки разрыва.

Данная функция \textbf{непрерывна}, так как знаменатель не может быть равен нулю.

Следовательно, вертикальных асимптот нет.

\hfill 

Найдем наклонные асимптоты: посчитаем пределы.

$k_{+} = \lim\limits_{x \to +\infty} \frac{x}{x(1+e^{-x})} = 1, k_{-} = \lim\limits_{k \to -\inf} \frac{x}{x(1+e^{-x})} = 0$

$b_{+} = \lim\limits_{x \to +\infty} (\frac{x}{1+e^{-x}} - x) = \lim\limits_{x \to +\infty} \frac{x - x - x e^{-x}}{1 + e^{-x}} = 0, b_{-} = \lim\limits_{x \to -\infty} (\frac{x}{1 + e^x}) = 0$

\hfill

При $x \to +\infty, y = x$ - наклонная асимптота.

При $x \to -\infty, y = 0$ - горизонтальная асимптота.

\subsection{Производные функции}

\textbf{Определение.} Если для $f(x)$ существует предел $\lim\limits_{\Delta x \to 0} \frac{\Delta y}{\Delta x} = \lim\limits_{\Delta x \to 0} \frac{f(x + \Delta x) - f(x)}{\Delta x}$, то он называется производной функции $y = f(x)$ в точке $x$, и обозначается $y' = f'(x) = \frac{\diff f(x)}{\diff x} = \frac{\diff}{\diff x}; f(x) = \frac{\diff y}{\diff x}$.

\subsubsection{Свойства производных функции}

Принятые обозначения: $c$ - константа, $u, v$ - функции.

\begin{multienumerate}
    \mitemxx{$(c)' = 0$}{$(c u)' = c * u'$}
    \mitemxx{$(u \pm v)' = u' \pm v'$}{$(u * v)' = u' v + u v'$}
    \mitemx{$(\frac{u}{v})' = \frac{u' v - u v'}{v^2}$}
    \mitemx{Если $y = f(u), u = \phi(x)$, то $(f(\phi(x)))' = f'(u) * u'$. \\ Пример: $\cos 3x = -\sin 3x * 3 = -3\sin x$ \\ Еще один пример: $\tg^{2x} e^{x} = 2 \tg e^x * \frac{1}{\cos^2 e^x} * e^x$}
\end{multienumerate}

\subsubsection{Таблица производных}

\begin{multienumerate}
    \mitemxx{$(u^{a})' = a * u^{a - 1} * u', a \in R$  \\
    $(\frac{1}{u}) = (u^{-1})' = -1 * \frac{1}{u^2} * u'$ \\
    $(\sqrt{u})' = (u^{\frac{1}{2}})'$ = $\frac{1}{2\sqrt{u}} * u'$}{$(a^{u}) = a^{u} * \ln a * u'$ \\
    $(e^{u})' = e^{u} * u'$}
    \mitemxx{$(\log_{a}{u})' = \frac{1}{u} \log_{a}{e} * u' = \frac{1}{u \ln a} * u'$ \\
    $(\ln u)' = \frac{1}{u} * u', (\ln |u|)' = \frac{1}{u} * u'$}{$(\sin u)' = \cos x$}
    \mitemxx{$(\cos u)' = -\sin x$}{$(\tg u)' = \frac{1}{\cos^2 u} * u'$}
    \mitemxx{$(\ctg u)' = - \frac{1}{\sin^2 u} * u'$}{$(\arcsin u)' = \frac{1}{\sqrt{1 - u^2}} * u'$}
    \mitemxx{$(\arccos u)' = - \frac{1}{\sqrt{1 - u^2}} * u'$}{$(\arctg u)' = \frac{1}{1 + u^2} * u'$}
    \mitemxx{$(\arcctg u)' = - \frac{1}{1 + u^2} * u'$}{$(\sinh u)' = \cosh u * u'$}
    \mitemxx{$(\cosh u)' = \sinh u * u'$}{$(\tanh u)' = \frac{1}{\cosh^2 u} * u'$}
    \mitemxx{$(\coth u)' = - \frac{1}{\sinh^2{u}} * u'$}{ ($u(x)^{v(x)})' = v(x) * u(x)^{v(x) - 1} * u'(x) + u(x)^{v(x)} * \ln u(x) * v'(x)$}
\end{multienumerate}

\subsubsection{Гиперболические функции}


\begin{multienumerate}
    \mitemxx{$\cosh u = \frac{e^u + e^{-u}}{2}$}{$\sinh u = \frac{e^u - u^{-u}}{2}$}
    \mitemxx{$\tanh u = \frac{\sinh u}{\cosh u}$}{$\coth u = \frac{\cosh u}{\sinh u}$}
\end{multienumerate}


\subsubsection{Уравнение гиперболы}

$$\frac{x^2}{a^2} - \frac{y^2}{b^2} = 1$$

\begin{equation}
    \begin{cases}
        x = a \coth t \\
        y = b \cosh t
    \end{cases}
\end{equation}

\subsubsection{Показательно-степенная функция}

Производную показательно-степенной функции можно найти следующим образом:

$(u^v)' = v * u^{v - 1} * u' + u^v \ln u * v'$

\subsubsection{Примеры}

\textbf{Пример 1.} $y = \tg 3x + 5x^2$, $y' = \frac{3}{\cos^{2} 3x} + 10x$

\hfill

\textbf{Пример 2.} $y = \cos (3x^2 + x)$, $y' = -\sin(3x^2 + x) * (6x + 1)$

\hfill

\textbf{Пример 3.} $y = x^3 * \cos x$, $y' = 3x^2 \cos x - x^3 \sin x$

\hfill

\textbf{Пример 4.} $y = \frac{x^2 + 1}{x^2 - 1}$, $y' = \frac{2x(x^2 - 1) - 2x(x^2 + 1)}{(x^2 - 1)^2} = \frac{-4x}{(x^2 - 1)^2}$

\hfill

\textbf{Пример 5.} $y = \ln (2x^2 + x - 1)$, $y' = \frac{1}{2x^2 + x - 1} * (4x + 1)$

\hfill

\textbf{Пример 6.} $y = \tg^3 (x + e^{-x^2})$, $y' = 2 \tg^3 (x + e^{-x^2}) * \frac{1}{\cos^2(x+e^{-x^2}} * (1 + e^{-x^2})$

\hfill

\textbf{Пример 7. } $y = (\cos x)^{x^2}$, $y' = x^2 (\cos x)^{x^2 - 1} * (-\sin x) + (\cos x)^{x^2} * \ln (\cos x) * 2x$

\hfill

\textbf{Пример 8.} $y = 2 \sqrt[3]{x} + \frac{3}{x^2}$, $y' = 2 * \frac{1}{3} * x^{\frac{1}{3} - 1} * x^{-1} = \frac{2}{3\sqrt[3]{x^2}} - \frac{6}{x^3}$

\hfill

\textbf{Пример 9.} $y = (x^2 + 5x + 7)^{8}$, $y' = 8(x^2 + 5x + 7)^{7} * (2x + 5)$

\end{flushleft}

\end{document}