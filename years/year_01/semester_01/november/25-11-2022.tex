\documentclass{article}
\usepackage[utf8]{inputenc}

\usepackage[T2A]{fontenc}
\usepackage[utf8]{inputenc}
\usepackage[russian]{babel}

\usepackage{amsmath}
\usepackage{pgfplots}
\usepackage{multienum}
\newcommand*\diff{\mathop{}\!\mathrm{d}}

\DeclareMathOperator{\sign}{sign}

\title{Высшая математика}
\author{Лисид Лаконский}
\date{November 2022}

\begin{document}

\maketitle

\tableofcontents
\pagebreak

\section{Высшая математика - 25.11.2022}

\subsection{Исследование функций с помощью первой и второй производной}

\subsubsection{Наибольшее и наименьшее значение}

\begin{flushleft}

Если $f(x)$ имеет производную и на $[a; b]$ возрастает, то $f'(x) > 0$, если убывает - $f'(x) < 0$

Если $x_1$ - точка максимума, то $f(x_1) > f(\text{в любой точке из } \epsilon \text{ - окрестности точки } x_1)$. Если $x_2$ - точка минимума, то $f(x_1) < f(\text{в любой точке из } \epsilon \text{ - окрестности точки } x_2)$

\paragraph{Теорема об необходимом условии существования экстремума}

Если дифференцируемая функция $f(x)$ имеет в точке $x_1$ максимум или минимум, то её производная в этой точке равна нулю или не существует.

\textbf{Замечание.} Не при всяком $x$, при котором производная равна нулю, существует максимум или минимум.

\paragraph{Теорема о достаточном условии существования экстремума} Пусть $f(x)$ непрерывна на некотором интервале, содержащем точку $x_1$, в которой $f'(x_1) = 0$ или не существует, и $f(x)$ дифференцируема во всех точках интервала (кроме, может, самой $x_1$).

Если при переходе через эту точку знак производной меняется с плюса на минус, то она называется точкой максимума. Если меняется знак меняется с минуса на плюс, то она называется точкой минимума.

\paragraph{Наибольшее и наименьшее значение функции на отрезке.}

Пусть $f(x)$ непрерывна на $[a; b]$, тогда функция достигает своего наибольшего (наименьшего) значения или на концах $[a; b]$, или в одной из точек экстремума внутри отрезка.

\paragraph{Примеры решения задач}

Найдем наибольшее значение функции $f(x) = 2x^2 - 3x^2 - 12x + 1$, определенной на отрезке $[ -2; \frac{5}{2}]$

$f'(x) = 6x^2 - 6x - 12$

Найдем экстремумы, приравняя производную к нулю: $x_1 = -1$, $x_2 = 2$

$f(-2) = -3$, $f(2) = -19$, $f(-1) = 8$ - \textbf{ответ}, $f(\frac{5}{2}) = -16 \frac{1}{2}$

\subsubsection{Исследование кривой на выпуклость и вогнутость, точки перегиба}

\paragraph{Выпуклость и вогнутость} Кривая обращена \textbf{выпуклостью вверх} (выпукла), если все точки кривой лежат ниже любой ее касательной на этом интервале. При этом $f''(x) < 0$

Кривая обращена \textbf{выпуклостью вниз} (вогнута), если все точки кривой лежат ниже любой ее касательной на этом интервале. При этом $f''(x) > 0$

\paragraph{Точка перегиба} Точка, отделяющая выпуклую часть непрерывной кривой от ее вогнутой части называется \textbf{точкой перегиба}.

Если $y = f(x)$ - непрерывна в точке $a$, $f''(a) = 0$ или не существует; при переходе через $a$ меняет знак, то $a$ - \textbf{точка перегиба}.

\paragraph{Теорема о втором достаточном условии существования экстремума}

Если $f'(x_1) = 0$ или не существует, $f''(x_1) > 0$, то в точке $x_1$ - минимум, иначе если $f''(x_1) < 0$, то в точке $x_1$ - максимум.

\textbf{Замечание} Если $f'(x_1) = f''(x_1) = ... = f^{(n - 1)}$, $n$ - нечетное, то производной не существует. Если $n$ - четное, то $f^{(n)} > 0$ - минимум, $f^{(n)} < 0$ - максимум. 

\paragraph{Примеры решения задач}

Пусть $y = \frac{x}{1 + x^2}$, исследуем ее на выпуклость и вогнутость; найдем точки перегиба.

$y' = \frac{1 + x^2 - x * 2x}{(1 + x^2)^2} = \frac{1 - x^2}{(1 + x^2)^2}$, $y'' = \frac{-2x(1 + x^2)^2 - (1 - x^2) * 2 (1 + x^2) 2x}{(1 + x^2)^4} = \frac{2x^3 - 6x}{(1 + x^2)^3} = \frac{2x (x^2 - 3)}{(x^2 + 1)^3} = \frac{2x(x - \sqrt{3})(x + \sqrt{3})}{(1 + x^2)^3}$

$\sign (-\infty; \sqrt{3}) = -1$, $\sign [-\sqrt{3}; 0] = 1$, $\sign (0; \sqrt{3}) = -1$ $\sign [\sqrt{3}; +\infty] = 1$

\subsection{Асимптоты функций}

Прямая называется \textbf{асимптотой} к кривой, если расстояние $\Delta$ от некоторой переменной точки кривой до этой прямой стремится к нулю при удалении данной точки в бесконечность.

\textbf{Виды асимптот}:

\begin{enumerate}
    \item Вертикальные: $\lim\limits_{x \to a + 0} = \infty$ (хотя бы справа или слева), $x = a$ - асимптота
    \item Горизонтальные: $\lim\limits_{x \to \infty} f(x) = b$, $y = b$
    \item Наклонные: $y = k x + b$, $k = \lim\limits_{x \to \infty} \frac{f(x)}{x}$, $b = \lim\limits_{x \to \infty} (f(x) - k x)$
\end{enumerate}

\subsection{Общий план исследования функции}

Общий план исследования функции:

\begin{enumerate}
    \item Находим область определения функции, нули функции, интервалы знакопостоянства
    \item Проверяем четность-нечетность, периодичность функции, находим точки пересечения с осями
    \item Исследуем функцию на непрерывность, точки разрыва, вертикальные асимптоты
    \item Находим первую производную, точки экстремума, вычисляем значение функции в этих точках
    \item Находим вторую производную, исследуем на выпуклость и вогнутость
    \item Проверяем наклонные асимптоты, можно проверить отдельные точки
\end{enumerate}

\subsubsection{Примеры применения общего плана исследования функции}

\textbf{Пример 1.} Рассмотрим функцию $y = \frac{x}{1 + x^2}$, $y = 0$ при $x = 0$, при положительных $x$ график располагается выше оси $x$, при отрицательных - ниже.

Проверим на четность и нечетность: $f(-x) = \frac{-x}{1 + x^2} = -y(x)$ - функция \textbf{является нечетной} - симметрична относительно нуля.

Найдем первую производную: $y' = \frac{1 - x^2}{(1 + x^2)^2}$, $y' = 0$ в точках $x_1 = 1$, $x_2 = -1$, вычислим значения функции в этих точках: $y(1) = \frac{1}{2}$, $y(-1) = -\frac{1}{2}$

Найдем вторую производную: $y'' = \frac{2x(x-\sqrt{3})(x+\sqrt{3})}{(1 + x^2)^3}$, исследуем ее на выпуклость и вогнутость.

Поищем наклонную асимптоту данной функции: $k = \lim\limits_{x \to \infty} \frac{f(x)}{x} = \lim\limits_{x \to \infty} \frac{1}{1 + x^2} = 0$, $b = \lim\limits_{x \to \infty} (f(x) - 0 * x) = \lim\limits_{x \to \infty} \frac{x}{1 + x^2} = 0$, $y = 0$ - \textbf{горизонтальная асимптота}.

\hfill

\textbf{Пример 2.}

Рассмотрим функцию $y = \frac{x}{x^2 - 1}$, $y = 0$ при $x = 0$, область определения данной функции: $(-\infty; -1) \cup (-1; 1) \cup (1; +\infty)$

Проверим на четность и нечетность: $f(-x) = \frac{-x}{x^2 - 1} = -y(x)$ - функция \textbf{является нечетной} - симметрична относительно нуля.

$\lim\limits_{x \to 1 + 0} \frac{x}{x^2 - 1} = +\infty$, $\lim\limits_{x \to 1 - 0} \frac{x}{x^2 - 1} = -\infty$

Найдем первую производную данной функции: $y' = - \frac{x^2 + 1}{(x^2 - 1)^2} < 0$ - убывающая 

Найдем вторую производную данной функции: $y'' = \frac{2x(x^2 + 3)}{(x^2 - 1)^3}$, исследуем ее на выпуклость и вогнутость.

\end{flushleft}

\end{document}