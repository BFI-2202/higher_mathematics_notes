\documentclass{article}
\usepackage[utf8]{inputenc}

\usepackage[T2A]{fontenc}
\usepackage[utf8]{inputenc}
\usepackage[russian]{babel}

\usepackage{amsmath}
\usepackage{pgfplots}
\usepackage{multienum}
\newcommand*\diff{\mathop{}\!\mathrm{d}}

\DeclareMathOperator{\sign}{sign}

\title{Высшая математика}
\author{Лисид Лаконский}
\date{November 2022}

\begin{document}

\maketitle

\tableofcontents
\pagebreak

\section{Высшая математика - 01.11.2022}

\subsection{Правило Лопиталя}

\begin{flushleft}

Если $\lim\limits_{x \to a} f(x) = 0 (\infty)$, $\lim\limits_{x \to a} g(x) = 0 (\infty)$, то $\lim\limits_{x \to a} \frac{f(x)}{g(x)}$ представляет собой неопределенность вида $\{ \frac{0}{0} \}$ или $\{ \frac{\infty}{\infty} \}$.

\hfill

$\lim\limits{x \to a} \frac{f(x)}{g(x)} = \lim\limits_{x \to a} \frac{f'(x)}{g'(x)}$

\subsubsection{Примеры}

\textbf{Пример 1.}

$\lim\limits_{x \to 1} \frac{3x^2 + 4x - 7}{2x^2 - x - 1} = \{ \frac{0}{0} \} = \lim\limits_{x \to 1} \frac{6x + 4}{4x - 1} = \frac{10}{3}$

\hfill

\textbf{Пример 2.}

$\lim\limits_{x \to 0} \frac{e^x - x - 1}{x^2} = \{ \frac{0}{0} \} = \lim\limits_{x \to 0} \frac{e^x - 1}{2x} = \{ \frac{0}{0} \} = \lim\limits_{x \to 0} \frac{e^x}{2} = \frac{1}{2}$

\hfill

\textbf{Пример 3.}

$\lim\limits_{x \to +\infty} \frac{x^3}{e^x} = \{ \frac{\infty}{\infty} \} = \lim\limits_{x \to +\infty} = \lim\limits_{x \to +\infty} \frac{3x^2}{e^x} = \lim\limits_{x \to +\infty} \frac{6x}{e^x} = \lim\limits_{x \to +\infty} \frac{6}{e^x} = 0$

\hfill

\textbf{Пример 4.}

$\lim\limits_{x \to -\infty} \frac{x^3}{e^x} =  \lim\limits_{x \to -\infty} \{ -\infty * \infty \} = -\infty$

\hfill

\textbf{Пример 5.}

$\lim\limits_{x \to 0} x * \ctg x = \{ 0 * \infty \} = \lim\limits_{x \to 0} \frac{x}{\tg x} = \{ \frac{0}{0} \} = \lim\limits_{x \to 0} \cos^2{x} = 1$

\hfill

\textbf{Пример 6.}

$\lim\limits_{x \to 0} (\tg x)^x = \{ 0^0 \} = \lim\limits_{x \to 0} \ln y = \lim\limits_{x \to 0} x \ln \tg x = \{ 0 * \infty \} = \lim\limits_{x \to 0} \frac{\ln \tg x}{\frac{1}{x}} = \{ \frac{\infty}{\infty} \} = \lim\limits_{x \to 0} \frac{\frac{1}{\tg x} * \frac{1}{\cos^2 x}}{-\frac{1}{x^2}} = \lim\limits_{x \to 0} - \frac{x^2}{\sin x \cos x} = \lim\limits_{x \to 0} -\frac{x^2}{0.5\sin 2x} = 0$

$y = (\tg x)^x \Longleftrightarrow \ln y = \ln (\tg x)^x$

\hfill

\textbf{Пример 7.}

$\lim\limits_{x \to 0} \ln y = \ln \lim\limits_{x \to 0} y = 0$,  $\ln y = 0 \Longleftrightarrow y = e^0 = 1$

\hfill

\textbf{Пример 8.}

$\lim\limits_{x \to 0} (\tg x)^x = \lim\limits_{x \to 0} e^{\ln(\tg x)^x} = \lim\limits_{x \to 0} e^{x \ln \tg x} = ... = 0$

\end{flushleft}

\pagebreak
\section{Высшая математика - 09.11.2022}

\subsection{Разбор контрольной работы}

\begin{flushleft}

Семь заданий в варианте, решается шесть заданий, потому что шестое задание - то, что не рассказывается на лекциях.

\subsubsection{Первое задание}

$
\lim\limits_{n \to \infty} \frac{\sqrt{n^4 + n + 1}}{3n^2 - 2\sqrt{n + 2}}
=
\lim\limits_{n \to \infty} [ \frac{\infty}{\infty} ]
=
\lim\limits_{n \to \infty} \frac{1 + \frac{1}{n^3} + \frac{1}{n^4}}{3 - 2 \sqrt { \frac{1}{n^3} + \frac{2}{n^4}}}
=
\frac{1}{3}
$

\subsubsection{Второе задание}

$
\lim\limits_{n \to \infty} \frac{(n + 2)^2}{n\sqrt{n} + \sqrt{n^4 + 2}}
=
\lim\limits_{n \to \infty} \frac{n^2 + 4n + 4}{n\sqrt{n} + \sqrt{n^4 + 2}}
=
\lim\limits_{n \to \infty} \frac{1 + \frac{4}{n} + \frac{4}{n^2}}{\frac{1}{\sqrt{n^3}} + \sqrt{1 + \frac{2}{n^4}}}
=
1
$

\subsubsection{Третье задание}

$
\lim\limits_{x \to 3} \frac{\sqrt{x + 13} - 2\sqrt{x + 1}}{x^2 - 9}
=
\lim\limits_{x \to 3} \frac{x + 13 - 4(x + 1)}{(x^2 - 9)(\sqrt{x + 13) + 2\sqrt{x + 1}}}
=
\lim\limits_{x \to 3} \frac{-3(x - 3)}{(x-3)(x+3)*a} (a = \sqrt{x+13} + 2\sqrt{x + 1})
=
\lim\limits_{x \to 3} \frac{-3}{(x + 3)(\sqrt{x + 13} + 2\sqrt{x + 1}}
=
-\frac{1}{16}
$

\subsubsection{Четвертое задание}

$
\lim\limits_{x \to 0} \frac{2^{3x} - 3^{2x}}{x + \arcsin x^3}
=
\lim\limits_{x \to 0} \frac{2^{3x} * \ln 2 * 3 - 3^{2x} * \ln 3 * 2}{1 + 3x^2}
=
3\ln 2 - 2\ln 3
$

\subsubsection{Пятое задание}

$
\lim\limits_{x \to 0} \frac{1 - \cos 2x}{\cos 7x - \cos 3x}
=
\lim\limits_{x \to 0} \frac{1 + 4x}{-49x + 9x}
=
\frac{x(\frac{1}{x} + 4}{-40x}
=
-\frac{1}{10}
$

\subsubsection{Седьмое задание}

Найти число и характер точек разрыва функции $f(x) = \frac{\frac{1}{x} - \frac{1}{x + 1}}{\frac{1}{x - 1} - \frac{1}{x}}$

Точек разрыва ровно три штуки: $x = -1$, $x = 0$, $x = 1$

\hfill

Рассмотрим точку разрыва $x = 0$:

$
\lim\limits_{x \to 0} f(x) = \frac{\frac{1}{0} - \frac{1}{0 + 1}}{\frac{1}{0 - 1} - \frac{1}{x}} = \{ \frac{\infty}{\infty} \} = \frac{x + 1 - x}{x(x+1)} * \frac{x(x - 1)}{x - x + 1} = \frac{\frac{1}{x} - \frac{1}{x + 1}}{\frac{1}{x - 1} - \frac{1}{x}} = \lim\limits_{x \to 0} \frac{x - 1}{x + 1} = -1
$

Рассмотрим два односторонних предела:

\begin{enumerate}
    \item $\lim\limits_{x \to 0 + 0} f(x) = -1$
    \item $\lim\limits_{x \to 0 - 0} f(x) = -1$
\end{enumerate}

Данные пределы равны между собой, следовательно, $x = 0$ - устранимая точка разрыва первого рода.

\hfill

По аналогии находится $x = 1$ - она тоже является устранимой точкой разрыва первого рода

\hfill

Рассмотрим точку разрыва $x = -1$:

\begin{enumerate}
    \item $f(-1)$ - не определена
    \item $\lim\limits_{x \to -1} f(x) = -\infty$
\end{enumerate}

Следовательно, $x = -1$ - точка разрыва второго рода.

\end{flushleft}

\pagebreak
\section{Высшая математика - 11.11.2022}

\subsection{Некоторые теоремы о дифференцируемых функциях}

\subsubsection{Теорема Ролля}

\begin{flushleft}

Пусть $f(x)$ непрерывна на $[ a; b ]$; дифференцируема во внутренних точках (a, b); $f(a) = f(b) = 0$, тогда внутри отрезка $[ a; b ]$ существует по крайней мере одна точка $c$, в которой производная обращается в ноль:

$\exists c: a < c < b$, такая что $f'(c) = 0$

Доказательство смотреть в Лакерник А. Р. "Краткий курс высшей математики", Пискунов.

\paragraph{Первое замечение} Эта теорема останется справедливой, если $f(a) = f(b) \ne 0$

\paragraph{Примеры}

Например, имеется функция: $f(x) = x^2$, рассмотрим ее на отрезке $[-2; 2]$. Для нее выполняются все условия, следовательно, существует такая точка, в которой производная обращается в ноль. Можно найти, что это точка $x = 0$

\subsubsection{Теорема Лагранжа о конечных приращениях}

Пусть $f(x)$ непрерывна на $[ a; b ]$; дифференцируема во внутренних точках ($a, b$), тогда $\exists c: a < c < b$, что $f(b) - f(a) = f'(c)(b-a)$

\paragraph{Доказательство}

$Q = \frac{f(b) - f(a)}{b - a}$ - число, введем вспомогательную функцию $F(x) = f(x) - f(a) - (x - a)Q$

Уравнение прямой, проходящей через $(a; f(a))$: $y - f(a) = \tg \alpha (x - a) \Longleftrightarrow y = f(a) + \frac{f(b) - f(a)}{b - a}(x - a)$

$F(x) = f(x) - y$, то есть мы получили, что $F(x)$ для каждого значения $x$ является разностью ординаты кривой и хорды.

\hfill

Функция $F(x)$ \textbf{удовлетворяет всем условиям теоремы Ролля}, $F'(c) = 0, F'(x) = f'(x) - Q, F'(c) = f'(c) - Q = 0, Q = f'(c)$.

\paragraph{Геометрический смысл} Геометрический смысл теоремы Лагранжа в том, что при выполнении требуемых условий на кривой найдется точка между $a$ и $b$, что касательная в которой параллельна хорде $ab$.

\subsubsection{Теорема Коши об отношении приращений двух функций}

Пусть есть $f(x)$ и $g(x)$, непрерывные на $[ a; b ]$ и дифференцируемы во внутренних точках $(a; b)$, $g'(x) \ne 0$ внутри $(a; b)$, \textbf{тогда} $\exists c: a < c < b$, что $\frac{f(b) - f(a)}{g(b) - g(a)} = \frac{f'(c)}{g'(c)}$

\subsubsection{Теорема Ферма}

Пусть $f(x)$ определена на $[a; b]$ и принимает во внутренней точке $c$ наибольшее или наименьшее значение.

Тогда $f'(c) = 0$, если в этой точке существует конечная производная.

\end{flushleft}

\subsection{Геометрические приложения производной}

\subsubsection{Уравнение касательной к кривой}

\begin{flushleft}

\textbf{Касательной} $y = y(x)$ в точке $A(x; y(x))$ \textbf{называется} прямая, к которой стремится секущая, проходящая через точку $A$ и точку $B (x + \Delta x, y(x + \Delta x))$ при условии, что $\Delta x \to 0$

\paragraph{Уравнение касательной}

Уравнение касательной в точке $(x_0; y_0)$: $y - y_0 = y'(x_0)(x - x_0)$

\subsubsection{Уравнение нормали к кривой}

$\textbf{Уравнение нормали:}$ $y - y_0 = -\frac{1}{y'(x_0)} (x - x_0)$

\subsection{Правило Лопиталя}

\subsubsection{Раскрытие неопределенностей вида $\frac{0}{0}$}

Пусть $f(x)$ и $g(x)$ \textbf{удовлетворяют} условию теоремы Коши на интервале $[ a; b ]$ и \textbf{выполняется} условие $f(a) = g(a) = 0$

Тогда если $\exists \lim\limits_{x \to a} \frac{f'(x)}{g'(x)}$, то $\exists \lim\limits_{x \to a} \frac{f(x)}{g(x)}$, и они равны между собой.

\paragraph{Замечание} Если $f'(a) = g'(a) = 0$, то можно рассматривать собственно производные в качестве функций и дальше применять правило Лопиталя .

\subsubsection{Раскрытие неопределенностей вида $\frac{\infty}{\infty}$}

Пусть $f(x)$ и $g(x)$ \textbf{непрерывны} и дифференцируемы на всех точках кроме, может, самой точки $a$, и $g'(x) \ne 0$, и $\lim\limits_{x \to a} f(x) = \infty$, $\lim\limits_{x \to a} g(x) = \infty$

Если $\exists \lim\limits_{x \to a} \frac{f'(x)}{g'(x)} = A \implies \exists \lim\limits_{x \to a} \frac{f(x)}{g(x)} = A$

\subsection{Формула Тейлора}

Пусть $f(x)$ имеет непр. произв. до  $(n + 1)$ пор.

$P_n(x)$ многочлен степени не выше $n$: $P_n(a) = f(a), P_n'(a) = f'(a)$

$P_n(x) = f(a) + \frac{f'(a)(x - a)}{1!} + \frac{f''(a) (x-a)^2}{2!} + ... + \frac{f^{(n)}(a)(x - a)^n}{n!}, R_n(x) = \frac{(x - a)^{n + 1}}{(n + 1)!} f^{(n + 1)} (\xi), a < \xi < x$

\paragraph{Формула Маклорена}

$P_n(x) = f(0) + \frac{f'(0)}{1!}x + \frac{f''(0)}{2!} x^2 + ... + \frac{f^{(n)}(0)}{n!}x^n$

\paragraph{Примеры вывода формулы}

\hfill

\rule{\textwidth}{0.4pt}

\parbox{0.3\textwidth}{
$y = \sin x, y(0) = 0$

$y'(0) = 1$

$y''(0) = 0$

$y'''(0) = -1 $

$y''''(0) = 0$

$y'''''(0) = 1$
}
\parbox{0.3\textwidth}{
$y = \cos x, y(0) = 1$

$y'(0) = $

$y''(0) = $

$y'''(0) = $

$y''''(0) = $

$y'''''(0) = $
}
\parbox{0.3\textwidth}{
$y = e^x, y(0) = 1$

$y'(0) = 1$

$y''(0) = 1$

$y'''(0) = 1$

$y''''(0) = 1$

$y'''''(0) = 1$
}

\rule{\textwidth}{0.4pt}

Отсюда $\sin x = 0 + x + 0x^2 - \frac{1}{3!}x^3 + 0x^4 + \frac{1}{5!}x^5 + ... = x - \frac{x^3}{3!} + \frac{x^5}{5!} - ...$, $\cos x = 1 - \frac{x^2}{2!} + \frac{x^4}{4!} - ...$, $e^{x} = 1 + x + \frac{x^2}{2!} + \frac{x^3}{3!} + ...$

Интересно рассмотреть натуральный логарифм: $\ln (1 + x) = x - \frac{x^2}{2} + \frac{x^3}{3} - \frac{x^4}{4}$

Также $(1 + x)^m = 1 + m X + \frac{m(m - 1}{2!}x^2 + \frac{m(m - 1)(m - 2)}{3!}x^3 + ...$, $\sqrt{1 + x} = 1 + \frac{x}{2} - \frac{1}{2^2} \frac{x^2}{2!} + \frac{1 * 3 * x^3}{2^3 * 3!} - \frac{1 * 3 * 5 * x^4}{2^4 * 4!}$

$\cos^2 x = \frac{1}{2} (\cos 2x + 1) = \frac{1}{2} (1 + 1 - \frac{(2x)^2}{2!} + \frac{(2x)^4}{4!} - \frac{(2x)^6}{6!} + ...$

\paragraph{Применение данных формул к вычислению пределов}

Допустим, имеем $\lim\limits_{x \to 0} \frac{x - \sin x}{x - \tg x} = ...$

Распишем разложение в Тейлора: $... = \lim\limits_{x \to 0} \frac{x - x + \frac{x^3}{3!} + o(x^3)}{x - x - \frac{x^3}{3} + o(x^3)} = -\frac{1}{2}$

\end{flushleft}

\pagebreak
\section{Высшая математика - 25.11.2022}

\subsection{Исследование функций с помощью первой и второй производной}

\subsubsection{Наибольшее и наименьшее значение}

\begin{flushleft}

Если $f(x)$ имеет производную и на $[a; b]$ возрастает, то $f'(x) > 0$, если убывает - $f'(x) < 0$

Если $x_1$ - точка максимума, то $f(x_1) > f(\text{в любой точке из } \epsilon \text{ - окрестности точки } x_1)$. Если $x_2$ - точка минимума, то $f(x_1) < f(\text{в любой точке из } \epsilon \text{ - окрестности точки } x_2)$

\paragraph{Теорема об необходимом условии существования экстремума}

Если дифференцируемая функция $f(x)$ имеет в точке $x_1$ максимум или минимум, то её производная в этой точке равна нулю или не существует.

\textbf{Замечание.} Не при всяком $x$, при котором производная равна нулю, существует максимум или минимум.

\paragraph{Теорема о достаточном условии существования экстремума} Пусть $f(x)$ непрерывна на некотором интервале, содержащем точку $x_1$, в которой $f'(x_1) = 0$ или не существует, и $f(x)$ дифференцируема во всех точках интервала (кроме, может, самой $x_1$).

Если при переходе через эту точку знак производной меняется с плюса на минус, то она называется точкой максимума. Если меняется знак меняется с минуса на плюс, то она называется точкой минимума.

\paragraph{Наибольшее и наименьшее значение функции на отрезке.}

Пусть $f(x)$ непрерывна на $[a; b]$, тогда функция достигает своего наибольшего (наименьшего) значения или на концах $[a; b]$, или в одной из точек экстремума внутри отрезка.

\paragraph{Примеры решения задач}

Найдем наибольшее значение функции $f(x) = 2x^2 - 3x^2 - 12x + 1$, определенной на отрезке $[ -2; \frac{5}{2}]$

$f'(x) = 6x^2 - 6x - 12$

Найдем экстремумы, приравняя производную к нулю: $x_1 = -1$, $x_2 = 2$

$f(-2) = -3$, $f(2) = -19$, $f(-1) = 8$ - \textbf{ответ}, $f(\frac{5}{2}) = -16 \frac{1}{2}$

\subsubsection{Исследование кривой на выпуклость и вогнутость, точки перегиба}

\paragraph{Выпуклость и вогнутость} Кривая обращена \textbf{выпуклостью вверх} (выпукла), если все точки кривой лежат ниже любой ее касательной на этом интервале. При этом $f''(x) < 0$

Кривая обращена \textbf{выпуклостью вниз} (вогнута), если все точки кривой лежат ниже любой ее касательной на этом интервале. При этом $f''(x) > 0$

\paragraph{Точка перегиба} Точка, отделяющая выпуклую часть непрерывной кривой от ее вогнутой части называется \textbf{точкой перегиба}.

Если $y = f(x)$ - непрерывна в точке $a$, $f''(a) = 0$ или не существует; при переходе через $a$ меняет знак, то $a$ - \textbf{точка перегиба}.

\paragraph{Теорема о втором достаточном условии существования экстремума}

Если $f'(x_1) = 0$ или не существует, $f''(x_1) > 0$, то в точке $x_1$ - минимум, иначе если $f''(x_1) < 0$, то в точке $x_1$ - максимум.

\textbf{Замечание} Если $f'(x_1) = f''(x_1) = ... = f^{(n - 1)}$, $n$ - нечетное, то производной не существует. Если $n$ - четное, то $f^{(n)} > 0$ - минимум, $f^{(n)} < 0$ - максимум. 

\paragraph{Примеры решения задач}

Пусть $y = \frac{x}{1 + x^2}$, исследуем ее на выпуклость и вогнутость; найдем точки перегиба.

$y' = \frac{1 + x^2 - x * 2x}{(1 + x^2)^2} = \frac{1 - x^2}{(1 + x^2)^2}$, $y'' = \frac{-2x(1 + x^2)^2 - (1 - x^2) * 2 (1 + x^2) 2x}{(1 + x^2)^4} = \frac{2x^3 - 6x}{(1 + x^2)^3} = \frac{2x (x^2 - 3)}{(x^2 + 1)^3} = \frac{2x(x - \sqrt{3})(x + \sqrt{3})}{(1 + x^2)^3}$

$\sign (-\infty; \sqrt{3}) = -1$, $\sign [-\sqrt{3}; 0] = 1$, $\sign (0; \sqrt{3}) = -1$ $\sign [\sqrt{3}; +\infty] = 1$

\subsection{Асимптоты функций}

Прямая называется \textbf{асимптотой} к кривой, если расстояние $\Delta$ от некоторой переменной точки кривой до этой прямой стремится к нулю при удалении данной точки в бесконечность.

\textbf{Виды асимптот}:

\begin{enumerate}
    \item Вертикальные: $\lim\limits_{x \to a + 0} = \infty$ (хотя бы справа или слева), $x = a$ - асимптота
    \item Горизонтальные: $\lim\limits_{x \to \infty} f(x) = b$, $y = b$
    \item Наклонные: $y = k x + b$, $k = \lim\limits_{x \to \infty} \frac{f(x)}{x}$, $b = \lim\limits_{x \to \infty} (f(x) - k x)$
\end{enumerate}

\subsection{Общий план исследования функции}

Общий план исследования функции:

\begin{enumerate}
    \item Находим область определения функции, нули функции, интервалы знакопостоянства
    \item Проверяем четность-нечетность, периодичность функции, находим точки пересечения с осями
    \item Исследуем функцию на непрерывность, точки разрыва, вертикальные асимптоты
    \item Находим первую производную, точки экстремума, вычисляем значение функции в этих точках
    \item Находим вторую производную, исследуем на выпуклость и вогнутость
    \item Проверяем наклонные асимптоты, можно проверить отдельные точки
\end{enumerate}

\subsubsection{Примеры применения общего плана исследования функции}

\textbf{Пример 1.} Рассмотрим функцию $y = \frac{x}{1 + x^2}$, $y = 0$ при $x = 0$, при положительных $x$ график располагается выше оси $x$, при отрицательных - ниже.

Проверим на четность и нечетность: $f(-x) = \frac{-x}{1 + x^2} = -y(x)$ - функция \textbf{является нечетной} - симметрична относительно нуля.

Найдем первую производную: $y' = \frac{1 - x^2}{(1 + x^2)^2}$, $y' = 0$ в точках $x_1 = 1$, $x_2 = -1$, вычислим значения функции в этих точках: $y(1) = \frac{1}{2}$, $y(-1) = -\frac{1}{2}$

Найдем вторую производную: $y'' = \frac{2x(x-\sqrt{3})(x+\sqrt{3})}{(1 + x^2)^3}$, исследуем ее на выпуклость и вогнутость.

Поищем наклонную асимптоту данной функции: $k = \lim\limits_{x \to \infty} \frac{f(x)}{x} = \lim\limits_{x \to \infty} \frac{1}{1 + x^2} = 0$, $b = \lim\limits_{x \to \infty} (f(x) - 0 * x) = \lim\limits_{x \to \infty} \frac{x}{1 + x^2} = 0$, $y = 0$ - \textbf{горизонтальная асимптота}.

\hfill

\textbf{Пример 2.}

Рассмотрим функцию $y = \frac{x}{x^2 - 1}$, $y = 0$ при $x = 0$, область определения данной функции: $(-\infty; -1) \cup (-1; 1) \cup (1; +\infty)$

Проверим на четность и нечетность: $f(-x) = \frac{-x}{x^2 - 1} = -y(x)$ - функция \textbf{является нечетной} - симметрична относительно нуля.

$\lim\limits_{x \to 1 + 0} \frac{x}{x^2 - 1} = +\infty$, $\lim\limits_{x \to 1 - 0} \frac{x}{x^2 - 1} = -\infty$

Найдем первую производную данной функции: $y' = - \frac{x^2 + 1}{(x^2 - 1)^2} < 0$ - убывающая 

Найдем вторую производную данной функции: $y'' = \frac{2x(x^2 + 3)}{(x^2 - 1)^3}$, исследуем ее на выпуклость и вогнутость.

\end{flushleft}

\pagebreak
\section{Высшая математика - 29.11.2022}

\subsection{Примеры исследования функций}

\subsubsection{Задание 1}

\begin{flushleft}

\textbf{1) } Пусть $f(x) = \frac{x^2 - x - 6}{x - 2}$, $D(f) = (-\infty; 2) \cup (2; +\inf)$, $E(f) = R$

\textbf{2) } Посчитаем разные всякие, в том числе односторонние, пределы: $\lim\limits_{x \to -\infty} \frac{x^2 - x - 6}{x - 2} = -\infty$, $\lim\limits_{x \to +\infty} = +\infty$, $\lim\limits_{x \to 2^{+}} \frac{x^2 - x - 6}{x - 2} = \frac{-4}{0} = -\infty$, $\lim\limits_{x \to 2^{-}} \frac{x^2 - x - 6}{x - 2} = \frac{-4}{-0} = +\infty$

\textbf{3) } Найдем точки разрыва: $x = 2$ - точка разрыва второго рода.

\textbf{4) } Найдем асимптоты функции: $x = 2$ (горизонтальная асимптота), $k_1 = \lim\limits_{x \to +\infty} \frac{f(x)}{x} = \lim\limits_{x \to +\infty} \frac{x^2 - x - 6}{x^2 - 2x} = 1$, $b_1 = \lim\limits_{x \to +\infty} (\frac{x^2 - x - 6}{x - 2} - x) = \lim\limits_{x \to +\infty} (\frac{x^2 - x - 6 - x^2 + 2x}{x - 2}) = \lim\limits_{x \to +\infty} \frac{x - 6}{x - 2} = 1$, $y_1 = x_1 + 1$ (наклонная асимптота), $k_2 = \lim\limits_{x \to -\infty} \frac{x^2 - x - 6}{x^2 - 2x} = -1$, $b_2 = \lim\limits_{x \to -\infty} (\frac{x^2 - x - 6}{x - 2} - x) = \lim\limits_{x \to -\infty} (\frac{x - 6}{x - 2}) = 1$, $y_2 = -x + 1$. Итого, у нас есть \textbf{одна горизонтальная асимптота и одна наклонная асимптота}.

\textbf{5) } Исследуем вид функции: четная она, нечетная, периодическая, или же общего вида: $f(-x) = \frac{x^2 + x - 6}{-x - 2}$ - ни четная, ни нечетная, ни периодичная - функция общего вида.

\textbf{6) } Найдем точки пересечения с осями координат: $f(0) = \frac{-6}{-2} = 3$, $\frac{x^2 - x - 6}{x - 2} = 0 \Longleftrightarrow x^2 - x - 6 = 0$, $x_1 = 3$, $x_2 = -2$. Итого, точки пересечения: $A(0; 3)$, $B(3; 0)$, $C(-2; 0)$

\textbf{7) } Наметить примерный ход графика

\textbf{8) } Найдем экстремумы и критические точки: $f'(x) = (\frac{x^2 - x - 6}{x - 2})' = \frac{(2x - 1)(x - 2)}{(x - 2)^2} - \frac{-x^2 - x - 6}{(x - 2)^2} = \frac{x^2 - 4x + 8}{(x - 2)^2}$, $x^2 - 4x + 8 = 0$, $x_{1, 2} = \frac{4 \pm \sqrt{-16}}{2}$, $x_1 = 2 + 2i$, $x_2 = 2 - 2i$. Функция возрастает на всей области определения, экстремумов и критических точек не имеет.

\textbf{9) } $f''(x) = \frac{(2x - 4)(x^2 - 4x + 4) - 2(x - 2)(x^2 - 4x + 8)}{(x - 2)^4} = \frac{2x^3 - 4x^2 - 8x^2 + 16x + 8x - 16 - 2x^3 + 4x^2 + 8x^2 - 16x - 16x + 32}{(x - 2)^4} = \frac{-8x + 16}{(x - 2)^4} = \frac{-8}{(x - 2)^3}$. Выходит так, что функция выпулка вверх на промежутке $(2; +\infty)$, а вогнута вниз на промежутке $(-\infty; 2)$

\textbf{10) } Начертим график функции. Оставлю это в качестве упражнения внимательному читателю.

\end{flushleft}

\end{document}