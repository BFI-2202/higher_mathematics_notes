\documentclass{article}
\usepackage[utf8]{inputenc}

\usepackage[T2A]{fontenc}
\usepackage[utf8]{inputenc}
\usepackage[russian]{babel}

\usepackage{amsmath}
\usepackage{pgfplots}
\usepackage{multienum}
\newcommand*\diff{\mathop{}\!\mathrm{d}}

\DeclareMathOperator{\sign}{sign}

\title{Высшая математика}
\author{Лисид Лаконский}
\date{November 2022}

\begin{document}

\maketitle

\tableofcontents
\pagebreak

\section{Высшая математика - 29.11.2022}

\subsection{Примеры исследования функций}

\subsubsection{Задание 1}

\begin{flushleft}

\textbf{1) } Пусть $f(x) = \frac{x^2 - x - 6}{x - 2}$, $D(f) = (-\infty; 2) \cup (2; +\inf)$, $E(f) = R$

\textbf{2) } Посчитаем разные всякие, в том числе односторонние, пределы: $\lim\limits_{x \to -\infty} \frac{x^2 - x - 6}{x - 2} = -\infty$, $\lim\limits_{x \to +\infty} = +\infty$, $\lim\limits_{x \to 2^{+}} \frac{x^2 - x - 6}{x - 2} = \frac{-4}{0} = -\infty$, $\lim\limits_{x \to 2^{-}} \frac{x^2 - x - 6}{x - 2} = \frac{-4}{-0} = +\infty$

\textbf{3) } Найдем точки разрыва: $x = 2$ - точка разрыва второго рода.

\textbf{4) } Найдем асимптоты функции: $x = 2$ (горизонтальная асимптота), $k_1 = \lim\limits_{x \to +\infty} \frac{f(x)}{x} = \lim\limits_{x \to +\infty} \frac{x^2 - x - 6}{x^2 - 2x} = 1$, $b_1 = \lim\limits_{x \to +\infty} (\frac{x^2 - x - 6}{x - 2} - x) = \lim\limits_{x \to +\infty} (\frac{x^2 - x - 6 - x^2 + 2x}{x - 2}) = \lim\limits_{x \to +\infty} \frac{x - 6}{x - 2} = 1$, $y_1 = x_1 + 1$ (наклонная асимптота), $k_2 = \lim\limits_{x \to -\infty} \frac{x^2 - x - 6}{x^2 - 2x} = -1$, $b_2 = \lim\limits_{x \to -\infty} (\frac{x^2 - x - 6}{x - 2} - x) = \lim\limits_{x \to -\infty} (\frac{x - 6}{x - 2}) = 1$, $y_2 = -x + 1$. Итого, у нас есть \textbf{одна горизонтальная асимптота и одна наклонная асимптота}.

\textbf{5) } Исследуем вид функции: четная она, нечетная, периодическая, или же общего вида: $f(-x) = \frac{x^2 + x - 6}{-x - 2}$ - ни четная, ни нечетная, ни периодичная - функция общего вида.

\textbf{6) } Найдем точки пересечения с осями координат: $f(0) = \frac{-6}{-2} = 3$, $\frac{x^2 - x - 6}{x - 2} = 0 \Longleftrightarrow x^2 - x - 6 = 0$, $x_1 = 3$, $x_2 = -2$. Итого, точки пересечения: $A(0; 3)$, $B(3; 0)$, $C(-2; 0)$

\textbf{7) } Наметить примерный ход графика

\textbf{8) } Найдем экстремумы и критические точки: $f'(x) = (\frac{x^2 - x - 6}{x - 2})' = \frac{(2x - 1)(x - 2)}{(x - 2)^2} - \frac{-x^2 - x - 6}{(x - 2)^2} = \frac{x^2 - 4x + 8}{(x - 2)^2}$, $x^2 - 4x + 8 = 0$, $x_{1, 2} = \frac{4 \pm \sqrt{-16}}{2}$, $x_1 = 2 + 2i$, $x_2 = 2 - 2i$. Функция возрастает на всей области определения, экстремумов и критических точек не имеет.

\textbf{9) } $f''(x) = \frac{(2x - 4)(x^2 - 4x + 4) - 2(x - 2)(x^2 - 4x + 8)}{(x - 2)^4} = \frac{2x^3 - 4x^2 - 8x^2 + 16x + 8x - 16 - 2x^3 + 4x^2 + 8x^2 - 16x - 16x + 32}{(x - 2)^4} = \frac{-8x + 16}{(x - 2)^4} = \frac{-8}{(x - 2)^3}$. Выходит так, что функция выпулка вверх на промежутке $(2; +\infty)$, а вогнута вниз на промежутке $(-\infty; 2)$

\textbf{10) } Начертим график функции. Оставлю это в качестве упражнения внимательному читателю.

\end{flushleft}

\end{document}