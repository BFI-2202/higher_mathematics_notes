\documentclass{article}
\usepackage[utf8]{inputenc}

\usepackage[T2A]{fontenc}
\usepackage[utf8]{inputenc}
\usepackage[russian]{babel}

\usepackage{amsmath}
\usepackage{pgfplots}
\usepackage{multienum}
\newcommand*\diff{\mathop{}\!\mathrm{d}}

\title{Высшая математика}
\author{Лисид Лаконский}
\date{November 2022}

\begin{document}

\maketitle

\tableofcontents
\pagebreak

\section{Высшая математика - 11.11.2022}

\subsection{Некоторые теоремы о дифференцируемых функциях}

\subsubsection{Теорема Ролля}

\begin{flushleft}

Пусть $f(x)$ непрерывна на $[ a; b ]$; дифференцируема во внутренних точках (a, b); $f(a) = f(b) = 0$, тогда внутри отрезка $[ a; b ]$ существует по крайней мере одна точка $c$, в которой производная обращается в ноль:

$\exists c: a < c < b$, такая что $f'(c) = 0$

Доказательство смотреть в Лакерник А. Р. "Краткий курс высшей математики", Пискунов.

\paragraph{Первое замечение} Эта теорема останется справедливой, если $f(a) = f(b) \ne 0$

\paragraph{Примеры}

Например, имеется функция: $f(x) = x^2$, рассмотрим ее на отрезке $[-2; 2]$. Для нее выполняются все условия, следовательно, существует такая точка, в которой производная обращается в ноль. Можно найти, что это точка $x = 0$

\subsubsection{Теорема Лагранжа о конечных приращениях}

Пусть $f(x)$ непрерывна на $[ a; b ]$; дифференцируема во внутренних точках ($a, b$), тогда $\exists c: a < c < b$, что $f(b) - f(a) = f'(c)(b-a)$

\paragraph{Доказательство}

$Q = \frac{f(b) - f(a)}{b - a}$ - число, введем вспомогательную функцию $F(x) = f(x) - f(a) - (x - a)Q$

Уравнение прямой, проходящей через $(a; f(a))$: $y - f(a) = \tg \alpha (x - a) \Longleftrightarrow y = f(a) + \frac{f(b) - f(a)}{b - a}(x - a)$

$F(x) = f(x) - y$, то есть мы получили, что $F(x)$ для каждого значения $x$ является разностью ординаты кривой и хорды.

\hfill

Функция $F(x)$ \textbf{удовлетворяет всем условиям теоремы Ролля}, $F'(c) = 0, F'(x) = f'(x) - Q, F'(c) = f'(c) - Q = 0, Q = f'(c)$.

\paragraph{Геометрический смысл} Геометрический смысл теоремы Лагранжа в том, что при выполнении требуемых условий на кривой найдется точка между $a$ и $b$, что касательная в которой параллельна хорде $ab$.

\subsubsection{Теорема Коши об отношении приращений двух функций}

Пусть есть $f(x)$ и $g(x)$, непрерывные на $[ a; b ]$ и дифференцируемы во внутренних точках $(a; b)$, $g'(x) \ne 0$ внутри $(a; b)$, \textbf{тогда} $\exists c: a < c < b$, что $\frac{f(b) - f(a)}{g(b) - g(a)} = \frac{f'(c)}{g'(c)}$

\subsubsection{Теорема Ферма}

Пусть $f(x)$ определена на $[a; b]$ и принимает во внутренней точке $c$ наибольшее или наименьшее значение.

Тогда $f'(c) = 0$, если в этой точке существует конечная производная.

\end{flushleft}

\subsection{Геометрические приложения производной}

\subsubsection{Уравнение касательной к кривой}

\begin{flushleft}

\textbf{Касательной} $y = y(x)$ в точке $A(x; y(x))$ \textbf{называется} прямая, к которой стремится секущая, проходящая через точку $A$ и точку $B (x + \Delta x, y(x + \Delta x))$ при условии, что $\Delta x \to 0$

\paragraph{Уравнение касательной}

Уравнение касательной в точке $(x_0; y_0)$: $y - y_0 = y'(x_0)(x - x_0)$

\subsubsection{Уравнение нормали к кривой}

$\textbf{Уравнение нормали:}$ $y - y_0 = -\frac{1}{y'(x_0)} (x - x_0)$

\subsection{Правило Лопиталя}

\subsubsection{Раскрытие неопределенностей вида $\frac{0}{0}$}

Пусть $f(x)$ и $g(x)$ \textbf{удовлетворяют} условию теоремы Коши на интервале $[ a; b ]$ и \textbf{выполняется} условие $f(a) = g(a) = 0$

Тогда если $\exists \lim\limits_{x \to a} \frac{f'(x)}{g'(x)}$, то $\exists \lim\limits_{x \to a} \frac{f(x)}{g(x)}$, и они равны между собой.

\paragraph{Замечание} Если $f'(a) = g'(a) = 0$, то можно рассматривать собственно производные в качестве функций и дальше применять правило Лопиталя .

\subsubsection{Раскрытие неопределенностей вида $\frac{\infty}{\infty}$}

Пусть $f(x)$ и $g(x)$ \textbf{непрерывны} и дифференцируемы на всех точках кроме, может, самой точки $a$, и $g'(x) \ne 0$, и $\lim\limits_{x \to a} f(x) = \infty$, $\lim\limits_{x \to a} g(x) = \infty$

Если $\exists \lim\limits_{x \to a} \frac{f'(x)}{g'(x)} = A \implies \exists \lim\limits_{x \to a} \frac{f(x)}{g(x)} = A$

\subsection{Формула Тейлора}

Пусть $f(x)$ имеет непр. произв. до  $(n + 1)$ пор.

$P_n(x)$ многочлен степени не выше $n$: $P_n(a) = f(a), P_n'(a) = f'(a)$

$P_n(x) = f(a) + \frac{f'(a)(x - a)}{1!} + \frac{f''(a) (x-a)^2}{2!} + ... + \frac{f^{(n)}(a)(x - a)^n}{n!}, R_n(x) = \frac{(x - a)^{n + 1}}{(n + 1)!} f^{(n + 1)} (\xi), a < \xi < x$

\paragraph{Формула Маклорена}

$P_n(x) = f(0) + \frac{f'(0)}{1!}x + \frac{f''(0)}{2!} x^2 + ... + \frac{f^{(n)}(0)}{n!}x^n$

\paragraph{Примеры вывода формулы}

\hfill

\rule{\textwidth}{0.4pt}

\parbox{0.3\textwidth}{
$y = \sin x, y(0) = 0$

$y'(0) = 1$

$y''(0) = 0$

$y'''(0) = -1 $

$y''''(0) = 0$

$y'''''(0) = 1$
}
\parbox{0.3\textwidth}{
$y = \cos x, y(0) = 1$

$y'(0) = $

$y''(0) = $

$y'''(0) = $

$y''''(0) = $

$y'''''(0) = $
}
\parbox{0.3\textwidth}{
$y = e^x, y(0) = 1$

$y'(0) = 1$

$y''(0) = 1$

$y'''(0) = 1$

$y''''(0) = 1$

$y'''''(0) = 1$
}

\rule{\textwidth}{0.4pt}

Отсюда $\sin x = 0 + x + 0x^2 - \frac{1}{3!}x^3 + 0x^4 + \frac{1}{5!}x^5 + ... = x - \frac{x^3}{3!} + \frac{x^5}{5!} - ...$, $\cos x = 1 - \frac{x^2}{2!} + \frac{x^4}{4!} - ...$, $e^{x} = 1 + x + \frac{x^2}{2!} + \frac{x^3}{3!} + ...$

Интересно рассмотреть натуральный логарифм: $\ln (1 + x) = x - \frac{x^2}{2} + \frac{x^3}{3} - \frac{x^4}{4}$

Также $(1 + x)^m = 1 + m X + \frac{m(m - 1}{2!}x^2 + \frac{m(m - 1)(m - 2)}{3!}x^3 + ...$, $\sqrt{1 + x} = 1 + \frac{x}{2} - \frac{1}{2^2} \frac{x^2}{2!} + \frac{1 * 3 * x^3}{2^3 * 3!} - \frac{1 * 3 * 5 * x^4}{2^4 * 4!}$

$\cos^2 x = \frac{1}{2} (\cos 2x + 1) = \frac{1}{2} (1 + 1 - \frac{(2x)^2}{2!} + \frac{(2x)^4}{4!} - \frac{(2x)^6}{6!} + ...$

\paragraph{Применение данных формул к вычислению пределов}

Допустим, имеем $\lim\limits_{x \to 0} \frac{x - \sin x}{x - \tg x} = ...$

Распишем разложение в Тейлора: $... = \lim\limits_{x \to 0} \frac{x - x + \frac{x^3}{3!} + o(x^3)}{x - x - \frac{x^3}{3} + o(x^3)} = -\frac{1}{2}$

\end{flushleft}

\end{document}