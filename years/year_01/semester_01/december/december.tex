\documentclass{article}
\usepackage[utf8]{inputenc}

\usepackage[T2A]{fontenc}
\usepackage[utf8]{inputenc}
\usepackage[russian]{babel}

\usepackage{amsmath}
\usepackage{pgfplots}
\usepackage{multienum}
\newcommand*\diff{\mathop{}\!\mathrm{d}}

\usepackage{hyperref}
\hypersetup{
    colorlinks, citecolor=black, filecolor=black, linkcolor=black, urlcolor=black
}

\DeclareMathOperator{\sign}{sign}
\newtheorem{definition}{Определение}
\newtheorem{theorem}{Теорема}

\title{Высшая математика}
\author{Лисид Лаконский}
\date{December 2022}

\begin{document}

\maketitle

\tableofcontents
\pagebreak

\section{Высшая математика - 09.12.2022}

\subsection{Функции нескольких переменных}

\begin{flushleft}

Для функций нескольких переменных существуют лишь частные производные; пусть $z = z(x; y)$, то $\frac{\delta z}{\delta x} = \lim\limits_{\Delta x \to 0} \frac{z(x + \Delta x, y) - z(x, y)}{\Delta x}$; $\frac{\delta z}{\delta y} = \lim\limits_{\Delta y \to 0} \frac{z(x, y + \Delta y) - z(x, y)}{\Delta y}$

\hfill

Пусть $z = 2xy + xy^3$. тогда $\frac{\delta z}{\delta x} = \lim\limits_{\Delta x \to 0} \frac{2(x + \Delta x)y + (x + \Delta x)y^3 - 2xy - xy^3}{\Delta x} = \lim\limits_{\Delta x to 0} \frac{2xy + 2 \Delta x y + xy^3 + \Delta x y^3 - 2xy - xy^3}{\Delta x} = 2y + y^3$; $\frac{\delta z}{\delta y} = \lim\limits_{\Delta y \to 0} \frac{2x(y + \Delta y) + x(y + \Delta y)^3 - 2xy - xy^3}{\Delta y} = \lim\limits_{\Delta y \to 0} \frac{2xy + 2x \Delta y + xy^3 + 3xy^2 \Delta y + 3xy (\Delta y)^2 + x (\Delta y)^3 - 2xy - xy^3}{\Delta y} = \lim\limits_{\Delta y \to 0} 2x + 3xy^2 + 3xy \Delta y + x (\Delta y)^2 = 2x + 3xy^2$

\subsubsection{Полный дифференциал}

$du = \frac{\delta u}{\delta x} \delta x + \frac{\delta u}{\delta y} \delta y$ — \textbf{полный дифференциал} функции от двух переменных

\paragraph{Применение полного дифференциала к вычислению приближенных значений}

\hfill

\hfill

$f(x + \Delta x) = f(x) + \delta f \approx f(x_0) + d f$

Например, нам нужно вычислить $1.04^{2.02}$, тогда составим функцию $z = x^y$, $x_0 = 1$, $y_0 = 2$, $\delta x = d x = 0.04$, $\delta y = d y = 0.02$

$z(1.04; 2.02) \approx z(1; 2) + d z = 1 + \frac{\delta z}{\delta x} dx + \frac{\delta z}{\delta y} dy = 1 + yx^{y - 1} dx + x^y \ln x d y = 1 + 2 * 1 * 0.04 + 1^2 \ln 1 * 0.02 = 1.08$

\subsubsection{Частные производные}

Частные производные $\frac{\delta z}{\delta x}$ и $\frac{\delta z}{\delta y}$ тоже являются функциями, и поэтому от них можно брать частные производные.

\begin{theorem}

Если функция и ее частные производные определены и непрерывны в точке $M$ и некоторой ее окрестности, то в этой точке выполняется условие: $\frac{\delta}{\delta x} (\frac{\delta z}{\delta y}) = \frac{\delta}{\delta y} (\frac{\delta z}{\delta x})$

\end{theorem}

Порядок взятия частных производных не имеет значения:

$$\frac{\delta^n z}{\delta x^k \delta y^{n - k}} = \frac{\delta^n z}{\delta y^{n - k} \delta x^{k}}$$

\subsubsection{Производная функции, заданной неявно}

\begin{theorem}
    Пусть непрерывная функция $y = y(x)$ задана неявно уравнением $F(x, y) = 0$, причем сама эта функция и ее первые производные — непрерывные функции в некоторой области, $F'_y \ne 0$ в интересующей нас точке, тогда $y'_x = \frac{-F'_x(x, y)}{F'_y(x, y)}$
\end{theorem}

Например, у нас есть функция $x^2 + x \sin y = 0$ ($F(x, y) = x^2 + x \sin y$), тогда возьмем производные по $x$ и $y$: $F'_x = 2 x y + \sin y$, $F'_y = x^2 + x \cos y$, $y'_x = -\frac{2 x y + \sin y}{x^2 + x \cos y}$

\subsubsection{Производная сложной функции и понятие полной производной}

Пусть $z = z(u; v)$, $u = u(x; y)$, $v = v(x; y)$, и существуют непрерывные частные производные $z$ по $u; v$, $u, v$ по $x; y$, тогда мы можем рассматривать $z$ как функцию от $x$ и $y$: $z = z(u(x, y), v(x, y))$, но не всегда так делать целесообразно и поступать лучше следующим образом:

$$\frac{\delta z}{\delta x} = \frac{\delta z}{\delta u} \frac{\delta u}{\delta x} + \frac{\delta z}{\delta v} \frac{\delta v}{\delta x}, \frac{\delta z}{\delta y} = \frac{\delta z}{\delta u} \frac{\delta u}{\delta y} + \frac{\delta z}{\delta v} \frac{\delta v}{\delta y}$$

Но функция может быть и от большего количества переменных: $z = z(x; y; t)$, $x = x(t)$, $y = y(t)$, по аналогии $z = z(x(t), y(t), t)$ — есть зависимость лишь от $t$, можно говорить о полной производной:

$$
\frac{d z}{d t} = \frac{\delta z}{\delta x} \frac{\delta x}{\delta t} + \frac{\delta z}{\delta y} \frac{\delta y}{\delta t} + \frac{\delta z}{\delta t}
$$

Например, $z = u\sqrt{v} + v u^2$, $u = \sin(x + y)$, $v = \sqrt{x^2 + y}$, $\frac{\delta z}{\delta u} = \sqrt{v} = 2v u$, $\frac{\delta z}{\delta v} = \frac{u}{2 \sqrt{v}} + u^2$, $\frac{\delta u}{\delta x} = \cos (x + y)$, $\frac{\delta u}{\delta y} = \cos (x + y)$, $\frac{\delta v}{\delta x} = \frac{2x}{2\sqrt{x^2 + y}}$, $\frac{\delta v}{\delta y} = \frac{1}{2\sqrt{x^2 + y}}$

$\frac{\delta z}{\delta x} = (\sqrt{v} + 2 v u) \cos (x + y) + (\frac{u}{2\sqrt{v}} + u^2) * \frac{x}{\sqrt{x^2 + y}}$

\hfill

Другой не менее славный пример: $z = t e^{x - 2y} + x t^2$, $x = \sin t$, $y = t^3$, мы желаем посчитать $\frac{\delta z}{\delta t}$, для этого посчитаем много различной фигни: $\frac{\delta z}{\delta x} = t e^{x - 2y} + t^2$, $\frac{\delta z}{\delta y} = t e^{x - 2y} (-2)$, $\frac{\delta z}{\delta t} = e^{x - 2y} + 2 x t$, $\frac{\delta z}{\delta t} = (t e^{x - 2y} + t^2) \cos t - 2 t e^{x - 2 y} * 3t^2 + e^{x -2 y} + 2 xt$

\subsubsection{Производная по направлению}

Пусть $D$ — некоторое пространство, определяющееся функцией $u = u(x; y; z)$, и т. $M(x; y; z)$, т. $M_1(x + \Delta x, y + \Delta y, z + \Delta z)$ лежат в данном пространстве, от нее отложен некоторый $\vec{S} = \{ \cos \alpha, \cos \beta, \cos \gamma \}$, при этом $\Delta S = \sqrt{\Delta x^2 + \Delta y^2 + \Delta z^2}$

$\Delta u = \frac{\delta u}{\delta x} \Delta x + \frac{\delta u}{\delta y} \Delta y + \frac{\delta u}{\delta z} \Delta z + E_1 \Delta x + E_2 \Delta y + E_3 \Delta z$

$\frac{\Delta u}{\Delta s} = \frac{\delta u}{\delta x} + \frac{\Delta x}{\Delta S} + \frac{\delta u}{\delta y} \frac{\Delta y}{\Delta S} + \frac{\delta u}{\delta z} \frac{\Delta z}{\Delta S} + \dots$

$\frac{\delta u}{\delta s} = \lim\limits_{\Delta S \to 0} \frac{\Delta u}{\Delta S} = \frac{\delta u}{\delta x} \cos \alpha + \frac{\delta u}{\delta y} \cos \beta + \frac{\delta u}{\delta z} \cos \gamma$

\paragraph{Пример} $M \vec{M_1} = \{ 3; 4 \} = \{\frac{3}{5}; \frac{4}{5} \}$, $\mu = \frac{1}{\sqrt{9 + 16}} = \frac{1}{5}$

Пусть $u(x; y) = x^2 + 3\sqrt{y}$, т. $M(1; 1)$, т. $M_1(4; 5)$; $M \to M_1$ в точке $M$, в этой точке $\frac{\delta u}{\delta x} = 2x = 2$, $\frac{\delta u}{\delta y} = \frac{3}{2\sqrt{y}} = \frac{3}{2}$

\subsubsection{Градиент функции}

В каждой точке области $D$, в которой задана функция $u = u(x, y, z)$, определим вектор, координатами которого являются значения частных производных, и назовем его \textbf{градиентом функции}:

$$grad u = \frac{\delta u}{\delta x} \overrightarrow{c} + \frac{\delta u}{\delta y} \overrightarrow{j} + \frac{\delta u}{\delta z} \overrightarrow{k} $$

$\frac{\delta u}{\delta s} = \text{Пр}_{\overrightarrow{s}} (grad u)$ — производная по направлению в данной точке имеет наибольшее значение, если направление $\overrightarrow{s}$ совпадает с направлением градиента функции, равное модулю этого градиента.

\subsubsection{Локальный экстремум функции двух переменных}

Функция $z = z(x; y)$ \textbf{имеет локальный максимум} в точке $M_0(x_0; y_0)$, если $z(x_0, y_0) > z(x, y)$ в окрестности точки $M(x, y) \ne M_0(x_0, y_0)$, но достаточно близких к ней.

Функция $z = z(x; y)$ \textbf{имеет локальный минимум} в точке $M_0(x_0; y_0)$, если $z(x_0, y_0) < z(x, y)$ в окрестности точки $M(x, y) \ne M_0(x_0, y_0)$, но достаточно близких к ней.

\begin{theorem}[необходимое условие локального экстремума]
    Если функция $z( = zx; y)$ достигает экстремума в точке $M_0(x_0, y_0)$, то каждая частная производная или обращается в ноль в этой точке, или не существует: $\frac{\delta z}{\delta x} = 0$, $\frac{\delta z}{\delta y} = 0$ (*), и такие точки называются стационарными (точками возможного экстремума)
\end{theorem}

\begin{theorem}[достаточное условие локального экстремума]
    Пусть в некоторой области, содержащей точку $M_0(x_0, y_0)$, выполнены условия (*), и функция $z = z(x; y)$ имеет непрерывные частные производные до третьего порядка включительно: $A = \frac{\delta^2 z}{\delta x^2}$, $B = \frac{\delta^2 z}{\delta x \delta y}$, $C = \frac{\delta^2 z}{\delta y^2}$, $\Delta = \begin{vmatrix}
        A & B \\
        B & C
    \end{vmatrix} = AC - B^2$, то если $\Delta > 0$ — экстремум есть ($A(C) < 0$ — максимум, $A(C) > 0$ — минмум), $\Delta < 0$ — экстремума не существует, $\Delta = 0$ — спорный случай, который стоит рассматривать отдельно
\end{theorem}

\end{flushleft}

\pagebreak
\section{Высшая математика - 13.12.2022}

\subsection{Функции нескольких переменных}

\begin{flushleft}

\begin{definition}
Если заданы два непустых множества $D$ и $G$ и каждому элементу $M$ множества $D$ по определенному правилу ставится в соответствие один и только один элемент множества $G$, то говорят, что на области определения задана функция со множеством значений $G$
\end{definition}

\textbf{Область определения} представляет собой часть координатной плоскости, ограниченной плоской кривой.

\hfill

$Z = \sqrt{1 - x^2 - y^2} \Longleftrightarrow Z^{2} = 1 - x^2 - y^2, 1 - x^2 - y^2 \ge 0 \Longleftrightarrow x^2 + y^2 \le 0$

\begin{definition}
    Линией уровня функции двух переменных называется линия на координатной плоскости, где функция сохраняет постоянное значение
\end{definition}

\begin{definition}
    Поверхностью уровня функции двух переменных называется поверхность, в точках которых функция сохраняет постоянное значение
\end{definition}

$f = \frac{y}{x}, \frac{y}{x} = c \Longleftrightarrow y = cx$ — линия уровня плоскости.

\subsubsection{Частные производные}

\paragraph{Пример 1}

\hfill

\hfill

$\frac{\delta^2 z}{\delta x \delta y}$, если $z(x, y) = \frac{x}{3y + 2x^2}$ в т. $V(1, 0)$

$\frac{\delta z}{\delta x} = \frac{1(3y + 2x^2) - x(4x)}{(3y + 2x^2)^2} = \frac{3y - 2x^2}{9y^2 + 12x^2 y + 4x^4}$

$\frac{\delta^2 z}{\delta x \delta y} = \frac{3(9y^2 + 12x^2 y + 4x^4) - (3y - 2x^2)(18y + 12x^2)}{(9y^2 + 12x^2 y + 4x^4)^2} = \frac{3 * 4 - (-2) * 12}{16} = \frac{9}{4}$

\paragraph{Пример 2}

\hfill

\hfill

Найти $\frac{\delta x}{\delta y}$ для $x^2 + 2xyz - \frac{z}{x} - 2yz^2 = 0$ в т. $M(2; 0; 8)$

$\frac{\delta f}{\delta y} = 2xz - 2z^2$; $\frac{\delta f}{\delta z} = 2zy - \frac{1}{x} - 4yz$; $\frac{\delta f}{\delta x} = 2x + 2yz - \frac{(x - z)}{x^2}$

$\frac{\delta z}{\delta y} = \frac{\frac{- \delta f}{\delta y}}{\frac{\delta f}{\delta z}}$


\paragraph{Пример 3}

\hfill

\hfill

Найти $\frac{\delta^2 z}{\delta x \delta y}$, если $z(x, y) = \frac{x^2}{x - 26}$ в т. $M(1; 0)$

$\frac{\delta z}{\delta x} = \frac{(x^2)' * (x - 2y) - x^2(x - 2y)'}{(x - 2y)^2} - \frac{2x * (x - 2y) - x^2}{(x - 2y)^2} = \frac{2x^2 - 4x y}{(x - 2y)^2}$

$\frac{\delta^2 z}{\delta x}{\delta y} = \frac{(2x^2 - 4xy)' * (x - 2y)^2 - (2x^2 - 4xy)((x - 2y)^2)'}{(x - 2y)^4} = \frac{(-4) * (x - 2y)^2 - (2x^2 - 4xy) * (x^2 - 4x y + 4y^2)'}{(x - 2y)^2} = \frac{-4 - 2 * (-4)}{1} = -4 + 8 = 4$

\end{flushleft}

\end{document}