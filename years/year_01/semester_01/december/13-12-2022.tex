\documentclass{article}
\usepackage[utf8]{inputenc}

\usepackage[T2A]{fontenc}
\usepackage[utf8]{inputenc}
\usepackage[russian]{babel}

\usepackage{amsmath}
\usepackage{pgfplots}
\usepackage{multienum}
\newcommand*\diff{\mathop{}\!\mathrm{d}}

\usepackage{hyperref}
\hypersetup{
    colorlinks, citecolor=black, filecolor=black, linkcolor=black, urlcolor=black
}

\DeclareMathOperator{\sign}{sign}
\newtheorem{definition}{Определение}
\newtheorem{theorem}{Теорема}

\title{Высшая математика}
\author{Лисид Лаконский}
\date{December 2022}

\begin{document}

\maketitle

\tableofcontents
\pagebreak

\section{Высшая математика - 13.12.2022}

\subsection{Функции нескольких переменных}

\begin{flushleft}

\begin{definition}
Если заданы два непустых множества $D$ и $G$ и каждому элементу $M$ множества $D$ по определенному правилу ставится в соответствие один и только один элемент множества $G$, то говорят, что на области определения задана функция со множеством значений $G$
\end{definition}

\textbf{Область определения} представляет собой часть координатной плоскости, ограниченной плоской кривой.

\hfill

$Z = \sqrt{1 - x^2 - y^2} \Longleftrightarrow Z^{2} = 1 - x^2 - y^2, 1 - x^2 - y^2 \ge 0 \Longleftrightarrow x^2 + y^2 \le 0$

\begin{definition}
    Линией уровня функции двух переменных называется линия на координатной плоскости, где функция сохраняет постоянное значение
\end{definition}

\begin{definition}
    Поверхностью уровня функции двух переменных называется поверхность, в точках которых функция сохраняет постоянное значение
\end{definition}

$f = \frac{y}{x}, \frac{y}{x} = c \Longleftrightarrow y = cx$ — линия уровня плоскости.

\subsubsection{Частные производные}

\paragraph{Пример 1}

\hfill

\hfill

$\frac{\delta^2 z}{\delta x \delta y}$, если $z(x, y) = \frac{x}{3y + 2x^2}$ в т. $V(1, 0)$

$\frac{\delta z}{\delta x} = \frac{1(3y + 2x^2) - x(4x)}{(3y + 2x^2)^2} = \frac{3y - 2x^2}{9y^2 + 12x^2 y + 4x^4}$

$\frac{\delta^2 z}{\delta x \delta y} = \frac{3(9y^2 + 12x^2 y + 4x^4) - (3y - 2x^2)(18y + 12x^2)}{(9y^2 + 12x^2 y + 4x^4)^2} = \frac{3 * 4 - (-2) * 12}{16} = \frac{9}{4}$

\paragraph{Пример 2}

\hfill

\hfill

Найти $\frac{\delta x}{\delta y}$ для $x^2 + 2xyz - \frac{z}{x} - 2yz^2 = 0$ в т. $M(2; 0; 8)$

$\frac{\delta f}{\delta y} = 2xz - 2z^2$; $\frac{\delta f}{\delta z} = 2zy - \frac{1}{x} - 4yz$; $\frac{\delta f}{\delta x} = 2x + 2yz - \frac{(x - z)}{x^2}$

$\frac{\delta z}{\delta y} = \frac{\frac{- \delta f}{\delta y}}{\frac{\delta f}{\delta z}}$


\paragraph{Пример 3}

\hfill

\hfill

Найти $\frac{\delta^2 z}{\delta x \delta y}$, если $z(x, y) = \frac{x^2}{x - 26}$ в т. $M(1; 0)$

$\frac{\delta z}{\delta x} = \frac{(x^2)' * (x - 2y) - x^2(x - 2y)'}{(x - 2y)^2} - \frac{2x * (x - 2y) - x^2}{(x - 2y)^2} = \frac{2x^2 - 4x y}{(x - 2y)^2}$

$\frac{\delta^2 z}{\delta x}{\delta y} = \frac{(2x^2 - 4xy)' * (x - 2y)^2 - (2x^2 - 4xy)((x - 2y)^2)'}{(x - 2y)^4} = \frac{(-4) * (x - 2y)^2 - (2x^2 - 4xy) * (x^2 - 4x y + 4y^2)'}{(x - 2y)^2} = \frac{-4 - 2 * (-4)}{1} = -4 + 8 = 4$

\end{flushleft}

\end{document}