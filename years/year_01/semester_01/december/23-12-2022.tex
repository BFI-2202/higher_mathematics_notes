\documentclass{article}
\usepackage[utf8]{inputenc}

\usepackage[T2A]{fontenc}
\usepackage[utf8]{inputenc}
\usepackage[russian]{babel}

\usepackage{amsmath}
\usepackage{pgfplots}
\usepackage{multienum}
\newcommand*\diff{\mathop{}\!\mathrm{d}}

\usepackage{hyperref}
\hypersetup{
    colorlinks, citecolor=black, filecolor=black, linkcolor=black, urlcolor=black
}

\DeclareMathOperator{\sign}{sign}
\newtheorem{definition}{Определение}
\newtheorem{theorem}{Теорема}

\title{Высшая математика}
\author{Лисид Лаконский}
\date{December 2022}

\begin{document}

\maketitle

\tableofcontents
\pagebreak

\section{Высшая математика - 23.12.2022}

\subsection{Условный экстремум}

\begin{flushleft}

$z = z(x, y)$, $\phi (x; y) = 0$ — условие связи

\subsubsection{Функция Лагранжа}

$L(x; y; \lambda) = z(x, y) + \lambda \phi (x, y)$

\begin{equation}
    \begin{cases}
        \frac{\delta L}{-\delta x} = 0 \\
        \frac{\delta L}{\delta y} = 0 \\
        \phi (x, y) = 0
    \end{cases}
\end{equation}

$\delta^2 L$ — дифференциал второго порядка, исследуем его значение и выясняем, есть ли экстремум или его нет.

\end{flushleft}

\subsection{Наибольшее и наименьшее значение в области}

\begin{flushleft}

Наибольшее и наименьшее значение в замкнутой области будет достигаться \textbf{либо в точках экстремума} в этой замкнутой области, \textbf{либо на границе} этой области.

\paragraph{Пример} Найти наибольшее и наименьшее значение функции $z = x^2 + 2 x y - 4 x + 8 y$ в прямоугольнике, ограниченном прямыми $x = 0$, $y = 0$, $x = 1$, $y = 2$

\begin{equation}
\begin{cases}
    \frac{\delta z}{\delta x} = 2x + 2y - 4 = 0 \\
    \frac{\delta z}{\delta y} = 2x + 8 = 0
\end{cases} \Longleftrightarrow
\begin{cases}
    x = -4 \\
    y = 6
\end{cases}
\end{equation}

Пусть точка $C(0; 2)$, $B(1; 2)$, $A(1; 0)$, $O(0; 0)$

\hfill

$OA$: $y = 0$, $z = x^2 - 4x$, $z(0; 0) = 0$, $z(1; 0) = -3$ — \textbf{наименьшее значение}

$AB$: $z = 10y - 3$, $z(1; 2) = 17$ — \textbf{наибольшее значение}

$BC$: $y = 2$, $z = x^2 + 16$, $z(0; 2) = 16$

$CO$: $x = 0$, $z = 8y$

\paragraph{Более сложный пример} Найти наибольшее и наименьшее значение функции $x^2 + y^2 = 4$

$L = x^2 + 2xy - 4x + 8y + \lambda (x^2 + y^2 - 4) = 0$

\begin{equation}
\begin{cases}
    \frac{\delta L}{\delta x} = 2x + 2y - 4 + 2 \lambda x = 0 \\
    \frac{\delta L}{\delta y} = 2x + 8 + 2 \lambda y = 0 \\
    x^2 + y^2 - 4 = 0
\end{cases}
\end{equation}

\end{flushleft}

\subsection{Уравнение касательной плоскости и нормали к поверхности}

\subsubsection{Уравнение касательной плоскости к поверхности}

\paragraph{При задании поверхности неявяно} Имеем поверхность, заданную в неявном виде $F(x, y, z) = 0$, чтобы найти касательную плоскость к поверхности в данной точке $M_0(x_0, y_0, z_0)$, запишем

$$A(x - x_0) + B(y - y_0) + C(z - z_0) = 0$$

Где $A = \frac{\delta F}{\delta x}$ в заданной точке, $B = \frac{\delta F}{\delta y}$ в заданной точке, $C = \frac{\delta F}{\delta z}$ в заданной точке

\paragraph{При задании поверхности неявно} Имеет поверхность, заданную $z = f(x, y)$, $F(x, y, z) = f(x, y) - z = 0$

\subsubsection{Уравнение нормали к поверхности}

$$
\frac{x - x_0}{\frac{\delta F}{\delta x}} = \frac{y - y_0}{\frac{\delta F}{\delta y}} = \frac{z - z_0}{\frac{\delta F}{\delta z}} \Longleftrightarrow
\frac{x - x_0}{\frac{\delta F}{\delta x}} = \frac{y - y_0}{\frac{\delta F}{\delta y}} = \frac{z - z_0}{-1}
$$

\end{document}